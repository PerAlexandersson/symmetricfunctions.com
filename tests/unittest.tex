\metatitle{Unittest}

\metadescription{A unit test for most features.}

% Latex comments are removed.

\section[testFamily]{Test polynomials}

\begin{polydata}{testFamily}
  Name   & Test polynomials \\
  Space  & Sym \\
  Basis  & Yes \\
  Rating & 8 \\
  Bib    & Jacobi1841 \\
  Year   & 1841 \\
\end{polydata}

\todo{This is a todo note and is removed.}


A \defin{definition}. \emph{Emphasis} \texttt{texttt}.

\textbf{En dashes:} Murnaghan--Nakaygama.
\textbf{Em dashes:} An example --- very nice.

\begin{itemize}
\item One single cite:  \cite{Cauchy1815}
\item One single cite: \cite[p. 15]{Schur1901}.
\item Several cites: \cite{Cauchy1815,Macdonald1995,StanleyEC2}.
\end{itemize}

Here \hyperref[testFamily]{is an internal link},
and here is a link with title: \href{https://en.wikipedia.org/wiki/Lindstr%C3%B6m%E2%80%93Gessel%E2%80%93Viennot_lemma}{TestURL}.
One can also do \url{www.google.com}.

We can link OEIS: \oeis{A000128}


\begin{blockquote}
The best unittest is the best.
\end{blockquote}

So I said: \enquote{This is my quote}.

\begin{figure}
\ytableaushort{{\none}{\none}23,{\none}{\none}4,11}
\ytableaushort{{\none}{\none}23,{\none}14,1}
\ytableaushort{{\none}{\none}23,114}
\ytableaushort{{\none}123,1{\none}4}
\ytableaushort{1123,{\none}{\none}4}
\ytableaushort{1123,4}
\end{figure}


\begin{conjecture}[Test conjecture, 2012]
All conjectures are either true or false.
\end{conjecture}
\begin{proof*}
Example of an expandable proof.
\end{proof*}



\begin{example}[Example environment]
Here is example. We have \defin{a thing} defined here.


Also a figure with \textbf{ytableau} inside

\begin{figure}
\begin{ytableau}
\none & \none & & & & & & h\\
\none & \none & \\
\none & \none & \\
\none &  & \\
\none &  \\
t &  \\
\end{ytableau}
\end{figure}

\end{example}

Here is an array:
\begin{array}{rrrrrr}
\toprule
 1  \\
 1 & 2  \\
 1 & 8 & 6 \\
 1 & 22 & 58 & 24 \\
 1 & 52 & 328 & 444 & 120 \\
 1 & 114 & 1452 & 4400 & 3708 & 720 \\
\bottomrule
\end{array}



\begin{example*}[Mathematica code]

Here is some inline code: \texttt{HypergeometricPFQ[{-n, -n - 1}, {2}, x]}
and a code block:

\begin{lstlisting}
NN[1] := t;
NN[2] := t (1 + t);
NN[n_] := NN[n] = Expand[
	(
		(2 n - 1) (1 + t) NN[n - 1] -
		(n - 2) (1 - t)^2 NN[n - 2]
)/(n + 1)];
\end{lstlisting}

\end{example*}


\subsection[testSubsection]{Subsection test}


\begin{example}[Example with tableaux]
Here is example.

\begin{figure}
\begin{ytableau}
1 & 1 & 1 \\
2 & 2
\end{ytableau}
\begin{ytableau}
1 & 1 & 1 \\
3 & 3
\end{ytableau}
\begin{ytableau}
1 & 1 & 1 \\
2 & 3
\end{ytableau}
\begin{ytableau}
1 & 1 & 2 \\
2 & 2
\end{ytableau}
\begin{ytableau}
1 & 1 & 3 \\
3 & 3
\end{ytableau}
\begin{ytableau}
1 & 1 & 2 \\
3 & 3
\end{ytableau}
\begin{ytableau}
1 & 1 & 3 \\
2 & 3
\end{ytableau}
\end{figure}
\end{example}


\begin{example}[Example with ytableaushort]
Blah

\begin{figure}
\ytableaushort{1234}
\ytableaushort{123,4}
\ytableaushort{124,3}
\ytableaushort{134,2}
\ytableaushort{12,34}
\ytableaushort{13,24}
\ytableaushort{12,3,4}
\ytableaushort{13,2,4}
\ytableaushort{14,2,3}
\ytableaushort{1,2,3,4}
\end{figure}

\begin{figure}
\begin{ytableau}
 \circ & & & & & & \\
 & \circ & & & \\
 & & \circ & & \\
 & & & \circ \\
 & &
\end{ytableau}
\begin{ytableau}
\none & \none & & & & & & h\\
\none & \none & \\
\none & \none & \\
\none &  & \\
\none &  \\
t &  \\
\end{ytableau}
\end{figure}

\end{example}



The following is a theorem.
\begin{theorem}
Let $|\nu| = |\lambda|+|\mu|$. Then
\[
g( (n,\nu),\; (n+|\lambda|,\mu),\; (n+|\mu|,\lambda) ) = c^{\nu}_{\lambda \mu}.
\]
\end{theorem}


\begin{problem}\label{problem:cool}
Open problem statement goes here.
\end{problem}
\begin{solution*}
Example of a solution.
\end{solution*}


\begin{theorem}[Alice and Bob, \cite{StanleyEC2}]
Statement of theorem.
\end{theorem}
\begin{proof}
Example of a proof.
\end{proof}



\begin{enumerate}
\item This is the first item.
\item This is the second.
\end{enumerate}


\begin{itemize}
\item This is the first bullet point.
\item This is the second bullet point.
\end{itemize}


\section[sectionID]{Section--test $\setN$}


Here is an array:

\begin{array}{rrrrrr}
\toprule
 \; & \textbf{4} & \textbf{31} & \textbf{22} & \textbf{211} & \textbf{1111} \\
\midrule
 \textbf{4} & 1 & 0 & 0 & 0 & 0 \\
 \textbf{31} & q & 1 & 0 & 0 & 0 \\
 \textbf{22} & q^2 & q & 1 & 0 & 0 \\
 \textbf{211} & q^3 & q^2+q & q & 1 & 0 \\
 \textbf{1111} & q^6 & q^5+q^4+q^3 & q^4+q^2 & q^3+q^2+q & 1 \\
 \bottomrule
\end{array}



