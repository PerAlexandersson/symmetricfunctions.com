\documentclass[tikz, border=10pt]{standalone}
\usepackage{amsmath,amssymb,amsthm,mathrsfs,latexsym,mathtools,mathdots,tikz}
\usetikzlibrary{arrows,matrix,decorations.pathmorphing,decorations.markings}

\newcommand{\atom}{\mathcal{A}}
\newcommand{\gessel}{\mathrm{L}}
\newcommand{\grothG}{\mathfrak{G}}%Grothendieck
\newcommand{\sgrothG}{\mathrm{G}} %Stable Grothendieck
\newcommand{\grothg}{\mathrm{g}}  %Dual stable  Grothendieck
\newcommand{\hlPolyP}{\mathrm{P}}
\newcommand{\hlPolyQ}{\mathrm{Q}}
\newcommand{\jackJ}{\mathrm{J}}
\newcommand{\key}{\mathcal{K}}
\newcommand{\LLT}{\mathrm{G}}
\newcommand{\modMacdonald}{\tilde{H}}
\newcommand{\nonSymMacdonaldE}{E}
\newcommand{\macdonaldJ}{\mathrm{J}}
\newcommand{\macdonaldP}{\mathrm{P}}
\newcommand{\monomM}{\mathrm{m}}
\newcommand{\schubert}{\mathfrak{S}}
\newcommand{\schurS}{\mathrm{s}}
\newcommand{\qSymSchur}{\mathcal{S}}
\newcommand{\schurP}{\mathrm{P}}
\newcommand{\schurQ}{\mathrm{Q}}
\newcommand{\stanley}{\mathrm{F}}
\newcommand{\superS}{\mathcal{S}}
\newcommand{\zonal}{\mathrm{Z}}
\newcommand{\xvec}{\mathbf{x}}
\newcommand{\yvec}{\mathbf{y}}
\newcommand{\nuvec}{{\boldsymbol\nu}}

\begin{document}


\begin{tikzpicture}[xscale=3.8,yscale=2.3]
\tikzset{
    vertex/.style = {
        draw,
	align=left,
	fill            = blue!50!white!30,
        outer sep = 2pt,
        inner sep = 3pt,
	minimum height = 1cm
    },
    symvertex/.style={fill= red!50!yellow!30},
    pluses/.style={
	dashed, decoration={markings,
	mark=between positions 1.5pt and 1 step 6pt with {
	\draw[-] (0,1.5pt) -- (0,-1.5pt);
       }
    },
    postaction=decorate,
  },

  ExpandsPositiveIn/.style = {thick,->,black},
  SpecializesTo/.style = {thick,-{>},black},
  SupersetOf/.style = {double equal sign distance,-implies},
  Misc/.style = {thick,-{>},blue,dotted},
  KAnalogueOf/.style = {dashed,<-{<}, black},

  subst/.style =  {<-{>},line join=round,decorate, decoration={zigzag,segment length=4,amplitude=.9,post=lineto,post length=2pt},thick}
}


\node[vertex] (quasiSchur)	at ( 4, 0) {Quasisymmetric Schur $\qSymSchur_\alpha(\xvec)$};
\node[vertex] (atom)		at ( 1, -1) {Demazure atom $\atom_\alpha(\xvec)$};
\node[vertex] (key)		at ( 1, 2) {Key $\key_\alpha(\xvec)$};
\node[vertex] (generalAtom)	at ( -0.5, 1) {General Atom $\atom^\sigma_\alpha(\xvec)$};
\node[vertex] (pbMacdonaldE)	at ( 2, 7) {Permuted basement Macdonald $\nonSymMacdonaldE^\sigma_\alpha(\xvec;q,t)$};
\node[vertex,symvertex] (schurS)		at ( 4, 1) {Schur $\schurS_\lambda(\xvec)$};
\node[vertex] (schubert)		at ( 2, 3) {Schubert, $\schubert_w(\xvec)$};
\node[vertex,symvertex] (stanleySym)	at ( 2, 2) {Stanley, $\stanley_w(\xvec)$};
\node[vertex,symvertex] (modmacdonaldH)	at ( 6, 4) {Modified Macdonald $\modMacdonald_\lambda(\xvec;q,t)$};
\node[vertex,symvertex] (LLT)		at ( 6, 2) {LLT, $\LLT_{\nuvec}(\xvec;q)$};
\node[vertex] (doubleGrothendieck) at ( 10, 4) {Double Grothendieck, $\grothG_w(\xvec,\yvec)$};
\node[vertex] (factorialGrothendieck)	at ( 8, 6) {Factorial Grothendieck, $\sgrothG_\lambda(\xvec|\yvec)$};
\node[vertex] (grothendieck)	at ( 10, 3) {Grothendieck, $\grothG_w(\xvec)$};
\node[vertex,symvertex] (stableGrothendieck)at ( 10, 2) {Stable Grothendieck, $\sgrothG_w(\xvec)$};
\node[vertex,symvertex] (stableGrothendieck2)at ( 10, 1) {Stable Grothendieck, $\sgrothG_\lambda(\xvec)$};
\node[vertex] (shiftedJack)	at ( 5, 3) {Shifted Jack, $\jackJ^\star_\lambda(\xvec;a)$};
\node[vertex] (shiftedSchur)	at ( 5, 2) {Shifted Schur $\schurS^\star_\lambda(\xvec)$};
\node[vertex,symvertex] (jackJ)		at ( 8, 2) {Jack, $\jackJ_\lambda(\xvec;a)$};
\node[vertex] (doubleSchubert)	at ( 3, 4) {Double Schubert $\schubert_w(\xvec,\yvec)$};
\node[vertex] (doubleSchur)	at ( 3, 3) {Double Schur, $\schurS_\lambda(\xvec||\yvec)$};
\node[vertex] (factorialSchur)	at ( 4.5 , 4) {Generalized factorial Schur, $\schurS_\lambda(\xvec|\yvec)$};
\node[vertex,symvertex] (kSchur)		at ( 4, 2) {$k$-Schur, $\schurS^{(k)}_\lambda(\xvec)$};
\node[vertex,symvertex] (macdonaldP)	at ( 8, 4) {Macdonald $\macdonaldP_\lambda(\xvec;q,t)$};
\node[vertex,symvertex] (hallLittlewoodP)	at ( 7, 2) {Hall--Littlewood, $\hlPolyP_\lambda(\xvec;t)$};
\node[vertex] (nonSymMacdonaldE)	at ( 5, 6) {Non-symmetric Macdonald $\nonSymMacdonaldE_\alpha(\xvec;q,t)$};
\node[vertex,symvertex] (zonal)		at ( 8, 1) {Zonal, $\zonal_\lambda(\xvec)$};
\node[vertex] (gesselFundamental)	at ( 4, -1) {Gessel Fundamental, $\gessel_D(\xvec)$};


\draw[supSet,bend right] (generalAtom) to (atom);
\draw[supSet,bend right,looseness=0.4] (generalAtom) to (key);
\draw[supSet,bend right,looseness=0.4] (key) to (schurS);
\draw[supSet] (schubert) to (schurS);
\draw[supSet] (pbMacdonaldE) to (nonSymMacdonaldE);
\draw[spec,bend right] (pbMacdonaldE) to node[sloped,above] {\small$t=q=0$}  (generalAtom);
% \draw[supSet,bend right,looseness=0.4] (quasiSchur) to (schurS);
\draw[supSet] (LLT) to (schurS);
\draw[spec,bend left,looseness=0.2] (nonSymMacdonaldE) to (macdonaldP);
\draw[supSet] (doubleSchubert) to (doubleSchur);
\draw[misc]  (grothendieck) to node[sloped,above] {\small limit} (stableGrothendieck);
\draw[supSet]  (stableGrothendieck) to (stableGrothendieck2);
\draw[spec,bend right] (doubleSchubert) to node[sloped,above] {\small$\yvec=0$} (schubert);
\draw[spec] (shiftedJack) to node[sloped,above] {\small$a=1$} (shiftedSchur);
\draw[spec,bend right,looseness=0.8] (factorialSchur) to (shiftedSchur);
\draw[spec] (doubleSchur) to node[sloped,below] {\small$\yvec=0$} (schurS);
\draw[misc] (schubert) to node[sloped,above] {\small limit} (stanleySym);
\draw[spec] (macdonaldP) to node[auto] {$\substack{t=q^a \\ q\to 1}$ } (jackJ);
\draw[spec] (jackJ) to node[sloped,above] {\small$a=2$} (zonal);
\draw[misc] (shiftedSchur) to (schurS);
\draw[misc,bend left,looseness=0.1] (shiftedJack) to (jackJ);
\draw[spec,bend left,looseness=0.8] (hallLittlewoodP) to node[sloped,below] {\small$t=0$} (schurS);
\draw[spec,bend left] (jackJ) to node[sloped,below] {\small$a=1$} (schurS);
\draw[subst] (modmacdonaldH) to (macdonaldP);
\draw[spec,bend right,looseness=0.8] (nonSymMacdonaldE.west) to[out=-45,in=-135] node[sloped,above] {\small$t=q=\infty$} (key);
\draw[spec,bend right,looseness=1.3] (nonSymMacdonaldE.west) to[out=-60,in=-135] node[sloped,above] {\small$t=q=0$} (atom);
\draw[spec] (macdonaldP) to node[above,sloped] {\small$q=0$} (hallLittlewoodP);
\draw[misc] (factorialSchur) to node[sloped,above] {\small limit} (doubleSchur);
\draw[spec,bend right,looseness=0.2] (factorialGrothendieck) to node[sloped, above] {\small$\beta=0$} (factorialSchur);
\draw[spec] (doubleGrothendieck)  to node[sloped, above] {\small$\yvec=0$} (grothendieck);
\draw[spec,bend right,looseness=0.1] (factorialGrothendieck) to node[sloped, above] {\small$\yvec=0$} (stableGrothendieck2);
\draw[misc] (factorialGrothendieck) to node[sloped,above] {\small limit} (doubleGrothendieck);
\draw[posExp]  (schubert) to (key);
\draw[posExp]  (key) to (atom);
\draw[posExp,bend right,looseness=0.4]  (key) to (generalAtom);
\draw[posExp]  (quasiSchur) to (atom);
\draw[posExp,bend right,looseness=0.1]  (schurS) to (hallLittlewoodP);
\draw[posExp]  (generalAtom) to (atom);
\draw[posExp]  (schurS) to (quasiSchur);
\draw[posExp]  (quasiSchur) to (gesselFundamental);
\draw[posExp] (modmacdonaldH) to (LLT);
\draw[posExp,bend left,looseness=0.1] (LLT) to (schurS);
\draw[posExp]  (stanleySym) to (schurS);
\draw[posExp,red]  (kSchur) to node[sloped,above] {\text{conj.}} (schurS);
\draw[ktheoretical,bend left,looseness=0.6]  (stableGrothendieck2) to (schurS);



\coordinate (Aspec) at (-1,8);
\coordinate (Bspec) at (0,8);
\draw[spec] (Aspec) to node[above] {\text{specializes to}} (Bspec);


\coordinate (Apos) at (-1,7.5);
\coordinate (Bpos) at (0,7.5);
\draw[posExp] (Apos) to node[above] {\text{expands positively}} (Bpos);

\coordinate (Asup) at (-1,7);
\coordinate (Bsup) at (0,7);
\draw[supSet] (Asup) to node[above] {\text{superset of}} (Bsup);

\coordinate (Asubst) at (-1,6.5);
\coordinate (Bsubst) at (0,6.5);
\draw[subst] (Asubst) to node[above] {$\substack{\text{change of coordinates} /\\ \text{plethysm}}$} (Bsubst);

\coordinate (Ak) at (-1,6);
\coordinate (Bk) at (0,6);
\draw[ktheoretical] (Ak) to node[above] {\text{$k$-theoretical analogue}} (Bk);



\end{tikzpicture}
\end{document}