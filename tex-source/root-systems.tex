\metatitle{Root systems}
\metadescription{Introduction to root systems}


\todo{Reflection groups->Root systems.  https://www.mat.univie.ac.at/~schlosse/pdf/ChristianStumpDissertation.pdf}

\section[root-systems]{Root systems}

A \defin{root system} $\Phi \subset V$ is a set of vectors in $V$ such that
\begin{itemize}
 \item $\Phi$ span $V$,
 \item only $x$ and $-x$ are the scalar multiples of $x$ in $\Phi$,
 \item $\Phi$ is closed under reflection in the hyperplane defined by $x$, for every $x \in \Phi$,
 \item for $x, y \in \Phi$, $x$ projected onto $y$ is a half-integral multiple of $y$.
\end{itemize}

The \defin{root lattice} is the lattice in $V$ which is generated by $\Phi$.
That is, all points which are integer linear combinations of roots.

A \defin{simple root} is a root that cannot be written as a positive sum of other roots.
Each root $\alpha$ defines a hyperplane, where $\pm \alpha$ are normals to this hyperplane.
These hyperplanes divides the space into \defin{Weyl chambers}.
By picking a vector $v$ in the inner of a Weyl chamber, we define a half-space.
The \defin{positive roots} $\Phi^+$ are the roots in this half-space 
and the base (with respect to $\Phi^+$) is the set of simple roots in $\Phi^+$.

Given $\Phi^+$, the \defin{root poset} is a graded partial order on $\Phi^+$, where $\alpha \geq \beta$ if $\alpha - \beta$
is a sum of simple positive roots. 
The grading is given by the number of simple positive roots needed to express the root as a sum.

The \defin{coroot} of a root is defined as $\alpha^\vee = 2\alpha/(\alpha,\alpha)$.
This defies the dual or inverse root system. 
A root system and its dual have the same Weyl group.

\subsection[root-system-classification]{Classification of simple root systems}

The simple root systems are given by the \defin{Dynkin diagrams}, 
$A_{n \geq 1}$, $B_{n \geq 2}$, $C_{n \geq 3}$, $D_{n \geq 4}$, $E_6$, $E_7$, $E_8$, $F_4$ and $G_2$.



\section[weyl-lie-groups]{Weyl groups and Lie groups}

\todo{Billey has a nice background in her thesis: https://sites.math.washington.edu/~billey/papers/thesis.pdf}


\begin{array}{lll}
\toprule
\text{Root system} & \text{Weyl group} & \text{Lie group} \\
\midrule
 A_{n-1} & \symS_n & Sl_{n}(\setC) \\
 B_n & \symB_n & SO_{2n+1}(\setC) \\
 C_n & \symB_n & Sp_{2n}(\setC) \\
 D_n & \symD_n & SO_{2n}(\setC) \\
 \bottomrule
 \end{array}
 

The Weyl group $\symS_n$ is the permutation group, 
which we can think of as $n \times n$ permutation matrices.

The group $\symB_n$ known as the \defin{hyperoctahedral group},
which can be seen as the \emph{signed permutations} --- 
 the $n\times n$ permutation matrices, but entries can be $\pm 1$ and not just $1$.
This can be thought of as the group of \hyperref[permutationsTypeB]{permutations of type $B$}.
The group $\symD_n \subset \symB_n$ is the subgroup of the signed permutation 
matrices where the number of negative entries is even.

The groups $\symS_n$, $\symB_n$ and $\symD_n$ act on vectors via matrix multiplication.
 
Let $W$ be a reflection group. The ring of polynomials in $\setR[x_1,\dotsc,x_n]$
invariant under $W$, can be spanned by homogeneous polynomials.
The \defin{degrees} of these polynomials only depend on $W$.
The order of the reflection group is the product of the degrees.

\begin{array}{lll}
\toprule
 \text{Reflection group $W$} & \text{Degrees} & \text{Order}\\
\midrule
 A_{n-1} & 2,3,\dotsc,n  & n!\\
 B_n/C_n & 2,4,\dotsc,2n & 2^n n!\\
 D_n & 2,4,\dotsc,2(n-1),n & 2^{n-1} n! \\
 \bottomrule
\end{array}

 

\section[coxeter-groups]{Coxeter groups}

\todo{
Add list of exponents for coxeter groups.
These are used to define general Catalan numbers.
\url{https://arxiv.org/pdf/1101.5082.pdf}
\url{https://math.stackexchange.com/questions/1080402/exponents-of-the-coxeter-group-an}
}

