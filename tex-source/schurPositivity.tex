\section[schurPositivity]{Schur positivity}

There are several different techniques to prove that a symmetric function is \hyperref[schurS]{Schur positive}.

For a brief survey on Schur polynomials and the "probability" that a symmetric function is Schur positive, see \cite{Patrias2019,PatriasWilligenburg2018}.


\subsection[schurPositivityRSK]{Robinson--Schensted--Knuth (RSK)}

Examples of results and papers that use \hyperref[rsk]{Robinson--Schensted--Knuth} correspondence:

\begin{itemize}
\item 
One can show that the skew Schur functions are 
Schur positive by using RSK.

\item 
In \cite{HuhNamYoo2020}, the authors use RSK to get the Schur expansion of 
certain \hyperref[unicellularLLT]{unicellular LLT polynomials}.

\item 
In \cite{AlexanderssonSawhney2017}, we use RSK to get the Schur-expansion of 
certain \hyperref[macdonaldE]{non-symmetric Macdonald polynomials} evaluated at $t=0$.
This is later extended to a skew generalization in \cite{AlexanderssonUhlin2020}.

\end{itemize}



\subsection[schurPositivityDualEquivalence]{Dual equivalence}

The idea behind dual equivalence is to start with the \hyperref[gessel]{Gessel fundamental expansion} of a 
symmetric function (as a sum over some set of combinatorial objects),
and define a graph structure on these objects such that connected components sum to Schur functions.

Austin Roberts give a modified list of axioms, allowing for a \emph{local} 
characterization of dual equivalence graphs \cite{Roberts2013}.
It is then enough to verify all graphs with at 
most six vertices, which usually can be done on a computer.

The following are extensions and analogues of dual equivalence:

\begin{itemize}
\item 
Type $B$ version of dual equivalence for shifted tableaux, \cite[Def. 4.5.1]{RobertsThesis}.

\item 
\hyperref[schurP]{Schur-P} and \hyperref[schurQ]{Schur-Q} dual equivalence.

\item 
Dual equivalence can be generalized to quasi-symmetric Schur functions,
see A. Roberts \cite{Roberts2016}.

\item 
Permutation classes that give rise to Schur positive expressions, see \cite{ElizaldeRoichman2016}.
\end{itemize}


\subsection[schurPositivityCrystals]{Crystal graphs}

The idea behind \hyperref[crystals]{crystal graphs} is to define a graph structure 
on the (combinatorial) objects that generate the monomial expansion.
By showing that the graph satisfies a set of axioms, it follows that each connected component sum to a Schur function.
For example, one can define a crystal graph on skew SSYT 
in order to prove that skew Schur functions are Schur positive.

A crystal graph also comes with an $\symS_n$-action on the combinatorial objects,
so that one obtains an $\symS_n$-module (a representation), 
whose \hyperref[frobeniusCharacteristic]{Frobenius image} is exactly the 
original symmetric function.

For example, one can define a crystal graph on words of length $n$, and 
the $\symS_n$-action is generated by the Lascoux--Schutzenberger involutions $s_i$,
acting on the words. This proves that $(x_1+x_2+x_3+\dotsb)^n$ is Schur-positive.

Notable examples include \hyperref[macdonaldE]{non-symmetric Macdonald polynomials}, \cite{AssafGonzalez2018} and
\hyperref[stanleySym]{Stanley symmetric functions}, \cite{MorseSchilling2015},
and \hyperref[grothendieckDualStable]{Dual stable Grothendieck polynomials}, \cite{Galashin2017}.


A flagged version of skew dual stable Grothendieck polynomials are shown to be \hyperref[key]{key positive} 
using crystals in \cite{Kundu2023x}.
hyperref[kohnert]{Kohnert polynomials} are key positive with crystal structure defined in \cite{Assaf2021}.


The action of $\GL_m \times \GL_n$ act on $\setC[z_{ij}] / I_k$ where $I_k$ is the \defin{determinantal variety},
generated by the vanishing of the $(k+1) \times (k+1)$ minors of the matrix $[z_{ij}]_{1 \leq i \leq m, 1 \leq j \leq n}$.
An example using crystals in this setting is \cite{PriceStelzerYong2025x}.




\subsection[schurPositivityCrystalSkeletons]{Crystal skeletons}

There is a notion that unifies dual equivalence graphs and crystal graphs,
called \defin{crystal skeletons}. These were introduced in \cite{MaasGariepy2023x},
and further developed in \cite{BraunerCorteelDaughertySchilling2025x}.

\subsection[schurPositivityRep]{Representation theory}

For a background, see the page on \hyperref[general-representation-theory]{representation theory}.

Examples: \hyperref[macdonaldH]{Modified Macdonald polynomials}, 
\hyperref[LLT]{LLT polynomials}, \hyperref[eulerianSymmetric]{Eulerian symmetric functions}.



\subsection[schurPositivityEG]{Edelman--Greene}

The Edelman--Greene bijection is used to show that the
\hyperref[stanleySymAffineSchurExpansion]{Stanley symmetric functions} are Schur positive,
see \cite{EdelmanGreene1987}.




\subsection[schurPositivitySlinky]{The slinky rule}

The idea is to use the \hyperref[gesselSlinkyRule]{slinky rule} to convert an expansion in 
the \hyperref[gessel]{Gessel fundamental quasisymmetric basis} into a (signed) Schur expansion, 
and then one must invent some sign-reversing involution.
An example of a paper that uses this strategy is \cite{Sergel2017}.



\subsection[schurPositivityPermutationClasses]{Positive classes of permutations}

In \cite{ElizaldeRoichman2016}, the authors characterize several classes of permutations $A \subseteq \symS_n$,
such that
\[
\sum_{\sigma \in A} \gessel_{n,\DES(\sigma)}(\xvec)
\]
is Schur-positive. 
We say that $A$ is \defin{Schur-positive} if the above sum is Schur-positive.


In \cite{AdinRoichman2015}, it is proved that $A \subseteq \symS_n$ is Schur-positive if and only if 
there are non-negative integers $a_\lambda$ such that
\[
\sum_{\sigma \in A} \yvec_{\DES(\sigma)} = \sum_{\lambda \vdash n} a_\lambda \sum_{T \in \SYT(\lambda)} \yvec_{\DES(T)}.
\]
Here, $\yvec_{S}  = y_{i_1} \dotsm y_{i_\ell}$ where $S = \{i_1,i_2,\dotsc,i_\ell\}$.


More recent results can be found in \cite{BloomElizaldeRoichman2020} where cyclic descent sets are used.
