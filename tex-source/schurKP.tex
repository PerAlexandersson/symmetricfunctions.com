
\metatitle{K-theoretic Schur P/Q polynomials}
\metadescription{K-theoretic analogs of Schur P and Schur Q functions, introduced by \name{Ikeda} and \name{Naruse}, with properties and Littlewood-Richardson rules.}


\todo{
From Z. Hamaker:
In this [paper], Ikeda and Naruse define K-theoretic analaogues of Schur P and Schur Q functions. 
They give a presentation for equivariant K-theory classes of the orthogonal and 
Lagrangian Grassmannian and are typically denoted GP_lambda or GQ_lambda. 
The paper has determinantal formulas as well as tableaux formulas for these objects.

An expansion of the GP_lambda's into fundamental quasisymmetric 
functions can be found in my paper with Rebecca Patrias and some undergrads we worked with. 
The Littlewood-Richardson rule for GP_lambda's is due to Clifford-Thomas-Yong based on a 
Pieri rule due to Buch and Ravikumar, while the Littlewood-Richardson rule for GQ_lambda's is open. 

Recently, Marberg and Pawloski showed that stable Grothendieck 
polynomial analogs for fixed-point-free involutions 
expand positively into GP_lambda's and conjectured that 
analogs for arbitrary involutions expand positively into GQ_lambda's (they proved this in the vexillary case).
}

\todo{Add more info on K-theoretic Schur P/Q polynomials}

\section[schurKPQ]{K-theoretic Schur P/Q polynomials}
\begin{polydata}{schurKPQ}
  Name   & K-theoretic Schur P/Q polynomials \\
  Space    &  Sym   \\
  Basis    &  True   \\
  Rating   &  1      \\
  Bib      &  IkedaNaruse2013 \\
  Year     &  2013 \\
  Keywords &  jacobi-trudi,tableaux,pieri,littlewood-richardson \\
  Symbol   &  $\schurKP_{\lambda}(\xvec)$ \\
  Category &  Schur \\
\end{polydata}


In \cite{IkedaNaruse2013}, Ikeda and Naruse introduce $K$-theoretic analogs of the \hyperref[schurP]{Schur $P$}
and \hyperref[schurQ]{Schur $Q$} functions, as well as factorial versions of these.
The factorial versions are denoted $\schurKP_{\lambda}(\xvec|b)$ and $\schurKQ_{\lambda}(\xvec|b)$,
where $b = (b_1,b_2,\dotsc)$ is a vector of parameters.
When all these are set to $0$, we obtain $\schurKP_{\lambda}(\xvec)$ and $\schurKQ_{\lambda}(\xvec)$. 
These are $K$-theoretic analogs of $\schurP_{\lambda}(\xvec)$ and  $\schurQ_{\lambda}(\xvec)$.

They give a presentation for equivariant $K$-theory classes of the orthogonal and Lagrangian Grassmannian.


\subsection[schurKPQProps]{Properties}

The fundamental quasisymmetric expansion of $\schurKP_{\lambda}(\xvec)$
and $\schurKQ_{\lambda}(\xvec)$ can be found in \cite{HamakerKeilthyPatriasWebsterZhangZhou2017}.


\subsection[schurKPQLRRule]{Littlewood--Richardson rule}

The Littlewood-Richardson rule for $\schurKP_{\lambda}(\xvec)$ is due to 
Clifford--Thomas--Yong \cite{CliffordThomasYong2014}.
Their result is based on a Pieri rule due to Buch and Ravikumar, \cite{BuchRavikumar2012}.

\begin{problem}[Littlewood--Richardson rule]
Find a Littlewood--Richardson rule for $\schurKQ_{\lambda}(\xvec)$.
\end{problem}
