\metatitle{The Yang--Baxter equations}
\metadescription{An introduction to the Yang--Baxter equations, vertex models, and their connections to symmetric functions.}

\section[yang-baxter]{The Yang--Baxter equations}


Let $V$ be a vector space. A linear operator $R$ on $V\otimes V$
is said to be a solution to the \defin{Yang--Baxter equation} if
\[
R^{12} \circ R^{23} \circ R^{12} = R^{23} \circ R^{12} \circ R^{23} 
\]
when acting on $V \times V \times V$,
where $R^{ij}$ means that we act on components $i$ and $j$.
Stated differently, we wish that
\[
(R \times Id)\circ 
(Id \times R)\circ (R \times Id) =
(Id \times R) \circ (R \times Id) \circ (Id \times R).
\]

Another form is to have some operators, $h_i$, so that
\[
h_i(x)h_{i+1}(x+y)h_i(y) = h_{i+1}(y) h_{i}(x+y) h_{i+1}(x) 
\]
See \href{https://users.mccme.ru/valya/Kirillov-Fomin_%20Schubert%20polynomials.pdf}{this pdf} for the connection with Schubert polynomials.

Yet another reference state the Yang--Baxter relation as
\[
R^{12} \circ R^{13} \circ R^{23} = R^{23} \circ R^{13} \circ R^{12}.
\]
\url{https://lapth.cnrs.fr/conferences/RAQIS/RAQIS12/pdfRAQIS12/fonseca.pdf}


Notions close to the Yang--Baxter equation are \defin{(integrable) vertex models},
\defin{R-matrix}, \defin{5-vertex model}, \defin{partition function}.

It is interesting to show that symmetric functions are partition functions for 
some particular choice of a vertex model.
This is closely related to Jacobi--Trudi identities.


\section[6vertexModel]{The 6 vertex model}


\todo{

\url{https://hal.archives-ouvertes.fr/hal-01417747/document} YB and RSK

\url{https://arxiv.org/pdf/0912.0911.pdf}

Book on Yang--Baxter: \url{https://doi.org/10.1142/9789814354448_0001}
}


\todo{Integrable vertex model: https://arxiv.org/pdf/2012.15011.pdf}