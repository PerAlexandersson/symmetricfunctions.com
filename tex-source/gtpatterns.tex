\metatitle{Gelfand-Tsetlin patterns and polytopes}
\metadescription{Definition of Gelfand-Tsetlin patterns and Gelfand-Tsetlin polytopes. Bijection with semi-standard Young tableaux. Related results.}


\todo{
http://www.kurims.kyoto-u.ac.jp/EMIS/journals/SLC/wpapers/s73vortrag/alexand.pdf 
}

\todo{Tokoyamas formula:
http://www-users.math.umn.edu/~reiner/REU/AlexandrComminsEmbryFrankLiVetter2018.pdf
}

\todo{
Describe it as projection of integer pt transform:
https://arxiv.org/pdf/1903.05548.pdf
}


\section[gtpatterns]{Gelfand--Tsetlin patterns}

Gelfand--Tsetlin (sometimes spelled Cetlin or Zetlin) patterns
were introduced by \name{Israel Gelfand} and \name{Michael Tsetlin} in \cite{GelfandTsetlin1950},
motivated by the close relationship with representation theory of $GL_n$.
A short introduction to GT-patterns can be found in \cite[p.103]{StanleyEC2}.

For a brief background, see \href{https://www.findstat.org/CollectionsDatabase/GelfandTsetlinPatterns/}{this page on FindStat}.


A \defin{Gelfand--Tsetlin pattern} or \defin{GT-pattern}
is a triangular (or sometimes parallelogram) arrangement of non-negative integers,
with certain constraints.
The triangular patterns are indexed as follows.
\[
\begin{matrix}
x_{n1} & & x_{n2} & & \cdots & & \cdots & & x_{nn} \\
 & \ddots & & \ddots &  & &  \iddots &  & &  \\
%   &  & x_{13} &  & x_{23}  &  & x_{33}\\
%   &  &  &  x_{12} &   & x_{22}  \\
   &  &   x_{21} &  & x_{22}\\
   &  &    & x_{11}
\end{matrix}
\]
The less common parallelogram patterns are indexed as follows.
\[
\begin{matrix}
x_{n1} & & x_{n2} & & \cdots & & \cdots & & x_{nm} \\
 & \ddots & & \ddots &  & &   & & & \ddots  \\
%   &  & x_{13} &  & x_{23}  &  & x_{33}   &  & \\
%   &  &  &  x_{12} &   & x_{22}  &     &  & \\
   &  &   x_{11} &  & x_{12} & & \cdots & & \cdots & & x_{1m} \\
   &  &    & x_{01} &    & x_{02} & & \cdots & & \cdots & & x_{0m}
\end{matrix}
\]
The entries must satisfy the conditions
\begin{equation*}
x_{i+1,j} \geq x_{ij} \text{ and } x_{ij} \geq x_{i+1,j+1}
\end{equation*}
for all values of $i$, $j$ where the indexing is defined.
The inequalities simply states that horizontal rows are weakly decreasing,
down-right diagonals are weakly decreasing and down-left diagonals are weakly increasing.

The \defin{weight} $w(G) = (w_1,\dotsc,w_n)$ of a GT-pattern $G$ is defined as
the vector where $w_i$ is the difference of the sum of row $i$ and $i-1$.
By convention, $x_{0j}=0$ if these entries are not specified.


\subsection[gtpatternsAsSSYT]{Bijection with semistandard Young tableaux}

The conditions ensures every row in the pattern defines an integer partition.
Furthermore, any two adjacent rows define a skew shape.
These properties enables us to biject triangular GT-patterns with Young tableaux
and parallelogram GT-patterns with skew Young tableaux.

The skew shape defined by row $j$ and $j+1$ in a GT-pattern $G$
describes which boxes in a tableau $T$ that have content $j$.
In particular, if $T$ has shape $\lambda/\mu$ then the first row in $G$ is $\mu$
and the last row is $\lambda$. 


\begin{example}
The following triangular GT-pattern
\[
\begin{matrix}
5 &   & 4 &   & 2 &   & 1 &   & 1 &  & 0\\
 & 5 &   & 3 &   & 2 &   & 1 &   & 0\\
 &  & \color{blue} 3 &   & \color{blue} 3 &   & \color{blue} 2 &   & \color{blue} 1  \\
 &  &  & 3 &   & 3 &   & 1 &  \\
 &  &  &  & 3 &   & 2 &  \\
 &  &  &  &  & 3 & \\
\end{matrix}
\]
is in bijection with the semi-standard Young tableau below.

\begin{figure}
\begin{ytableau}
*(lightblue)1 & *(lightblue)1 & *(lightblue)1 & 5 & 5 \\
*(lightblue)2 & *(lightblue)2 & *(lightblue)3 & 6 \\
*(lightblue)3 & *(lightblue)4 \\
*(lightblue)4 \\
6
\end{ytableau}
\end{figure}

Notice that the fourth row in the GT-pattern is $3321$ and 
the shape of all entries less than or equal to four in the tableau is also $3321$.
\end{example}

\begin{example}
The skew Young tableau
\begin{figure}
\begin{ytableau}
\none  & \none & 1 & 3 \\
\none  & 1 & 2 \\
\none  & 4 \\
2
\end{ytableau}
\end{figure}
is in bijection with the GT-pattern
\[
\begin{matrix}
4 &   & 3 &   & 2 &   & 1\\
 & 4 &   & 3 &   & 1 &   & 1\\
 &  & 3 &   & 3 &   & 1 &   & 1\\
 &  &  & 3 &   & 2 &   & 1 &   & 0\\
 &  &  &  & 2 &   & 1 &   & 1 &   & 0\\
\end{matrix}
\]
Observe that in any GT-pattern, $x_{i+1,j} - x_{i,j}$ counts the
number of boxes with content $i$ in row $j$ in the corresponding tableau.
Thus, the weight of a GT-pattern is the same as the weight of the corresponding tableau.
\end{example}

There is a \hyperref[crystals]{crystal graph structure} on semistandard Young tableaux.
The corresponding crystal graph on GT-patterns is described explicitly in
\url{https://arxiv.org/pdf/2005.06639.pdf}.


\subsection[gtpatternsTokuyama]{Tokuyama's formula}

Let $SGT(\lambda)\subset \gtp(\lambda)$ be the set of \defin{strict Gelfand--Tsetlin patterns}.
This is the subset of patterns where each row is \emph{strictly decreasing}.
Consider tuples $(a,b,c)$ in the pattern arranged as
$\begin{smallmatrix}
a && b\\ & c \end{smallmatrix}$.
Let $S(G)$ be the number of entries $c$ such that $a\gt c \gt b$,
and let $L(G)$ be the number of entries $c$ such that $a=c$.

Let $\rho = (n-1,n-2,\dotsc,1,0)$. Then (see \cite{Tokuyama1988})
\begin{equation*}
\schurS_{\lambda}(x_1,\dotsc,x_n) \cdot \prod_{1 \leq i \lt j \leq n} (x_i + t x_j)  = 
\sum_{G \in SGT(\lambda + \rho)} (1+t)^{S(G)} t^{L(G)}\xvec^{w(G)}.
\end{equation*}

Note that this interpolates between the \hyperref[schurWeylformula]{Weyl character formula}, and the SSYT definition of Schur polynomials.


See also \cite{Stanley1986} for some special cases, and connection with plane partitions
and \hyperref[alternatingSignMatrix]{alternating sign matrices}.

\section[gtpolytopes]{Gelfand--Tsetlin polytopes}

By specifying the top row in a triangular GT-pattern as $\lambda$
and imposing the inequalities above we get a convex polytope. 
This is a \defin{Geltand--Tsetlin polytope}, $\gtp_{\lambda} \subset \setR^{n(n+1)/2}$.
By construction, the integer lattice points in $\gtp_{\lambda}$
is in bijection with $\SSYT(\lambda)$.
The analogous statement holds for parallelogram GT-patterns where 
also the bottom row $\mu$ is a fixed integer partition.
These polytopes are denoted $\gtp_{\lambda\mu} \subset \setR^{nm}$.
Below, we use $d$ to denote the dimension ($n(n+1)/2$ or $mn$) of the ambient space.


\subsection[gtpolytopesMarkedPoset]{Unimodular triangulation and marked posets}

The polytopes $\gtp_{\lambda\mu}$ admit \emph{unimodular triangulations}.
This means that the polytopes can be decomposed into simplices with integer vertices and 
normalized volume $1$. For a short proof, see \cite{Alexandersson2019CounterExamples}.

As a consequence, Gelfand--Tsetlin polytopes have integer vertices,
and have the \hyperref[polytopeIDP]{integer decomposition property} (IDP). This means that 
if $p$ is a lattice point in $k \gtp_{\lambda\mu}$ for some integer $k\geq 1$,
then $p$ can be expressed as $p_1 + \dotsb + p_k$ where each $p_i$ is a lattice point in $\gtp_{\lambda\mu}$.
This can be proved directly by bijecting to semi-standard tableaux,
see \cite[Subsection 2.3]{Alexandersson2016GTPoly}.

\name{Akiyoshi Tsuchiya} generalize the notion of IDP to $k$-tuples of 
polytopes (the $k=2$ case was introduced in \cite{HaaseHofmann2017}).
The following theorem follows very easy from the bijection with SSYT.

\begin{theorem}[Alexandersson, 2018]
Let $\gtp_{\lambda^1},\dotsc,\gtp_{\lambda^k}$ 
be GT-polytopes, embedded in the same ambient space 
by padding the partitions with zeros of needed.
Then
\[
\setZ^d \cap \left( \gtp_{\lambda^1} + \dotsb + \gtp_{\lambda^k} \right)
=
\left( \setZ^d \cap \gtp_{\lambda^1} \right) + \dotsb + \left(\setZ^d \cap \gtp_{\lambda^k} \right).
\]
By letting all $\lambda^i$ be equal, we recover the usual IDP.
\end{theorem}

This theorem does not generalize to skew shapes. The pair $\gtp_{3}$
and $\gtp_{2/1}$ yields a counter-example.


Gelfand--Tsetlin polytopes are special cases of so-called \hyperref[orderPolytope]{marked order polytopes},
which generalizes order polytopes considered by \name{Richard Stanley}, \cite{Stanley86TwoPosetPolytopes}.
In \cite{ArdilaBliemSalazar2011}, the authors generalize the bijection 
between order polytopes and chain polytopes and give a marked chain polytope analog of GT-polytopes.
In type $A$ these analogs are so called \defin{Feigin--Fourier--Littelmann polytopes}.

The Gelfand--Tsetlin polytopes are also special cases of so called \emph{flow polytopes},
see \cite{LiuMeszarosDizier2019flow}.

\subsection[gtEhrhart]{Ehrhart polynomials}

By definition, it follows that for a partition of length $m$,
\[
 \left| \setZ^d \cap \gtp_{\lambda} \right| = \schurS_{\lambda}(1^m)  = 
 \prod_{1 \leq i \lt j \leq m } \frac{\lambda_i - \lambda_j + j-i}{j-i},
\]
where the last identity is due to Stanley's hook-content formula.
Furthermore, the \hyperref[ehrhart]{Ehrhart polynomial} of $\gtp_{\lambda}$ is given by
\[
Ehr(\gtp_{\lambda};k) = \prod_{1 \leq i \lt j \leq m } \frac{k(\lambda_i - \lambda_j) + j-i}{j-i}.
\]

\begin{proposition}
We have that
 \[
 1 + \sum_{n\geq 1} t^n |\schurS_{\lambda/\mu}(1^m)| = 
  \frac{ \sum_{T \in \SYT(\lambda/\mu)} z^{\des(T)} }{(1-t)^{1+|\lambda/\mu|}}.
 \]
In particular, the $h^*$-polynomial of $\gtp_{a^b}$ is the descent-generating polynomial
of rectangular standard Young tableaux; $\sum_{T \in \SYT(a^b)} z^{\des(T)}$.
\end{proposition}
\begin{proof}
Use \cite[Eqs. (7.96), (7.108)]{StanleyEC2}, where one needs to note that 
GT-patterns for rectangular SSYTs can be made in bijection with rectangular plane partitions.
\end{proof}


\begin{conjecture}[Alexandersson, Alhajjar \cite{AlexanderssonAlhajjar2018}]
The coefficients of the Ehrhart polynomial of $\gtp_{\lambda\mu}$ has non-negative coefficients.
Equivalently, the function
\[
f(k) = \schurS_{k\lambda/k\mu}(1^m)
\]
is a polynomial in $k$ with non-negative coefficients.
\end{conjecture}
For example, $m=3$, $\lambda=321$ and $\mu=1$ gives the Ehrhart polynomial
\[
Ehr(\gtp_{321/1},k) = \frac{1}{2} (k^5+6 k^4+14 k^3+16 k^2+9 k+2).
\]

By adding an extra row to $\lambda/\mu$, one can show that 
\[
\schurS_{k\lambda/k\mu}(1^n) = K_{k\nu/k\tau,(km)^n},
\]
for appropriate $\nu/\tau$ and $m = |\lambda/\mu|$. 
The idea is that one can freely fill the shape $\lambda/\mu$,
and in a unique way fill the remaining boxes in the added row,
so that the weight becomes $m^n$.
A similar construction works for flagged skew Schur 
functions and \hyperref[schurFlagged]{flagged skew Kostka coefficients}.



\begin{question}
Do all Kogan faces of Gelfand--Tsetlin polytopes have non-negative Ehrhart polynomial?
There is an example (see \cite{Alexandersson2019CounterExamples}) 
of a (non-Kogan) face where the Ehrhart polynomial 
has a negative coefficient. However, this polynomial is of degree 20.
\end{question}


\subsection[gtpolytopesFVectors]{Vertices, $f$-vectors and differential equations}

Ther has been some effort in counting the number of vertices of GT-polytopes.
In \cite{GusevKiritchenkoTimorin2013}, the authors 
describe a generating function that enumerates the vertices of $\gtp_{\lambda}$.

Let $V(1^{i_1},2^{i_2},\dotsc,k^{i_k})$ denote the number of vertices of
$\gtp_{\lambda}$ where $\lambda$ is the partition with \hyperref[prelimPartitions]{type} $(i_1,i_2,\dotsc,i_k)$.
Consider the generating function
\[
E_k(z_1,\dotsc,z_k) \coloneqq 
\sum_{i_1,\dotsc,i_k \geq 0} V(1^{i_1},2^{i_2},\dotsc,k^{i_k}) \frac{z_1^{i_1}}{i_1!} \dotsm \frac{z_k^{i_k}}{i_k!}.
\]
\begin{theorem}[See \cite{GusevKiritchenkoTimorin2013}]
The formal power series $E_k(z_1,\dotsc,z_k)$ satisfies the partial differential equation
\[
\left( \frac{\partial^k}{\partial z_1 \dotsm \partial z_k} -
\left(\frac{\partial}{\partial z_1}-\frac{\partial}{\partial z_2}\right)
\dotsm
\left(\frac{\partial}{\partial z_{k-1}}-\frac{\partial}{\partial z_k}\right) \right)E_k = 0
\]
\end{theorem}

This result is later improved in \cite[Thm. 1.14]{AnChoKim2018},
where the authors provide a differential equation 
for the full $f$-vector of GT-polytopes.
They consider the generating function
\[
F_k(z_1,\dotsc,z_k;t) \coloneqq 
\sum_{d\geq 0} \sum_{i_1,\dotsc,i_k \geq 0} t^d f_d(1^{i_1},2^{i_2},\dotsc,k^{i_k}) \frac{z_1^{i_1}}{i_1!} \dotsm \frac{z_k^{i_k}}{i_k!}.
\]
where $f_d$ is the number of $d$-dimensional faces of $\gtp_{\lambda}$.


\subsection[gtpolytopesTypeBCD]{Type $B$ and $C$}

There are analogs of GT-polytopes for type $B$ and $C$, see \cite{ArdilaBliemSalazar2011}.


\section[gtKoskta]{Weight-restricted Gelfand--Tsetlin polytopes}

Let $\gtp_{\lambda,w}\subseteq \gtp_{\lambda}$ be the Gelfand--Tsetlin polytope defined 
as $\gtp_{\lambda}$ but 
with the additional constraint that the difference of 
the row row sums of row $j+1$ and $j$ is exactly $w_j$.

By construction, lattice points in $\gtp_{\lambda,w}$ are in bijection with $\SSYT(\lambda,w)$.
Therefore, the number of lattice points in $\gtp_{\lambda,w}$ is 
given by the \hyperref[schurCombinatorialFormula]{Kostka coefficient} $K_{\lambda w}$.

We do the analogous definitions for parallelogram GT-patterns and skew shapes.


\begin{theorem}[See \cite[p.179]{Kirillov2001}]
The dimension of the polytope $\gtp_{\lambda,\mu}$ (assuming non-empty) 
where $\lambda$ has $r$ parts and $\mu$ has $s$ parts is given by
\[
 \dim \gtp_{\lambda,\mu} = (r-1)(s-1) - \binom{r}{2} - \sum_{i=1}^{r} \binom{\lambda'_i - \lambda'_{i+1}}{2}.
\]
\end{theorem}


\subsection[gtKostkaVertices]{Non-integrality and vertices}

Contrary to $\gtp_{\lambda}$, the polytopes (also called Gelfand--Tsetlin polytopes) $\gtp_{\lambda,w}$
are not always integral. In fact, the denominators of the 
vertices can get arbitrary large, see \cite{DeLoeraMcAllister2004}.

In \cite{Alexandersson2016GTPoly}, we prove that for $\lambda \vdash n$,
the polytope $\gtp_{\lambda,1^n}$ is non-integral unless $\lambda=(2,2)$ or a hook shape.


\subsection[gtKostkaSaturation]{Saturation theorem}

The following theorem is a GT-polytope analog of the celebrated Saturation theorem,
proved by \name{Allen Knutson} and \name{Terence Tau} \cite{KnutsonTau1999,KnutsonTaoWoodward2004}.
They proved a similar statement for so called \emph{Berenstein--Zelevinsky polytopes}.

\begin{theorem}[See \cite{Alexandersson2015KSaturation}]

Every \emph{non-empty} Gelfand--Tsetlin polytope $\gtp_{\lambda/\mu,w}$
contains at least one vertex with integer coordinates.

\end{theorem}



\subsection[gtKostkaEhrhart]{Ehrhart polynomials and period collapse}

Since the $\gtp_{\lambda/\mu,w}$ are not lattice \hyperref[polytope]{polytopes}, 
one does not expect the \hyperref[ehrhart]{Ehrhart function} to be a polynomial. 
However, these polytopes exhibit a \defin{period collapse} \cite{HaaseMcAllister2007}
and they do have a polynomial \hyperref[ehrhart]{Ehrhart function}. 
See \cite{Rassart2004} for a proof of this statement.


\begin{conjecture}[See \cite[p.170]{Kirillov2001} and later \cite{KingTolluToumazet2004}]
The coefficients of the Ehrhart polynomial of $\gtp_{\lambda/\mu,w}$ are non-negative.
\end{conjecture}

See also \cite{Alexandersson2015KSaturation} for related open conjectures.


We could also consider a fixed face of the Gelfand--Tsetlin polytope intersected 
with the hyperplane defined by a weight.
Suppose $P$ is the polytope obtained from the non-skew shape $\lambda = (2,2,2)$
and $w = (1,1,\dotsc,1)$, intersected with the face requiring 
that there the $3$ is not in the first row of the corresponding SSYTs. 
Then the number of lattice points in $kP$ is
\[
  3, 7, 13, 22, 34, 50, 70, 95, 125, 161, 203 \dotsc, \text{ for } k=1,2,\dotsc,
\]
which is a quasipolynomial of degree 3 with period 2.



\section[gtASM]{Monotone triangles and alternating sign matrices}\label{alternatingSignMatrix}

\defin{Monotone triangles} are special types of triangular GT-patterns with additional constraints.
First, the top row is $(n,n-1,\dotsc,1)$ and all rows must be \emph{strictly} decreasing.

Monotone triangles of size $n$ are in bijection with $n \times n$ \emph{alternating sign matrices}.
\defin{Alternating sign matrices} are matrices with entries in $\{-1,0,1\}$ 
such that in each row and column, the nonzero elements alternate in sign.
Furthermore, the sum of the entries in each row (and column) is $1$.

\begin{example}
Consider the following $5\times 5$ alternating sign matrix, where we omit the zeros,
and $-1$ is written as $\color{blue}{\overline{1}}$.
The rows are accumulated from the bottom, so that row $k$ from the bottom
is the sum of the first $k$ rows.
\[
\begin{bmatrix}
 &1& & & \\
1&\color{blue}{\overline{1}}& &1& \\
 &1& &\color{blue}{\overline{1}}&1\\
 & &1& & \\ 
 & & &1& 
\end{bmatrix}
\to
\begin{bmatrix}
1&1&1&1&1 \\
1& &1&1&1 \\
 &1&1& &1 \\
 & &1&1&  \\ 
 & & &1& 
\end{bmatrix}
\]
For each row, we then record the columns that contains a $1$
and write the column indices in a decreasing fashion.
The result is a monotone triangle.
\[
\begin{matrix}
5&&4&&3&&2&&1 \\
&5&&4&&3&&1 \\
&&5&&3&&2 \\
&&&4&&3 \\
&&&&4& \\
\end{matrix}
\]
\end{example}

\name{Doron Zeilberger} \cite{Zeilberger1995} proved that the number of 
monotone triangles of size $n$, or equivalently, number of $n\times n$
\hyperref[alternatingSignMatrix]{alternating sign matrices} is given by
\[
\prod_{j=0}^{n-1}\frac{(3j+1)!}{(n+j)!}.
\]

A generalization of this result is given in \cite{Fischer2006},
where difference operators are used in a formula to 
count the number of monotone triangles with prescribed top row $\lambda$.


\section[gtApplications]{Applications}

\subsection[gtSchur]{Schur polynomials}

The close connection between GT-patterns and \hyperref[schurS]{Schur polynomials}
is quite apparent. The Schur polynomial $\schurS_\lambda(x_1,\dotsc,x_n)$ can be expressed 
as a sum over $GT(\lambda)$, the set of triangular GT-patterns with top row
equal to $\lambda$.
We have that
\[
\schurS_\lambda(x_1,\dotsc,x_n) = \sum_{G \in GT(\lambda) } \xvec^{w(G)}
\]
where $w$ depends on the row-sums of the pattern $G$.

\subsection[gtHL]{Hall--Littlewood polynomials}

The identity for Schur functions can be 
generalized to \hyperref[hallLittlewoodP]{Hall--Littlewood polynomials},
via a weighted version of Brion's theorem.
\begin{theorem}[See  \cite{FeiginMakhlin2016}]
Let $\lambda$ be a partition. Then
\begin{equation*}
 \hallLittlewoodP_\lambda(\xvec;t) = \sum_{G \in GT(\lambda) } p_G(t) \xvec^{w(G)}
\end{equation*}
where $p_G(t)$ is a certain statistic that depends on $G$.
\end{theorem}
This theorem can also be proved via other means, see \cite{Macdonald1995}.



\subsection[gtKey]{Key polynomials}

The \hyperref[key]{key polynomials} generalize the Schur polynomials.
In \cite{KiritchenkoSmirnovTimorin2010}, a formula for key polynomials is proved,
where the sum runs over lattice points in a union of certain faces of $GT(\lambda)$,
see \hyperref[keyDefinitionGT]{key polynomials from GT-patterns} for details.


\subsection[gtSchubert]{Schubert polynomials}

It was recently proved in \cite{LiuMeszarosDizier2019} 
that \hyperref[schubertGTFormula]{Schubert polynomials}
can be expressed as a sum over lattice points in a Minkowski-sum of GT-polytopes.


