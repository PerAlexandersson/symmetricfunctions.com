\metatitle{Hook Schur polynomials}
\metadescription{An introduction to hook Schur polynomials, their definitions via tableaux, Jacobi--Trudi identity, plethysm, and various properties including Cauchy identity and Littlewood formula.} 

\section[hookSchur]{Hook Schur polynomials}


\begin{polydata}{hookSchur}
  Name   & Hook Schur polynomials \\
  Space    & SuperSym \\
  Basis    & True \\
  Rating   & 5 \\
  Bib      & BereleRegev1983 \\
  Year     & 1983 \\
  Keywords & jacobi-trudi,weyl,tableaux,character,plethysm,skew \\
  Symbol   & $\schurHook_\lambda(\xvec/\yvec)$ \\
  Category & Schur \\
\end{polydata}


The \emph{hook Schur functions} also known as \emph{supersymmetric Schur functions},
are characters of the Lie superalgebra $\mathrm{gl}(m/n)$, see \cite{Kac1977}.
The hook Schur functions were introduced by \name{Allan Berele} and \name{Amitai Regev} 1983, see \cite{BereleRegev1983}.
A good overview of this area can be found in the PhD thesis by \name[E. M. Moens]{Els M. Moens}, \cite{Moens2007}.

This is also the 6th variant of Schur functions considered in \cite[Eq. 6.19]{Macdonald1992}.

\subsection[superSymmetricFunctions]{Supersymmetric functions}

We follow the definitions in \cite{MoensJeugt2003}. We let $\xvec = (x_1,\dotsc,x_m)$
and $\yvec = (y_1,\dotsc,y_n)$. A function $f(\xvec,\yvec)$
is \defin{doubly symmetric} if it is symmetric in each alphabet.
We let $\spaceSym(\xvec/\yvec)$ denote the subspace of doubly symmetric functions
with the property that substituting $x_1=t$, $y_1=-t$ results in an expression independent of $t$.
We refer to functions in this subspace as the \defin{supersymmetric functions}.
For example, $x_1+x_2-y_1-y_2$ is doubly symmetric, (in $2+2$ variables),
while $x_1+x_2 + y_1 + y_2$ is also supersymmetric.


The \defin{complete homogeneous supersymmetric functions} are defined as
\[
\completeH_r(\xvec/\yvec) \coloneqq \sum_{j=0}^r \completeH_{j}(\xvec)\elementaryE_{r-j}(\yvec).
\]
Similarly, the \defin{elementary supersymmetric functions} are defined as
\[
\elementaryE_r(\xvec/\yvec) \coloneqq \sum_{j=0}^r \elementaryE_j(\xvec)\completeH_{r-j}(\yvec).
\]
The \defin{supersymmetric powersum polynomials} are defined as
\[
\powerSum_r(\xvec/\yvec) \coloneqq \powerSum_r(\xvec) + (-1)^{r-1} \powerSum_r(\yvec).
\]
The \defin{supersymmetric monomial polynomials} are defined as
\[
\monomial_\lambda(\xvec/\yvec) = \sum_{\mu\cup \nu = \lambda} \monomial_{\mu}(\xvec) \omega(\monomial_{\nu}(\yvec)).
\]
All these give bases for $\spaceSym(\xvec/\yvec)$.

\subsection[hookSchurTableauDef]{Tableau definition}

There is also a definition in terms of fillings of a Ferrers diagram of shape $\lambda$.
We fill the shape with entries  
\[
1 \lt 2 \lt \dotsb \lt k \lt 1' \lt 2' \lt \dotsb \lt l'.
\]
A filling in $SST(\lambda)$ is defined as a filling of $\lambda$ with entries in the alphabet above
such that rows and columns are weakly increasing. 
Furthermore, the unprimed entries must be strictly increasing with row index,
and the primed entries must be strictly increasing with column index.
The weight $x^{w_x(T)} y^{w_y(T)}$ is what you expect, keeping track of the primed and 
the unprimed alphabet. We have
\[
\schurHook_{\lambda/\mu}(\xvec/\yvec) \coloneqq \sum_{T \in SST(\lambda/\mu)} x^{w_x(T)} y^{w_y(T)}
\]
where the expansion in the last sum is in terms of the 
classical \hyperref[schurS]{schur functions}.

\begin{example}
The following tableau is an element in $SST(7,6,5,2)$, contributing with 
$x_1^2 x_2^2 x_3^3 y_1^4 y_2 y_3^2 y_4 y_5^2 y_6^2 y_7$.

\begin{figure}
\begin{ytableau}
1 & 1 & 2 & 3 & 1' & 6' & 7' \\
2 & 3 & 3 & 1'& 5' & 6' \\
1'& 2'& 3'& 4'& 5' \\
1'& 3' 
\end{ytableau}
\end{figure}

\end{example}

\subsection[schurHookDeterminant]{Weyl type formula}

See 1.17 in  \url{http://igm.univ-mlv.fr/~fpsac/FPSAC02/ARTICLES/Moens.pdf}

\subsection[schurHookJacobiTrudi]{Jacobi--Trudi identity}

A Jacobi--Trudi type formula for hook Schur functions was proved in \cite{PragaczThorup1992},
and it also follows from \cite[Thm. 4.5]{Kwon2008}.
The \defin{$(m,n)$-hook Schur functions} are then given as
\[
\schurHook_{\lambda/\mu}(\xvec/\yvec) \coloneqq 
\det[ \completeH_{\lambda_i-\mu_j + j - i}(\xvec/\yvec) ]_{1\leq i,j \leq \length(\lambda)}.
\]
There is also the dual version of this identity.

The functions $\schurHook_\lambda(\xvec/\yvec)$ are identically zero whenever
$\lambda_{m+1} \geq n$.


\todo{Add Goulden--Hamel version for flagged: https://arxiv.org/pdf/2512.13265}

\subsection[hookSchurPlethDef]{Plethysm definition}

The \defin{$(m,n)$-hook Schur functions} can be 
defined in \hyperref[plethysm]{plethystic notation} as 
\[
\schurHook_\lambda(x_1,\dotsc,x_m/y_1,\dotsc,y_n) \coloneqq \schurS_\lambda(X - t Y) \vert_{t=-1}
\]
where $X= x_1+\dotsb+x_m$ and $Y= y_1+\dotsb+y_n$, see \cite[Eq. (55)]{YangRemmel1998}.



\subsection[schurHookProperties]{Properties}

The following four properties uniquely characterize the hook Schur functions,
see \cite{Macdonald1995} and \cite{MoensJeugt2003}.

\begin{itemize}

\item
(Homogeniety) 
The polynomial $\schurHook_{\lambda}(\xvec/\yvec)$ is a homogeneous of degree $|\lambda|$.

\item 
(Factorization) If $\lambda_m\geq n \geq  \lambda_{m+1}$ 
so that $\lambda = (n^m + \tau)\cup \eta$, then
\[
\schurHook_{\lambda}(\xvec/\yvec) = 
\schurS_{\tau}(\xvec)\schurS_{\eta'}(\yvec) \prod_{i=1}^m\prod_{j=1}^n (x+i+y_j).
\]

\item 
(Cancellation) We have that
\[
\schurHook_{\lambda}(x_1,\dotsc,x_{m-1},t/y_1,\dotsc,y_{n-1},-t)
=\schurHook_{\lambda}(x_1,\dotsc,x_{m-1}/y_1,\dotsc,y_{n-1}).
\]

\item 
(Restriction) We have that
\[
\schurHook_{\lambda}(x_1,\dotsc,x_{m-1},0/\yvec)
=\schurHook_{\lambda}(x_1,\dotsc,x_{m-1}/\yvec)
\]
and 
\[
\schurHook_{\lambda}(\xvec/y_1,\dotsc,y_{n-1},0)
=\schurHook_{\lambda}(\xvec/y_1,\dotsc,y_{n-1}).
\]
\end{itemize}

We have the following properties of the hook Schur functions,
see e.g., \cite{YangRemmel1998}.

\begin{itemize}
\item 
$\schurHook_{\lambda/\mu}(\xvec/\emptyset) = \schurS_{\lambda/\mu}(\xvec)$.

\item 
$\schurHook_{\lambda/\mu}(\emptyset/\yvec) = \schurS_{\lambda'/\mu'}(\yvec)$.

\item 
$\schurHook_{\lambda/\mu}(\xvec/\yvec) = \schurHook_{\lambda'/\mu'}(\yvec/\xvec)$.

\item 
$\schurHook_{\lambda/\mu}(\xvec/ \yvec) = 
\sum_{\mu \subseteq \nu \subseteq \lambda} \schurS_{\nu/\mu}(\xvec) \schurS_{\lambda'/\nu'}(\yvec)$.
\end{itemize}

A Weyl-type formula for $\schurHook_{\lambda}(\xvec/\yvec)$ as a quotient 
of determinants is given in \cite[Eq. (1.17)]{MoensJeugt2003}.
This is referred to as the \defin{Sergeev--Pragacz} formula,
proved by \name{Sergeev} and independently in \cite{JeugtHughesKingThierry1990}.
A skew version was proved later in 1995 by \name{Hamel} and \name{Goulden}.


In \cite[Thm. 4.4]{Remmel1987}, a version of the
hook-content formula is  proved where an expression for
\[
\sum_{k,l\geq 0}
t^k s^l \schurHook_\lambda(1,q,q^2,\dotsc,q^k / 1,p,p^2,\dotsc,p^l) 
\]
is given.




\subsection[schurHookCauchy]{Cauchy identity}

In \cite{BereleRemmel1985}, the following Cauchy-type identity is proved.
\[
\sum_{\lambda}
\schurHook_\lambda(\xvec / \svec) 
\schurHook_\lambda(\yvec / \tvec) = 
\prod_{i,j} \frac{1+x_i t_j}{1-x_iy_j} \frac{1+y_i s_j}{1-s_i t_j}
\]
A bijective proof can be found  in \cite{YangRemmel1998}.


\subsection[schurHookLittlewoodFormula]{Littlewood formula}
M. Yang and J. Remmel \cite{YangRemmel1998} prove that
\[
\prod_{i\lt j} (1-x_i x_j)
\prod_{i\gt j} (1+y_i y_j)
\prod_{i, j} \frac{1}{1-x_iy_j} = 1 +\sum_{\alpha} \schurHook_\alpha(x_1,\dotsc,x_m/y_1,\dotsc,y_n)
\]
where we sum over all partitions of the form
\[
\alpha = 
\begin{pmatrix}
a_1  & a_2 & \dotsc & a_r \\
a_1+1& a_2+1& \dotsc & a_r+1
\end{pmatrix}
\]
in \hyperref[prelimPartitions]{Frobenius notation}.


\subsection[schurHookGeneralizations]{Generalizations}

A quasisymmetric refinement is introduced in \cite{MasonNiese2018},
and the symmetric hook Schur functions can be 
decomposed into such quasi-symmetric counterparts.
That is
\[
\schurHook_\lambda(x_1,\dotsc,x_m/y_1,\dotsc,y_n) = 
\sum_{\alpha \sim \lambda} \schurHookQS_\alpha(x_1,\dotsc,x_m/y_1,\dotsc,y_n)
\]

\todo{Make as a separate family}

The quasisymmetric hook Schur functions are positive in the 
super Gessel fundamental basis, see \cite[Theorem 4.2]{MasonNiese2018}.
They conjecture that the structure constants for 
quasisymmetric hook Schur functions are the same as for the 
quasisymmetric Schur functions.

\bigskip

There is also a generalization in the direction of supersymmetric Schur functions
indexed by \emph{composite partitions}. A conjectured Jacobi--Trudi formula 
was presented in \cite{Moens2007} and later proved in \cite{BinhDungHai2018}.





\section[bigSchur]{Big Schur functions}

\begin{polydata}{bigSchur}
  Name   & Big Schur functions \\
  Space    & Sym \\
  Basis    & Yes \\
  Rating   & 1 \\
  Bib      & Shigechi2017x \\
  Keywords & jacobi-trudi, tableau \\
  Year     & 2017 \\
  Categoty & Schur \\
\end{polydata}


In \cite{Shigechi2017x}, K. Shigechi introduces the big Schur functions.
These are closely related to \hyperref[schurP]{Schur's P functions},
and the \hyperref[hookSchur]{supersymmetric Schur functions}.

We consider fillings of $\lambda$ with entries in 
the alphabet $1' \lt 1 \lt 2' \lt 2 \lt \dotsb $
such that 
\begin{itemize}
\item 
	each row has at most one marked $i$ for every $i=1,2,\dotsc$ 
\item 
	each column has at most one unmarked $i$, for every $i=1,2,\dotsc$,
\item 
	entries in rows and columns are weakly increasing and
\end{itemize}
We let $SSShYT(\lambda)$ denote the set of such fillings.

Then, the \defin{big Schur function} $\bigSchur_{\lambda}(\xvec)$
is defined as 
\[
\bigSchur_{\lambda}(\xvec) = \sum_{T \in SSShYT(\lambda)} \xvec_T
\]
where the weight of a tableau is obtained by treating primed entries as unprimed.

We can also realize $\bigSchur_{\lambda}(\xvec)$ as the specialization $\schurHook_{\lambda}(\xvec/\xvec)$.


\begin{example}
We have
\[
\bigSchur_{211} = 12 \monomial_{32}+4 \monomial_{41}+40 \monomial_{221}+24 \monomial_{311}+80 \monomial_{2111}+160 \monomial_{11111}.
\]
\end{example}


Of course, there is a Jacobi--Trudi identity (also valid on the skew case):
\[
\bigSchur_{\lambda}(\xvec) = \det\left[ r_{\lambda_i -i +j}(\xvec) \right]_{1\leq i, j \leq \length(\lambda)}
\]
where $r_k(\xvec) = \sum_{\mu \vdash r} 2^{\length(\mu)} \monomial_{\mu}(\xvec)$.
Alternatively, $\sum_{k \geq 0} t^k r_k(\xvec) = \prod_{i} \frac{1+x_i t}{1-x_i t}$.


\todo{
Another note: The big Schur functions can also be realized as 
skew Schur's Q functions. 
}




\section[factorialHookSchur]{Factorial supersymmetric Schur polynomials}

\begin{polydata}{factorialHookSchur}
  Name   & Factorial supersymmetric Schur polynomials \\
  Space    & SuperSym \\
  Basis    & True \\
  Rating   & 1 \\
  Bib      & Molev1996 \\
  Year     & 1996 \\
  Keywords & jacobi-trudi,tableaux,skew \\
  Symbol   & $\schurS_\lambda(\xvec/\yvec|a)$ \\
  Category & Schur \\
\end{polydata}


In \cite[Def. 1.1]{Molev1996}, \name{Alexander Molev} introduces a factorial version of supersymmetric Schur functions.
They can be described  via tableaux and Jacobi--Trudi identities. 
Molev also proves a characterization theorem and a Sergeev--Pragacz type formula,
and introduces the \defin{shifted supersymmetric Schur polynomials}.


In \cite{FoleyKing2020x}, the authors present determinant identities 
for skew factorial supersymmetric Schur functions.
