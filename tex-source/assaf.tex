\metatitle{Monomial slide polynomials and fundamental slide polynomials}
\metadescription{An introduction to monomial slide polynomials and fundamental slide polynomials, their definitions, properties, and connections to quasisymmetric functions and Schubert polynomials.} 

\section[slideM]{Monomial slide polynomials}

\begin{polydata}{slideM}
  Name  & Monomial slide polynomials \\
  Space    & All \\
  Basis    & True \\
  Rating   & 1 \\
  Bib      & AssafSearles2016 \\
  Year     & 2016 \\
  Symbol   & $\slideM_\alpha(\xvec)$ \\
  Category & Schur \\
\end{polydata}

The \defin{monomial slide polynomials} were introduced by S. Assaf and D. Searles in \cite{AssafSearles2016}.
The slide polynomials form a basis for the space of polynomials, and can be seen as a lift 
of the \hyperref[qmonom]{monomial quasisymmetric functions}.


\begin{definition}
For $\alpha$ being a weak composition, we set
\[
\slideM_\alpha(\xvec) \coloneqq \sum_{\substack{b \trianglerighteq a \\ \mathrm{flat}(b)=\mathrm{flat}(a)}} \xvec^b
\]
where $\mathrm{flat}(\beta)$ denotes the composition obtained by removing all 0s from $\beta$,
and $\trianglerighteq$ denotes \hyperref[partialOrderingsOnPartitions]{dominance order}.
\end{definition}

As an example (from \cite[Eq. 3.7]{AssafSearles2016}), $\slideM_{(0,2,0,3)}(\xvec) = x_1^2x_2^3 + x_1^2x_3^3 + x_1^2 x_4^3 + x_2^2 x_3^3  + x_2^2 x_4^3$.



The monomial slide polynomials have the \hyperref[qmonom]{monomial quasisymmetric functions}
as stable limit:
\[
 \lim_{m \to \infty} \slideM_{0^m\times \alpha}(\xvec) = \qmonom_{\mathrm{flat}(\alpha)}(\xvec).
\]


\section[slideF]{Fundamental slide polynomials}

\begin{polydata}{slideF}
  Name &  Fundamental slide polynomials \\
  Space    & All \\
  Basis    & True \\
  Rating   & 1 \\
  Bib      & AssafSearles2016 \\
  Year     & 2016 \\
  Symbol   & $\slideF_\alpha(\xvec)$ \\
  Category & Schur \\
\end{polydata}

The \emph{fundamental slide polynomials} (or sometimes just \defin{slide polynomials})
were introduced by S. Assaf and D. Searles in \cite{AssafSearles2016}.
The slide polynomials form a basis for the space of polynomials, and can be seen as a lift 
of the \hyperref[gessel]{gessel quasisymmetric functions}.

The $K$-theoretical analog of fundamental slide 
polynomials are the \emph{glide polynomials}, see \cite{PechenikSearles2017}.
\todo{add glide polynomials from \cite{PechenikSearles2017}}

The fundamental slide polynomials expand positively 
in the \hyperref[slideM]{monomial slide polynomials}.
Moreover, the \hyperref[schubert]{Schubert polynomials}
expand positively in the fundamental slide polynomials, \cite[thm. 3.13]{AssafSearles2016}.
The main motivation for introducing the fundamental slide polynomials,
is that products of Schubert polynomials can be expanded (with a combinatorial formula) into
fundamental slide polynomials.

\begin{definition}
For $\alpha$ being a weak composition, \defin{fundamental slide polynomial} $\slideF_\alpha$
is defined as
\[
\slideF_\alpha(\xvec) \coloneqq \sum_{\substack{b \trianglerighteq a \\ \mathrm{flat}(b)\text{ refines }\mathrm{flat}(a)}} \xvec^b
\]
where $\mathrm{flat}(\beta)$ denotes the composition obtained by removing all 0s from $\beta$,
and $\trianglerighteq$ denotes \hyperref[partialOrderingsOnPartitions]{dominance order}.
\end{definition}


The fundamental slide polynomials have the \hyperref[gessel]{gessel quasisymmetric functions}
as stable limit:
\[
 \lim_{m \to \infty} \slideF_{0^m\times \alpha}(\xvec) = \gessel_{\mathrm{flat}(\alpha)}(\xvec).
\]

\subsection[slidePositive]{Slide positive families}

\hyperref[key]{Key polynomials} expand positively in the 
fundamental slide basis, \cite[Thm. 2.13]{AssafSearles2018}.
In \cite{ChoWilligenburg2022}, the authors determine for which $\alpha$,
the key polynomials $\key_\alpha$ expanded into slide polynomials are multiplicity free.

In \cite{AssafBergeron2019}, the authors consider a flagged version of 
$(P,w)$-partitions, and show that these are slide-positive.
In \cite{TewariWilsonZhang2022}, it is shown that certain polynomials similar to 
chromatic symmetric functions are slide-positive.


See \cite{SmirnovTutubalina2021} the notion of \emph{slide complexes}.


\section[lock]{Lock polynomials}

\begin{polydata}{lock}
  Name   & Lock polynomials \\
  Space    & All \\
  Basis    & True \\
  Rating   & 1 \\
  Bib      & AssafSearles2019 \\
  Year     & 2019 \\
  Symbol   & $\lock_\alpha(\xvec)$ \\
  Category & Schur \\
\end{polydata}

Lock polynomials were introduced by S. Assaf and D. Searles in \cite{AssafSearles2019}.
The Lock polynomials form a basis for the polynomial ring, 
and are indexed by weak compositions.
The combinatorial formula for these is
\[
\lock_\alpha(\xvec) \coloneqq \sum_{T \in LT(\alpha)} \xvec^T
\]
where the sum is over all \emph{lock tableaux}.

As an example (from \cite{Wang2020}), we have
\[
\lock_{(0,2,3)}(\xvec) = x_2^2 x_3^2 + x_1x_2x_3^3 + x_1^2x_3^3 + x_1x_2^2x_3^2 + 
x_1^2x_2x_3^2 + x_1^2x_2^2x_3 + x_1^2 x_2^3
\]

Whenever the non-zero parts of $\alpha$ are weakly decreasing,
we have that the lock polynomial coincides with a \hyperref[key]{key polynomial};
$\lock_\alpha(\xvec)=\key_\alpha(\xvec)$, see \cite[Thm. 6.12]{AssafSearles2019}.

There is a \hyperref[crystals]{crystal structure} on lock polynomials, 
explored by G. Wang in \cite{Wang2020}.
This crystal structure embeds naturally into Demazure crystals.


