\metatitle{Lindström--Gessel--Viennot lemma}
\metadescription{The Lindström--Gessel--Viennot lemma and applications}



\section[lgvLemma]{The Lindström--Gessel--Viennot lemma}

See \cite{Lindstrom1973,GesselViennot1989}.

\todo{https://xavierviennot.org/xavier/articles_files/determinant_89.pdf}

\todo{https://arxiv.org/pdf/1503.05934}

\todo{https://www.irif.fr/~chapuy/chapuyCombinatoricsNotesMPRI.pdf}

\todo{https://www.sciencedirect.com/science/article/pii/0001870885901215}


\subsection[LGVssyt]{Semistandard tableaux}


Semistandard Young tableaux can be put in correspondence with sets of non-intersecting lattice paths,
and this allows us to prove the two \hyperref[schurJacobiTrudi]{Jacobi--Trudi identities},
by using the Lindström--Gessel--Viennot lemma \cite{Lindstrom1973,GesselViennot1989}.

Let $T \in \SSYT(\lambda/\mu)$ where the entries are bounded by $m$.

\emph{Row bijection}: 
Each row $i$ of $T$ is mapped to a lattice path $P_i$, with steps in the set $\{ (0,1), (1,0)\}$.
The starting vertex of $P_i$ is $(-i + \mu_i,1)$,
the ending vertex is $(-i + \lambda_i, m)$,
and its number of horizontal steps on $y$-coordinate $j$ is given by the number of entries in
row $i$ equal to $j$. The length of path $P_i$ is $m-1 + \lambda_i-\mu_i$.
This bijection can be used to prove the first \hyperref[skewJacobiTrudi]{Jacobi--Trudi identity}.


\emph{Column bijection}: 
Each column of $T$ is mapped to a path $P'_i$,
with steps in the set $\{ (-1,1), (1,1)\}$.
The starting vertex of $P'_i$ is given by $(2i - 2\mu'_i, 0)$,
the ending vertex is $(2i - 2\lambda'_i + m, m)$
and the $j$th step of $P'_i$ is $(-1,1)$ if $j$ is present in column $i$,
otherwise it is $(1,1)$. Each path has length $m$.
This bijection can be used to prove the second Jacobi--Trudi identity.


\begin{example}[Bijection to lattice paths]

Consider the following skew semistandad Young tableau $T$.
\begin{figure}
\ytableaushort{{\none}{\none}{\none}1122346,{\none}{\none}1223455,{\none}{\none}2344566,113556,2346,456,6}
\end{figure}

The rows give rise to the following set of non-intersecting lattice paths,
where the topmost row in $T$ corresponds to the rightmost path.
\begin{figure}
\svgimg[width=0.75\textwidth]{./svg-images/ssyt-lattice-paths-1.svg}{SSYT rows as lattice paths.}  
\end{figure}

The columns in $T$ give rise to the following set of non-intersecting lattice paths,
where the leftmost column corresponds to the leftmost path.

\begin{figure}
  \svgimg[width=0.95\textwidth]{./svg-images/ssyt-lattice-paths-2.svg}{SSYT columns as lattice paths.}
\end{figure}
\end{example}


The following theorem appears in \cite{McDowell2023} and is proved using lattice paths.
\begin{theorem}
Let $A$ and $B$ be subsets of $\{0,1,\dotsc,n\}$ of equal size and let $A^c$, $B^c$ be their complements.
Then
\[
\det \left[ \completeH_{b-a}( x_1,x_2,\dotsc,x_{a+1} ) \right]_{a \in A, b \in B} =
\det \left[ \elementaryE_{a'-b'}( x_1,x_2,\dotsc,x_{a'} ) \right]_{a' \in A^c, b' \in B^c}.
\]
Moreover, this is related to \name{C. A. Aitken}'s identity;
\[
\det \left[ \completeH_{b-a}(\xvec ) \right]_{a \in A, b \in B} =
\det \left[ \elementaryE_{a'-b'}( \xvec ) \right]_{a' \in A^c, b' \in B^c}.
\]
\end{theorem}



See also \cite{Stembridge1990} for background on counting lattice paths with Pfaffians.
One can also prove the Weyl alternant formula by LGV, see \cite{Xiong2020x}.

See \href{https://www.youtube.com/watch?v=H1wEiEIznaA}{this GOCC talk from 2023} on a generalization of the LGV lemma.



\section[LGVplanePartitions]{Plane partitions as lattice paths}


See \cite[Thm. 10.7.1]{Krattenthaler2015} and \cite[Eq. (6)]{StanleyPitman2002} for the following
application of the Lindström--Gessel--Viennot lemma.

\begin{theorem}
Let $\lambda$ and $\mu$ be partitions with at most $\ell$ parts such that 
$\lambda/\mu$ defines a skew shape.
Then the number of lattice paths contained in $\lambda / \mu$ equals
\[
  \det \left[ \binom{\lambda_i - \mu_j+1}{i-j+1} \right]_{1\leq i,j \leq \ell.}
\]
\end{theorem}
For non-skew shapes, we can also interpret such partitions $\nu$ as lattice
points in Stanley--Pitman \hyperref[polytope]{polytopes} \cite{StanleyPitman2002}.
This is also the number of plane partitions of skew shape $\lambda/\mu$, with binary entries,
or the number of non-nesting rook placements on the Ferrers board of shape $\lambda / \mu$.
There is also an interpretation of these quantities using certain linear extensions, see \cite{AlexanderssonJal2024x}.


\todo{https://www.mat.univie.ac.at/~kratt/artikel/generat.pdf}


\todo{https://arxiv.org/pdf/2509.04789  Cramer's rule via LGV}
