\metatitle{Tutte symmetric functions}
\metadescription{Background on Tutte symmetric functions.}



\section[tutteSymmetric]{Tutte symmetric functions}


\begin{polydata}{tutteSymmetric}
  Name   & Tutte symmetric functions \\
  Space  & Sym \\
  Basis  & False \\
  Rating & 1 \\
  Bib    & Stanley1998 \\
  Year   & 1998 \\
\end{polydata}

\todo{Find earliest reference}

\todo{add arXiv:2007.11042 }

The \defin{Tutte symmetric functions} were introduced by \name{R. Stanley} \cite[Def. 3.1]{Stanley1998},
and generalize the chromatic symmetric functions.

The Tutte symmetric functions are indexed by graphs, and defined as
\begin{equation}
 \tutte_G(\xvec;t) \coloneqq \sum_{\pi \vdash V(G)} (1+t)^{a(\pi)} \tilde{\monomial}_{\lambda(\pi)}
\end{equation}
where the $\tilde{\monomial}$ denote augmented monomial symmetric functions,
and the sum is taken over all set-partitions of the vertex set of $G$.
Here, $a(\pi)$ denote the number of \defin{attacking edges} --- edges where both endpoints are in the 
same block of $\pi$.

Alternatively, for a graph on $n$ vertices, we have
\begin{equation}
 \tutte_G(\xvec;t) \coloneqq \sum_{\kappa : V(G) \to \setN} (1+t)^{m(\kappa)} x_{\kappa(1)} \dotsm 
 x_{\kappa(n)}
\end{equation}
where the sum is over all vertex colorings of $G$, and $m(\kappa)$ counts
the number of monochromatic edges in the coloring.

Note that $\tutte_G(\xvec; -1) = \chrom_G(\xvec)$, that is, we recover the
\hyperref[chromaticQuasisymmetric]{chromatic symmetric function} at $t=-1$.


A quasisymmetric version of the Tutte symmetric functions 
were introduced in \cite{AwanBernardi2016}.
In \cite[Thm. 7.15]{AlexanderssonSulzgruber2019},
we give the expansion of these polynomials in the quasisymmetric powersum basis.


See \cite{CrewSpirkl2020x} for more results on the Tutte symmetric functions.


\subsection[pExpansionTutte]{Powersum expansion}

We have that (see \cite{Stanley1998})
\[
 \tutte_G(\xvec;t) = \sum_{S \subseteq E(G)} t^{|S|} \powerSum_{\lambda(S)}(\xvec)
\]
where $\lambda(S)$ denotes the sizes of the connected components induced by $S$.


\subsection[tutteSpanningTrees]{Spanning tree expansion}

Formula for computing $\tutte_G(\xvec;t)$ using spanning trees and spanning forests is given in \cite{McDonaldMoffatt2012}.
This generalizes the classical formula for computing Tutte polynomials using spanning trees.
A vertex-weighted version is stated in \cite[Eq. (15)]{AlistePrietoCrewSpirklZamora2020x}.

