\metatitle{Kohnert polynomials}
\metadescription{An introduction to Kohnert polynomials, their definition via Kohnert diagrams, and their properties including monomial slide positivity and key positivity.}
\metakeywords{Kohnert polynomials,key-positive,divided-difference,fillings}

\section[kohnert]{Kohnert polynomials}

\begin{polydata}{kohnert}
  Name   & Kohnert polynomials \\
  Space    & All \\
  Basis    & False \\
  Rating   & 1 \\
  Bib      &  AssafSearles2019 \\
  Year     & 2019  \\
  Symbol   & $\kohnert_{D}(z)$ \\
  Keywords & key-positive, divided-difference, fillings \\
  Category & Schur \\
\end{polydata}

\defin{Kohnert polynomials} were introduced by \name{Sami Assaf} and \name{Dominic Searles} in \cite{AssafSearles2019}.
This is a large family of polynomials, which contains several 
of the classical families as special cases, in particular
the family of \hyperref[schubert]{Schubert polynomials}
and \hyperref[key]{key polynomials}.

All Kohnert polynomials are \hyperref[slideM]{monomial slide positive}, \cite[Thm. 3.7]{AssafSearles2019},
and they were later proved to be key positive in \cite{Assaf2021}.



See \cite{ArmonAssafBowlingEhrhard2023} for a representation-theoretical interpretation of Kohnert polynomials.
