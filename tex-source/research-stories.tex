\metatitle{Research stories}
\metadescription{Research stories}


\section[research-story-table]{Big table of data}

Boris Shapiro sent me an email with the following table of data --- it was actually a few lines longer.
He of course wanted a formula for the entries.
I first went to the \href{https://oeis.org/}{OEIS} but no luck there. 
Boris had already tried this resource as well but one shall always do a double check.

\begin{array}{cccccc}
 1 & 1 \\
 1 & 3 \\
 1 & 4 & 3  \\
 1 & 5 & 10  \\
 1 & 6 & 16 & 9  \\
 1 & 7 & 23 & 33  \\
 1 & 8 & 31 & 61 & 27 \\
 1 & 9 & 40 & 98 & 108 \\
 1 & 10 & 50 & 145 & 225 \\
 1 & 11 & 61 & 203 & 397 & 351
\end{array}

So the rows add up to powers of two, and Boris suggested 
that this is might be related to the celebrated Pascals triangle.
I tried to subtract various versions of Pascals triangle from the entries, but not much luck.

Boris had already noted that the columns seemed to be (eventually) polynomial in the row-index,
so I started to write these polynomials in different polynomial bases.
There is a notion of binomial transforms of sequences, so I tried this on the columns.
This did not lead anywhere either and I decided to give up.


About three weeks later, I met with Boris and we discussed different research topics.
I got some more context on the table above --- 
there is a more general story, and every family of graphs give rise to a table as above.
It is natural to then think of the rows as a refinement of graph orientations,
so I went through my favorite list of statistics 
on graph orientations (which at the moment is very short).

No luck, but I did the next natural thing and encoded the rows as $q$-analogs (of say binary words),
and then check for statistics in FindStat. No hit there either.

\begin{array}{l}
 q+1 \\
 3 q+1 \\
 3 q^2+4 q+1 \\
 10 q^2+5 q+1 \\
 9 q^3+16 q^2+6 q+1 \\
 33 q^3+23 q^2+7 q+1 \\
 27 q^4+61 q^3+31 q^2+8 q+1 \\
 108 q^4+98 q^3+40 q^2+9 q+1 \\
 81 q^5+225 q^4+145 q^3+50 q^2+10 q+1 \\
 351 q^5+397 q^4+203 q^3+61 q^2+11 q+1 \\
\end{array}

Boris and I had for some reason talked about orthogonal polynomials and linear recursions,
while discussing a different project. I figured I give this test a shot.

\subsection[research-story-linrec]{Method}

Given a sequence $p_n(q)$ as above, 
there is a quick way to see if it is likely to satisfy a linear recursion.
We simply compute a Toeplitz determinant. 
I first computed the following simple $3\times 3$-determinant
\[
\begin{vmatrix}
p_k(q) & p_{k+1}(q) & p_{k+2}(q) \\
p_{k-1}(q) & p_{k}(q) & p_{k+1}(q) \\
p_{k-2}(q) & p_{k-1}(q) & p_{k}(q)
\end{vmatrix}
\]
for $k=3,4,\dotsc,11$ and some surprisingly simple expressions. 
Increasing the dimension,
\[
\begin{vmatrix}
p_k(q) & p_{k+1}(q) & p_{k+2}(q) & p_{k+3}(q) \\
p_{k-1}(q) & p_{k}(q) & p_{k+1}(q) & p_{k+2}(q)\\
p_{k-2}(q) & p_{k-1}(q) & p_{k}(q) & p_{k+1}(q) \\
p_{k-3}(q) & p_{k-2}(q) & p_{k-1}(q) & p_{k}(q)
\end{vmatrix}
\]
gave $0$ for every value of $k$. Bingo! 

This means that the sequence of polynomials $\{p_k(q)\}_{k=1}^{\infty}$
should satisfy a linear recursion involving $4$ terms (so they do not form an orthogonal family).
In Mathematica, it looked something like this:

\begin{lstlisting}
eqns = Table[
    pp[n] - (AA pp[n - 1] + BB pp[n - 2] + CC pp[n - 3]) == 0
    , {n, 4, 12}] /. pp -> p;
Solve[And @@ eqns, {AA, BB, CC}]

{{AA -> 1, BB -> 3 q, CC -> -2 q}}
\end{lstlisting}


Hence, the initial conditions $p_0(q) = 1+q$, $p_1(q) = 1+3q$, $p_2(q) = 1 + 4 q + 3 q^2$
together with the recursion
\[
p_n(q) = p_{n-1} + 3q \cdot  p_{n-1} - 2q \cdot p_{n-2}
\]
should give the sequence of polynomials. This checks out!

Finally, standard generating function trickery gives that
\[
\sum_{k=0}^\infty p_{k}(q) t^k = 
\frac{(1 + q) + 2 q t + (-2 q) t^2}{
 1 - t - 3 q t^2 + 2 q t^3}.
\]

From this, we can (with computer), see that the roots of $p_{k}(q)=0$ as $k \to \infty$
accumulate on a curve between the two points $\frac{1}{24}(-1,\pm 5 \sqrt{5/3})$.

