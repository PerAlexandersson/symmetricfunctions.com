\metatitle{Miscellaneous cyclic sieving}
\metadescription{A list of miscellaneous cyclic sieving phenomena}

See the \hyperref[cyclicSieving]{cyclic sieving phenomenon} 
page for the definition and related theorems.




\todo{Add the conjectures in https://arxiv.org/pdf/1907.09337.pdf}


\todo{Add conjectures in https://arxiv.org/pdf/2006.01568.pdf}



\section[cspMisc]{Miscellaneous}


\subsection[cspListPosetsRoot]{Root posets}


Let $\Phi$ be a \hyperref[root-systems]{root system} and $\Phi^+$ its poset of positive roots.
Armstong--Stump--Thomas \cite{ArmstrongStumpThomas2013} studied the action of the
rowmotion operator (a.k.a. the \enquote{Fon-Der-Flaass/Panyushev map}) on the antichains of $\Phi^+$.
Addressing conjectures of Panyushev and Bessis--Reiner, they showed that this action is
in equivariant bijection with the Kreweras complementation action on
the $\Phi$-noncrossing partition lattice.
Moreover, they showed that the $q$-$\Phi$-Catalan polynomial is a cyclic sieving polynomial
for the action of rowmotion on the  antichains of $\Phi^+$ (equivalently, for the Kreweras
complement acting on $\Phi$-noncrossing partitions).
Note that the Type $A$ case of this CSP is equivalent to
the \hyperref[cspNonCrossingMatchingsI]{Non-crossing matchings} CSP.


Rowmotion can be realized as a certain composition of toggles.
By making these toggles piecewise-linear, one obtains a piecewise-linear action
of rowmotion on height $m$ \hyperref[pPartition]{$P$-partitions} of a poset $P$.
Hopkins (see \cite{Hopkins2020}) conjectured a CSP for the action
of piecewise-linear rowmotion on these $P$-partitions when $P$ is
root poset \enquote{of coincidental type}, where the sieving polynomial is
the so-called \emph{$q$-$\Phi$-multi-Catalan number}.
This conjecture extend the aforementioned result of Armstong--Stump--Thomas, which is the case $m=1$.


\subsection[cspListFrieze]{Polygonal dissections and Frieze patterns}

In \cite{AdamsBanian2025x}, the authors prove CSP for 
the set $A_\mu$, consisting of non-crossing dissections of an $(n+2)$-gon, where the number of $j$-gons in
the dissection is $\mu_j$. The group action is rotation of the $(n+2)$-gon.
This is gives 
\[
\left( A_\mu, C_{n+2},  \frac{1}{[n+1]_q} \qbinom{n+k}{k}\qbinom{k}{\mu_1,\mu_2,\dotsc,\mu_n} \right)
\]
where $k-1$ is the number of non-crossing diagonals in the dissection.

There is then a certain bijection between non-crossing dissections of $(n+2)-gons$ and classes of Frieze patterns,
see \cite{HolmJorgensen2018}.



\subsection[cspListPosetsMinuscule]{Minuscule posets}



In \cite{RushShi2012}, the authors consider rowmotion acting on the antichains of minuscule posets.
The corresponding CSP-polynomial is the rank-generating function of the poset,
which for minuscule posets enjoys some particularly nice properties, see \cite[Exercise 170]{StanleyEC1}.

For the product of two chains, this action is a special case of the
\hyperref[cspPlanePartitions]{plane partitions CSP}.

Hopkins (see \cite{Hopkins2020}) conjectured a CSP for the action of piecewise-linear
rowmotion on height $m$ $P$-partitions when $P$ is a minuscule poset, where the
sieving polynomial is the rank-generating of $P\times[m]$,
which again has nice properties (in particular, a product formula).
This conjecture extends the aforementioned result of Rush--Shi, which is the case $m=1$.



\subsection[csp01Matrices]{BiCSP on 01-matrices and Hall--Littlewood}

In \cite{BarceloReinerStanton2008}, a bi-cyclic sieving phenomenon on permutation matrices is described.
Let $X_n$ be the set of $n\times n$ permutation matrices,
and let $\grpc_n \times \grpc_n $ act on $X_n$ by cyclic shift on rows, and columns.
Then
\[
\left(X_n, \grpc_n \times \grpc_n, \epsilon(q,t)\sum_{\lambda \vdash n} f^\lambda(q) f^\lambda(t) \right)
\]
exhibit the bi-cyclic sieving phenomenon. 
Here, $\epsilon(q,t)$ is $(qt)^{n/2}$ if $n$ is even, and$1$ if $n$ is odd.
In fact, they prove a much stronger statement regarding complex reflection groups.

This is further generalized in \cite{Rhoades2010b} to $01$-matrices,
where the column sums and row sums in the matrices are given by fixed integer partitions.

\name[Brendon Rhoades]{B. Rhoades} further generalizes this to matrices with non-negative 
integer entries, and prove a bi-cyclic phenomenon using
\[
\epsilon(q,t)\sum_{\lambda \vdash n} K_{\lambda,\mu}(q) K_{\lambda,\nu}(t)
\]
as biCSP polynomial. Note that this is closely 
related to the \hyperref[rsk]{Robinson--Schensted--Knuth algorithm}.



\subsection[cspTensors]{Tensor products}

In \cite{Westbury2016}, cyclic sieving phenomena are created via representation theory
and crystals, generalizing the promotion CSP by \name[Brendon Rhoades]{Rhoades}.
This is related to the principal specialization of \hyperref[frobeniusCharacteristic]{Frobenius image} of 
characters of the symmetric group.



\subsection[cspLatticePolytopes]{Lattice polytopes}

\name[J. Propp]{J. Propp} asks for a solution to the following problem.
Let $P \subset \setR^d$ be a lattice polytope 
and $g$ be a \emph{linear map} $\setR^d \to \setR^d$, such that $\langle g \rangle$ 
is a cyclic group of order $n$ where $g(P) = P$.

Can we find a linear function $\sigma:\setZ^d \to \setZ$ such that for 
the $q$-analogue of the \hyperref[ehrhart]{Ehrhart function} of $P$, defined as
\[
p_m(q) \coloneqq \sum_{x \in mP \cap \setZ^d} q^{\sigma(x)},
\]
we have that $(mP \cap \setZ^d, \langle g \rangle, p_m(q))$ is a CSP-triple, for all $m \in \setN$?
That is, we get a family of cyclic sieving phenomena on the lattice points of dilations of $P$.


\subsection[cspCyclicCodes]{Hamming codes}

\name[Alexander Mason]{A. Mason}, \name[Vic Reiner]{V. Reiner} and \name{S. Sridhar} have some results on 
cyclic codes and (dual) Hamming codes in \cite{MasonReinerSridhar2020x}.


\subsection[cspMaximalWeights]{Dominant maximal weights}

A cyclic sieving phenomenon on dominant maximal weights in affine Kac--Moody algebras
is described in \url{https://arxiv.org/pdf/1909.07010.pdf}.
