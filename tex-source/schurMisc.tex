\metatitle{Symplectic and Orthogonal Schur polynomials}
\metadescription{An introduction to symplectic and orthogonal Schur polynomials, their definitions via tableaux, Jacobi--Trudi identities, Cauchy identities, and other properties.}  

\section[schurSymplectic]{Symplectic Schur polynomials}


\begin{polydata}{schurSymplectic}
  Name     & Symplectic Schur polynomials \\
  Space    &  Other   \\
  Basis    &  False   \\
  Rating   &  1      \\
  Bib      &  King1971\\
  Year     &  1971? \\
  Keywords & jacobi-trudi,tableaux,cauchy-identity,skew \\
  Symbol   & $\schurSp_{\lambda/\mu}(\xvec)$ \\
  Category & Schur \\
\end{polydata}


\todo{
Combinatorial def: https://pdfs.semanticscholar.org/f1e7/d5a08b9ff7745473403237f170795f1b32e2.pdf 
}
\todo{http://de.arxiv.org/pdf/1804.08495.pdf}


\todo{
Orthosymplectic pieri rule
https://www.google.com/url?sa=t&rct=j&q=&esrc=s&source=web&cd=22&ved=2ahUKEwjJ1-28vtvdAhWkiaYKHd7pDks4FBAWMAF6BAgIEAI&url=http%3A%2F%2Fwww.combinatorics.org%2Fojs%2Findex.php%2Feljc%2Farticle%2Fdownload%2Fv25i3p37%2Fpdf&usg=AOvVaw2KLWUbv88cUcV-wRPc4FQI
}


\todo{
Operator and Giambelli formulas
http://citeseerx.ist.psu.edu/viewdoc/download?doi=10.1.1.11.6964&rep=rep1&type=pdf
}


The \emph{even symplectic Schur polynomials} were explicitly described combinatorially in \cite{King1976},
and are characters for the lie group $Sp(2n)$. See also \cite{Krattenthaler1998} for a good reference.

We work in the alphabet $\xvec = (x_1,x_1^{-1},x_2,x_2^{-1},\dotsc,x_n,x_n^{-1})$.
A \defin{symplectic tableau} (also known as King tableau) of shape $\lambda/\mu$ is a filling of the shape $\lambda/\mu$
with entries in 
\[
1 \lt \overline{1} \lt 2 \lt \overline 2 \lt \dotsb \lt n \lt \overline{n}
\]
such that (a) entries are weakly increasing along rows, and strictly increasing along columns and
(b) entries in row $i$ are greater than or equal to $i$.
Let $SP(\lambda/\mu)$ denote the set of such tableaux.
We then define the \defin{even symplectic Schur polynomials} 
(using the definition in \cite{Hamel1997}) as
\[
\schurSp_{\lambda/\mu}(\xvec) = \sum_{T \in SP(\lambda/\mu)} \xvec^{T},
\]
where entries $i$ in $T$ contribute with $x_i$, and entries $\overline{j}$ contribute with $x_j^{-1}$.

\begin{example}
The following tableau is an element in $SP(\lambda/\mu)$
\begin{figure}
\begin{ytableau}
1 & 1 & \overline{1} & \overline{1} & \overline{2} & 3 \\
\overline{1} & \overline{2} & 3 & 4 \\
3 & 3
\end{ytableau}
\end{figure}
\end{example}


To prove that these are symmetric functions, see \cite[Thm. 6.12]{SundaramThesis}
where a type of Bender-Knuth involutions are defined.
The proof however \href{https://mathoverflow.net/questions/362997/bender-knuth-involutions-for-symplectic-king-tableaux}{seems to have a gap},
a complete proof is available in \cite{Gutirrez2024}.


There are also altnerative tableaux-descriptions. For example, the \defin{Kashiwara--Nakashima tableaux},
see for example \cite{AzenhasFellerTorres2022}. There is a $U_q(\mathfrak{sp}(2m,\setC))$-crystal structure
defined on these KN-tableaux.


\subsection[schurSymplecticBialternant]{Bialternant formula}

We have that the \emph{even symplectic Schur functions} can be expressed as 
\[
\schurSp_\lambda(x^{\pm 1}_{1},x^{\pm 1}_{2},\dotsc, x^{\pm 1}_{n})
=
\frac{
\left| x_i^{\lambda_j+n-j+1} - x_i^{-(\lambda_j+n-j+1)}\right|_{1 \leq i,j \leq n}
}{
\left| x_i^{n-j+1} - x_i^{-(n-j+1)}\right|_{1 \leq i,j \leq n}
}
\]

The \emph{odd symplectic Schur functions} has a more complicated bialternant formula,
see \url{http://de.arxiv.org/pdf/1905.12964.pdf}.


\subsection[schurSymplecticCauchy]{Cauchy-identity}


The following Cauchy-identity is proved by S. Sundaram in her thesis, see \cite{SundaramThesis}.
\[
\sum_{\lambda} \schurSp_\lambda(x^{\pm 1}_{1},x^{\pm 1}_{2},\dotsc, x^{\pm 1}_{n}) \schurS_\lambda(y_1,\dotsc,y_n)
= \prod_{1\leq i \lt j \leq n} (1-y_i y_j)
\prod_{1\leq i \lt j \leq n} \frac{1}{(1-y_i x_j)(1-y_i x^{-1}_j)}.
\]





\subsection[schurSymplecticJacobiTrudy]{Jacobi--Trudi}

In \cite{FulmekKrattenthaler1997}, the following analogues of the Jacobi-Trudi identities are proved:
\begin{align}
\schurSp_\lambda(x^{\pm 1}_{1},x^{\pm 1}_{2},\dotsc, x^{\pm 1}_{n}) 
& = \frac{1}{2}\det[ \completeH_{\lambda_i - i+j}(\xvec) + \completeH_{\lambda_i - i- j + 2}(\xvec) ]_{1\leq i,j \leq \length(\lambda)} \\
& = \det[ \elementaryE_{\lambda'_j - j+i}(\xvec) - \elementaryE_{\lambda'_j - j- i}(\xvec) ]_{1\leq i,j \leq \lambda_1}.
\end{align}
Note that these formulas allow us to consider $\schurSp_\lambda(\xvec)$ as a symmetric function, 
by instead using the infinite alphabet $x_1,x_2,\dotsc$.

\begin{example}
We have the following expansions in the monomial basis:
\begin{align}
\schurSp_{4}  &= \monomial_{4} + \monomial_{22} + \monomial_{31}+\monomial_{211}+\monomial_{1111} \\
\schurSp_{31} &= -\monomial_{2} - \monomial_{11} + \monomial_{22}+\monomial_{31}+2\monomial_{211}+3\monomial_{1111} \\
\schurSp_{22} &= - \monomial_{11} + \monomial_{22} + \monomial_{211}+2\monomial_{1111} \\
\schurSp_{211} &= 1 - \monomial_{2} -2 \monomial_{11} + \monomial_{211} + 3\monomial_{1111} \\
\schurSp_{1111} &= - \monomial_{11} + \monomial_{1111} 
\end{align}
Observe that one needs to substitute the finite symplectic alphabet 
in these expressions in order for these formulas to agree with the tableau definition.
For example,
\begin{align}
\schurSp_{11}(\xvec) &= - 1 + \monomial_{11}(\xvec)\\
\schurSp_{11}(x_1,x_1^{-1},x_2,x_1^{-2}) &=
1 + x_1^{-1}x_2^{-1} + x_1x^{-1}_2+x^{-1}_1x_2 + x_1x_2
\end{align}
\end{example}


\subsection[schurSymplecticGiambelli]{Giambelli formula}

In Frobenius coordinates, we have that
\[
\schurSp_{(\alpha|\beta)} = \left| \schurSp_{(\alpha_i|\beta_j)} \right|_{s \times s}
\]
see e.g. \cite[Eq. (3.11)]{FulmekKrattenthaler1997}.



\subsection[schurSymplecticCrystal]{Kashiwara crystals}

In \cite{Lee2019} S. J. Lee defines a \hyperref[crystals]{crystal graph} structure on King tableaux,
so that connected components correspond to single symplectic Schur functions.
The crystal structure on $K(\mu,n)$ --- the set of symplectic tableaux of shape $\mu$ and maximal entry at most $n$ ---
is isomorphic to the crystal $B(\mu)$, the irreducible representation of $\mathfrak{sp}_{2n}$ indexed by $\mu$.

Lee also shows a bijection between symplectic tableaux and so called \emph{oscillating tableaux}, preserving the crystal structure.

An explicit conjecture regarding the expansion
\[
 \schurSp_\lambda(\xvec^{\pm}) \schurS_\mu(\xvec^{\pm}) = \sum_{\nu} c^{\nu}_{\lambda\mu} \schurSp_\nu(\xvec^{\pm})
\]
is also given in \cite[Conj. 6.2]{Lee2019}.


\subsection[schurSymplecticOtherProperties]{Further properties}

There are analogs of Gelfand--Tsetlin patterns for these functions, see e.g. \cite{AyyerFischer2019}.

One can express the even symplectic Schur functions as a sum of skew Schur functions as
\[
\schurSp_\lambda(\xvec) = \sum_\alpha (-1)^{|\alpha|/2} \schurS_{\lambda/\alpha}(\xvec)
\]
where $\alpha$ ranges over the \hyperref[prelimPartitions]{Frobenius coordinates} $(a_1,a_2,\dotsc, | a_1+1,a_2+1,\dotsc)$,
including the empty partition.


The following branching rule is due to Littlewood (\cite[p.295]{Littlewood1977}), see also \cite[p.54]{Krattenthaler1998}.
\begin{theorem}[Littlewood]
We have that
\[
\schurS_{\lambda}(\xvec) = \sum_{\nu} \schurSp_\nu(\xvec) \sum_{\mu, \text{ $\mu'$ even}} c^{\lambda}_{\mu,\nu}
\]
where the sum is over all partitions $\mu$ where the columns are even.
\end{theorem}





\section[schurOrthogonal]{Orthogonal Schur polynomials}




\begin{polydata}{schurOrthogonal}
  Name   & Orthogonal Schur polynomials \\
  Space    & Other \\
  Basis    & False \\
  Rating   & 1 \\
  Bib      & King1971\\
  Year     & 1971? \\
  Keywords & jacobi-trudi,tableaux,cauchy-identity,skew \\
  Symbol   & $\schurOr_\lambda(\xvec)$ \\
  Category & Schur \\
\end{polydata}


A \defin{special orthogonal tableau} of shape $\lambda/\mu$ is a filling of the shape $\lambda/\mu$
with entries in 
\[
1 \lt \overline{1} \lt 2 \lt \overline 2 \lt \dotsb \lt n \lt \overline{n} \lt \infty
\]
such that (a) entries are weakly increasing along rows 
and all finite entries are strictly increasing along columns,
(b) entries in row $i$ are greater than or equal to $i$,
and (c) there are at most one $\infty$ in every row.
Let $SO(\lambda/\mu)$ denote the set of such tableaux.

We then define the \defin{even orthogonal Schur polynomials} (using the definition in \cite{Hamel1997}) as
\[
\schurSp_{\lambda/\mu}(\xvec) = \sum_{T \in SO(\lambda/\mu)} \xvec^{T},
\]
where entries $i$ in $T$ contribute with $x_i$ and entries $\overline{j}$ contributes with $x_j^{-1}$.
The $\infty$-elements do not contribute.

We note that $\omega(\schurSp_\lambda)=\schurOr_{\lambda'}$, where $\omega$ is the standard involution on the symmetric group.


\subsection[schurOrthogonalJacobiTrudy]{Jacobi--Trudi}

We have the following Jacobi--Trudi identities for the 
orthogonal Schur polynomials, see \cite{FulmekKrattenthaler1997,Hamel1997}.
These follow from the symplectic ones, using the identity $\omega(\schurSp_\lambda)=\schurOr_{\lambda'}$.

\begin{align}
\schurOr_\lambda(x^{\pm 1}_{1},x^{\pm 1}_{2},\dotsc, x^{\pm 1}_{n}) 
& 
=\det[ \completeH_{\lambda_i - i+j}(\xvec) - \completeH_{\lambda_i - i- j }(\xvec) ]_{1\leq i,j \leq \length(\lambda)} \\
& =\frac{1}{2}\det[ \elementaryE_{\lambda'_i - i+j}(\xvec) + 
\elementaryE_{\lambda'_i - i- j + 2}(\xvec) ]_{1\leq i,j \leq \lambda_1}.
\end{align}
Note that the functions in the right hand side are also evaluated in 
the alphabet $\xvec = (x^{\pm 1}_{1},x^{\pm 1}_{2},\dotsc, x^{\pm 1}_{n})$.


There are more general identities, analogous to the Hamel--Goulden identities, see \cite{Hamel1997}.



As with the symplectic characters, we have the following expansion.
\begin{theorem}[Littlewood]
We have that
\[
\schurS_{\lambda}(\xvec) = \sum_{\nu} \schurOr_\nu(\xvec) \sum_{\mu, \text{ $\mu$ even}} c^{\lambda}_{\mu,\nu}
\]
where the sum is over all partitions $\mu$ where the rows are even.
\end{theorem}


\todo{Add https://core.ac.uk/download/pdf/157582001.pdf this has a great intro and correct formulas}
% 
% \subsection[schurOrthogonalBialternant]{Bialternant formula}
% 
% We have that the \emph{orthogonal Schur functions} can be expressed as 
% \[
% \schurOr_\lambda(x^{\pm 1}_{1},x^{\pm 1}_{2},\dotsc, x^{\pm 1}_{n})
% =
% \frac{
% \left| x_i^{\lambda_j+n-j+1} - x_i^{-(\lambda_j+n-j+1)}\right|_{1 \leq i,j \leq n}
% }{
% \left| x_i^{n-j+1} - x_i^{-(n-j+1)}\right|_{1 \leq i,j \leq n}
% }
% \]



\subsection[schurOrthogonalCauchy]{Cauchy identity}

\[
\sum_{\lambda} \schurOr_\lambda(x^{\pm 1}_{1},x^{\pm 1}_{2},\dotsc, x^{\pm 1}_{n}) \schurS_\lambda(y_1,\dotsc,y_n)
=
\prod_{1\leq i \leq j \leq n} (1-y_i y_j)
\prod_{1\leq i \lt j \leq n} \frac{1}{(1-y_i x_j)(1-y_i x^{-1}_j)}.
\]

\subsection[schurOrthogonalOtherProperties]{Further properties}

One can express the \defin{orthogonal Schur functions} as a sum of skew Schur functions as
\[
\schurOr_\lambda(\xvec) = \sum_\beta (-1)^{|\beta|/2} \schurS_{\lambda/\beta}(\xvec)
\]
where $\beta$ ranges over the Frobenius coordinates $(b_1+1,b_2+1,\dotsc, | b_1,b_2,\dotsc)$,
including the empty partition.



\section[schurP]{Schur's $P$ functions}


\begin{polydata}{schurP}
  Name   & Schur's $P$ functions \\
  Space    & Sym \\
  Basis    & False \\
  Rating   & 4 \\
  Bib      & Schur1911\\
  Year     & 1911 \\
  Keywords & pfaffian,tableaux,schur-positive \\
  Symbol   & $\schurP_\lambda(\xvec)$ \\
  Category & Schur \\
\end{polydata}

\todo{ add this http://math.sfsu.edu/federico/Articles/staircaseschur.pdf }
\todo{ Add this rule: \url{https://mathoverflow.net/questions/337891/pieri-type-rule-for-schur-p-functions} }

Schur's $P$ functions were introduced by Issai Schur in \cite{Schur1911},
to stody projective resentations of the symmetric and alternating groups.
Schur's $P$ functions $\schurP_\lambda(\xvec)$ are indexed by partitions with \emph{distinct parts} and are given 
as the specialization of the \hyperref[hallLittlewoodP]{Hall--Littlewood $P$ functions} at $t=-1$.


\subsection[schursPPfaffian]{Pfaffian formula}

Let $\lambda$ be a partition with $\ell$ distinct parts and let $n \geq \ell$.
Then
\[
\schurP_\lambda(x_1,\dotsc,x_n) \coloneqq \frac{1}{(n-\ell)!} \sum_{\sigma \in \symS_n} \sigma \left(
\xvec^{\lambda} \prod_{\substack{i \leq \ell \\  i \lt i \leq n } } \frac{x_i+x_j}{x_i-x_j}
\right).
\]

\subsection[schursPTableauformula]{Combinatorial formula}

Given a partition $\lambda$ with distinct parts, consider the \defin{shifted tableau} of shape $\lambda$.
Then consider fillings of this diagram with entries in the alphabet $1' \lt 1 \lt 2' \lt 2 \lt \dotsb $
such that 
\begin{itemize}
\item 
	each row has at most one marked $i$ for every $i=1,2,\dotsc$ 
\item 
	each column has at most one unmarked $i$, for every $i=1,2,\dotsc$,
\item 
	entries in rows and columns are weakly increasing and
\item 
	there are no marked entries on the main diagonal.
\end{itemize}
We let $ShSSYT(\lambda)$ denote the set of such fillings.

\begin{example}
For $\lambda=9531$ we have the shifted diagram with an element in $ShSSYT(\lambda)$.
\begin{figure}
\begin{ytableau}
 &   &  & & & & & & \\
\none & & & & & \\
\none & \none & & & \\
\none & \none & \none & 
\end{ytableau}
\begin{ytableau}
1 & 1  & 2' & 3' & 3 & 3 & 4 & 4 & 4 \\
\none & 2 & 2 & 3 & 4' & 4 \\
\none & \none & 2 & 3 & 4 \\
\none & \none & \none & 4 
\end{ytableau}
\end{figure}
\end{example}

\defin{Schur's $P$ function} can then be defined as
\[
\schurP_\lambda(\xvec) \coloneqq \sum_{T \in ShSSYT(\lambda)} \xvec^T
\]
where $\xvec^T$ is the product over all entries in $T$ and $i$ and $i'$ both contribute with $x_i$.

\todo{
For other Lie types, see http://igm.univ-mlv.fr/~fpsac/FPSAC07/SITE07/PDF-Proceedings/Posters/75.pdf 
}

An alternative combinatorial model uses semi-standard decomposition tableaux, introduced in \cite{Serrano2010}.
See \cite{Cho2012} for additional results.

\subsection[schurPSchurExpansion]{Schur expansion}

Schur's $P$ functions are Schur-positive, see \cite{WorleyThesis,Sagan1987,Assaf2018ShiftedDual}.
B. Sagan gives a proof using an insertion algorithm similar to RSK insertion 
that respects the Knuth relations. The proof by S. Assaf uses the notion of dual 
equivalence which in some sense is very close to RSK.

S. Choi and J-H Kwon gives a proof of Schur positivity using \hyperref[crystals]{crystals}, see \cite{ChoiKwon2018}.
A different proof using crystals is provided in \cite{AssafOguz2018}.


\begin{example*}[Schur expansions]
Here are the Schur expansions of $\schurP_\lambda(\xvec)$ for various $\lambda$.

\begin{array}{lll}
\toprule
\text{Partition} & \text{Schur expansion} \\
\midrule
1	& \schurS_{1} \\
2	& \schurS_{2} + \schurS_{11} \\
21	& \schurS_{21} \\
3	& \schurS_{3} + \schurS_{21} + \schurS_{111} \\
31	& \schurS_{22} + \schurS_{31} + \schurS_{211} \\
321	& \schurS_{321} \\
4	& \schurS_{4} + \schurS_{31} + \schurS_{211} + \schurS_{1111} \\
41	& \schurS_{32} + \schurS_{41} + \schurS_{221} + \schurS_{311} + \schurS_{2111} \\
32	& \schurS_{32} + \schurS_{221} + \schurS_{311} \\
421	& \schurS_{322} + \schurS_{331} + \schurS_{421} + \schurS_{3211} \\
\bottomrule
\end{array}
\end{example*}

\subsection[schurPLRRule]{Littlewood--Richardson rule}

The coefficients $f_{\lambda\mu}^{\nu}$ in
\[
\schurP_\lambda(\xvec) \schurP_\mu(\xvec) = \sum_{\nu} f_{\lambda\mu}^{\nu} \schurP_\nu(\xvec)
\]
are non-negative integers, see \cite{Stembridge1989}. A combinatoral description of these coefficients
was first given by L. Serrano \cite{Serrano2009}.
See also Another proof is given in \cite{Cho2012}.
S. Assaf gave a proof using \hyperref[schurPositivityDualEquivalence]{dual equivalence}, see \cite{Assaf2018ShiftedDual}.

The coefficients $f_{\lambda\mu}^{\nu}$ appear in the study of the type $B$ and type $D$
Schubert calculus of Grassmannians.

A new combinatorial model for $f_{\lambda\mu}^{\nu}$ is given in \cite{Nguyen2022}.
He also give a new model for the coefficients in the Schur expansion of $\schurQ_\lambda$.
An easier formula for for $f_{\lambda\mu}^{\nu}$ using a type of Yamanouchiwords is given in \cite{EstupinanSalamancaPechenik2025x}.


\subsection[schurPEvacuation]{Evacuation}

For evacuation, see \url{https://arxiv.org/pdf/1701.08950.pdf}.


\section[schurQ]{Schur's $Q$ functions}


\begin{polydata}{schurQ}
  Name   & Schur's $Q$ functions \\
  Space    & Sym \\
  Basis    & False \\
  Rating   & 3 \\
  Bib      & Schur1911\\
  Year     & 1911 \\
  Keywords & pfaffian,tableaux,schur-positive \\
  Symbol   & $\schurQ_\lambda(\xvec)$ \\
  Category & Schur \\
\end{polydata}


Schur's $Q$ functions $\schurQ_\lambda(\xvec)$ are indexed by partitions with distinct parts, and are given 
as the specialization of the \hyperref[hallLittlewoodQ]{Hall--Littlewood $Q$ functions} at $t=-1$.
A combinatorial formula using \hyperref[gtpatterns]{Gelfand--Tsetlin patterns} is provided by R. Stanley in
\cite{Stanley1986}.

Alternatively, $\schurQ_\lambda(\xvec) = 2^{\length(\lambda)} \schurP_\lambda(\xvec)$.
Hence, $\schurQ_\lambda(\xvec)$ can be realized as a sum over tableau objects by modifying the 
formula for Schur's $P$ functions by allowing diagonal entries in $ShSSYT(\lambda)$
to be marked.


The Schur $Q$ functions are related to so called \defin{spin characters},
and this was perhaps the original motivation for studying these functions.
The following is due to Schur. 

\begin{theorem}[See \cite[3.3.6]{Morris1962} and \cite[(7.1)]{Stembridge1989}]

Let $\lambda$ be a partition of $n$ with $m$ distinct parts.
We have that the spin characters $\zeta_{\mu}^{\lambda}$
\[
\schurQ_\lambda(\xvec) = \sum_{\mu}
2^{(\length(\lambda)+ \length(\mu)+\epsilon)/2} 
\zeta_{\lambda}(\mu) \frac{ \powerSum_{\mu} }{z_\mu} .
\]
Here, $\mu$ runs over partitions with odd parts,
and $\epsilon$ is $0$ or $1$ depending on if $n-m$ is even or odd.
\end{theorem}

The one-part partitions have a rather explicit expression:
\[
\schurQ_{(n)}(\xvec) = \sum_{\lambda \vdash n} 2^{\length(\lambda)} \monomial_{\lambda}(\xvec).
\]


H. Rosengren gives formulas for $\schurQ_{\lambda}(1,q,q^2,\dotsc,q^n)$
and also shows \cite[Prop.3.1]{Rosengren2008}, that for a partition $\lambda$ of length $m$,
\[
\schurQ_{\lambda}(1,q,q^2,\dotsc) = \prod_{i=1}^m \frac{(-1;q)_{\lambda_i}}{ (q;q)_{\lambda_i} } 
\prod_{1 \leq i \lt \leq j} \frac{q^{\lambda_j} - q^{\lambda_i}}{1-q^{\lambda_i+\lambda_j}}.
\]
Rosengren mentions that this formula is in fact equivalent with \hyperref[schurKawanaka]{Kawanaka's identity} for 
Schur functions.


A \hyperref[murnaghanNakayamaRule]{Murnaghan--Nakayama} type formula for computing
the spin characters $\zeta_{\mu}^{\lambda}$  is given in \cite{MorrisOlsson1988}.

In \cite[Cor. 4.3]{MorrisOlsson1988}, they 
give a formula which seem to be a spin 
version of the \hyperref[borderStripHookFormulas]{Fomin--Lulov}
hook formula for ribbon tableaux. 
This is similar to the properties of the classical irreducible 
character values for the symmetric group.
Morris and Olsson use the abacus model to prove their results.


A Hall--Littlewood analog of $\schurQ_\lambda(\xvec)$ is introduced in \cite{TudoseZabrocki2003}.

In \cite{ChoHuhNam2020}, the authors conjecture a classification
of ribbon shapes, such that the functions $\schurQ_{\lambda/\mu}(\xvec)$ are \hyperref[powerSum]{power-sum} positive.

For more on spin characters, see \cite{MatsumotoSniady2020}.

Plethysm stability for Schur's $Q$-functions is considered in \cite{GrafJing2024x}.



\subsection[schurQCauchy]{Cauchy identity}

In \cite[Corollary 8.3]{Sagan1987}, it is proved that
\begin{equation*}
\prod_{i,j=1}^\infty \frac{1+x_iy_j}{1-x_iy_j} = \sum_{\lambda} 2^{-\length(\lambda)} \schurQ_\lambda(\xvec)\schurQ_\lambda(\yvec),
\end{equation*}
where the sum is over all strict partitions.




\section[schurBottom]{Bottom Schur functions}

\begin{polydata}{schurBottom}
  Name   & Bottom Schur functions \\
  Space  & Sym \\
  Basis  & False \\
  Rating & 2 \\
  Bib    & CliffordStanley2004 \\
  Year   & 2004 \\
  Category & Schur \\
\end{polydata}


The bottom Schur functions were introduced by P. Clifford and R. Stanley in \cite{CliffordStanley2004}.

Let the degree of the power-sum symmetric functions be $1$, so that $deg(\powerSum_i)=1$
and $deg(\powerSum_\nu) = \length(\nu)$. The Murnaghan--Nakayama rule states that
\[
\schurS_\lambda = \sum_{\mu} \frac{\chi^\lambda(\mu)}{z_\mu} \powerSum_\mu
\]
The \defin{bottom Schur function} $\hat{\schurS}_\lambda$ is defined as
\[
\schurS_\lambda = \sum_{\mu : \length(\mu)=\text{lowest}} \frac{\chi^\lambda(\mu)}{z_\mu} \powerSum_\mu,
\]
where we only sum over partitions contributing to the lowest-degree part in the power-sum expansion.

The bottom Schur functions can be expressed as the lowest-degree term in a Jacobi--Trudi type 
determinant of power-sum symmetric functions.


\begin{conjecture}[Alexandersson, 2019]

Let $R=(R_{\lambda\mu})$ with $\lambda,\mu \vdash n$ be the matrix with coefficients 
defined via $\powerSum_{\lambda} = \sum_\mu R_{\lambda\mu} \monomial_\mu$.


In \cite{CliffordStanley2004}, the authors consider $R-D$, where $D$ is the diagonal of $R$,
and compute the dimension of the null-space of $R-D$ as $n$ increases. They computed the beginning of this sequence,
and got
\[
1,1,1,2,2,3,4,5,7,9,11,15,19,24,\dotsc
\]
They did not guess any combinatorial interpretation of these coefficients.

In 2019, I did a search in the OEIS and the sequence $a(n)$ given by \oeis{A129528} seems to be a match.
The sequence $a(n)$ is defined as follows:
 Consider the set of all Dyck paths, consisting of steps $(1,1)$ and $(1,−1)$. 
 The \defin{peak-abscissa-sum} of a path is the sum of the $x$-coordinates of all peaks. 
 Then $a(n)$ is the number of Dyck paths of \emph{any length} with peak-abscissa-sum $n$.

\end{conjecture}

\todo{
Bottom Schur functions: (not monomial positive, but has 'sort of' JT). \cite{CliffordStanley2004}
}


\section[schurLectureHall]{Lecture hall Schur functions}


\begin{polydata}{schurLectureHall}
  Name   & Lecture hall Schur functions \\
  Space  & All \\
  Basis  & False \\
  Rating & 1 \\
  Bib    & CorteelKim2018 \\
  Year   & 2018 \\
  Category & Schur \\
\end{polydata}

\todo{Write more about lecture hall Schur funcs}


Introduced in \cite{CorteelKim2018}. The authors prove analogues of the Jacobi--Trudi identity and its dual.
The lecture hall Schur functions are related to multivariate little $q$-Jacobi polynomials.

See also \url{http://de.arxiv.org/pdf/1904.10602.pdf}

For asymptotics of lecture hall tableaux, se \url{http://de.arxiv.org/pdf/1905.02881.pdf}.



\section[schurWreath]{Wreath product Schur polynomials}


\begin{polydata}{schurWreath}
  Name   & Wreath product Schur polynomials \\
  Space  & Other \\
  Basis  & True \\
  Rating & 1 \\
  Bib    & IngramJingStitzinger2009 \\
  Year   & 2009 \\
  Symbol & $\schurS_{\underline{\lambda}}$ \\
  Category & Schur \\
\end{polydata}

Wreath product Schur polynomials show up in the study of $G \wr \symS_n$,
see for example in \cite{Macdonald1980} and \cite{MendesRemmelWagner2004}.

A good intruduction for the combinatorics is given in \cite{IngramJingStitzinger2009}. 
These are indexed by $r$-tuples of partitions, $\underline{\lambda}$, and live in the space $\spaceSym^{\otimes r}$.
We then define the \defin{wreath product Schur polynomials} as
\[
\schurS_{\underline{\lambda}}(\xvec^{(1)},\dotsc,\xvec^{(r)}) = \prod_{i} \schurS_{\lambda^i}(\xvec^{(i)}).
\]
This can be realized as a sum over $r$-tuples of semi-standard Young tableaux of shape $\lambda^i$, where tableau $i$ only uses 
entries from the alphabet $\xvec^{(i)}$.


A Murnaghan--Nakayama rule is described in \cite{MendesRemmelWagner2004}.

\todo{
See also 
http://www.math.wm.edu/~vinroot/KostkaSeq.pdf
}
