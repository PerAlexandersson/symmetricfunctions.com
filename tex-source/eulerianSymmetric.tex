\metatitle{Eulerian quasisymmetric functions}
\metadescription{An introduction to Eulerian quasisymmetric functions, including their definition, plethystic formulas, Schur positivity, power-sum positivity, and connections to cyclic sieving phenomena.}
\metakeywords{Eulerian quasisymmetric functions,schur-positive,plethysm,cyclic-sieving}

\section[eulerianSymmetric]{Eulerian quasisymmetric functions}


\begin{polydata}{eulerianSymmetric}
  Name   & Eulerian quasisymmetric functions \\
  Space    & Sym \\
  Basis    & False \\
  Rating   & 3 \\
  Bib      & ShareshianWachs2010 \\
  Year     & 2010 \\
  Symbol   & $\eulerianQ_{\lambda,j}(\xvec)$ \\
  Keywords & schur-positive, plethysm, cyclic-sieving \\
  Category & Other \\
\end{polydata}



The Eulerain quasisymmetric functions are actually symmetric.
They were introduced in \cite{ShareshianWachs2010}.

\subsection[eulerianSymmetricDefinition]{Definition}

Let $\bar{1} \lt \bar{2} \lt \dotsb \lt \bar{n} \lt 1 \lt 2 \lt \dotsb \lt n$.
For $\sigma \in \symS_n$, let $\overline{\sigma}$ be define as the word 
obtained from $\sigma$ by replacing $\sigma_i$ with $\overline{\sigma}_i$ whenever $\sigma_i \gt i$.
Let
\[
\DEX(\sigma) = \{ i \in [n-1] : \overline{\sigma}(i+1)>\overline{\sigma}(i) \}
\]
that is, the set of descents in the modified alphabet.

The \defin{Eulerian quasisymmetric function} indexed by $n$ and $j$ is then 
described using the \hyperref[gessel]{Gessel quasisymmetric functions}:
\[
\eulerianQ_{\lambda,j}(\xvec) \coloneqq \sum_{\substack{\sigma \in \symS_n \\ \type(\sigma) = \lambda}} \gessel_{n,\DEX(\sigma)}(\xvec).
\]

\begin{example}
We have that 
\begin{align*}
\eulerianQ_{33,2}(\xvec) &= 
\gessel_{\emptyset,6} + 
\gessel_{\{2\},6} + 
\gessel_{\{3\},6} + 
\gessel_{\{4\},6} + 
\gessel_{\{1,3\},6}  \\
&+\gessel_{\{1,4\},6} + 
\gessel_{\{1,5\},6} + 
\gessel_{\{2,4\},6} + 
\gessel_{\{2,5\},6} + 
\gessel_{\{3,5\},6}.
\end{align*}
\end{example}

\subsection[eulerianSymmetricPlethystic]{Plethystic formula}


\begin{theorem}[See \cite{ShareshianWachs2010}]
Let $\lambda$ have $m_i$ parts of size $i$. 
Then
\[
\sum_{j=0}^{|\lambda|-1} \eulerianQ_{\lambda,j}(\xvec)t^j = 
\prod_{i \geq 1} \completeH_{m_i}\left[ \sum_{j=0}^{i-1} \eulerianQ_{(i),j} t^j \right].
\]
\end{theorem}


\subsection[eulerianSymmetricSchurPositivity]{Schur positivity}

In \cite{HendersonWachs2012} it is proved that $\eulerianQ_{\lambda,j} = \frobChar V_{\lambda,j}$,
that is, it is the \hyperref[frobeniusCharacteristic]{Frobenous characteristic} of a 
certain vector space $V_{\lambda,j}$ spanned by forests of marked trees.
This implies Schur positivity.

\subsection[eulerianSymmetricPowerSumPositivity]{Power-sum positivity}

In \cite{SaganShareshianWachs2011}, it is proved that $\eulerianQ_{(n),j}$ expand positively in the power-sum basis.

\begin{conjecture}[Alexandersson (2018)]
 It seems like all $\eulerianQ_{\lambda,j}$ are $\powerSum_\mu$-positive.
\end{conjecture}

The stronger statement, being positive in the complete homogenous basis, is not true.
For example, 
\[
\eulerianQ_{(6),3} = \completeH_{321} - \completeH_{411}+2\completeH_{42} + \completeH_{51}.
\]

\subsection[eulerianSymmetricCSP]{Cyclic sieving phenomena}

Let $\symS_{\lambda,j}$ be the subset of permutations in $\symS_n$ with cycle type $\lambda$
and exactly $j$ excedances. In \cite{SaganShareshianWachs2011}, it is shown that
\[
\left(\symS_{\lambda,j}, C_n, \sum_{\sigma \in \symS_{\lambda,j}} q^{\maj(\sigma) - \exc(\sigma)} \right)
\]
exhibit the cyclic sieving phenomena, where $C_n$ act by conjugation.


\begin{conjecture}[Alexandersson (2018)]
Let $\lambda$ be a partition of $n$. Then 
\[
\eulerianQ_{\lambda,j}(1,q,q^2,\dotsc,q^{n-1})
\]
evaluates to non-negative integers at $q = e^{2 \pi i k/n}$, and there should be some action $C_n$
that completes this to a cyclic sieving phenomena.
\end{conjecture}


\section[eulerianSymmetricColored]{Colored Eulerian quasisymmetric functions}


\begin{polydata}{eulerianSymmetricColored}
  Name   & Colored Eulerian quasisymmetric functions \\
  Space    & Sym \\
  Basis    & False \\
  Rating   & 1 \\
  Bib      & Hyatt2012 \\
  Year     & 2012 \\
  Keywords & cyclic-sieving \\
  Category & Other \\
\end{polydata}



In \cite{Hyatt2012}, the notion of Eulerian quasisymmetric functions is generalized to the wreath product of 
a cyclic group and the symmetric group, also known as the group of \emph{colored permutations}.
The notion of descents, fixed points and so on generalizes naturally to this setting,
and thus allow for a definition of \emph{colored Eulerian quasisymmetric functions}.

A special case would be to choose the cyclic group to be $\setZ_2$, which can then be 
called a \defin{type $B$ Eulerian quasisymmetric function}.

In \href{https://arxiv.org/abs/1309.6368}{this preprint they are shown to be symmetric}.
These are believed to be symmetric and satisfy a cyclic sieving phenomena as above.


\todo{Wreath product Schur polys}
% There is a notion of Wreath product Schur polys. Are these WS-positive? 
% http://www.m-hikari.com/ija/ija-password-2009/ija-password1-4-2009/jingIJA1-4-2009.pdf 

