
\metatitle{Polytopes}
\metadescription{Introduction to polytopes}



\section[polytope]{Polytopes}


A polytope is called \defin{integral} if all its vertices have integer coordinates.



\section[polytope-families]{Families of polytopes}


\subsection[orderPolytope]{Order polytopes}

Given a partial order $P$ on the set $[n]$, we associate a polytope $\mathcal{O}(P)$, 
called the \defin{order polytope} of $P$.
It is the polytope in $\setR^n$ defined via the inequalities
\begin{align}
x_i \leq x_j \text{ if } i <_P j \qquad \text{and} \qquad 0 \leq x_i \leq 1 \text{ for } 1\leq i \leq n.
\end{align}

There are order polytopes with negative Ehrhart coefficients, see \cite[Figure 3.87]{StanleyEC1} and \cite{Alexandersson2019CounterExamples}.


\subsection[markedOrderPolytope]{Order polytopes}

There is a slight generalization of order polytopes called \defin{marked order polytopes}, see \cite{ArdilaBliemSalazar2011},
where we replace the conditions $0 \leq x_i \leq 1$ with $a_i \leq x_i \leq b_i$ for $1\leq i \leq n$,
where the $a_i$ and $b_i$ are constants.

The \hyperref[gtpolytopes]{Geltand--Tsetlin polytopes} are marked order polytopes.


\todo{Cross polytope: https://arxiv.org/pdf/2512.08669 and Ehrhart coeffs.}

% \subsection[flowPolytope]{Flow polytopes}
\todo{Define flow polytopes}

%\subsection[permutohedron]{The permutohedron}
\todo{Define permutohedron}

\todo{Parking function polytope: https://arxiv.org/pdf/2512.14199}

\section[ehrhart]{The Ehrhart polynomial and $h^*$-vector}

Let $P$ be a lattice polytope. The number of lattice points in the dilation $n\cdot P$ is a polynomial in $n$.
The \defin{Ehrhart polynomial of $P$} is the polynomial $E_P(t) \coloneqq |n P|$.
The \defin{Ehrhart series} is 
\[
	\sum_{n \geq 0} |n P| t^n = \frac{H(t)}{(1-t)^{d+1}}
\]
where $H_P(t) = h^*_0 + h^*_1 t + \dotsb + h^*_d t^d$ is the \defin{$h^*$-polynomial} of $P$.
The vector $(h^*_0 , h^*_1, \dotsc,  h^*_d )$ is the \defin{$h^*$-vector} and one can show that all entries of this are non-negative integers.

For a $d$-dimensional polytope, we have the following relation between entries in the $h^*$-vector and the Ehrhart polynomial:
\[
 E_P(t) = \sum_{j=0}^d \binom{t+d-j}{d} h^*_j.
\]
There is a similar relation to express $H_P(t)$ in terms of the Ehrhart polynomial coefficients involving the Eulerian polynomials.


\section[polytopeIDP]{The integer decomposition property}

An integral polytope $\mathcal{P} \subset \setR^d$ is said to have the \defin{integer decomposition property} (IDP) 
if for every positive integer $k$ and $\xvec \in k \mathcal{P} \cap \setZ^d$,
we can find $\xvec_1,\xvec_2,\dots,\xvec_k \in \mathcal{P} \cap \setZ^d$ such that $\xvec_1 + \dots + \xvec_k = \xvec$.

In the following list, each entry implies the next:
\begin{itemize}
	\item $\mathcal{P}$ has a unimodular triangulation.
	\item $\mathcal{P}$ has the integer decomposition property.
	\item $\mathcal{P}$ is integral.
\end{itemize}



\section[polytope-papers]{Further reading}

Here we mainly discuss polytopes that arises in the area of algebraic combinatorics.

\begin{itemize}


\item \url{https://arxiv.org/pdf/2511.10373} Ehrhart theory over Abelian groups.

\item \url{https://arxiv.org/pdf/2512.20575} Flow Polytopes

\item \url{https://arxiv.org/pdf/2406.15803} Root polytopes, flow polytopes

\item \url{https://arxiv.org/pdf/2404.02136.pdf} Symmetric Edge polytopes, and Ehrhart roots interlacing on the canonical line:

\item \url{https://arxiv.org/pdf/2504.07505} Birkhoff polytope, and some unimodular equivalence with another.

\end{itemize}
