
\section[whittaker]{Whittaker functions}


\begin{polydata}{whittaker}
  Name     &  Whittaker functions \\
  Space    &  Sym   \\
  Basis    &  True   \\
  Rating   &  1      \\
  Bib      &  Jacquet1967\\
  Year     &  1967 \\
  Keywords & eigenfunction \\
  Symbol   & $\whittaker_{\lambda}(\xvec)$ \\
  Category & Schur \\
\end{polydata}

The Whittaker functions where introduced by H. Jacquet in 1967 \cite{Jacquet1967}.

T. Lam has a nice intro in \url{https://arxiv.org/abs/1308.5451v2}

Iwahori Whittaker functions: \url{https://arxiv.org/pdf/1906.04140.pdf}

Spin q-Whittaker: \url{https://arxiv.org/pdf/2003.14260.pdf}

(p,q)-Whittaker: \url{https://arxiv.org/pdf/1710.07196.pdf}

Metaplectic Whittaker and connection with LLT polynomials 
\url{https://arxiv.org/pdf/1806.07776.pdf}

Whittaker from non-symmetric Macdonalds: \url{dx.doi.org/10.1016/j.jnt.2014.01.001}




\section[qWhittaker]{$q$-Whittaker functions}


\begin{polydata}{qWhittaker}
  Name     & q-Whittaker functions \\
  Space    & All \\
  Basis    & True \\
  Rating   & 2 \\
  Bib      & Opdam1995\\
  Year     & 1995\\
  Symbol   & $\macdonaldE_\mu(\xvec;q,0)$ \\
  Keywords & fillings, key-positive \\
  Category & Schur \\
\end{polydata}

The $q$-Whittaker functions $\qWhittaker_\lambda(\xvec;q)$ are eigenfunctions of the quantum Toda lattice.

For a recent survey on this topic, see \cite{Bergeron2024}.
The \defin{$q$-Whittaker function $\qWhittaker_\lambda(\xvec;q)$}
can be defined as any of the quantities:
\begin{itemize}
\item 
the \hyperref[macdonaldH]{modified Macdonald polynomial} $[t^{\partitionN(\lambda)}] \macdonaldH_\lambda(\xvec;q,t)$,
\item 
the \hyperref[macdonaldP]{Macdonald $P$ polynomial} $\macdonaldP_\lambda(\xvec;q,0)$
\item 
the \hyperref[hallLittlewoodT]{transformed Hall--Littlewood polynomial} $\omega \hallLittlewoodT_{\lambda'}(\xvec;q)$,
\item 
the \hyperref[macdonaldE]{non-symmetric Macdonald polynomial} $\macdonaldE_\lambda(\xvec;q,0)$.
\end{itemize}

We have that 
\[
  \qWhittaker_\mu(\xvec;q) = \sum_{\lambda} K_{\lambda'\mu'}(q) \schurS_\lambda,
\]
where the coefficients are given by the \hyperref[kostkaFoulkes]{Kostka--Foulkes polynomials}.
Proofs of this can be found in \cite{Assaf2018Kostka,AssafGonzalez2018},
where RSK ans a crystal structure is given.


See \cite{Uhlin2019,AlexanderssonUhlin2020} and \hyperref[cspMacdonaldE]{the cyclic sieving page}
for a cyclic sieving phenomena on non-attacking fillings associated with
$q$-Whittaker polynomials.


\subsection[skewqWhittaker]{Skew $q$-Whittaker functions}

In \cite{AlexanderssonUhlin2020}, we introduce a skew version, $\qWhittaker_{\lambda/\mu}(\xvec;q)$
which is symmetric and Schur positive for partitions $\mu \subseteq \lambda$.



\subsection[geometricWhittaker]{Relation with geometric RSK and crystals}

The Whittaker functions show up when 
considering a geometric lift of \hyperref[rsk]{RSK}.
There is also a notion of geometric crystals.
\todo{ \url{http://www.math.lsa.umich.edu/~tfylam/CDM2014talk1.pdf} }

Geometric RSK \url{https://maths.ucd.ie/~noconnell/pubs/cosz.pdf}
Reda's thesis: \url{https://arxiv.org/abs/1302.0902}
