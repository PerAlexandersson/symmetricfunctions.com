\metatitle{RSK, the Robinson--Schensted--Knuth correspondence}

\metadescription{Background and properties of the Robinson--Schensted--Knuth correspondence.}



\section[rsk]{The Robinson--Schensted--Knuth correspondence}

The \defin{Robinson--Schensted--Knuth correspondence} (RSK),
is a bijection between matrices with non-negative integer entries and pairs of
semi-standard Young tableaux of the same shape.
Let $M(\mu,\nu)$ denote the set of matrices with entries in $\setN$,
row sums are determined by $\mu$ and column sums determined by $\nu$.
The RSK correspondence provides the following bijections,
where $\lambda$, $\mu$ are integer partitions of $n$:

\begin{align}
M(\mu,\nu) & \qquad \rskArrow \qquad \bigcup_{\lambda \vdash n} \SSYT(\lambda,\nu) \times \SSYT(\lambda,\mu) \\
[d]^n & \qquad \rskArrow \qquad \bigcup_{\lambda \vdash n} \SSYT(\lambda,d) \times \SYT(\lambda)  \\
\symS_n  & \qquad\rskArrow \qquad \bigcup_{\lambda \vdash n} \SYT(\lambda) \times \SYT(\lambda) \\
\{\sigma \in \symS_n : \sigma^2 = e \}  & \qquad \rskArrow \qquad \bigcup_{\lambda \vdash n} \SYT(\lambda).
\end{align}

In particular, the first bijection proves \hyperref[schurCauchyFormula]{the first Cauchy identity}
for Schur functions. The second variant of RSK proves the second Cauchy identity.


For an excellent overview on the different variants of RSK, see C. Krattenthaler's survey \cite{Krattenthaler2006}.
We shall cover the same four variants of RSK as in his paper and keep the same numbering.
See also \cite[App. A.4.3]{Fulton1997} and \cite{Sagan2001} for more resources on RSK.

See the \hyperref[evacuation]{evacuation} section on how RSK interacts with evacuation.

\todo{Add the symmetries in Fulton1997! }



\subsection[rowInsertion]{Row insertion}

In order to describe RSK, we need the notion of row insertion.
Given a SSYT $T$ and $k \in \setN$, the \emph{insertion} $T \leftarrow k$ recursively as follows,
where $T_1$ is the first row of $T$ and $T_{2\sim}$ denotes the SSYT with the first row of $T$
removed.
\begin{enumerate}
\item If $T = \emptyset$, then $T \leftarrow k$ is just the tableau \ytableaushort{k}.
\item If $k$ is not smaller than the largest entry in $T_1$, 
then $T  \leftarrow k$ is given by appending $k$ at the end of the first row of $T$.
\item Otherwise, find the leftmost entry in $T_1$ larger than $k$, and let this be $k'$.
The insertion $T \leftarrow k$ is then given as the tableau with first row $T_1$,
except that $k'$ has been replaced by $k$, and the remaining rows are given by
the insertion $T_{2\sim}  \leftarrow k'$.
\end{enumerate}
Given a word $w$, inserting the letters one by one gives the \defin{insertion tableau}, $\ins(w)$.
This is sometimes denoted $P(w)$.

\begin{example}
Inserting the entries $w = 34112312$ from left to right 
gives the following sequence of tableaux, where $\ins(w)$ is the last.

\begin{figure}
\ytableaushort{3}
\ytableaushort{34}
\ytableaushort{14,3}
\ytableaushort{11,34}
\ytableaushort{112,34}
\ytableaushort{1123,34}
\ytableaushort{1113,24,3}
\ytableaushort{1112,23,34}
\end{figure}

\end{example}

\begin{theorem}[Greene's theorem, \cite{Greene1974,Sagan2001}]
Let $\pi$ be a permutation and let $I_k(\pi)$
denote the length of the longest sub-word of $\pi$ that can be 
expressed as a disjoint union of $k$ increasing subsequences of $\pi$.
Similarly, let $D_k(\pi)$ be the length of the 
longest subword of $\pi$ that can be expressed as a disjoint union of $k$ 
decreasing subsequences of $\pi$.

Let $\lambda$ be the shape of $P(\pi)$. Then 
\[
I_k(\pi) = \lambda_1+\dotsb + \lambda_k \qquad
D_k(\pi) = \lambda'_1+\dotsb + \lambda'_k.
\]
\end{theorem}


\begin{example*}[An application:  Erdős--Szekeres theorem]

The \defin{Erdős--Szekeres theorem}\cite{ErdosSzckeres2009} 
states that any sequence of distinct real numbers of length $(r-1)(s-1)+1$
contains a monotonically increasing subsequence of length $r$, 
or a monotonically decreasing subsequence of length $s$.

Suppose we have a sequence $w$ of such numbers. We apply row insertion on this sequence,
and obtain a tableau $P$ (with real numbers).
Note that this tableau cannot be contained in an $(r-1)\times (s-1)$-rectangle,
so either the first row is longer than $r$, or the first column is longer than $s$.

Application of Greene's theorem finishes the proof.
\end{example*}



\subsection[rskOne]{RSK (variant I)}

The RSK correspondence is done in two steps.
A matrix $A \in M(\lambda,\mu)$ is first turned into a \defin{biword} $W$ of length $n$ as follows. 
For each entry $A_{ij}$, we have $A_{ij}$ columns in $W$ equal to $\binom{i}{j}$.
Furthermore, the columns are ordered lexicographically, with respect to the top-most value first.

Given a pair $(P,Q)$ of SSYT of the same shape,
the insertion $(P,Q) \leftarrow \binom{i}{j} $ is defines as the pair $(P',Q')$,
where $P' =  P \leftarrow j$ and $Q'$ is obtained from $Q$ by putting $i$
in the box where the last insertion took place to get $P'$.

Finally, the pair $(P,Q)$ in the correspondence $W \rskArrow (P,Q)$
is obtained by inserting each column $\binom{i}{j}$ from left to right in $W$.
One can prove that $(P,Q)$ is indeed a pair of semi-standard Young tableaux,
provided that $W$ correspond to some matrix $A$.
We refer to $P = \ins(W)$ as the \defin{insertion tableau} 
and $Q=\rec(W)$ as the \defin{recording tableau}.

\begin{example}
The word
\begin{equation*}
W=
\begin{pmatrix}
1& 1& 1& 2& 2& 3& 3& 4 \\
1& 3& 4& 1& 2& 1& 3& 2
\end{pmatrix}
\end{equation*}
% BiwordRSK[{1, 1, 1, 2, 2, 3, 3, 4}, {1, 3, 4, 1, 2, 1, 3, 2}] 
is mapped to the pair $(P,Q)$

\begin{figure}
\ytableaushort{1112,23,34}\ytableaushort{1113,22,34}
\end{figure}

\end{example}


\begin{theorem}[Greene's theorem for words, \cite{Greene1974,Sagan2001}]
Let $W$ be a biword with lexicographically sorted columns.
We use $I_k(W)$ and $D_k(W)$ to denote $k$-weakly increasing
and $k$-strictly decreasing subwords of the bottom row of $W$.

Let $\lambda$ be the shape of $\ins(W)$. Then 
\[
I_k(W) = \lambda_1+\dotsb + \lambda_k \qquad
D_k(W) = \lambda'_1+\dotsb + \lambda'_k.
\]
\end{theorem}


\subsection[rskDual]{Dual RSK (variant II)}


There is a dual notion of row insertion, called \defin{column insertion}, 
traditionally indicated with a right-pointing arrow.
We now insert into dual SSYT. 
A \defin{dual SSYT} is a filling that is the transpose of an SSYT.

Given a \emph{dual SSYT} $T$ and $k \in \setN$, the \emph{dual insertion} $k \to T $ 
is defined recursively as follows where $T_1$ is the first 
row of $T$ and $T_{2\sim}$ denotes the dual SSYT with the first row of $T$
removed.
\begin{enumerate}
\item If $T = \emptyset$, then $k \to T$ is just the tableau \ytableaushort{k}.
\item If $k$ is \emph{greater} than the largest entry in $T_1$, 
then $k \to T$ is given by appending $k$ at the end of the first row of $T$.
\item Otherwise, find the leftmost entry in $T_1$ \emph{larger than or equal to} $k$, 
and let this be $k'$.
The insertion $k \to T$ is then given as the tableau with first row $T_1$,
except that $k'$ has been replaced by $k$, and the remaining rows are given by
the insertion $k' \to T_{2\sim}$.
\end{enumerate}
This series of operations correspond to inserting boxes in the \emph{columns} of $T^t$,
hence the name.

\todo{Should we do it right to left instead?}

\begin{example}
Dual-inserting the entries $w=34112312$ from left to right 
gives the following sequence of tableaux, where $P$ is the last.

\begin{figure}
\ytableaushort{3}
\ytableaushort{34}
\ytableaushort{14,3}
\ytableaushort{14,1,3}
\ytableaushort{12,14,3}
\ytableaushort{123,14,4}
\ytableaushort{123,14,1,3}
\ytableaushort{123,12,14,3}
\end{figure}

\end{example}

Dual RSK is defined on biwords such that there are no duplicate columns.
We denote the dual RSK map as $W \rskDualArrow (P,Q)$,
and we obtain a pair of tableaux such that $P$ is dual semistandard and $Q$ is semistandard.
Let $B(\mu,\nu)$ denote the set of \emph{binary} matrices with given row- and column-sums.
Dual RSK provides the following bijections:

\begin{align}
B(\mu,\nu) & \qquad \rskDualArrow \qquad \bigcup_{\lambda \vdash n} \SSYT(\lambda',\nu) \times \SSYT(\lambda,\mu) \\
[d]^n & \qquad \rskDualArrow \qquad \bigcup_{\lambda \vdash n} \SSYT(\lambda',d) \times \SYT(\lambda) \\ 
\symS_n  & \qquad\rskDualArrow \qquad \bigcup_{\lambda \vdash n} \SYT(\lambda') \times \SYT(\lambda) \\
\{\sigma \in \symS_n : \sigma^2 = e \}  & \qquad \rskDualArrow \qquad \bigcup_{\lambda \vdash n} \SYT(\lambda).
\end{align}

\begin{example*}[Dual RSK biword insertion]
The matrix and corresponding word
\begin{equation*}
A = \begin{pmatrix}
1 & 0 & 1 & 1 \\
1 & 1 & 0 & 0 \\
1 & 0 & 1 & 0 \\
0 & 1 & 0 & 0
\end{pmatrix}
\qquad
W=
\begin{pmatrix}
1& 1& 1& 2& 2& 3& 3& 4 \\
1& 3& 4& 1& 2& 1& 3& 2
\end{pmatrix}
\end{equation*}
% BiwordRSKDual[{1, 1, 1, 2, 2, 3, 3, 4}, {1, 3, 4, 1, 2, 1, 3, 2}] 
is mapped to the pair $(P,Q)$

\begin{figure}
\ytableaushort{123,124,13}\ytableaushort{111,223,34}
\end{figure}
\end{example*}


\begin{proposition}[RSK vs. dual RSK]
Let $\pi \in \symS_n$.
Then \cite[Thm. 4.1.1]{Leeuwen1996} states that the following are equivalent:
\begin{itemize}
\item $\pi \rskArrow (P,Q)$
\item $n+1-\pi \rskDualArrow (\evac(P^t),Q^t)$
\item $\rev(\pi) \rskDualArrow (P^t,\evac(Q^t))$
\item $\revCompl(\pi) \rskArrow (\evac(P^t),\evac(Q^t))$
\item $\revCompl(\pi) \rskDualArrow (\evac(P),\evac(Q))$
\end{itemize}

The following extension is proved in \cite[Prop. 2.3.14]{Butler1994}.
Let $w \in [d]^n$ be a word. We have that
\[
w \rskArrow P \iff \rev(w) \rskDualArrow P^t.
\]
\end{proposition}
This property generalizes to biwords, see \hyperref[RSKvsDualRSK]{this section}.



\subsection[rskBurge]{RSK (variant III)}


\todo{See also 
\href{https://www.combinatorics.org/ojs/index.php/eljc/article/view/v12i1r10/pdf}{Spin-preserving Knuth correspondences for ribbon tableaux}
\href{https://core.ac.uk/download/pdf/82527725.pdf}{RSK vs Burge correspondence}
}

In the usual RSK we require that the biword $W$ is sorted in a 
lexicographical fashion. Suppose instead the columns $\binom{i}{j}$ are sorted first 
on $i$ and in case of equality, sort \emph{decreasingly} with respect to $j$.
Furthermore, we impose the same condition as in dual RSK that two columns may not be identical.
We call such biwords \defin{Burge words} as this ordering was studied in Burge insertion.

Each binary matrix $B$ give rise to a Burge word by recording the row and column coordinates of the ones.
That is, if  $B_{ij}=1$ then $\binom{i}{j}$ appear as a column in the corresponding biword.
For example,
\[
\begin{pmatrix}
1 & 0 & 0 & 1 \\
1 & 1 & 1 & 0 \\
0 & 0 & 1 & 0
\end{pmatrix}
\qquad 
\longleftrightarrow
\qquad 
\begin{pmatrix}
1 & 1 & 2 & 2 & 2 & 3 \\
4 & 1 & 3 & 2 & 1 & 3
\end{pmatrix}
\]

In this setting, we write $W \rskbArrow (P,Q)$ if the biword $W$ is first sorted in the Burge 
fashion and the entries are inserted using row insertion.

For this map, we have that $W \rskbArrow (P,Q)$ where $P$ and $Q^T$ 
are semistandard tableaux of the same shape
and we get a second set of bijections. 
We let $B(\mu,\nu)$ denote the 
set of binary matrices with row sums $\mu$ and column sums $\nu$.

\begin{align}
B(\mu,\nu) & \qquad \rskbArrow \qquad \bigcup_{\lambda \vdash n} \SSYT(\lambda,\nu) \times \SSYT(\lambda',\mu) \\
[n]^n & \qquad \rskbArrow \qquad \bigcup_{\lambda \vdash n} \SYT(\lambda) \times \SSYT(\lambda') \\
\symS_n  & \qquad \rskbArrow \qquad \bigcup_{\lambda \vdash n} \SYT(\lambda) \times \SYT(\lambda') 
\end{align}


\begin{example*}[Burge biword insertion]
The matrix and corresponding word
\begin{equation*}
A = \begin{pmatrix}
1 & 0 & 1 & 1 \\
1 & 0 & 1 & 0 \\
1 & 1 & 0 & 0 \\
0 & 1 & 0 & 0
\end{pmatrix}
\qquad
W=
\begin{pmatrix}
1& 1& 1& 2& 2& 3& 3& 4 \\
4& 3& 1& 3& 1& 2& 1& 2
\end{pmatrix}
\end{equation*}
is mapped to the pair $(P,Q)$ below.

\begin{figure}
\ytableaushort{1112,23,3,4}\ytableaushort{1234,12,1,3}
\end{figure}
\end{example*}


\begin{proposition}[See \cite{Krattenthaler2006} and \cite[p. 200]{Fulton1997}]
Let $B$ be a binary matrix.
Then 
\[
B \rskbArrow (P,Q) \iff B^T \rskDualArrow (Q,P).
\]
Using biwords, this can be phrased as
\[
\binom{w_1}{w_2} \rskbArrow (P,Q) \iff \binom{w_2}{w_1} \rskDualArrow (Q,P).
\]
\end{proposition}



\subsection[knuthRelations]{Knuth relations}

Let $A$ be an alphabet. The \defin{plactic monoid} is the quotient $A^*/\equiv$,
where $\equiv$ is the equivalence class generated by the \defin{Knuth relations}

\begin{align*}
 xzy \equiv zxy &\quad (x\leq y \lt z) \\
 yxz \equiv yzx &\quad (x\lt y \leq z).
\end{align*}

Two words $w$ and $w'$ are \defin{Knuth-equivalent}
if and only if they have the same insertion tableau, $T$.
Moreover, the reading word of $T$ is also Knuth-equivalent to $w$,
and it is the unique word in the equivalence class which is the 
reading word of a semistandard tableau,
see \cite[p.22]{Fulton1997} or \cite[Cor. 2.3.21]{Butler1994}.


\subsection[knuthRelationsDual]{Dual Knuth relations}


Two words $w$ and $w'$ are \defin{dual Knuth-equivalent}
if and only if they have the same \defin{recording tableau}.

An \defin{elementary dual Knuth transformation} on a permutation
on a permutation $\sigma_1,\dotsc,\sigma_n$
interchanges $\sigma_i =k$ and $\sigma_j =k+1$
if and only if $k-1$ or $k+2$ occur somewhere between.

For example, $\underline{5}34\underline{6}7182$ and $\underline{6}34\underline{5}7182$
are equivalent, since $4$ appear between them.

Two words are dual Knuth equivalent if and only if one can be transformed to the other 
via elementary dual Knuth transformations, see \cite[p.200]{Fulton1997}.



\section[rsk-properties]{Properties of RSK}

Here is an overview of the properties of RSK.
Note that RSK is closely connected to \hyperref[chargeProperties]{charge}.

\begin{itemize}

\item For $A \in M(\mu,\nu)$,
\[
A \quad \rskArrow \quad (P,Q) \qquad  \iff  \qquad A^{T} \quad \rskArrow \quad (Q,P).
\]

\item 
For a permutation $\pi \in \symS_n$, the following statements are equivalent, see 
\cite[A.1.2.11]{StanleyEC2}.

\begin{align}
\pi \quad &\rskArrow \quad (P,Q) \\
\pi^{-1} \quad &\rskArrow \quad (Q,P) \\
\rev(\pi) \quad &\rskArrow \quad (P^t, \evac(Q)^t)
\end{align}

For dual RSK, we have the same properties:
\begin{align}
\pi \quad &\rskDualArrow \quad (P,Q) \\
\pi^{-1} \quad &\rskDualArrow \quad (Q,P) \\
\rev(\pi) \quad &\rskDualArrow \quad (P^t, \evac(Q)^t)
\end{align}


\item For words, $w \equiv w'$ if and only if $\ins(w)=\ins(w')$.

\item If $w \equiv w'$, then $\charge(w)=\charge(w')$.

\item We have that
\[
\sum_{\sigma \in \symS_n} t^{\des(\sigma)} = \sum_{\lambda}
f^{\lambda} \sum_{Q\in\SYT(\lambda)} t^{\des(T)},
\]
since RSK maps descents on permutations to descents in tableaux.


\item 

If $\pi \rskArrow (P,Q)$ then $\sign(\pi) = \sign(P)\sign(Q)(-1)^e$,
where $\sign(T)$ is defined as $(-1)^{\inv(\rw(T))}$ and $e$ is the total length
of the even-indexed rows in $P$, see \cite{Reifegerste2004}.
\end{itemize}



\subsection[rsk-standardization]{RSK and standardization}

RSK commutes with standardization, so if $W\rskArrow (P,Q)$
then $W'=\std(W)$ is mapped to $(\std(P),\std(Q))$.

\begin{example*}
We have that
\[
W=\begin{pmatrix}
1& 1& 1& 2& 2& 3 \\
1& 2& 2& 1& 1& 2
\end{pmatrix}
\]
is mapped to the pair
\begin{figure}
\begin{ytableau}
1 & 1 & 1 & 2 \\
2 & 2
\end{ytableau}
\begin{ytableau}
1 & 1 & 1 & 3 \\
2 & 2
\end{ytableau}
\end{figure}

and the standardization of the word 
\[
W'=\begin{pmatrix}
1& 2& 3& 4& 5& 6 \\
1& 4& 5& 2& 3& 6
\end{pmatrix}
\]
is mapped to the following pair of standard Young tableaux.
\begin{figure}
\begin{ytableau}
1 & 2 & 3 & 6 \\
4 & 5
\end{ytableau}
\begin{ytableau}
1 & 2 & 3 & 6 \\
4 & 5
\end{ytableau}
\end{figure}

\end{example*}

\begin{lemma}
We have that for $w \in [d]^n$,
\[
\rev(w) \rskDualArrow P \implies \std(w) \rskArrow \std(P^t),
\]
and 
\[
(\rev \circ w) \rskDualArrow P
\implies
(\rev \circ \std \circ w) \rskDualArrow \std(P^t)^t.
\]
\end{lemma}
\begin{proof*}
From \cite[Prop. 2.3.14]{Butler1994} we have that 
\[
w \rskArrow P \iff \rev(w) \rskDualArrow P^t.
\]
This together with the fact that usual RSK commutes 
with standardization gives the first statement.
It is then easy to derive the second statement as well.
\end{proof*}


\section[skewRSK]{Skew RSK}

B. Sagan and R. Stanley consider a skew extension of RSK 
\url{http://www-math.mit.edu/~rstan/pubs/pubfiles/76.pdf}.

Also, I recommend the FPSAC2020 talk, \url{https://www.youtube.com/watch?v=SeFLyM3zDV0}
on injections (seen as generalizations of permutations).



\section[RSKstatistics]{Statistical properties of RSK}

See \url{https://arxiv.org/pdf/2005.03147.pdf}
and \url{https://arxiv.org/pdf/2005.14397.pdf}


\section[patternsAndRSK]{RSK and pattern avoidance}

See this preprint, \url{https://arxiv.org/pdf/1907.09451.pdf}
on how RSK can be applied to solve problems regarding pattern avoidance in permutations.

\section[rimHookRSK]{Rimhook RSK}

In \cite[Thm. 3]{StantonWhite1985}, the authors define a bijection,
\[
\begin{pmatrix}
H(1)& H(2)& \dotsc & H(\ell) \\
H_1& H_2& \dotsc & H_\ell \\
\end{pmatrix}
\rskArrow_{d}
(P,Q)
\]
where $H(1), H(2),\dotsc, H(\ell)$ is a sequence of hook shapes of size $d$,
filled with $1,2,\dotsc,\ell$ and $H_1, H_2, \dotsc, H_\ell$
is the same sequence of hook shapes, but the values are permuted.
The output is a pair $(P,Q)$ of $d$-rim-hook tableaux of the same shape.
The map $\rskArrow_{d}$ can be decomposed into a $d$-tuple of usual RSK-maps.


As a corollary, they have that $d$-hook tableaux of shape $\lambda$ are in 
bijection with $d$-tuples of standard Young tableaux 
(with different entries!) whose shapes are given by the $d$-quotient of $\lambda$.


\section[hillmanGrassl]{Hillman--Grassl correspondence}

A $\lambda$-array is a filling of the Young diagram of shape $\lambda$
with non-negative integers. There are no other conditions.
A reverse plane partition of shape $\lambda$ is a $\lambda$-array
with weakly increasing rows and columns.

The \defin{Hillman--Grassl correspondence} is a bijection from the set of $\lambda$-arrays
to the set of reverse plane partitions of shape $\lambda$.

See 
\url{http://sporadic.stanford.edu/reference/combinat/sage/combinat/tableau.html#sage.combinat.tableau.Tableau.hillman_grassl}
for definition.


\subsection[sulzgruberInsertion]{The Pak--Sulzgruber correspondence}

A different bijection is defined by I. Pak \cite{Pak2001}, from reverse plane partitions to $\lambda$-arrays.
The inverse of this map is called the Sulzgruber correspondence.


\todo{ Add also Sulzgruber insertion https://www.combinatorics.org/ojs/index.php/eljc/article/view/v26i3p25/7885
https://www.mat.univie.ac.at/~slc/wpapers/FPSAC2017/65%20Sulzgruber.pdf
}





\section[signedPermutationsRSK]{Signed permutations and RSK}

\todo{Insertion of type B signed permutations: https://arxiv.org/pdf/math/0308265.pdf}

See \url{https://arxiv.org/pdf/math/0308265.pdf}


\section[bereleRSK]{Berele insertion for the symplectic case}


In \cite{Berele1986}, a version of Schensted insertion is described, which proves the identity
\[
(a_1+ a_1^{-1} + \dotsb + a_n + a^{-1}_{n})^m = \sum_{\lambda} Q^\lambda_m(n)\schurSp_\lambda(a_1,\dotsc,a_n) 
\]
where $\schurSp_\lambda(a_1,\dotsc,a_n)$ is a \hyperref[schurSymplectic]{symplectic Schur function}
and $Q^\lambda_m(n)$ is the number of oscillating sequences 
of diagrams $\emptyset = \lambda^0,\lambda^1,\dotsc,\lambda^n = \lambda$ with length at most $n$.


\url{https://arxiv.org/pdf/1705.05454.pdf}


See also \url{http://fpsac2019.fmf.uni-lj.si/resources/Proceedings/67.pdf}
for $SO(2n+1)$.


\section[mulitSetRSK]{Multiset RSK}


See  \url{https://arxiv.org/pdf/1905.02071.pdf},
\url{https://arxiv.org/abs/math/0601514},
\url{https://arxiv.org/pdf/0705.2915.pdf},
and \url{https://www.macalester.edu/~halverson/papers/rsk-partition.pdf}

\todo{Add info on Multiset RSK}

\todo{Type B RSK? Terada, JOURNAL OF COMBINATORIAL THEORY, Series A 63, 90-109 (1993)}

\todo{RSK on set partitions: https://www.macalester.edu/~halverson/papers/rsk-partition.pdf
See  also \url{https://arxiv.org/pdf/1905.02071.pdf}
}

\todo{Add https://math.mit.edu/research/undergraduate/urop-plus/documents/Rowan.pdf
}

\todo{K-theoretic JDT:
\url{https://arxiv.org/pdf/0705.2915.pdf}
Hecke insertion (This paper is good!) \url{https://arxiv.org/abs/math/0601514}
}

\todo{Super Multiset RSK: https://arxiv.org/pdf/2308.07238.pdf}


\section[probabilisticRSK]{Probabilistic RSK}

Frieden and Schreier-Aigner \cite{FriedenSchreierAigner2021} introduced a $qt$-deformation of the Robinson--Schensted correspondence.
The output consists of several tableaux (with some probabilities), and this proves a (squarefree)
version of the Cauchy identity for \hyperref[macdonaldP]{Macdonald polynomials}.
In a more recent paper, they define a $qt$-version of RSK from where one can prove the dual Cauchy identity for 
\hyperref[macdonaldP]{Macdonald polynomials}, see \cite{FriedenSchreierAigner2024x}.


\section[geometricRSK]{Geometric RSK}

There is a geometric lift of RSK, where the classical RSK is given by the tropicalization of this lift.
The Whittaker functions then take the place of Schur functions.
See \url{https://maths.ucd.ie/~noconnell/pubs/cosz.pdf}


For a different geometrical aspect, see \cite{GillespieReimerBerg2023}
where Schubert-geometric type questions (points on a curve) are considered.


\section[nonsymmetricRSK]{Key polynomials and RSK}


\todo{Nonsymmetric: (keys and atoms) http://www.mat.uc.pt/~oazenhas/final--azemami_ejc_november_17_2014.pdf}


\todo{Supersymmetric Schur polynomials and RSK https://arxiv.org/pdf/1910.13710.pdf}

\todo{More RSK here: https://arxiv.org/pdf/1410.7683.pdf}



\section[otherRSK]{Other types of RSK}

\begin{itemize}
\item 
Strictly increasing and decreasing sequences in subintervals of words and a conjecture of Guo and Poznanović, \cite{BloomSaracino2022x}.

\item There is a type of RSK involving vacillating tableaux,
\[
  B(2k) = \sum_{\lambda} m^{\lambda}_k \cdot m^{\lambda}_k
\]
where $m^{\lambda}_k$ is the number of vacillating tableaux of shape $\lambda$, of length $k$ and $B(2k)$
is a Bell number (number of set partitions of $[2k]$), see \cite{HalversonLewandowski2005}.
For a generalization of this, see \cite{BerikkyzyHarrisPunYanZhao2022x}.

One can ask if this can be interpreted for run-sorted permutations,
see \cite{AlexanderssonNabawanda2021}. The number of run-sorted permutations of size $2k+1$ is given by $B(2k)$---
can one do some type of RSK on such permutations to obtain a pair of vacillating tableaux?


RSK has a close connection with promotion and the shadow construction by Viennot, see \cite{PfannererSwanson2025x}.

\end{itemize}
