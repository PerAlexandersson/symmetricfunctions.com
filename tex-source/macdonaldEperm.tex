\metatitle{Permuted basement Macdonald E polynomials}
\metadescription{Permuted basement Macdonald E polynomials and related symmetric functions.}


\section[macdonaldEperm]{Permuted basement Macdonald E polynomials}

\begin{polydata}{macdonaldEperm}
  Name     & Permuted basement Macdonald E polynomials \\
  Space    & All \\
  Basis    & True \\
  Rating   & 1 \\
  Bib      & Ferreira2011\\
  Year     & 2011 \\
  Symbol   & $\macdonaldE^{\sigma}_\mu(\xvec;q,t)$ \\
  Keywords & fillings \\
  Category & Schur \\
\end{polydata}



The \defin{permuted-basement Macdonald polynomials} 
(also called \emph{relative Macdonald polynomials}, \cite{GuoRam2021x}) generalize the 
non-symmetric Macdonald polynomials, by introducing an additional parameter $\sigma \in \symS_n$,
the \emph{basement}.
They were introduced in \cite{Ferreira2011} by J. Ferreira, as eigenpolynomials of certain operators.
Later in \cite{Alexandersson2015gbMacdonald}, a combinatorial model was introduced. 
For some properties of these formulas, see \cite{AlexanderssonSawhney2017,AlexanderssonSawhney2019}.

A good overview and introduction to this topic, is \cite{GuoRam2021x,GuoRam2021xSupp}.

\todo{
Multiline queue relation (Olya)
http://de.arxiv.org/pdf/1811.01024.pdf
}



\subsection[macdonaldEpermDefinition]{Definitions}


The following section is mainly from \cite{Alexandersson2015gbMacdonald}.

Let $\sigma = (\sigma_1,\dots,\sigma_n)$ be a list of $n$ different positive integers 
and let $\alpha=(\alpha_1,\dots,\alpha_n)$ be a weak integer composition.
An \defin{augmented filling} of shape $\alpha$ and \defin{basement} $\sigma$
is a filling of a Young diagram of shape $(\alpha_1,\dotsc,\alpha_n)$ with positive integers,
augmented with a zeroth column filled from top to bottom with $\sigma_1,\dotsc,\sigma_n$.

Note that we use \emph{English notation} rather than the 
skyline fillings used in \cite{HaglundHaimanLoehr2008,Mason2009}.
For example, the following figure illustrates the difference,
where the English notation is used in the left diagram, while the skyline convention is 
used in the right diagram.
\begin{figure}
\ytableaushort{655,5,4,3342,22,116}
\ytableaushort{{\none}{\none}2,6{\none}4{\none}{\none}5,123{\none}{\none}5,123456}
\end{figure}
In the skyline convention, the basement appears in the bottom of the diagram,
thus explaining the peculiar choice of terminology.


\begin{definition}
Let $F$ be an augmented filling. Two boxes $a$, $b$, are \defin{attacking}
if $F(a)=F(b)$ and the boxes are either in the same column,
or they are in adjacent columns, with the rightmost box in a row strictly below the other box.

\begin{figure}
\begin{ytableau}
a  \\
\none\vdots \\
b  \\
\end{ytableau}
or
\begin{ytableau}
 a & \none \\
 \none\vdots \\
  & b \\
\end{ytableau}
\end{figure}

\end{definition}

A filling is \defin{non-attacking} if there are no attacking pairs of boxes. 


\begin{definition}
A \defin{triple of type $A$} is an arrangement of boxes, $a$, $b$, $c$,
located such that $a$ is immediately to the left of $b$, and $c$ is somewhere below $b$,
and the row containing $a$ and $b$ is at least as long as the row containing $c$.
Similarly, a \defin{triple of type $B$} is an arrangement of boxes, $a$, $b$, $c$,
located such that $a$ is immediately to the left of $b$, and $c$ is somewhere above $a$,
and the row containing $a$ and $b$ is \emph{strictly} longer than the row containing $c$.

A type $A$ triple is an \defin{inversion triple} if the entries ordered increasingly
form a \emph{counter-clockwise} orientation. Similarly, a type $B$ triple is an inversion triple
if the entries ordered increasingly form a \emph{clockwise} orientation.
If two entries are equal, the one with largest subscript in the figures below
is considered largest.

\begin{figure}
Type $A$:
\begin{ytableau}
 a_3 & b_1 \\
 \none  & \none\vdots \\
\none & c_2 \\
\end{ytableau}

Type $B$:
\begin{ytableau}
c_2 & \none \\
\none\vdots  & \none \\
a_3 & b_1 \\
\end{ytableau}
\end{figure}

\end{definition}


If $u = (i,j)$ let $d(u)$ denote $(i,j-1)$.
A \defin{descent} in $F$ is a non-basement box $u$ such that $F(d(u)) \lt F(u)$.
The set of descents in $F$ is denoted $\Des(F)$.



\begin{example}
Here is a non-attacking filling of shape $(4,1,3,0,1)$ and basement $(4,5,3,2,1)$.
The bold entries are descents and the underlined entries form a type $A$ inversion triple.
There are in total $7$ inversion triples (of type $A$ and $B$).
\begin{figure}
\begin{ytableau}
\underline{4} & \underline{2} & 1 & \textbf{2} & 4\\
5 & 5\\
3 & 3 & \textbf{4} & 3\\
2\\
1 & \underline{1} \\
\end{ytableau}
\end{figure}
\end{example}


The \defin{leg}, $\leg(u)$, of a box $u$ in a diagram is the number of boxes
to the right of $u$ in the diagram.
The \defin{arm}, denoted $\arm(u)$, of a box $u = (r,c)$ in 
a diagram $\alpha$ is defined as the cardinality of
the sets
\begin{align*}
\{ (r', c) \in \alpha : r < r' \text{ and } \alpha_{r'} \leq \alpha_r \} \text{ and } \\
\{ (r', c-1) \in \alpha : r' < r \text{ and } \alpha_{r'} < \alpha_r \}.
\end{align*}

The \defin{major index} of an augmented filling $F$ is defined as
\begin{align*}
\maj(F) = \sum_{ u \in \Des(F) } \leg(u)+1.
\end{align*}
The \defin{number of inversions}, $\inv(F)$ of a filling is the number of inversion triples of either type.
The number of \defin{coinversions}, $\coinv(F)$, is the number of type $A$ and type $B$ triples which are \emph{not}
inversion triples.


Let $\mathrm{NAF}_\sigma(\alpha)$ denote all non-attacking fillings of 
shape $\alpha$, augmented with the basement $\sigma \in \symS_n$,
and all entries in the fillings are from $[n]$.


\begin{definition}
Let $\sigma \in \symS_n$ and let $\alpha$ be a weak composition with $n$ parts.
The \defin{non-symmetric permuted basement Macdonald polynomial} 
$\macdonaldE^\sigma_\alpha(\xvec;q,t)$ is defined as

\begin{equation}
\macdonaldE^\sigma_\alpha(\xvec; q,t) = \sum_{ F \in \mathrm{NAF}_\sigma(\alpha)} 
\xvec^F q^{\maj(F)} t^{\coinv(F)} 
\prod_{ \substack{ u \in F \\ u \text{ is in the basement or} \\ F(d(u))\neq F(u) }}  \frac{1-t}{1-q^{1+\leg(u)} t^{1+\arm(u)}}.
\end{equation}
The product is over all boxes $u$ in $F$ 
such that either $u$ is in the basement or $F(d(u))\neq F(u)$.
\end{definition}

When $\sigma = \omega_0$, we recover
the \hyperref[macdonaldE]{non-symmetric Macdonald polynomials}
defined in \cite{HaglundHaimanLoehr2008}, $\macdonaldE_\alpha(\xvec;q,t)$.
\emph{There is a slight difference in notation, the index $\alpha$ is reversed compared to \cite{HaglundHaimanLoehr2008}.}


A formula for the $\macdonaldE^\sigma_\alpha$ using 
set-valued tableaux can be found in \cite[Thm. 2.2]{DaughertyRam2022x}.
The proof relies on a bijection with alcove walks.


\subsection[macdonaldEpermProperties]{Properties}




\begin{proposition}[See \cite{CorteelMandelshtamWilliams2018}]
Let $\mu$ be a partition, of length at most $n$
and let $\macdonaldP_\mu(\xvec;q,t)$ denote the \hyperref[macdonaldP]{Macdonald P polynomials}.
Then
\[
 \macdonaldP_\mu(\xvec;q,t) = \sum_{\sigma \in \symS_n(\mu)} \macdonaldE^{\rev(\sigma)}_{inc(\mu)}(\xvec;q,t),
\]
where $inc(\mu)$ are the entries of $\mu$ sorted in increasing order.
\end{proposition}


\begin{conjecture}[Olya Mandelshtam, 2019 personal communication]
For any fixed \emph{composition} $\mu$, we can find
coefficients $R_{\mu}(q,t) \in \setQ(q,t)$ and some subset $M \subseteq \symS_n$ such that
\[
R_{\mu}(q,t) \macdonaldP_{\lambda(\mu)}(\xvec;q,t) = \sum_{\sigma \in M} \macdonaldE^{\sigma}_{\mu}(\xvec;q,t).
\]
\end{conjecture}
Combining \cite[Prop 1.1]{GuoRam2021xSupp} with \cite[Eq 1.10]{GuoRam2021xSupp}, 
one can more or less find the expansion 
\[
\macdonaldP_{\lambda{\mu}}(\xvec;q,t) = \sum_{\sigma \in \symS_n} R'_{\mu,\alpha}(q,t) \macdonaldE^{\sigma}_\mu(\xvec;q,t),
\]
see also \cite[Eq. 5.7.8]{Macdonald1996}.



\section[macdonaldEQuasi]{Quasisymmetric Macdonald E polynomials}


\begin{polydata}{macdonaldEQuasi}
  Name     & Quasisymmetric Macdonald E polynomials \\
  Space    & QSym \\
  Basis    & True \\
  Rating   & 1 \\
  Bib      & CorteelMandelshtamMasonWilliams2019\\
  Year     & 2019 \\
  Symbol   & $\macdonaldEQuasi_\alpha(\xvec;q,t)$ \\
  Keywords & fillings \\
  Category & Schur \\
\end{polydata}

A quasisymmetric version of the non-symmetric Macdonald 
polynomials were introduced in \cite{CorteelMandelshtamMasonWilliams2019x}.

They specialize to the quasisymmetric Schur polynomials at $q=t=0$.



\begin{conjecture}[Alexandersson 2020]
Let $\alpha$ be a composition. Then the coefficients $K_{\alpha\gamma}(q)$ in the expansion
\[
\macdonaldEQuasi_\alpha(\xvec;q,0) = \sum_{\gamma} K_{\alpha\gamma}(q) \schurQS_\gamma(\xvec) 
\]
are in $\setN[q]$. Note that this resemblence the 
fact that $\macdonaldE_\alpha(\xvec;q,0)$ are key-positive, with versions of \hyperref[kostkaFoulkes]{Kostka--Foulkes polynomials} as coefficients.
\end{conjecture}
