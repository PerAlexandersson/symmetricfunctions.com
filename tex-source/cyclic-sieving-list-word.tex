\metatitle{Word cyclic sieving}
\metadescription{A list of instances of the cyclic sieving phenomenon on words}

See the \hyperref[cyclicSieving]{cyclic sieving phenomenon} 
page for the definition and related theorems.


\section[cspListWords]{Words and permutations}


\subsection[cspWords]{Words}

Let $X_n(\alpha)$ be the set of words of length $n$ and content $\alpha$, that is, each word has $a_i$ entries equal to $i$.
The cyclic group $\grpc_n$ act on $X$ by cyclic shift, and let 
\[
f_n(\alpha;q) \coloneqq \qbinom{n}{\alpha}_q = \sum_{w \in X_n(\alpha)} q^{\maj(w)} = \sum_{w \in X_n(\alpha)} q^{\inv(w)}.
\]
Then $(X_n,\grpc_n,f_n(\alpha;q))$ exhibits the CSP, see \cite{ReinerStantonWhite2004}.


\subsection[cspMultisets]{Multisets}

Let $X_{n,k}$ be the set of $k$-multisets of $[n]$. Note that $|X_{n,k}| = \binom{n+k-1}{k}$.
For $n=3$, $k=2$ we have 
\[
X = \{11,12,13,22,23,33 \}.
\]
Let $\grpc_n = \langle g \rangle$ act on $X_{n,k}$ by the cycle $g=(123\dotsb n)$, so that $g\circ 23 = 31 = 13$ in the example above.
Then 
\[
\left( X_{n,k}, \grpc_n,  \qbinom{n+k-1}{k}_q \right)
\]
exhibits the CSP, see \cite{ReinerStantonWhite2004}.


\subsection[cspBinaryWordsTwist]{Binary words with a twist}

Let $X_n$ be the set of binary words and let $\eta$ 
act on $X_n$ by a \emph{twisted cyclic shift}
\[
\eta(b_1,b_2,\dotsc,b_{n-1},b_n) = (1-b_n,b_1,b_2,\dotsc,b_{n-1}),
\]
which generates a cyclic group of order $2n$, so $\langle \eta^2 \rangle$ is a cyclic 
group of order $n$.
Then
\[
\left(X_n, \langle \eta^2 \rangle, \prod_{j=0}^{n-1}(1+q^j) \right)
\]
exhibits the cyclic sieving phenomenon. This is proved in \cite{AlexanderssonLinussonPotka2019}.

By adjusting the $q$-analog a bit, one can show that
\[
\left(X_n, \langle \eta \rangle, \prod_{j=1}^{n}(1+q^j) \right)
\]
is an instance of CSP. Note that the $q$-binomial theorem implies that 
$\prod_{j=1}^{n}(1+q^j) = \sum_{k=0}^n q^{\binom{k+1}{2}} \qbinom{n}{k}$.


See also \hyperref[subsetCyclicSieving]{the example on subset-cyclic sieving},
for restricting the set of words we act on.

It was pointed out to me by S. Hopkins that this also follows from \cite{RushShi2012},
with \hyperref[cspListPosetsMinuscule]{cyclic sieving on minuscule posets of type $B$}.
The idea is to biject order ideals to binary words.
This in turn is related to \cite[Cor. 8.5]{ReinerStantonWhite2004}, by taking $k=0$ in their statement.


\subsection[cspWordsTwist]{General words with a twist}

We can generalize the twisted binary words CSP.

\begin{definition}
Let $W(n,k)$ denote the set of words of length $n$ with entries in $\{1,2,\dotsc,k\}$.
Given $w=(w_1,\dotsc,w_n) \in W(n,k)$, let $\phi(w) \coloneqq (w_n+1,w_1,w_2,\dotsc,w_{n-1})$,
where the addition is taken modulo $k$.
Note that $\phi^{\circ n}(w) = w+1$ and that it follows that $\phi^{\circ k}$ 
generate a cyclic group of order $n$.
\end{definition}

\begin{conjecture}[Alexandersson, 2022]
For every $n,k\geq 1$,
\[
 \left(
W(n,k), \left\langle \phi \right\rangle,
\prod_{j=1}^{n}\left(1+q^j+q^{2j}+q^{3j}+\dotsb+q^{(k-1)j}\right)  
 \right)
\]
is a CSP-triple.
\end{conjecture}



\subsection[cspKrewerasWords]{Kreweras words}

In this preprint, \url{https://arxiv.org/pdf/2005.14031.pdf}
a CSP is conjectured on the set of Kreweras words of length $3n$.
These are in bijection with linear extensions of a certain poset,
and also with so-called Kreweras webs.
The group action is promotion in the linear extensions, which is equivalent with rotation of the webs.


\subsection[cspLucasWords]{Lucas binary words}

The following results are proved in \cite{Gorodetsky2019}.

Let $LW_{n}$ be the set of binary words of length $n$,
such that there are no two consecutive ones, not even cyclically.
We let $\grpc_n$ act on $LW_{n}$ by rotation.
We note that $|LW_{n}|= L_n$, a Lucas number, \oeis{A000032}.
Define the $2\times 2$-matrix $A(x,t)$ with entries in $\setZ[q,t]$ as
\[
A(x,t)\coloneqq 
\begin{bmatrix}
1 & t \\
x & 0
\end{bmatrix}
\]
and let $|LW_{n}|_q \coloneqq \trace \left(  A(q^{n-1},1)  A(q^{n-2},1) \dotsm  A(1,1) \right)$.
This is a $q$-analog of $C_{n}$.
Then the following is a CSP-triple:
\[
\left( LW_n, \grpc_n, |LW_{n}|_q \right).
\]

This result can be refined in the following manner.
Let $LW_{n,k}\subset LW_{n}$ be the subset of binary words with exactly $k$ ones.
Let $|LW_{n,k}|_q \coloneqq [t^k] \trace \left(  A(q^{n-1},t)  A(q^{n-2},t) \dotsm  A(1,t) \right)$.
This is a $q$-analog of $LW_{n,k}$ and in fact,
\[
|LW_{n,k}|_q = \sum_{w \in LW_{n,k}}  q^{\stat(w)} 
= q^{k^2-k}\qbinom{n-k+1}{k}_q - 
q^{n+(k-1)^2-k}\qbinom{n-k-1}{k-2}_q.
\]
where $\stat(w) = \sum_{i=1}^n [w_i=1] (n-i)$, using the Iverson bracket.
We have the CSP-triple 
\[
\left( LW_{n,k}, \grpc_n, |LW_{n,k}|_q \right).
\]



\subsection[cspLongestTypeBReducedWords]{Reduced words of $w_0$ in type B}

Let $X_n$ be the set of all reduced words for the longest element in the type $B_n$ coxeter group.
We can define major index on these. Let $C_{n^2}$ act on such reduced words by rotation. 
Then 
\[
\left( X_n, C_{n^2}, q^{-n\binom{n}{2}}\sum_{w \in X_n} q^{\maj(w)} \right)
\]
is a CSP-triple. 
The proof in \cite{PetersenSerrano2010} is done via 
bijecting to $n\times n$-SYT and then using \hyperref[cspRectangularSYT]{B. Rhoades result}.



\subsection[cspPermutations]{Permutations of fixed type and exceedances}

The following CSP is proved in \cite{SaganShareshianWachs2011}.
Let $S_{\lambda,j} \subseteq \symS_n$ be the set of permutations of 
cycle type $\lambda$ and exactly $j$ exceedances.
Let $\grpc_n$ be generated by the long cycle $(1\,2\,3\,\dotsb\,n)$ and act on $S_{\lambda,j}$ by conjugation.
Define 
\[
f_{\lambda,j}(q) \coloneqq  \sum_{\pi \in S_{\lambda,j} } q^{\maj(\pi) - \exc(\pi)}.
\]
Then $(S_{\lambda,j}, \grpc_n, f_{\lambda,j}(q))$ is a CSP-triple.



\subsection[cspPermutations2]{Permutations}

The following CSP is a special case of \hyperref[csp01Matrices]{the CSP on binary matrices}.
Let $\symS_n$ be the set of permutations, seen as a list of pairs $\{i,\pi(i)\}$, $1 \leq i \leq n$.
We let $C_n = \langle g \rangle$ act on such pairs by $g \cdot \{i,\pi(i)\} = \{i+1,\pi(i)+1\}$,
where addition is performed modulo $n$.

Then
\[
\left( \symS_n, C_n, \sum_{\lambda \vdash n} (f^{\lambda}(q))^2 \right)
\]
is a CSP-triple, where $f^{\lambda}(q)$ is the $q$-analog of the \hyperref[prelimQHook]{hook formula}.


\subsection[cspPermutations3]{Permutations of fixed shape}

By using \hyperref[rsk]{RSK}, we have that
the set $\SYT(\lambda) \times \SYT(\lambda)$ is in bijection with permutations in $\symS_n$,
which insert to a pair of Young tableaux of shape $\lambda$.
In \cite{AlexanderssonPfannererRubeyUhlin2020x}, we show that
there exists a cyclic group action of order $n$, such that
for any $\lambda \vdash n$,
\[
 \left( \SYT(\lambda) \times \SYT(\lambda), C_n, \left(f^{\lambda}(q)\right)^2 \right) 
\]
is a CSP-triple. It is unclear what this group action might be.

Note that this refines the CSP on \hyperref[cspPermutations2]{permutation matrices}.


\subsection[cspPathGraphLabelings]{Labelings of the path graph}

The path graph $P_n$ on $n$ vertices has $n!$ labelings, using labels from $\setZ/n\setZ$.
The authors of \cite{DefantMadhukaraThomas2023x} study a version of promotion, named \defin{toric promotion}.
They introduce \defin{permutoric promotion} which is an operator determined by a permutation $\pi$ in $\symS_n$.
The role of $\pi$ is to decide in which order to perform certain local toggles.

Permutoric promotion, $\mathrm{TPro}_{\pi}$, 
acts on labelings with order $d(n-d)$ where $d$ is the number of descents of $\pi^{-1}$.
They show that 
\[
 \left( \text{Labelings of $P_n$}, \mathrm{TPro}_{\pi}, n(d-1)!(n-d-1)! [n-d]_{q^d} \qbinom{n-1}{d-1}_q \right)
\]
is a CSP-triple.
