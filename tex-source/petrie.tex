
\metatitle{Petrie symmetric functions}
\metadescription{An introduction to Petrie symmetric functions, their definition as sums of monomial symmetric functions with bounded parts, their Schur expansions, and various properties including recursions, combinatorial interpretations, and connections to modular Schur functions.}
\metakeywords{Petrie symmetric functions,modular Schur functions,Schur expansion,recurrence relations,combinatorial interpretation,principal specialization,Hopf algebra,Pieri rule}

\section[petrie]{Petrie symmetric functions}

\begin{polydata}{petrie}
  Name   & Petrie symmetric functions \\
  Space  & Sym \\
  Basis  & False \\
  Rating & 1 \\
  Bib    & DotyWalker1992 \\
  Year   & 1992 \\
  Category & Other \\
\end{polydata}


The \emph{Petrie symmetric functions} is a family of symmetric functions indexed by 
two non-negative integers. 
This family is the main topic in the work by D. Grinberg \cite{Grinberg2020x,Grinberg2020}.
They are also studied independently by H. Fu and Z. Mei \cite{FuMei2020}
under the name \emph{truncated homogeneous symmetric functions}.

The earliest reference we have found (so far) is \cite[Eq. (3.11)]{DotyWalker1992}, 
under the name \defin{modular complete symmetric functions}.
A corresponding family of \defin{modular elementary symmetric functions} is also defined.

A more recent paper, \cite{BazeniarAhmiaBelbachir2018}, studies the 
Petrie symmetric functions under the name \emph{generalized elementary symmetric functions},
when studying variants of $q$-binomial coefficients.
See also the recent reference is M. Ahmira and M, Merca \cite{AhmiaMerca2020x}.

The Petrie symmetric functions are special cases of the \hyperref[modularSchur]{modular Schur functions}.



The \defin{Petrie symmetric function} $G(k,m)$ is defined as
\[
G(k,m) \coloneqq \sum_{\alpha : |\alpha|=m, \max(\alpha) \lt k} \xvec^{\alpha}
= \sum_{\lambda \vdash m, \lambda_1 \lt k}
\monomial_\lambda(\xvec)
\]
where the first sum is over \hyperref[weakCompositions]{weak compositions} $\alpha$
of $m$, where all entries are less than $k$.
Note that for $k=2$, we have that $G(2,m) =\elementaryE_m$.
Hence, the alternative notation $E^{(k)}_m(\xvec)$ as seen in 
\cite{BazeniarAhmiaBelbachir2018,AhmiaMerca2020x} makes sense --- however there is a 
shift in indexing so that the elementary symmetric functions are recovered at $k=1$
in their notation.


Note that $G(k,m)$ is homogeneous degree $m$.
The non-homogeneous version is defined as
\[
G(k) \coloneqq \sum_{m \geq 0} G(k,m) = \sum_{\lambda : \lambda_1 \lt k} \monomial_\lambda(\xvec),
\]
where the second sum is over all integer partitions with all parts less than $k$.


The Petrie symmetric functions also appear as a special case in \cite{Numata2007},
as a special case of \defin{weighted complete homogeneous symmetric polynomials},
$\completeH^{(a)}_i(x_1,\dotsc,x_n)$ indexed by an integer $i\geq 0$ and a vector $a$.
We then let $a(t)\coloneqq \sum_{t\geq 0} a_i t^i$ and set
\[
\completeH^{\{a\}}_i(x_1,\dotsc,x_n) \coloneqq 
[t^i] a(t x_1) a(t x_2)\dotsm a(t x_n).
\]
This is clearly a symmetric polynomial. 
Now we can see that $G(k,m) = \completeH^{\{a\}}_m(\xvec)$ for the sequence 
$a_i =1$ if $i \lt k$ and $0$ otherwise.


\subsection[petrieRecursion]{Recurrence relations and identities}

For this section, see \cite{BazeniarAhmiaBelbachir2018}.
Define $E^{(s)}_k(\xvec)$ and $H^{(s)}_k(\xvec)$ via
\[
  \sum_{k=0}^\infty H^{(s)}_k(x_1,\dotsc,x_n) t^k = 
  \prod_{i=1}^n \left(  1-x_i t + \dotsb + (-x_i t)^s  \right)^{-1},
\]
and
\[
  \sum_{k=0}^\infty E^{(s)}_k(x_1,\dotsc,x_n) t^k = 
  \prod_{i=1}^n \left(  1 + x_i t + \dotsb + (x_i t)^s  \right)^{-1}.
\]
Note that $E^{(s)}_k(\xvec)$ is equal to $G(s+1,k)$.

\[
	E^{(s)}_k(x_1,\dotsc,x_n) = \sum_{j=0}^s x_n^j E^{(s)}_{k-j}(x_1,\dotsc,x_{n-1})
\]
A similar recursion is given for 
\[
	H^{(s)}_k(x_1,\dotsc,x_n) = \sum_{j=0}^s (-1)^j x_n^j H^{(s)}_{k-j}(x_1,\dotsc,x_{n-1}).
\]


In \cite{AhmiaMerca2020x} the following generalized Newton identity is proved,
where $s,k$ and $n$ are positive integers:
\[
   \sum_{j=0}^k (-1)^j E^{(s)}_j(x_1,\dotsc,x_n) H^{(s)}_{k-j}(x_1,\dotsc,x_n) = \delta_{0,n}.
\]

There are some interesting relations related to evaluations of monomial symmetric functions 
at roots of unity given in \cite{AhmiaMerca2020x}.

\subsection[petrieCombinatorial]{Combinatorial interpretation}

There is a combinatorial interpretation of the $E^{(s)}_j(x_1,\dotsc,x_n)$
in terms of lattice paths.

\subsection[petriePrincipal]{Principal specialization}

We have \cite[Cor. 4.1]{AhmiaMerca2020x} that
\[
E^{(s-1)}_k(1,q,q^2,\dotsc,q^{n-1}) = 
\sum_{j=0}^{\lfloor k/s \rfloor} (-1)^j q^{s \binom{j}{2}} \qbinom{n}{j}_{q^s} \qbinom{n+k-sj-1}{k-sj}_q.
\]


\subsection[petrieInSchur]{Schur expansion}

The Petrie symmetric functions expand with coefficients in $\{-1,0,1\}$
in the Schur basis:
\[
G(k) = \sum_\lambda H_k[ \schurS_\lambda ] \schurS_\lambda(\xvec)
\]
where $H_k$ is an algebra homomorphism defined via
\[
H_k[\completeH_j] \coloneqq 
\begin{cases}
1 \text{ if } 0 \leq j \lt k \\
0 \text{ otherwise}.
\end{cases}
\]
Note that the \hyperref[schurJacobiTrudi]{Jacobi--Trudi identity}
for Schur functions imply that $H_k[ \schurS_\lambda ]$
is the determinant of a Petrie matrix, 
which always have value $\pm 1$ or $0$, see \cite{GordonWilkinson1974} for more background.

\begin{example*}[Schur expansion of $G(4,8)$]
We have that $G(4,8)$ is given by 
\[
\monomial_{332}+\monomial_{2222}+\monomial_{3221}+\monomial_{3311}+\monomial_{22211}+\monomial_{32111}+\monomial_{221111}+\monomial_{311111}+\monomial_{2111111}+\monomial_{11111111}
\]
and this has Schur expansion
\[
\schurS_{332}+\schurS_{2222}-\schurS_{3221}+\schurS_{311111}-\schurS_{2111111}+\schurS_{11111111}.
\]
\end{example*}


An explicit formula for the coefficients in the Schur expansion is proved in \cite{Grinberg2020}.
A more explicit formula using \hyperref[partitionQuotient]{$k$-cores} is proved in \cite{ChengChouEuFuYao2022x}.
These formulas are used to prove the Liu--Polo conjecture, \cite[Remark 1.4.5]{LiuPolo2021}.


The Petrie symmetric functions have the following nice property:
For every $\lambda$, consider the expansion
\[
\schurS_\lambda \cdot G(k) = \sum_\mu \epsilon(\lambda,k,\mu) \schurS_\mu.
\]
The coefficients $\epsilon(\lambda,k,\mu)$ 
then have the property that they lie in $\{-1,0,1\}$.

\begin{problem}[D. Grinberg (2020), Institute Mittag--Leffler]
Classify all symmetric functions $F$, such that
for every partition $\lambda$, the product $\schurS_\lambda \cdot F$
has coefficients in $\{-1,0,1\}$ when expanded in the Schur basis.
\end{problem}


\subsection[petrieMN]{Multiplication by $\powerSum_2$}

In February 2022, I conjectured that for positive $k$, the product $G(k,m) \powerSum_2$
is signed multiplicity-free in the Schur basis. 
That is,
\[
G(k,m) \powerSum_2 = \sum_\mu u_\mu \schurS_\mu
\]
with $u_\mu \in \{-1,0,1\}$. 
This conjecture has now been proved in \cite[Thm. 1.5]{ChengChouEuFuYao2022x}.


\subsection[petrieHopf]{Hopf algebra}

There is a nice \hyperref[hopfAlgebra]{coproduct formula} for the Petrie symmetric functions:
\[
\Delta( G(k,m) ) = \sum_{j=0}^m  G(k,j) \otimes G(k,m-j).
\]


\subsection[pieriPetrie]{Pieri rule}

In \cite{JinJingLiu2024x}, the authors give a combinatorial proof of
the Pieri-type rule for the expansion 
\[
 G(k,m) \schurS_\mu = \sum_\lambda Pet_k(\lambda,\mu) \schurS_\lambda.
\]
The $Pet_k(\lambda,\mu)$ are in $\{-1,0,1\}$ and the paper above gives a combinatorial 
interpretation of these using ribbon tilings (similar to the Murnaghan--Nakayama rule).


More transparent proofs are given in \cite{WuEuKuShih2025x} as well as generalizing to
skew shapes.


\subsection[petrieDeterminant]{Generalization}

The following definition is done in \cite[Eq. (3.20)]{DotyWalker1992}.

It is natural to extend the definition of Petrie symmetric functions as follows.
Let
\[
G(k,\lambda) \coloneqq \det[ G(k, \lambda_i-i+j)  ]_{1\leq i,j \leq \length(\lambda)}.
\]
Then $G(k,\lambda) = \schurS_\lambda$ whenever $k > \lambda_1$,
and $G(k,\lambda)$ are the so called \hyperref[modularSchur]{modular Schur functions}.

The $G(k,\lambda)$ are not Schur positive in general.
For example,
\[
G(4,44) = 
\schurS_{4 4} + \schurS_{3 3 2} - \schurS_{4 3 1} + \schurS_{6 1 1} + 
 2 \schurS_{2 2 2 2} - \schurS_{3 2 2 1} + \schurS_{4 2 1 1} - 
 \schurS_{5 1 1 1} - \schurS_{2 1 1 1 1 1 1} + 
 \schurS_{1 1 1 1 1 1 1 1}.
\]
In fact, $G(5,321)$ is not even monomial-positive.




\section[modularSchur]{Modular Schur functions}

\begin{polydata}{modularSchur}
  Name   & Modular Schur functions \\
  Space  & Sym \\
  Basis  & False \\
  Rating & 1 \\
  Bib    & DotyWalker1992\\
  Year   & 1992\\
  Category & Schur \\
\end{polydata}

The \defin{modular Schur functions} were introduced 
by S. Doty and G. Walker \cite[Eq. (3.20)]{DotyWalker1992}, 
and further studied by G. Walker \cite{Walker1994}.

They are indexed by two parameters, a partition $\lambda$ and a positive integer $m$.
When $m$ is a prime number, the modular Schur function is a character of $\GL_n(K)$
where $K$ has characteristic $m$.



