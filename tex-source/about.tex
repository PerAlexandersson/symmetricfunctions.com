\metatitle{About}
\metadescription{Brief background on the catalog of symmetric functions.}


\section[about]{About}


I first came in contact with symmetric functions at \emph{Formal power series and algebraic combinatorics}, 
(FPSAC), back in 2011.
There is a plethora of generalizations of Schur functions and somewhere I decided that 
I wanted a good overview of all the definitions, relations and results.
The project is very much inspired by \emph{Schur functions: Theme and variations} by I.G. Macdonald \cite{Macdonald1992}.

The initial effort consists of the relation graphs made with TikZ and Mathematica,
and after showing this to people I got plenty of positive responses.
I then decided to put it online. 
Note that I am biased towards research I am more familiar with.

Please send me an email if you find this page useful or have any feedback.


\subsection[about-contributions]{Contributions}

I am grateful for the content contributions and suggestions by 
\name{Olga Azenhas},
\name{Darij Grinberg},
\name{Sam Hopkins},
\name[M. van Leeuwen]{Marc van Leeuwen}
and \name{Michelle Wachs}.

I am also grateful for some web design help provided by \href{https://alexrasch.github.io/}{Rasch Alexander} (cookie notice, mobile user interface).

\subsection[about-content]{Content}

The following types of content are given priority.

\begin{itemize}
\item  Basic definitions and references, even recent references on arxiv.
\item  Open problems and conjectures. In particular, make it easy for new researchers to start working on open problems.
\item  Brief overview of formulas such as analogs of Jacobi--Trudi identities.
\item  Some extra pages on topics that I would have found useful when doing my PhD.
\end{itemize}

If you have a particular formula, problem or result you want to promote, please send me an email.
Even better, provide a DOI or .bib-file with the reference and write the text and send it to me.

The entire website covers currently about 800 topics (by counting sections and subsections),
and uses a .bib file with over 1200 references. 
It was first published online around February 2019.


\subsection[howToCite]{How to cite}

You can use the following BibTeX entry when citing:
\begin{lstlisting}
@misc{Alexandersson2020,
 author = {Per Alexandersson},
 title = {The symmetric functions catalog},
 howpublished = {Online},
 url = {https://www.symmetricfunctions.com}
} 
\end{lstlisting}



\subsection[about-webpage]{Website}

There are no fancy techniques used on the website.
I write the information in plain files using LaTeX syntax and convert this
to HTML using Pandoc and Lua before uploading.
The references are automatically incorporated through this system using BibTeX.
I have use AI extensively to develop this build process.
The full code is available on \href{https://github.com/PerAlexandersson/symmetricfunctions.com}{\icon{github} GitHub}.


The math is rendered using the javascript library \href{https://katex.org//}{KaTeX} 
and I use \href{https://fontawesome.com/}{FontAwesome} for some of the icons.
It is my intention to keep the website mobile friendly as 
about $30\%$ of the users access the page using a phone.
