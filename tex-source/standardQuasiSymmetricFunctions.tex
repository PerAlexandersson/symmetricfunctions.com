\metatitle{Quasisymmetric functions}
\metadescription{An introduction to quasisymmetric functions, including monomial quasisymmetric functions, quasisymmetric power sums, and the involution on quasisymmetric functions.}  
\metakeywords{Quasisymmetric functions,monomial quasisymmetric functions,quasisymmetric power sums,involution on quasisymmetric functions}


\section[quasiSymmetricFunctions]{Quasisymmetric functions}

Quasisymmetric functions were formally introduced by \name{I. Gessel} in \cite{Gessel1984}.
Earlier work on \hyperref[pPartition]{P-partitions} anticipated this development.
For an introduction to quasisymmetric functions, see \cite{LuotoEtAl2013IntroQSymSchur}.


A function $f$ is \defin{quasisymmetric} if for every composition $\alpha$ of length $\ell$,
the coefficient of $x_1^{\alpha_1} \dotsm x_\ell^{\alpha_\ell}$
is the same as the coefficient of $x_{i_1}^{\alpha_1} \dotsm x_{i_\ell}^{\alpha_\ell}$,
for any $0 \lt i_1 \lt i_2 \lt \dotsb \lt i_{\ell}$.
The set of quasisymmetric functions form a graded ring, $\spaceQSym$.


Quasisymmetric functions of degree $n$ are usually indexed by either integer compositions of $n$,
or subsets of $[n-1]$.
Given a composition $\alpha \vDash n$ with $\ell$ parts, define
\[
S_\alpha \coloneqq \{\alpha_1, \alpha_1+\alpha_2,\dotsc, \alpha_1+\alpha_2+\dotsb+\alpha_{\ell-1}\}.
\]
The bijection $\alpha \mapsto S_\alpha$ maps compositions of $n$ to subsets of $[n-1]$.
The partial order $\alpha \leq \beta$ denotes \emph{refinement}. 
That is, $\beta$ can be obtained from $\alpha$ by adding consecutive parts of $\alpha$.
When $\alpha \leq \beta$, we illustrate this relationship with bars between parts of $\alpha$,
such that parts between bars add to parts of $\beta$.

\begin{example*}[Refinement and bars]
Taken from \cite{AlexanderssonSulzgruber2019}.
\[
112|341|21|34|2 \quad\text{corresponds to}\quad  \alpha = 11234121342, \quad \beta = 48372.
\]
\end{example*}


The most prevalent quasisymmetric function is perhaps the \hyperref[gessel]{Gessel quasisymmetric functions}.


\section[qmonom]{Monomial quasisymmetric functions}

\begin{polydata}{qmonom}
  Name   & Monomial quasisymmetric functions \\
  Space    & QSym \\
  Basis    & Yes \\
  Rating   & 5 \\
  Bib      & Gessel1984 \\
  Symbol   & $\qmonom_\alpha(\xvec)$ \\
  Year     & 1984 \\
  Category & QuasiElementary \\
\end{polydata}

Given a composition $\alpha$ with $\ell$ parts, 
we define the \defin{monomial quasisymmetric functions} as
\[
\qmonom_\alpha(\xvec) \coloneqq 
\sum_{i_1 < i_2 < \dotsb < i_\ell} 
x_{i_1}^{\alpha_1} x_{i_2}^{\alpha_2} \dotsm x_{i_\ell}^{\alpha_\ell}.
\]
The functions $\qmonom_\alpha$ constitute a basis for the space of 
homogeneous quasisymmetric functions of degree $n$ as $\alpha$ ranges over all compositions of $n$.

The monomial quasisymmetric functions refine the \hyperref[monomial]{monomial symmetric functions},
\[
\monomial_{\lambda}(\xvec) = \sum_{\alpha \sim \lambda}\qmonom_{\alpha}(\xvec)
\]
where we sum over all compositions $\alpha$ that are a permutation of $\lambda$.


\section[qPsi]{Powersum quasisymmetric functions (Psi)}

\begin{polydata}{qPsi}
  Name   & Powersum quasisymmetric functions (Psi) \\
  Space    & QSym \\
  Basis    & Yes \\
  Rating   & 3 \\
  Bib      & BallantineDaughertyHicksMason2020 \\
  Symbol   & $\qPsi_\alpha(\xvec)$ \\
  Year     & 2017 \\
  Category & QuasiElementary \\
\end{polydata}

There are two quasisymmetric refinements of the power-sum symmetric functions 
introduced in \cite{BallantineDaughertyHicksMason2020}, denoted $\qPsi_\alpha$ and $\qPhi_\alpha$.


Given a pair of compositions of $n$, $\alpha \leq \beta$, related by
\begin{equation*}
 \alpha_{11} \alpha_{12} \dotsc \alpha_{1,i_1}| \alpha_{21} \alpha_{22} \dotsc \alpha_{2,i_2} | \dotsb |
 \alpha_{k1} \alpha_{k2} \dotsc \alpha_{k,i_k}
\end{equation*}
let 
\begin{equation*}
\pi(\alpha,\beta)
\coloneqq
\prod_{j=1}^k (\alpha_{j1})(\alpha_{j1}+\alpha_{j2})\dotsb (\alpha_{j1}+\alpha_{j2}+\dotsb +\alpha_{j,i_j}).
\end{equation*}
The \defin{quasisymmetric power sum} $\qPsi_\alpha$ is defined as
\begin{equation*}
\qPsi_\alpha(\xvec) \coloneqq z_\alpha \sum_{\beta \geq \alpha } \frac{1}{\pi(\alpha,\beta)} \qmonom_\beta(\xvec).
\end{equation*}
For example,
$\Psi_{231} = 
\frac{1}{10}\qmonom_6 + 
\frac{1}{4}\qmonom_{24}+
\frac{3}{5}\qmonom_{51}+\qmonom_{231}$.
It was shown in \cite{BallantineDaughertyHicksMason2020} 
that quasisymmetric power sums refine the usual power sums as
\[
\powerSum_{\lambda}(\xvec) = \sum_{\alpha \sim \lambda}\qPsi_{\alpha}(\xvec).
\]


The $\qPsi_\alpha$ have a nice relationship with \hyperref[gesselQuasiPowerSum]{Gessel quasisymmetric functions}.




\section[qPhi]{Powersum quasisymmetric functions (Phi)}

\begin{polydata}{qPhi}
  Name   & Powersum quasisymmetric functions (Phi) \\
  Space    & QSym \\
  Basis    & Yes \\
  Rating   & 3 \\
  Bib      & BallantineDaughertyHicksMason2020 \\
  Symbol   & $\qPhi_\alpha(\xvec)$ \\
  Year     & 2017 \\
  Category & QuasiElementary \\
\end{polydata}



\section[qRho]{Powersum quasisymmetric functions (p)}

\begin{polydata}{qRho}
  Name   & Powersum quasisymmetric functions (p) \\
  Space    & QSym \\
  Basis    & Yes \\
  Rating   & 2 \\
  Bib      & AliniaeifardWangWilligenburg2021x \\
  Symbol   & $\qRho_\alpha(\xvec)$ \\
  Year     & 2021 \\
  Category & QuasiElementary \\
\end{polydata}


In \cite{AliniaeifardWangWilligenburg2021x}, the authors introduce
a third quasisymmetric refinement of the power-sum basis, denoted $\qRho_\alpha$.
They refer to this basis as the \emph{combinatorial quasisymmetric power-sum basis}.

Let $\alpha$ be a composition and let $\lambda$
be the partition obtained from $\alpha$ by rearranging the parts in decreasing order.
The monomial expansion of $\qRho_\alpha$ is given by
\[
\qRho_\alpha(\xvec) = \sum_{\beta} R_{\alpha \beta} \qmonom_\beta(\xvec)
\]
where $R_{\alpha \beta}$ is the number of ordered set-partitions (see \oeis{A000670})
\[
 \gamma_1 | \gamma_2 | \dotsb | \gamma_k
\]
with the following two properties. The word
\[
\lambda_{\gamma_{11}}, \lambda_{\gamma_{12}},\dotsc,\;
\lambda_{\gamma_{21}}, \lambda_{\gamma_{22}},\dotsc,\; \dotsc,
\lambda_{\gamma_{k1}}, \lambda_{\gamma_{k2}},\dotsc,
\]
is equal to $\alpha$, and 
$
\lambda_{\gamma_{i1}}+ \lambda_{\gamma_{i2}}+\dotsb = \beta_i
$
for all $i$. For example, if $\lambda = 322211$, then
the set-partition $235|16|4$ contributes to $R_{\alpha \beta}$ where
\[
\alpha = (2,2,1,3,1,2), \qquad \beta = (5,4,2).
\]
The original article \cite{AliniaeifardWangWilligenburg2021x}
uses the enumeration of certain matrices to define the $R_{\alpha \beta}$,
but the above definition is equivalent to theirs.

\begin{example*}[Table of $\qRho_\alpha$, for size $4$]

The following table shows the monomial expansion of $\qRho_\alpha$.
\begin{array}{ll}
\toprule
\alpha & \qRho_\alpha \\
 \bottomrule
 4 & \qmonom_4 \\
 13 & \qmonom_{13} \\
 22 & \qmonom_4+2 \qmonom_{22} \\
 31 & \qmonom_4+\qmonom_{31} \\
 112 & \qmonom_{22}+2 \qmonom_{112} \\
 121 & 2 \qmonom_{13}+2 \qmonom_{121} \\
 211 & \qmonom_4+\qmonom_{22}+2 \qmonom_{31}+2 \qmonom_{211} \\
 1111 & \qmonom_4+4 \qmonom_{13}+6 \qmonom_{22}+4 \qmonom_{31}+12 \qmonom_{112}+12
   \qmonom_{121}+12 \qmonom_{211}+24 \qmonom_{1111} \\
 \bottomrule
\end{array}
\end{example*}

The authors also give formulas for computing product and coproduct.


\section[standardQuasiInvolution]{Involution on symmetric functions}

\todo{
See section 3.6 on involutions!
https://sites.math.washington.edu/~billey/colombia/references/QuasiSchurBook.pdf
}

The \hyperref[standardInvolution]{standard involution on symmetric functions} 
that sends $\schurS_\lambda$ to $\schurS_{\lambda'}$ 
can be extended to quasisymmetric functions in two ways,
here defined via the \hyperref[gessel]{Gessel quasisymmetric functions}.
\[
\omega \gessel_{n,S}(x) = \gessel_{n,[n-1]\setminus (n-S)}(x) \qquad \text{ or }  \qquad
\psi \gessel_{n,S}(x) = \gessel_{n, [n-1]\setminus S}(x).
\]
Here, $n-S$ is the set $\{n-s : s \in S\}$.

In \cite{BallantineDaughertyHicksMason2020}, it is shown that
$
\omega\left( \qPsi_\alpha \right) = (-1)^{|\alpha|-\length(\alpha)}\qPsi_{\alpha^r},
$
where $\alpha^r$ denotes the reverse of $\alpha$.
The same property holds for $\qRho_\alpha$, see \cite[Thm. 5.7]{AliniaeifardWangWilligenburg2021x}.



