


\section[jackPi]{Jack interpolation polynomials}


\begin{polydata}{jackPi}
  Name   & Jack interpolation polynomials \\
  Space    & All \\
  Basis    & True \\
  Rating   & 2 \\
  Bib      & Sahi1994 \\
  Year     & 1994 \\
  Symbol   & $\jackP^{\rho}_\lambda(\xvec)$ \\
  Keywords & fillings \\
  Category & Schur \\
\end{polydata}

The \emph{Jack interpolation polynomials} $\jackP^{\rho}_\lambda(\xvec)$
are non-homogeneous extensions of the \hyperref[jackP]{Jack polynomials},
where $\rho = (\rho_1,\rho_2,\dotsc)$ is a sequence of indeterminates.
These polynomials were first introduced by S. Sahi in 1994, see \cite{Sahi1994}.
Similar to the \hyperref[jackShifted]{shifted Jack polynomials}, the 
Jack interpolation polynomials are characterized by relatively simple vanishing conditions.


\subsection[jackPiDef]{Definition}

The polynomial $\jackP^{\rho}_\lambda(x_1,\dotsc,x_n)$, is the unique
symmetric function of total degree $|\lambda|$ which satisfies the following.
\begin{itemize}
\item For every integer partitions $\mu$, $|\mu| \leq |\lambda|$ and $\mu \neq \lambda$,
we have $\jackP^{\rho}_\lambda(\mu + \rho) = 0$.

\item The coefficient of $\monomial_\lambda$ in $\jackP^{\rho}_\lambda$ is $1$.

\end{itemize}

Let $\delta = (n-1,n-2,\dotsc,1,0)$.
In the case $\rho = r \delta$, one have that 
\[
 \jackP^{r \delta}_\lambda(\xvec) = \jackP_\lambda(\xvec; 1/r) + \text{ lower order terms}.
\]
That is, a particular choice of $\rho$ allows us to recover the classical 
Jack $P$ polynomial $\jackP_\lambda(\xvec; 1/r)$, as the highest term.

\subsection[jackJiMonomialPositivity]{Monomial positivity}

In \cite{NaqviSahiSergel2021x}, the authors study the normalized version
\[
 \jackJ^{r \delta}_\lambda(\xvec) \coloneqq 
 H_\lambda(a) \cdot (-1)^{|\lambda|} \jackP^{r \delta}_\lambda(-\xvec),
\]
where $H_\lambda$ is an \hyperref[jackJ]{$a$-deformation of hook products},
and $r = 1/a$.

They show that the coefficients $c_{\mu}(a)$ in the expansion
\[
\jackJ^{r \delta}_\lambda(\xvec) = \sum_{\mu} a^{|\mu|-|\lambda|} c_{\mu}(a) \monomial_\mu(\xvec)
\]
are elements in $\setN[a]$.

In fact, in \cite[Thm. 5.4]{NaqviSahiSergel2021x} the authors give a combinatorial expansion,
\[
\jackJ^{r \delta}_{\gamma^+}(\xvec) = \sum_{\text{$T$ admissible}} d_T(a) \xvec^{\underline{w(T)}},
\]
where $ \xvec^{\underline{w(T)}}$ is a certain \emph{bar monomial}, which is monomial positive.
This formula is the interpolation analog of the \hyperref[jackCombinatorialFormula]{Knop--Sahi formula}.
A non-symmetric version is also given in \cite{NaqviSahiSergel2021x}.



% \section[jackF]{Jack non-symmetric interpolation polynomials}

% 
% \begin{polydata}{jackF}
%   Name   & Jack non-symmetric interpolation polynomials \\
%   Space    & All \\
%   Basis    & True \\
%   Rating   & 2 \\
%   Bib      & Jack1970 \\
%   Year     & 1970 \\
%   Symbol   & $\jackF_\alpha(\xvec)$ \\
%   Keywords & fillings \\
%   Category & Schur \\
% \end{polydata}
% 
% \todo{ Add defs and info from {NaqviSahiSergel2021x}  }


