\metatitle{Shifted Jack polynomials}
\metadescription{An introduction to shifted Jack polynomials, their definitions via vanishing conditions and reverse semistandard Young tableaux, as well as their Littlewood--Richardson coefficients and related conjectures.}

\section[jackShifted]{Shifted Jack polynomials}


\begin{polydata}{jackShifted}
  Name   & Shifted Jack polynomials \\
  Space    & ShiftedSym \\
  Basis    & True \\
  Rating   & 3 \\
  Bib      & Sahi1994 \\
  Year     & 1994 \\
  Symbol   & $\jackShifted_\lambda(\xvec)$ \\
  Keywords & inner-product\\
  Category & Schur \\
\end{polydata}

The shifted Jack polynomials generalize the \hyperref[jackP]{Jack polynomials},
as well as the \hyperref[schurShifted]{shifted Schur polynomials}.
The shifted Jack polynomials constitute a basis for the ring of \defin{shifted symmetric functions} $\Lambda^{a}$.
This is the ring of polyomials symmetric in variables $x_i - i/a$.
Note that these functions are not homogeneous.

They are given as a limit of \hyperref[macdonaldPi]{interpolation Macdonald P polynomials}.

\subsection[jackShiftedDef]{Definition}

Define two $a$-deformations of the hook products, $H_\lambda$ and $H'_\lambda$, as
\[
H_\lambda = \prod_{s \in \lambda} (a \cdot \arm_\lambda(s) + \leg_\lambda(s) + 1), \quad
H'_\lambda = \prod_{s \in \lambda} (a \cdot  \arm_\lambda(s) + \leg_\lambda(s) + a).
\]
These are denoted $c_\lambda(a)$ and $c'_\lambda(a)$ in \cite[(10.22)]{Macdonald1995}.


The polynomials $\jackShifted_\mu$ lie in $\Lambda^{a}(x_1,\dotsc,x_n)$ and it 
was shown by F. Knop and S. Sahi in \cite{KnopSahi1996} that $\jackShifted_\mu$ 
is the unique polynomial in $\Lambda^{a}$ of degree at most $|\mu|$ such that
\begin{equation*}
\jackShifted_\mu(\lambda) = 
\begin{cases}
a^{-|\mu|} H'_\mu,   &\lambda = \mu \\
0, \quad &|\mu| \geq |\lambda|, \; \mu \neq \lambda.
\end{cases}
\end{equation*}
(Note that the inquality is in the wrong direction in 
\cite[p.70]{OkounkovOlshanski1997}.)

We also define the shifted version of $\jackJ_\mu(\xvec;a)$ as $\jackShiftedJ_\mu(\xvec,a) \coloneqq H_\mu \jackShifted_\mu(\lambda)$.



\begin{example*}[Expression for $\jackShifted_{(2,1)}(\xvec,a)$]
The polynomials get rather complicated.
\[
(a^2 x_1 x_2^2+a^2 x_1 x_3^2+a^2 x_2 x_3^2+a^2 x_1^2
   x_2-2 a^2 x_1 x_2+a^2 x_1^2 x_3+a^2 x_2^2 x_3-2
   a^2 x_1 x_3-2 a^2 x_2 x_3+2 a x_1 x_2^2+a x_2^2+2
   a x_1 x_3^2+2 a x_2 x_3^2+2 a x_3^2+2 a x_1^2
   x_2-3 a x_1 x_2-2 a x_2+2 a x_1^2 x_3+2 a x_2^2
   x_3-5 a x_1 x_3+6 a x_1 x_2 x_3-7 a x_2 x_3-4 a
   x_3+2 x_2^2+4 x_3^2+2 x_1 x_2-4 x_2+4 x_1 x_3+6
   x_2 x_3-14 x_3)/(a(a+2))
\]

\end{example*}

\subsection[jackShiftedRSSYTDef]{RSSYT definition}

Similar to how the Jack polynomials can be \hyperref[jackTableauFormula]{defined} as a sum over reverse semistandard Young tableaux,
there is an analogous formula for the shifted Jack polynomials, see \cite{OkounkovOlshanski1997}.
\begin{equation*}
\jackShifted_\mu(\xvec;a) = \sum_{T \in \mathrm{RSSYT}(\mu)} \psi_T(a) \prod_{s \in \mu} (x_{T(s)} -  \arm'(s) + \leg'(s)/a),
\end{equation*}
where $\arm'(s) = j-1$ and $\leg'(s) = i-1$ for the box $s=(i,j)$.
Comparing this with the \hyperref[jackTableauFormula]{similar-looking formula} for the Jack P polynomials,
we see that the top degree component of $\jackShifted_\mu(\xvec;a)$ is given by the ordinary Jack function $\jackP_\mu(\xvec;a)$.

See also the corresponding formula for 
the \hyperref[macdonaldPiRSSYTFormula]{interpolation Macdonald P polynomials}.


\section[jackShiftedLittlewoodRichardson]{Littlewood--Richardson coefficients}

Define the Littlewood--Richardson type structure constants, $c^{\lambda}_{\mu\nu}$, 
which are rational functions in $a$, by the relation
\[
\jackShifted_\mu \jackShifted_\nu = \sum_\lambda c^{\lambda}_{\mu\nu}(a) \jackShifted_\lambda.
\]
If $|\lambda|=|\mu|+|\nu|$, then $c^{\lambda}_{\mu\nu}(a) = \langle \jackP_\mu \jackP_\nu, \jackP_\lambda \rangle_a$,
the \hyperref[jackLittlewoodRichardsonConjecture]{ordinary Jack structure constants}.
For $a = 1$, $c^{\lambda}_{\mu\nu}(1)$ coincide with the usual Littlewood--Richardson coefficients.

From the vanishing result by F. Knop and S. Sahi \cite{KnopSahi1996} and a simple contradiction argument
(see \cite[top of page 4434]{MolevSagan1999} for the case $\alpha=1$), it follows that
$c^{\lambda}_{\mu\nu}(a)$ is identically zero unless $\lambda \supseteq \mu$ and $\lambda \supseteq \nu$. 



\subsection[jackShiftedLittlewoodRichardsonConjecture]{Generalization of Stanley's conjecture}

The following section contains information from \cite{AlexanderssonFeray2019}.

Let $g^{\lambda}_{\mu \nu}(a) \coloneqq  H'_\lambda H_\mu H_\nu c^{\lambda}_{\mu\nu}(a)$.

\begin{conjecture}[Alexandersson, Féray (2014)]
The expression $a^{|\lambda|-|\mu|-|\nu|-2} g^{\lambda}_{\mu\nu}(a)$ is a polynomial in $\setN[a]$.
\end{conjecture}
Note that whenever $|\lambda|=|\mu|+|\nu|$, this conjecture 
implies \hyperref[jackLittlewoodRichardsonConjecture]{Stanley's conjecture}
\cite{Stanley1989} regarding structure constants for Jack polynomials.


\begin{proposition}[P. Alexandersson, V. Féray (2019)]
The expression $a^{|\mu|+|\nu|-|\lambda|-2} g^{\lambda}_{\mu\nu}$ is a polynomial in $a$.
\end{proposition}

\subsection[jackShiftedLittlewoodRichardsonFormula]{Recursion for the $c^{\lambda}_{\mu\nu}(a)$}

One can prove that $c^{\lambda}_{\mu\lambda}(a) = \jackShifted_\mu(\lambda,a)$.


Define $\psi'_T(\alpha)$, which is similar to $\psi_T(\alpha)$ and calculated as
\begin{equation*}
\psi'_T(\alpha) \coloneqq \prod_{i=1}^n \psi'_{\rho^i/\rho^{i-1}}(\alpha)
\end{equation*}
and where $\psi'_{\lambda/\mu}(\alpha) \coloneqq \psi_{\lambda'/\mu'}(1/\alpha)$.

The following proposition can be proved by applying the same technique as in \cite{MolevSagan1999}.
\begin{proposition}[See \cite{Sahi2011}]
Let $\mu, \nu \subseteq \lambda$. Then
\begin{equation*}
c^{\lambda}_{\mu\nu} = 
\frac{1}{|\lambda|-|\nu|}\left( 
 \sum_{\nu \to \nu^+} \psi'_{\nu^+ / \nu} c^{\lambda}_{\mu \nu^+} -
 \sum_{\lambda^- \to \lambda } \psi'_{\lambda / \lambda^-} c^{\lambda^-}_{\mu \nu}
\right)
\end{equation*}
where the first sum is taken over all possible ways to add one box to the diagram $\nu$,
and the second sum is over all ways to remove one box from $\lambda$.
\end{proposition}
This together with the identity $c^{\lambda}_{\mu \lambda} = \jackShifted_\mu(\lambda)$
gives a recursive method to compute the $c^{\lambda}_{\mu\nu}$.

\subsection[jackShiftedLittlewoodRichardsonData]{Data for structure constants}

Let
\[
\mathtt{JackStructureConstant}_{\mu,\nu,\lambda}(a) \coloneqq  H'_\lambda H_\mu H_\nu c^{\lambda}_{\mu\nu}(a)
\]
be the structure constants $g^{\lambda}_{\mu \nu}(a)$ appearing 
in \hyperref[jackShiftedLittlewoodRichardsonConjecture]{the generalized Stanley conjecture}.
We have computed these for all $\mu,\nu$ with size at most $6$, 
and <a href="jackStructureConstants6.m" class="dataFile">the data is available for download here (1.4Mb)</a>.

For example, in the data you can find 
that $\mathtt{JackStructureConstant}_{311,222,42211}(a)$ is equal to 
\begin{align}
& 192a^4(1 + a)^2(3 + a)(4 + a)(1 + 2a)(2 + 3a) \cdot \\
& (45 + 276a + 547a^2 + 426a^3 + 134a^4 + 12a^5).
\end{align}
Note that some of these constants are Laurent polynomials in $a$.

