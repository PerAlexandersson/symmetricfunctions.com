\metatitle{Tableaux cyclic sieving}
\metadescription{A list of instances of the cyclic sieving phenomenon on tableaux  and fillings}

See the \hyperref[cyclicSieving]{cyclic sieving phenomenon} 
page for the definition and related theorems.


\section[cspTableau]{Standard Young tableaux}


\subsection[cspRectangularSYT]{Rectangular SYT}


Let $\SYT(a^b)$ the set set of standard Young tableaux with $b$ rows of length $a$,
and let $\langle \partial \rangle = \grpc_{ab}$ be the cyclic group generated by \hyperref[promotion]{promotion}.
Finally, for a partition $\lambda \vdash n$ define the $q$-analogue of the hook formula
\[
f^{\lambda}(q) \coloneqq \frac{[n]_q!}{\prod_{\square \in \lambda} [h(\square)]_q}.
\]
Then 
$
(\SYT(a^b), \grpc_{ab}, f^{a^b}(q))
$
is a CSP-triple, see B. Rhoades \cite{Rhoades2010}.

For a different proof using Wronskians, see \cite{Purbhoo2013}.
An overview of different approaches is given in the master's thesis by D. Rhee \cite{Rhee2019}.
See also C. Ahlbach \cite{Ahlbach2019} for some related results and open conjectures.
In particular, Ahlbach explicitly describes the fixed-points under the promotion.

Other alternative proofs can be found in \cite{FontaineKamnitzer2013}, \cite{Westbury2016} and \cite{Westbury2019}.

The 3-row case can also be seen as rotation of certain graphs introuced in \cite{Kuperberg1996}.
The cyclic sieving-result is proved in \cite{PetersenPylyavskyyRhoades2008}. 
These graphs constitute a so-called web basis for $U_q(\mathfrak{sl}_3)$.

The 4-row case can be interpreted as rotation of \emph{hourglass plabic graphs} introduced 
in \cite{GaetzPechenikPfannererStrikerSwanson2023x}.
That is, there is a bijection between 4-row rectanguar SYT and hourglass plabic graphs,
such that promotion on the SYTs maps to rotation of these planar graphs.
These graphs give rise to a web basis for $U_q(\mathfrak{sl}_4)$.




\subsection[cspStretchedSYT]{Stretched SYT}

In \cite{AlexanderssonPfannererRubeyUhlin2020x}, we show that
there exists a cyclic group action of order $n$, such that
for any $\lambda$,
\[
 \left( \SYT( n \lambda), C_n, f^{n \lambda}(q) \right) 
\]
is a CSP-triple. 
It is unclear what this group action might be.



\subsection[cspStaircaseSYT]{Staircase and other SYT}



Let $sc_k$ be the staircase shape for the partition $(k,k-1,\dotsc,2,1)$. 
In \cite{PonWang2011}, the authors show that promotion on $\SYT(sc_k)$ is
acting in a nice fashion, but they do not manage to show a cyclic sieving phenomenon.
In \cite{Purbhoo2018}, building on his earlier work using Wronskians to understand promotion,
and by considering the Lagrangian Grassmannian, Purbhoo showed that every
promotion symmetry class for the staircase is counted
by the number of certain ribbon tableaux.
N. Williams conjectured a product formula for a polynomial
which should be a CSP polynomial for the action of
promotion on $\SYT(sc_k)$: see \cite{Hopkins2020} for this conjecture.

Now let $ssc_k$ be the shifted staircase shape for the strict
partition $(k,k-1,\dotsc,2,1)$. By considering the orthogonal Grassmannian,
Purbhoo \cite{Purbhoo2018} showed that every symmetry class of
promotion on $\SYT(ssc_k)$ can again be counted by the number of certain ribbon tableaux.
Using this result of Purbhoo, Sekheri--Schank--Djermane \cite{SekheriSchankDjermane2014}
verified that 
\[
(\SYT(ssc_k),\langle\partial\rangle,f(q))
\]
is a CSP-triple, where 
\[
f(q)\coloneqq \frac{[k(k+1)/2]_q!}{\prod_{1\leq i \leq j \leq k}[i+j-1]_q},
\]
which is again the major index generating function for these SYTs.

For more results/conjectures on CSPs for promotion acting on
linear extensions of various posets (including the shifted trapezoid shape,
the shifted double staircase shape, and the "$\mathsf{V}\times[n]$" poset),
see \cite{Hopkins2020}.



\section[cspListSSTableaux]{Semistandard type tableaux}

\subsection[cspRectangularSSYT]{Rectangular SSYT}

Let $\SSYT(a^b,k)$ the set set of semi-standard Young tableaux with $b$ rows of length $a$ and entries less then 
or equal to $k$.
Let $\langle \partial \rangle = \grpc_{k}$ be the cyclic group generated by \hyperref[kpromotion]{$k$-promotion}.
Consider the $q$-analogue of the \hyperref[schurSpecializations]{hook-content formula}:
\[
X_{\lambda,k}(q) \coloneqq q^{-n(\lambda)} \schurS_\lambda(1,q,q^2,\dotsc,q^{k-1}) = 
\prod_{(i,j) \in \lambda} \frac{[n+c(i,j)]_q}{ [h(i,j)]_q}.
\]
Then 
$
(\SSYT(a^b,k), \grpc_{k}, X_{\lambda,k}(q))
$
is a CSP-triple, \cite{Rhoades2010}. 
This result can also be deduced from \cite{ShenWeng2020} by mapping promotion to toggles.


In \cite{FontaineKamnitzer2013}, B. Rhoades result is refined in the following manner.
Let $\gamma$ be a composition of $ab$, such that $\gamma$ is invariant under $\ell$-th cyclic shift,
and let $\partial$ denote the \hyperref[kpromotion]{$k$-promotion} operator.
Then 
\[
\left( \SSYT(a^b,\gamma), \langle \partial^\ell \rangle, q^{\frac12(a^2 b- (i_1^2+i_2^2+\dotsb + i^2_m) } K_{a^b,\gamma}(q)  \right)
\]
is a CSP-triple. 
Here $K_{\lambda,\nu}(q)$ is a \hyperref[kostkaFoulkes]{Kostka--Foulkes polynomial}.


\subsection[cspHookSSYT]{Hook SSYT}

The following results are proved by M. Bennett, B. Madill and A. Stokke, \cite{BennettMadillStokke2014}.
Let $\alpha$ be a weak composition with $k$ parts
and let $\lambda = (n-m,1^m)$ be a hook shape with $n$ boxes.
Then 
\[
|\SSYT(\lambda,\alpha)| = \binom{ nz(\alpha)-1}{m}
\]
where $nz(\alpha)$ is the number of non-zero parts in $\alpha$.

The \emph{cyclic symmetry} of a weak composition $\alpha$ is defined as 
the smallest positive integer $p$ such that $\alpha$ is expressible as the concatenation
$(\beta,\beta,\dotsc,\beta)$ where $\beta$ is a composition with $p$ parts. 

\begin{theorem}
Let $\lambda = (n-m,1^m)$, and suppose $\alpha$ is a weak composition of $n$
with $k$ parts and cyclic symmetry $p$.
Furthermore, let $\promotion_k$ denote \hyperref[kpromotion]{$k$-promotion}.
Then 
\[
\left( \SSYT(\lambda,\alpha), \langle \promotion_k^p \rangle, \qbinom{ nz(\alpha)-1}{m}_q 
\right)
\]
is a CSP-triple, and the cyclic group has order $nz(\alpha)-1$.
\end{theorem}


\subsection[cspGeneralSSYT]{SSYT of arbitrary shape}

In \cite{OhPark2019}, Y.-T. Oh and E. Park provide a new CSP on $\SSYT(\lambda,n)$ with a group action of 
order $n$ whenever $\gcd(n,|\lambda|)=1$.
The element $c \coloneqq \cryss_1 \cryss_2 \dotsb \cryss_{n-1} \in \symS_n$ act on $\SSYT(\lambda,n)$
as a product of \hyperref[crystal-operators]{crystal reflection operators}.
Note that this is similar to how the promotion operator is defined,
but in contrast with promotion, the element $c$ has order $n$ for all partitions $\lambda$
when acting on $\SSYT(\lambda,n)$.
The authors then prove that whenever $\gcd(n,|\lambda|)=1$,
\[
(\SSYT(\lambda,n), \langle c \rangle , X_{\lambda,k}(q))
\]
is a CSP-triple, where $X_{\lambda,n}(q) = q^{-n(\lambda)} \schurS_\lambda(1,q,q^2,\dotsc,q^{n-1})$.
They prove that every orbit is \emph{free}, meaning that all orbits under $c$
has size $n$. This means that $X_{\lambda,k}(\xi)=0$ for $\xi$ being a primitive $n$th root of unity,
and that 
\[
X_{\lambda,k}(q) \equiv \frac{|\SSYT(\lambda,n)|}{n} [n]_q \mod (q^n -1 ).
\]

Their result generalizes to skew shapes, see \cite{Alexandersson2023} (where a shorter proof is given as well).


\begin{example*}[Orbits for $\lambda=32$ and $n=4$]
As an example, the following are some of the orbits under $\cryss_{1}\cryss_{2}\cryss_{3}$ on the set $\SSYT(32,4)$.
Since $\gcd(4,5)=1$, the result above applies, and all orbits have size $n=4$.
Each row here is an orbit.
\begin{figure}
\ytableaushort{113,22}
\ytableaushort{224,33}
\ytableaushort{114,24}
\ytableaushort{134,34}
\end{figure}
\begin{figure}
\ytableaushort{112,23}
\ytableaushort{223,34}
\ytableaushort{112,44}
\ytableaushort{133,44}
\end{figure} 
\begin{figure}
\ytableaushort{114,22}
\ytableaushort{123,23}
\ytableaushort{114,34}
\ytableaushort{234,34}
\end{figure}
\begin{figure}
\ytableaushort{112,24}
\ytableaushort{122,33}
\ytableaushort{113,44}
\ytableaushort{233,44}
\end{figure}
\begin{figure}
\ytableaushort{113,23}
\ytableaushort{224,34}
\ytableaushort{114,33}
\ytableaushort{124,24}
\end{figure}
\begin{figure}
\ytableaushort{112,33}
\ytableaushort{223,44}
\ytableaushort{113,34}
\ytableaushort{122,44}
\end{figure}
\end{example*}


In \cite[Cor. 3.4]{OhPark2021}, a solution to the Alexandersson--Amini conjecture \cite{AlexanderssonAmini2018},
is proved in the non-skew case.
\begin{theorem}[Oh--Park 2020]
Let $\lambda$ be a partition so that $n$ divides  $\lambda_i - \lambda_j$ for all $i,j$.
Then there is a cyclic group $\grpc_n$ of order $n$, such that
\[
(\SSYT(\lambda,m),\grpc_n ,q^{-\partitionN(\lambda)}\schurS_{\lambda}(1,q,q^2,\dotsc,q^{m-1}))
\]
is a CSP-triple. 
\end{theorem}
The cyclic group is not described explicitly.
The result by Oh and Park is a semistandard analog of \cite[Thm. 46]{AlexanderssonPfannererRubeyUhlin2020x},
which concerns standard Young tableaux.


The skew case is partially resolved in \cite{LeeOh2022}.
\begin{theorem}[Lee--Oh, 2022]
Let $\lambda/\mu$ be a skew shape such that $\lambda_i - \mu_i$ is a multiple of $m$ for all $i$.
Then there is a cyclic group $\grpc_m$ of order $m$, such that
\[
(\SSYT(\lambda/\mu,km),\grpc_m , \schurS_{\lambda/\mu}(1,q,q^2,\dotsc,q^{km-1}))
\]
is a CSP-triple, for any $k \in \setN$.
\end{theorem}


They also prove a similar theorem for ribbons.
\begin{theorem}[Lee--Oh, 2022]
Let $\lambda/\mu$ be a \emph{ribbon shape}, such that $\lambda_i - \mu_i$ is a multiple of $m$ for all $i$.
Then there is a cyclic group $\grpc_m$ of order $m$, such that
\[
(\SSYT(\lambda/\mu,k),\grpc_m ,\schurS_{\lambda/\mu}(1,q,q^2,\dotsc,q^{k-1}))
\]
is a CSP-triple. 
\end{theorem}

Moreover, in \cite{LeeOh2022} it is shown that the conjecture in \cite{AlexanderssonAmini2018}, 
regarding stretching of general skew shapes, is not true in general.
Taking $\lambda/\mu = 3321/21$, and $m=9$, $k=4$, and looking at $\schurS_{9\lambda/9\mu}(1,q,q^2,q^3)$,
gives an example where CSP cannot hold.


The skew shape case is treated further by N. Kumari in \cite{Kumari2022x},
but the results there has not yet been peer-reviewed.

\todo{Update on status on this paper?}

%
% \begin{theorem}[N. Kumari, 2022, \cite{Kumari2022x}]
% Let $\lambda/\mu$ be any skew shape. Then there is a cyclic group $\grpc_m$ of order $m$,
% such that
% \[
% (\SSYT(\lambda/\mu,km),\grpc_m , \schurS_{\lambda/\mu}(1,q,q^2,\dotsc,q^{km-1}))
% \]
% is a CSP-triple, \emph{if and only if} a certain sign defined in \cite[Eq. (2.10)]{Kumari2022x}
% associated with $\lambda$  and $\mu$ coincide.
% This theorem also has an analog in the \hyperref[cspHookSchur]{skew hook Schur polynomials}.
% \end{theorem}
%
%
% \subsection[cspHookSchur]{Hook Schur polynomials}
%
% In \cite{Kumari2022x}, N. Kumari shows the following result involving
% \hyperref[hookSchur]{skew hook Schur functions}.
%
% \begin{theorem}[N. Kumari, 2022, \cite{Kumari2022x}]
%
% Let $k$ be odd and suppose $\lambda$ and $\mu$ are partitions of length at most $kn$,
% and such that they have the same \emph{sign}, as in \cite[Eq. (2.10)]{Kumari2022x}.
% Then there is a cyclic group $\grpc_k$ of order $k$, such that
% \[
% (\SSYT(\lambda/\mu,kn/km),\grpc_k , \schurHook_{\lambda/\mu}(1,q,q^2,\dotsc,q^{kn-1} / 1,q,q^2,\dotsc,q^{km-1})  )
% \]
% is a CSP-triple.
% \end{theorem}



\section[cspListMiscTableaux]{Other types of fillings}



\subsection[cspPlanePartitions]{Plane partitions}

The following is proved in \cite{ShenWeng2020}, using representation theory, cluster variables
and theory of the Grassmanian.

Let 
\[
P_{a,b,c}(q) \coloneqq \prod_{i=1}^a \prod_{j=1}^b \prod_{k=1}^c \frac{[i+j+k-1]_q}{[i+j+k-2]_q}.
\]
This is a $q$-analogue of the number of plane partitions $P_{a,b,c}$ that
fit in a $a \times b$-rectangle, and entries $\leq c$.

Let $\pi \in P_{a,b,c}$ and let us for convenience augment the plane partition with values 
$\pi_{i,0}=\pi_{0,j}=c$ and $\pi_{a+1,j}=\pi_{i,b+1}=0$.
The \defin{linear toggle} $\tau_{ij}$ produces a new plane partition $\tau_{ij} \pi$ from $\pi$
as follows:
\[
(\tau_{ij} \pi)_{kl} =
\begin{cases}
\pi_{kl} \text{ if } (i,j)\neq (k,l) \\
\max(\pi_{i,j+1},\pi_{i+1,j} )
-
\min(\pi_{i-1,j},\pi_{i,j-1} ) - \pi_{i,j} \text{ if } (i,j) = (k,l).
\end{cases}
\]

Let $\eta$ be the product of linear toggles that hit each square 
exactly once, from bottom to top, from left to right.
That is,
\[
\eta \coloneqq (\tau_{a,1}\tau_{a-1,1} \dotsm \tau_{1,1})\dotsm (\tau_{a,2}\tau_{a-1,2}\dotsm \tau_{1,2})\dotsm
\]
Then
\[
\left(P_{a,b,c}, \langle \eta \rangle,  P_{a,b,c}(q) \right)
\]
is a CSP-triple, where $\eta$ is of order $a+b$.
This extends an earlier result given in \cite{RushShi2012}, where they consider the case $c=1$. 

It turns out that this CSP obtained by Shen--Weng is equivalent
to the aforementioned CSP for promotion of rectangular
SSYT obtained by B. Rhoades; see \cite{Hopkins2020} for the details of this equivalence.

In \cite{Hopkins2020}, S. Hopkins consider plane partitions with additional symmetry,
counting plane partitions in a square under the promotion and transposition-complement operator.
The paper contains several interesting conjectures regarding CSP on plane partitions with additional symmetry.
See the sections \hyperref[cspListPosetsRoot]{Root posets} and \hyperref[cspListPosetsMinuscule]{Minuscule posets} 
for more discussion on these conjectures.



\subsection[cspPlethysmCoefficients]{Plethysm coefficients}

In \cite{Rush2018}, the author give several instances of cyclic sieving related to 
\hyperref[plethysm]{plethysm} coefficients and promotion.
For example, if $\lambda$ is a rectangular partition,
\[
\left( \mathrm{PYTab}(\lambda,\mu^n), \partial, \pm \langle \hallLittlewoodT_{1^n}(\xvec;q) \circ \schurS_\mu,  \schurS_\lambda \rangle \right)
\]
is a CSP-triple.
Here, $\mathrm{PYTab}$ is a set of semistandard Young tableaux with given shape and weight, plus a Yamanouchi condition.
The function $\hallLittlewoodT_{1^n}(\xvec;q)$ is a \hyperref[hallLittlewoodT]{transformed Hall--Littlewood polynomial}.



\subsection[cspMacdonaldE]{Macdonald $\macdonaldE$ fillings}

Consider the specialized \hyperref[macdonaldE]{non-symmetric Macdonald polynomial} 
$\macdonaldE_{\lambda}(\xvec;q;0)$, where $\lambda$ is a partition.
This makes $\macdonaldE_{\lambda}(\xvec;q;0)$ into a symmetric polynomial ---
in fact it is more or less a \hyperref[hallLittlewoodH]{modified Hall--Littlewood polynomial}.
These polynomials in $k$ variables can be realized as a sum over certain non-attacking fillings, $NAF(\lambda,k)$.

In \cite{AlexanderssonUhlin2020}, we prove the following results.

\begin{theorem}[Alexandersson, Uhlin 2020]
Let $\lambda$ be an integer partition, and $n$ a positive integer. 
Then
\[
\left( 
NAF(n\lambda,k), \langle \phi \rangle ,\macdonaldE_{n\lambda}(\underbrace{1,1,\dotsc,1}_k;q;0)
\right)
\]
is a CSP-triple, where $\phi$ act by cyclically shifting blocks of $n$ consecutive columns, 
and redistributing the entries within columns.
Moreover, as $n=1,2,3,\dotsc$, this is a \hyperref[lyndonCSP]{Lyndon-like family}.

Furthermore, this can be refined to the case when we let the content $\nu$ be fixed,
i.e.,
\[
\left( 
NAF(n\lambda,\nu), \langle \phi \rangle  ,[\monomial_\nu]\macdonaldE_{n\lambda}(\xvec;q;0)
\right)
\]
is a CSP-triple for every choice of partition $\nu$.
\end{theorem}
We prove this theorem by using a result by B. Rhoades, \cite{Rhoades2010b},
regarding \hyperref[csp01Matrices]{cyclic sieving on matrices}.

In the earlier work, \cite{Uhlin2019}, several related results are proved.
Note that the case $\lambda = (1)$, gives the 
the CSP with \hyperref[notationPermutations]{major index} 
on words of length $n$ with entries in $[k]$.



\subsection[cspIncreasingTableau]{Increasing tableaux}

In \cite{Pechenik2014}, the following CSP is described.
Let $\lambda\vdash n$ and let $Inc_k(\lambda)$ denote the set of 
tableaux with strictly increasing rows and columns and with maximal value $n-k$,
such that every number in $[n-k]$ is present at least once.
Note that $Inc_0(\lambda) = \SYT(\lambda)$.

For $T \in Inc_k(2\times n)$, let $\maj(T)$ be the sum of all entries $j$ in row $1$,
such that $j+1$ appear in row $2$.
It is then proved that 
\[
f_{n,k}(q) \coloneqq \sum_{T \in Inc_k(2\times n)} q^{\maj(T)} = 
q^{n+\binom{k}{2}} \frac{\qbinom{n-1}{k}_q \qbinom{2n-k}{n-k-1}_q }{[n-k]_q}.
\]
Let $\grpc_{2n-k}$ act by a variant of $k$-promotion on increasing tableaux. Then
\[
\left( Inc_k(2\times n), \grpc_{2n-k}, f_{n,k}(q) \right)
\]
is a CSP-triple. Note that $Inc_k(2\times n)$ is in bijection with $\SYT(n-k,n-k,1^k)$.


The result by Pechenik is generalized in \cite{PresseyStokkeVisentin2016},
and the authors show that 
\[
\left( Inc_k(N-r,1^r), \grpc_{N-k-1}, \qbinom{N-k-1}{r}_q \qbinom{r}{k}_q \right)
\]
is a CSP-triple, where promotion is used.


In \cite{Chen2020}, formulas for two-row skew shapes 
are given, and perhaps these exhibit the CSP.



In \cite{GaetzPechenikStrikerSwanson2021x}, the authors prove 
that packed, increasing tableaux of shape $3 \times k$ and maximal entry 
$3+k$, are equinumerous with $\SYT(2^3,1^{k-2})$.
They prove that the set of increasing tableaux under $K$-promotion,
together with $f^{(2^3,1^{k-2})}(q)$ is a CSP-triple.
Interestingly, the promotion has order $k$ here, even though 
the number of boxes in the SYTs is $k+1$.



\subsection[cspASM]{Alternating sign matrices}

There is a canonical $q$-analog of the set of alternating sign matrices,
\[
ASM(n,q) \coloneqq \prod_{k=0}^{n-1} \frac{(3k+1)_q!}{(n+k)_q!}.
\]

In \cite{StephensCloninger2007}, the authors attribute a CSP to Stanton.
Let $\grpc_4$ act on $ASM(n)$ by a quarter-turn.
Then $(ASM(n),\grpc_4,ASM(n,q))$ is a CSP-triple.
See also \href{https://www.youtube.com/watch?v=q5iW1ghHtTI}{this Youtube video by V. Reiner} discussing
this result.


The authors ask in Question 6.4 if there is a map of order 3, such that $ASM(n,q)$ exhibits CSP.



