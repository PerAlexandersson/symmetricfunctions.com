\section[lah]{Lah symmetric functions}

\begin{polydata}{lah}
  Name   & Lah symmetric functions \\
  Space    & Sym \\
  Basis    & False \\
  Rating   & 1 \\
  Bib      & PetreolleSokal2019x \\
  Year     & 2019 \\
  Symbol   & $\lah_{n,k}(\xvec)$ \\
  Keywords & monomial-positive, e-positive \\
  Category & Other \\
\end{polydata}



The \emph{Lah symmetric functions} were introduced in \cite{PetreolleSokal2019x}.
They are indexed by two integers, $n \geq k \geq 1$.
The combinatorial description is as follows.

We first need the notion of an \emph{increasing ordered tree} on vertex set $[n]$.
Each node has label smaller than its children (so vertex labeled $1$ is a root),
and the children are ordered linearly.
The number of such trees on $n$ nodes are given by $1$, $1$, $3$, $15$, $105,\dotsc$,
see \oeis{A001147}.

An \emph{unordered forest on increasing ordered trees}
is a forest on vertex set $[n]$, where each component is an increasing ordered tree.
The number of such forests on $n$ vertices with $k$ trees is given by \oeis{A001497}:
\begin{array}{lllll}
\toprule
 n \backslash k & \textbf{1} & \textbf{2} & \textbf{3} & \textbf{4} & \textbf{5} \\
\midrule
\textbf{1}& 1 \\
\textbf{2}& 1 & 1 \\
\textbf{3}& 3 & 3 & 1\\
\textbf{4}& 15 & 15 & 6 & 1  \\
\textbf{5}& 105 & 105 & 45 & 10 & 1 \\
\bottomrule
\end{array}
A closed-form formula is given by $\frac{(2n-k-1)!}{2^{n-k}(n-k)!(k-1)!}$,
see \cite[Eq. 7.11]{PetreolleSokal2019x}.

The \defin{Lah symmetric function} is then defined as
\[
 \lah_{n,k}(\xvec) \coloneqq \sum_{F \in \mathrm{Forests}(n,k)} \prod_{j=1}^n
\elementaryE_{c(j)}(\xvec)
\]
where $c(j)$ is the number of children of vertex $j$,
and the sum is taken over all unordered forest on increasing ordered trees,
with $n$ vertices and exactly $k$ trees.
By definition, the $\lah_{n,k}(\xvec)$ are positive
in the \hyperref[elementaryE]{elementary symmetric function basis}.

\emph{Note:} the $\lah_{n,k}(\xvec)$ is denoted $L^{(\infty)+}_{n,k}(X)$
and $L^{(\infty)-}_{n,k}(X) = \omega \lah_{n,k}(\xvec)$ in 
\cite{PetreolleSokal2019x}.

\begin{example*}[Lah symmetric functions for $n=4$]
We have that
\begin{align*}
\lah_{4,1} &= 6 \elementaryE_3 + 8 \elementaryE_{21} + \elementaryE_{3} = 
\monomial_{3} + 11\monomial_{21}+36\monomial_{111}, \\
\lah_{4,2} &= 8 \elementaryE_2 + 7 \elementaryE_{11} = 
7\monomial_{2} + 22\monomial_{11} \\
\lah_{4,3} &= 6 \elementaryE_{1}= 6\monomial_{1} \\
\lah_{4,4} &= 1.
\end{align*}
\end{example*}


