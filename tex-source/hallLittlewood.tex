\metatitle{Hall--Littlewood polynomials}
\metadescription{An introduction to Hall--Littlewood polynomials, their definitions, properties, products, conjectures, and connections to Kostka--Foulkes polynomials.}

\section[hallLittlewoodP]{Hall--Littlewood P}


\begin{polydata}{hallLittlewoodP}
  Name   & Hall--Littlewood P \\
  Space    & Sym \\
  Basis    & True \\
  Rating   & 5 \\
  Bib      & Littlewood1961 \\
  Year     & 1961 \\
  Keywords & cauchy-identity,skew \\
  Symbol   & $\hallLittlewoodP_\lambda(\xvec;t)$ \\
  Category & Schur \\
\end{polydata}

\todo{
Add info in \cite{WheelerZinnJustin2018}
}

\todo{
Non-symmetric : https://arxiv.org/pdf/math/0609512.pdf
}


\todo{
Carbonara: A combinatorial interpretation of the inverse t-Kostka matrix
}

\todo{
Via GT-patterns, Tokuyamas formula:https://www.combinatorics.org/ojs/index.php/eljc/article/view/v22i2p11/pdf

}

The Hall--Littlewood $P$ polynomals interpolates between
the Schur polynomals and the monomial symmetric functions.
They were introduced by \name{Philip Hall} (implicitly) and later by \name{Dudley Ernest Littlewood} in \cite{Littlewood1961}.


\subsection[hallLittleWoodPWeylformula]{Definition}

For a partition $\lambda$, with $\alpha_i$ parts equal to $i$, we define the Hall--Littlewood functions as
\begin{equation*}
\hallLittlewoodP_\lambda(x_1,\dotsc,x_n;t) = 
\frac{1}{\prod_{i\geq 0} [\alpha_i]_t! }\sum_{\sigma \in \symS_n} 
\sigma\left( \xvec^\lambda \frac{ \prod_{i \lt j} 1-tx_j/x_i }{\prod_{i \lt j} 1-x_j/x_i }  \right).
\end{equation*}
Compare with the \hyperref[schurWeylformula]{Weyl formula for Schur polynomials}.
Note that 
\[
\hallLittlewoodP_\lambda(\xvec;1) = \monomial_\lambda(\xvec) \text{ and }
\hallLittlewoodP_\lambda(\xvec;0) = \schurS_\lambda(\xvec).
\]
Furthermore, $\hallLittlewoodP_\lambda(\xvec;-1)$ is equal to the \hyperref[schurP]{Schur-$P$ function}.

The Hall--Littlewood polynomials are orthogonal with respect to the inner product $\langle - , - \rangle_t$
defined via
\[
\langle \powerSum_\lambda , \powerSum_\mu \rangle_t = \delta_{\lambda\mu} z_{\lambda} \prod_{i=1}^{\length(\lambda)}
\frac{1}{1-t^{\lambda_i} },
\]
see \cite[III. §4]{Macdonald1995}. This is a $t$-deformation of the usual Hall inner product,
which is recovered at $t=0$.


\begin{example*}[The Hall--Littlewood polynomial indexed by 221]
We have that $\hallLittlewoodP_{221}(\xvec;t)$ is equal to
\[
\monomial_{221} + (2-t-t^2)\monomial_{2111} + (5-4 t-4 t^2+t^3+t^4+t^5)\monomial_{11111}.
\]
\end{example*}


The (skew) Hall--Littlewood $\hallLittlewoodP$ functions can be computed via a branching rule as follows.
First let the one-variable specialization be $\hallLittlewoodP_{\lambda/\mu}(z;t) = z^{|\lambda/\mu|} \psi_{\lambda/\mu}(t)$,
where
\[
\psi_{\lambda/\mu}(t) = \prod_{\substack{ i \geq 1 \\ m_i(\mu) = m_i(\lambda)+1  }} \left( 1-t^{m_i(\mu)} \right).
\]
Then the \defin{skew Hall--Littlewood}  $\hallLittlewoodP$ polynomial is given by
\[
\hallLittlewoodP_{\lambda/\mu}(x_1,\dotsc,x_n;t) 
= \sum_{ \mu = \nu^{0} \prec \dotsb \prec \nu^n = \lambda}
\prod_{i=1}^n \psi_{\nu^{i}/\nu^{i-1}}(t) x_i^{|\nu^{i}/\nu^{i-1}|}.
\]
Here, we write $\mu \prec \lambda$ if $\lambda/\mu$ is a skew shape.

We have that the Hall--Littlewood $\hallLittlewoodP$ polynomials is the dual basis to the Hall--Littlewood $\hallLittlewoodQ$
basis, $\langle \hallLittlewoodQ_\lambda, \hallLittlewoodP_\lambda \rangle_t = \delta_{\lambda\mu}$.


\subsection[hallLittlewoodCauchy]{Cauchy identity}

We have the Cauchy identity (see \cite[Chapter III, Eq. (4.4)]{Macdonald1995}),
\[
\prod_{i=1}^n \prod_{j=1}^n \left( \frac{1-tx_i y_j}{1-x_iy_j } \right) = 
\sum_{\mu} \hallLittlewoodP_\mu(x_1,\dotsc,x_n;t) \hallLittlewoodQ_\mu(y_1,\dotsc,y_n;t).
\]

\subsection[hallLittlewoodGT]{Gelfand--Tsetlin patterns}


In \cite{FeiginMakhlin2016}, the authors give a combinatorial formula of the form 
\begin{equation}
 \hallLittlewoodP_\lambda(\xvec) = \sum_{G \in \GT(\lambda) } p_G(t) \xvec^{w(G)}
\end{equation}
where $p_G(t)$ is a certain polynomial defined 
via a statistic on the \hyperref[gtpatterns]{Gelfand--Tsetlin pattern} $G$.

Another formula using GT-patterns is given in \cite[Thm. 11]{GuptaRoyPeski2015}.
They generalize \hyperref[gtpatternsTokuyama]{Tokuyama's formula}, \cite{Tokuyama1988} 
and recover Stanley's formula \cite{Stanley1986} for Schurs-Q functions as a special case.


\subsection[hallLittlewoodPieri]{Pieri Rule}

In \cite{KonvalinkaLauve2012}, the authors prove a skew Pieri rule for the Hall--Littlewood $\hallLittlewoodP$
functions. Below is the simpler non-skew version.

\begin{theorem}[See \cite[Thm. 1]{KonvalinkaLauve2012}]
Let $\lambda$ be a partition and $r \geq 0$. Then
\[
\hallLittlewoodP_\mu(\xvec;t) \elementaryE_r(\xvec) = 
\sum_{\substack{ \lambda \\ |\lambda/\mu| = r }} sk_{\lambda/\mu}(t) \hallLittlewoodP_\lambda(\xvec;t)
\]
where 
\[
sk_{\lambda/\mu}(t) = 
t^{\sum_{j} \binom{\lambda'_j - \mu'_j}{2} } \prod_{i\geq i} \qbinom{ \lambda'_j - \mu'_{j+1}}{ m_j(\mu) }_t.
\]
\end{theorem}


\subsection[kostkaFoulkesPolynomials]{Kostka--Foulkes polynomials}

The polynomials $K_{\lambda\mu}(t)$ in the expansion
\[
\schurS_\lambda(\xvec) = \sum_\mu K_{\lambda\mu}(t) \hallLittlewoodP_\mu(\xvec)
\]
are called \defin{Kostka--Foulkes polynomials} and are 
certain $t$-analogs of the \hyperref[schurCombinatorialFormula]{Kostka coefficients}.
It was shown by Lascoux and Schutzenberger \cite{LascouxSchutzenberger78} that
\[
K_{\lambda\mu}(t) = \sum_{T \in \mathrm{SSYT}(\lambda,\mu)} t^{\charge(T)},
\]
where the sum is over semi-standard Young tableaux and \hyperref[kostkaFoulkes]{charge} is a certain statistic on SSYTs.
There is an alternative way to define Kostka--Foulkes polynomials 
using \hyperref[riggedConfigurations]{rigged configurations},
introduced by \name{Anatoly Kirillov} and \name{Nicolai Reshetikhin} \cite{KirillovReshetikhin1988thebethe}.
See the page on \hyperref[kostkaFoulkes]{Kostka--Foulkes polynomials} for more information.



\section[hallLittlewoodProducts]{Products}


There are various different products that lead to $t$-deformations of known structure constants.

\subsection[hallPolynomials]{Hall polynomials}

The \defin{Hall polynomials}, $c^{\nu}_{\lambda\mu}(t) \in \setZ[t]$
deform the \hyperref[littlewoodRichardsonModels]{Littlewood--Richardson coefficients}.
They are defined via the expansion
\[
\hallLittlewoodP_\lambda(\xvec) \hallLittlewoodP_\mu(\xvec) 
= \sum_\nu c^{\nu}_{\lambda\mu}(t) \hallLittlewoodP_\nu(\xvec),
\]
and at $t=0$, the Littlewood--Richardson coefficients for Schur polynomials are recovered.
Note that the polynomials $c^{\nu}_{\lambda\mu}(t)$ may have negative coefficients.

A combinatorial model for $c^{\nu}_{\lambda\mu}(t)$ can be recovered 
from the more general theory in \cite{Yip2012}. 
An earlier proof appear in \cite{Schwer2006}.

In \cite{ZinnJustin2019}, a honeycomb model for the Hall polynomials is described.
This extends the \hyperref[knutsonTaoHoneycomb]{honeycomb model} 
by Knutson--Tao for the \hyperref[littlewoodRichardsonModels]{Littlewood--Richardson coefficients}.


The Hall polynomials have significance in the study of \emph{abelian $p$-groups}.
Let $p$ be a prime number. 
An abelian $p$-group of \emph{type} $\lambda$ is isomorphic to 
\[
G = \setZ/p^{\lambda_1} \oplus  \setZ/p^{\lambda_2} \oplus  \dotsb \oplus \setZ/p^{\lambda_\ell}.
\]
A subgroup $H \subseteq G$ has \defin{cotype} $\nu$ if $G/H$ has type $\nu$.

\begin{theorem}
The number of subgroups of type $\mu$ and cotype $\nu$ in 
a finite abelian $p$-group of type $\lambda$ is given by $c^{\lambda}_{\mu\nu}(p)$.
\end{theorem}

For fixed $\lambda$, the Hall polynomial $c^{\lambda}_{\mu\nu}(p)$ has non-negative
coefficients for all $\mu$ and $\nu$ if and only if no two parts of $\lambda$ 
differ by more than one.

The parameters for which all coefficients are positive is determined exactly in \cite{ButlerHales1993}.
This result later inspired a proof of the following theorem.
\begin{theorem}[See \cite{Maley1996}]
The shifted polynomial $c^{\lambda}_{\mu\nu}(p+1)$ lie in $\setN[p]$.
\end{theorem}


\subsection[hallLittlewoodOtherCoeffs]{Other deformations}

In \cite[Thm. 4]{WheelerZinnJustin2018}, the authors give a 
(signed) combinatorial formula for $\overline{K}^{\nu}_{\lambda\mu}(t)$
in the product 
\[
\hallLittlewoodP_\lambda(\xvec;t) \schurS_\mu(\xvec) 
= \sum_{\nu} \overline{K}^{\nu}_{\lambda\mu}(t) \schurS_\nu(\xvec).
\]


\section[hallLittlewoodConjectures]{Conjectures}


\subsection[hallLittlewoodNegativeT]{MathOverflow conjecture}

Let $\lambda = (\lambda_1,\dotsc,\lambda_\ell)$ 
be a partition satisfying $\lambda_i \geq \lambda_{i+1} +2$ for all $1 \leq i \leq \ell-1$.
In the MathOveflow post \cite{MO190646}, it is conjectured that 
$\hallLittlewoodP_\lambda(\xvec,-t)$ 
expands positively in the Schur polynomial basis under this condition on $\lambda$.
This conjecture should be attributed to Igor Makhlin (monomial positivity) and Richard Stanley (Schur positivity).


\subsection[hallLittlewoodBhattacharya]{Bhattacharya conjecture}

Aritra Bhattacharya (personal communication, FPSAC 2022 Bangalore) 
suggests the following conjecture:
\begin{conjecture}
Consider the expansion 
\[
\hallLittlewoodP_\lambda(\xvec,t) = \sum_\mu B_{\lambda\mu}(t) \hallLittlewoodP_\mu(\xvec,t+1).
\]
Then $B_{\lambda\mu}(t)$ are unimodular polynomials with non-negative integer coefficients.

Similarly, we fix a non-negative integer $m$ and consider
\[
\hallLittlewoodP_\lambda(\xvec,m) = \sum_\mu B^m_{\lambda\mu}(t) \hallLittlewoodP_\mu(\xvec,t+m).
\]
Then $B^m_{\lambda\mu}(t)$ are unimodular polynomials with non-negative integer coefficients.
\end{conjecture}
Note that the first conjecture implies that expanding $\hallLittlewoodP_\lambda(\xvec,t)$ into 
$\hallLittlewoodP_\mu(\xvec,t+m)$ for any $m \geq 1$, gives non-negative coefficients also.

\begin{example*}[Case $n=4$, $m=2$]
The transition matrix $\{B^2_{\lambda\mu}(t)\}$ for $\lambda, \mu \vdash 4$ is 
\[
\begin{pmatrix}
 1 & t & t^2+2 t & t^3+4 t^2+2 t & t^6+10 t^5+38
   t^4+66 t^3+48 t^2+12 t \\
 0 & 1 & t & t^2+3 t & t^5+9 t^4+31 t^3+46 t^2+18 t
   \\
 0 & 0 & 1 & t & t^4+6 t^3+11 t^2+2 t \\
 0 & 0 & 0 & 1 & t^3+7 t^2+17 t \\
 0 & 0 & 0 & 0 & 1 \\
\end{pmatrix}
\]
Rows and columns are indexed by the partitions $4,31,22,211,1111$ in this order.
\end{example*}


This conjecture does not generalize to 
quasisymmetric Hall--Littlewood polynomials, there is a counterexample for $n=5$.
For the definition of quasisymmetric Schur functions, see \cite{HaglundLuotoMasonWilligenburg2011}.



\section[hallLittlewoodQ]{Hall--Littlewood Q}


\begin{polydata}{hallLittlewoodQ}
  Name   & Hall--Littlewood Q \\
  Space    & Sym \\
  Basis    & True \\
  Rating   & 4 \\
  Bib      & Littlewood1961 \\
  Year     & 1961 \\
  Keywords & cauchy-identity,skew \\
  Symbol   & $\hallLittlewoodQ_\lambda(\xvec;t)$ \\
  Category & Schur \\
\end{polydata}


The \defin{Hall--Littlewood $Q$ polynomials} can be defined as 
\begin{equation*}
\hallLittlewoodQ_\lambda(x_1,\dotsc,x_n;t) = 
(1-t)^{\length(\lambda)}[n-\length(\lambda)]_t!
\sum_{\sigma \in \symS_n} 
\sigma\left( \xvec^\lambda \frac{ \prod_{i \lt j} 1-tx_j/x_i }{\prod_{i \lt j} 1-x_j/x_i }  \right).
\end{equation*}
Thus, $\hallLittlewoodQ_\lambda(\xvec;t)$ and $\hallLittlewoodP_\lambda(\xvec;t)$ differ only by a factor in $\setZ[t]$.


\begin{theorem}[See \cite[Eq. (17)]{DesarmenienLeclercThibon1994} ]

Littlewood proved that
\[
\hallLittlewoodQ_\lambda(\xvec;q) = \prod_{i \lt j} (1-qR_{ij})^{-1}\schurS_\lambda[(1-q)X]
\]
where we use \hyperref[schurRaisingFormula]{raising operators} and \hyperref[plethysm]{plethystic notation}.
\end{theorem}

See also \cite{Jing1991} for a similar way of creating Hall--Littlewood functions via a modification of the 
\hyperref[mapsOnSymmetricFunctions]{Bernstein operator}.


\section[hallLittlewoodT]{Transformed Hall--Littlewood}


\begin{polydata}{hallLittlewoodT}
  Name   & Transformed Hall--Littlewood \\
  Space    & Sym \\
  Basis    & True \\
  Rating   & 5 \\
  Bib      & Littlewood1961 \\
  Year     & 1961 \\
  Keywords & schur-positive, skew \\
  Symbol   & $\hallLittlewoodT_\lambda(\xvec;t)$ \\
  Category & Schur \\
\end{polydata}


The \defin{transformed Hall--Littlewood polynomials} are defined via the Kostka--Foulkes polynomials as
\[
\hallLittlewoodT_\mu(\xvec;q) \coloneqq \sum_{\lambda} K_{\lambda\mu}(q) \schurS_\lambda(\xvec).
\]
These are sometimes also denoted $Q'_\mu(\xvec;q)$, some references are \cite{DesarmenienLeclercThibon1994}
and \cite{TudoseZabrocki2003}. 
It is shown that
\[
\hallLittlewoodT_\mu(\xvec;q) = \prod_{i \lt j} \frac{1-R_{ij}}{1-q R_{ij}} \completeH_{\mu}(\xvec)
=
\prod_{i \lt j} (1-q R_{ij})^{-1} \schurS_{\mu}(\xvec)
\]
where we use the same notation as in \hyperref[schurRaisingFormula]{the raising operator formula}
for Schur polynomials. In particular, from this deinition it is 
clear that $\hallLittlewoodT_\mu(\xvec;0) = \schurS_\mu(\xvec)$.

The relationship between the classical and transformed Hall--Littlewood functions is given by
the \hyperref[plethysm]{plethystic relationship}
\[
\hallLittlewoodQ_\mu(\xvec;q) = \hallLittlewoodT_\mu[(1-q)\xvec;q].
\]


In \cite{TudoseZabrocki2003}, a $q$-deformation of \hyperref[schurQ]{Schur's Q-functions} is introduced
in a manner analogously to the construction above.

\begin{example}
The transformed Hall--Littlewood polynomials indexed by partitions of size four are
given as
\begin{align*}
\hallLittlewoodT_{4}(\xvec;q) &= \schurS_{4} \\
\hallLittlewoodT_{31}(\xvec;q) &= q \schurS_{4} + \schurS_{31} \\
\hallLittlewoodT_{22}(\xvec;q) &=q^2 \schurS_{4}+\schurS_{22} + q\schurS_{31} \\
\hallLittlewoodT_{211}(\xvec;q) &=q^3 \schurS_{4} + q \schurS_{22} + (q+q^2)\schurS_{31}+ \schurS_{211} \\
\hallLittlewoodT_{1111}(\xvec;q) &=q^6 \schurS_{4} + (q^2+q^4)\schurS_{22} + (q^3+q^4+q^5)\schurS_{31}+(q+q^2+q^3) \schurS_{211} + \schurS_{1111}
\end{align*}
The coefficients are the \hyperref[kostkaFoulkes]{Kostka--Foulkes polynomials}.
\end{example}

The transformed Hall--Littlewood polynomials are 
special cases of the \hyperref[catalanSymmetric]{Catalan symmetric functions}.
Using these, one can define 
\defin{compositional Hall--Littlewood polynomials}, $\hallLittlewoodT_{\gamma}(\xvec;q)$ where $\gamma$
is a composition.

We also have the following relation with \hyperref[macdonaldE]{non-symmetric Macdonald polynomials}:
\[
\omega\hallLittlewoodT_{\lambda'}(x_1,\dotsc,x_n;q) = \macdonaldE_{\rev(\lambda)}(x_1,\dotsc,x_n;q,0).
\]
A consequence of this relation is the following identity involving LLT polynomials:
\[
q^{\sum_{i\geq 2} \binom{\lambda_i}{2} } \omega \hallLittlewoodH_\lambda(\xvec;t) = \LLT_{\nuvec}(\xvec;q)
\]
where $\nuvec = (1^{\lambda_1}, 1^{\lambda_2},\dotsc,1^{\lambda_\ell})$.


\begin{theorem}[See proof of \cite[Cor. 40]{Alexandersson2019llt}]
The function $\hallLittlewoodT_\mu(\xvec;q+1)$ expands in the 
complete homogeneous symmetric basis with coefficients in $\setN[q]$.

An explicit combinatorial expansion is given in \cite{AlexanderssonSulzgruber2020x}:
\[
 \hallLittlewoodT_{\mu'}(\xvec;q+1) = (q+1)^{-\sum_{i\geq 2} \binom{\mu_i}{2} }
 \sum_{\theta \in \mathcal{O}(P_\mu)} q^{\asc(\theta)} \completeH_{\lambda(\theta)}(\xvec).
\]
Here, $\lambda(\theta)$ is a certain statistic on orientations of a graph,
indexed by a Schroder path $P_\mu$. 
\end{theorem}




\section[hallLittlewoodH]{Modified Hall--Littlewood polynomials}

\begin{polydata}{hallLittlewoodH}
  Name   & Modified Hall--Littlewood polynomials \\
  Space    & Sym \\
  Basis    & True \\
  Rating   & 5 \\
  Bib      & Littlewood1961 \\
  Year     & 1961 \\
  Keywords & schur-positive, skew \\
  Symbol   & $\hallLittlewoodH_\lambda(\xvec;t)$ \\
  Category & Schur \\
\end{polydata}

\todo{  State uniqueness properties }

The \defin{modified Hall--Littlewood polynomials} are related to 
the \hyperref[hallLittlewoodT]{transformed Hall--Littlewood polynomials} via
\[
\hallLittlewoodH_\lambda(\xvec;t) = t^{\partitionN(\lambda)} \hallLittlewoodT_\lambda(\xvec;t^{-1}). 
\]
The modified Hall--Littlewood polynomials are more combinatorial in nature.
They are Schur-positive and are in fact 
the \hyperref[frobeniusCharacteristic]{Frobenius image} of a certain graded vector space.

The modified Hall--Littlewood polynomials are 
specializations of the \hyperref[macdonaldH]{the modified Macdonald polynomials}.


\subsection[modifiedKostkaFoulkesPolynomials]{Modified Kostka--Foulkes polynomials}

The \defin{modified Kostka--Foulkes polynomials} are defined as
\[
\tilde{K}_{\lambda\mu}(t) = t^{\partitionN(\mu)}K_{\lambda\mu}(1/t) = \sum_{T \in \mathrm{SSYT}(\lambda,\mu)} t^{\cocharge(T)},
\]
where the sum is over semi-standard Young tableaux and \hyperref[notationPermutations]{cocharge} is 
defined via the \hyperref[computingKostkaFoulkes]{charge statistic}.

We then have that
\[
\hallLittlewoodH_\mu(\xvec;t) = \sum_{\lambda} \tilde{K}_{\lambda\mu}(t) \schurS_{\lambda}(\xvec). 
\]



\section[hallLittlewoodOtherRootSystems]{In other root systems}

\todo{Add more info}

Hall--Littlewood polynomials can be define in general for 
symmetrizable Kac--Moody algebras, see \url{http://www.kurims.kyoto-u.ac.jp/EMIS/journals/SLC/wpapers/s58viswa.pdf}



\todo{Spin Hall-Littlewood and Yang-Baxter: https://doi.org/10.1063/1.5001687}
