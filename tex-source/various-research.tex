\metatitle{Various problems}
\metadescription{Problems that I find interesting}


\section[personal-minor-problems]{Minor problems}


\subsection[keyRealRootedHstar]{Interlacing roots and key polynomials}

Let $\key_{\lambda,\sigma}$ denote a \hyperref[key]{key polynomial},
and define
\[
P_{\sigma}(\lambda_1,\dotsc,\lambda_n;k) \coloneqq \key_{k\lambda,\sigma}(1^n).
\]
This is a polynomial in $\setQ[\lambda_1,\dotsc,\lambda_n,k]$,
and for fixed $\lambda$, this is an Ehrhart polynomial of 
a certain union of faces in a GT-polytope.
See \cite{AlexanderssonAlhajjar2018} for more background.


We may compute the corresponding $h^*$-polynomial, and get some interesting properties.
\begin{conjecture}[Alexandersson (May 2020)]
Let $\lambda$ and $\sigma$ be fixed and define $H_{\lambda,\sigma}(t) \in \setN[t]$ via
\[
\sum_{k\geq 0} P_{\lambda,\sigma}(k) t^k = \frac{H_{\lambda,\sigma}(t)}{(1-t)^{d+1}}
\]
where $d$ is the degree (in $k$) of $P_{\lambda,\sigma}(k)$.
Then $H_{\lambda,\sigma}(t)$ is a real-rooted polynomial.
Moreover, if $\sigma_1,\sigma_2,\dotsc,\sigma_\ell$ is a saturated 
chain in the \hyperref[bruhatOrder]{Bruhat order}, then 
\[
P_{\lambda,\sigma_1}(k), P_{\lambda,\sigma_2}(k),\dotsc,P_{\lambda,\sigma_\ell}(k)
\]
is a sequence of interlacing polynomials.
\end{conjecture}
I have checked this for some small cases.



\subsection[qInequality]{A $q$-generalization of an inequality}

In \cite{AlexanderssonAmini2018}, we used the following inequality,
where $a,b\geq 0$ and $k \geq j \ge 0$.
\[
\binom{ka+kb}{ka}^j \geq \binom{ja+jb}{ja}^k. 
\]
This is not very hard to prove. I realized that there might be a $q$-analogue of this inequality.

\begin{conjecture}[Alexandersson, 2019]
Suppose $a,b\geq 0$ and $k \geq j \ge 0$. Then
\[
q^{kab \binom{j}{2}} \qbinom{ka+kb}{ka}_q^j -  q^{jab \binom{k}{2}} \qbinom{ja+jb}{ja}_q^k
\]
is a polynomial in $\setN[q]$. 
\end{conjecture}

This was poset on \href{https://mathoverflow.net/questions/330620/q-analogue-of-an-inequality}{MathOverflow},
and there I sketched a proof that shows that this is true whenever $j$ divides $k$.


\subsection[schurPositive]{A Schur-positive expansion?}

Let $\BST(\lambda,\mu)$ be the set of border-strip tableaux of shape $\lambda$
and strip-sizes $\mu$.
Define
\[
T_\lambda(\xvec) \coloneqq \sum_{\mu} |\BST(\lambda,\mu)| \powerSum_\mu(\xvec).
\]
Show that $T_\lambda(\xvec)$ is Schur-positive.
Note the close resemblance with the usual power-sum expansion of Schur polynomials. 

\href{https://mathoverflow.net/questions/353150/schur-positive-expression-involving-border-strip-tableaux}{See this MO-post also}


\subsection[A189912]{On A189912}

The sequence \oeis{A189912} is defined as
\[
a_n \coloneqq \sum_{k=0}^n 
\frac{n!}{(n-k)! (\lfloor k/2 \rfloor!)^2 (\lfloor k/2 \rfloor +1)}.
\]
Let us split this sum into even and odd $k$.
We get
\begin{align*}
&\sum_{k=0}^n
\frac{n!}{(n-2k)! (\lfloor 2k/2 \rfloor!)^2 (\lfloor 2k/2 \rfloor +1)}
+ \\
&\sum_{k=0}^n 
\frac{n!}{(n-(2k+1))! (\lfloor (2k+1)/2 \rfloor!)^2 (\lfloor (2k+1)/2 \rfloor +1)}.
\end{align*}
Simplification and reindexing leads to
\[
\sum_{k=0}^n \left(
\frac{n!}{(n-2k)! (k!)^2 (k+1)}
+
\frac{n!}{(n-2k-1)! (k!)^2 (k+1)}
\right).
\]
Rewriting gives
\[
\sum_{k=0}^n 
\frac{n!}{ (k!) (k+1)!}\left(
\frac{1}{(n-2k)!} + \frac{1}{(n-2k-1)!}
\right)
=
\sum_{k=0}^n 
\frac{n!}{(k!) (k+1)!}\left(
\frac{1}{(n-2k)!} + \frac{n-2k}{(n-2k)!}
\right)
\]
so we end up with
\[
\sum_{k=0}^n (n+1-2k) \frac{n!}{ (k!) (k+1)! (n-2k)!}.
\]
The expression $\frac{n!}{(k!) (k+1)! (n-2k)!}$
is exactly \oeis{A055151}, so this verifies 
the conjecture by Werner Schulte, Oct 23 2016.


\subsection[schubertCharge]{Schubert charge}

The Schubert polynomials generalize the Schur polynomials,
so for Grassmann permutations, there should be a bijection 
to SSYT from pipe dreams. 
What is the notion of (co)charge on pipe dreams?
Does it generalize to arbitrary permutations?



\subsection[intersections]{Combinatorial characterization of interval intersections}

See \url{https://mathoverflow.net/questions/251560/combinatorial-characterization-of-intersecting-intervals-in-the-plane}


\subsection[staircaseSchurRoots]{Roots of staircase Schur polynomials (solved)}

I wrote down the following observation here,
and Vasu Tewari pointed out the straightforward proof presented below.
\begin{proposition}
Let $\delta_n$ be the staircase partition $(n,n-1,\dotsc,1,0)$.
Consider the Schur polynomial indexed by the stretched straircase,
\[
P_{n,k}(t) \coloneqq \schurS_{k \delta_n}(t,1^{n}).
\]
Then
\[
P_{n,k}(t) = (k+1)^{\binom{n}{2}} ([k+1]_t)^{n}.
\]
\end{proposition}
\begin{proof}
By the Vandermonde determinant formula,
\[
\schurS_{k \delta_n}(x_1,\dotsc,x_n) = \prod_{1 \leq i \lt j \leq n+1}
\frac{x_i^{k+1}-x_j^{k+1}}{x_i-x_j}.
\]
This then implies the claim.
\end{proof}

