\metatitle{Non-crossing cyclic sieving}
\metadescription{A list of instances of the cyclic sieving phenomenon on non-crossing objects}

See the \hyperref[cyclicSieving]{cyclic sieving phenomenon} 
page for the definition and related theorems.


\section[cspMatch]{Matchings and crossings}


\subsection[cspPolygon]{Polygon dissections with $k$ edges}

In \cite{ReinerStantonWhite2004}, the authors consider dissections of polygons with $k$ non-crossing edges,
where each edge connects two vertices, $X_{n,k}$. Let $\grpc_{n}$ act by rotation.
This gives the CSP-triple
\[
\left( X_{n,k}, \grpc_{n}, \frac{1}{[n+k]_q}\qbinom{n+k}{k+1}_q \qbinom{n-3}{k}_q \right).
\]
This can also be turned into a CSP on the set $\SYT((k+1)^2 1^{n-k-3})$, but 
it is not explained what the corresponding group action is.


See also \cite{BowmanRegev2014} for some related results.

\subsection[cspMatchings]{Matchings with $k$ crossings}

In \cite{LiangBowling2017x}, matchings on $[2n]$ with $k$ crossings is studied.
Let $X_{n,k}$ be the set of such matchings, and let $\grpc_{2n}$ act on such matchings via rotation.
The case $k=0$ correspond to the \hyperref[cspNonCrossingMatchingsI]{Catalan case}.
For $k=1,2,3$ they prove the following instances of cyclic sieving:

\[
\left( X_{n,1}, \grpc_{2n}, \qbinom{2n}{n-2}_q \right)
,
\quad
\left( X_{n,2}, \grpc_{2n},  \frac{[n+3]_q}{[2]_q}  \qbinom{2n}{n-3}_q \right)
\]
and
\[
\left( X_{n,3}, \grpc_{2n},  \frac{1}{[3]_q} \qbinom{n+5}{2}_q \qbinom{2n}{n-4}_q + \qbinom{2n}{n-3}_q  \right).
\]

\begin{problem}
Find a cyclic sieving phenomena on matchings with more than $3$ crossings.
\end{problem}


\subsection[cspAnnularNonCrossingPermutations]{Annular non-crossing permutations}

Let $\pi$ be a permutation on $[n]$ and draw the \emph{directed} edges $i\to \pi(i)$ on a circle
with vertices $1,\dotsc,n$ on the boundary. If all edges are non-crossing,
and all cycles are oriented clockwise, we say that $\pi$ is a \defin{non-crossing permutation}.
Such permutations are in natural bijection with Catalan objects.

In \cite{Kim2013}, a generalization of non-crossing permutations are studied.
Instead of a circle, one chooses $n$ and $m$ and places \emph{exterior} vertices $1,\dotsc,n$ on a circle
clockwise, and \emph{interior} vertices $n+1,\dotsc,n+m$ counter-clockwise on an interior circle.

A permutation is $(n,m)$-non-crossing if one can draw the edges as before in a non-crossing manner,
such that every cycle is oriented clockwise. A non-crossing permutation is called \emph{connected} 
if it contains at least one cycle with both exterior and interior vertices. 
The paper \cite{Kim2013} only consider non-crossing permutations with at least one connected cycle.
Let $ANC(n,m)$ denote the set of connected $(n,m)$-non-crossing permutations.

Let $C$ act on the annulus by rotation, where the order of $C$ is $\gcd(m,n)$.
Then
\[
\left( ANC(n,m), C, \frac{[2nm]_q}{[n+m]_q}\qbinom{2n-1}{n}_q\qbinom{2m-1}{m}_q \right)
\]
is a CSP-triple.

Several refinements, such as the $q$-Narayana and $q$-Kreweras analogue are also considered 
in \cite{Kim2013} and proved to have CSP.
The type $B$ variants, where all non-crossing partitions are symmetric with respect to rotation by half a turn,
are also covered. For example,
\[
\left( ANC_B(n,m), C, \frac{[2nm]_q}{[n+m]_q}\qbinom{2n}{n}_q\qbinom{2m}{m}_q \right)
\]
is a CSP-triple.

\subsection[cspNonCrossingForests]{Non-crossing trees, forests and graphs}

In \cite{Kluge2012}, we have the following CSP.
Let $X_{n,k}$ be the set of non-crossing forests with $n$ vertices and $k$ components.
These are forests drawn with vertices on a circle and edges inside, such that edges do not cross.

Let
\[
f_{n,k}(q) \coloneqq \frac{1}{[2n-k]_q} \qbinom{n}{k-1}_q \qbinom{3n-2k-1}{n-k}_q,
\]
and let $\grpc_n$ act on $X_{n,k}$ by $2\pi/n$ rotation. Then $(X_{n,k},\grpc_n,f_{n,k}(q))$
is a CSP triple. This is also proved in \cite{Poznanovic2011}.


We can also count graphs according to edges. In \cite{Poznanovic2011},
the following $q$-analogue is considered, which enumerates the number of 
\emph{non-crossing graphs with $n$ vertices and $k$ edges}, $Y_{n,k}$:
\[
g_{n,k}(q) \coloneqq \frac{1}{[n-1]_q} \sum_{j=0}^{n-2} 
q^{j(j+n-k+2)}\qbinom{n-1}{k-j}_q  \qbinom{n-1}{j+1}_q \qbinom{n-2+j}{n-2}_q.
\]
Then $(Y_{n,k},\grpc_n,g_{n,k}(q))$ is a CSP-triple, under rotation.
