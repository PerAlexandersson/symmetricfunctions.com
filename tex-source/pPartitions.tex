\section[pPartition]{P-partitions}

\begin{polydata}{pPartition}
  Name   & P-partitions \\
  Space    & QSym \\
  Basis    & False \\
  Rating   & 1 \\
  Bib      & Gessel1984 \\
  Year     & 1984 \\
  Symbol   & $\pPartition_P(x)$ \\
  Category & QuasiElementary \\
\end{polydata}


\todo{ https://arxiv.org/pdf/2006.10087.pdf McNamara -- when are differences F-positive?}


The elements in the family of $P$-partitions --- or rather 
generating functions of $P$-partitions --- are indexed by labeled posets.
These functions, $\pPartition_{(P,w)}(\xvec)$, 
are \hyperref[quasiSymmetricFunctions]{quasisymmetric}
and positive in the \hyperref[gessel]{fundamental quasisymmetric basis}.
It is a rich family, and includes the skew Schur functions,
the elementary and complete homogeneous symmetric functions 
as well as the fundamental quasisymmetric functions.

For a historical overview of the 
theory of $P$-partitions, see Ira Gessel's survey, \cite{Gessel2016}.


\subsection[pPartitionDefinition]{Definition}

Let $P$ be a poset and $w$ a labeling of $P$. 
A \defin{$(P,w)$-partition} is a map $f:P \to \setP$ such that 
\[
x \lt_P y \implies f(x) \leq f(y)
\]
and 
\[
x \lt_P y \text{ and } w(x)\gt w(y) \implies f(x) \lt f(y).
\]

The $(P,w)$-partition generating function $\pPartition_{(P,w)}(\xvec)$ is then defined as
\[
\pPartition_{(P,w)}(\xvec) = \sum_{f\in \text{$(P,w)$-partition }} \prod_{y \in P} x_{f(y)}.
\]
When the labeling $w$ is \emph{order-preserving}, 
we simply write $\pPartition_{P}(\xvec)$, as this function is then independent of the choice of labeling.

See also \hyperref[enrichedPPartitions]{enriched P-partitions}, where peaks are used instead of descents.
The analog of \hyperref[gessel]{fundamental quasisymmetric functions} are the 
\hyperref[peakQSym]{peak quasisymmetric functions}.


\begin{conjecture}[Stanley, \cite{Stanley1972}]
The function $\pPartition_{(P,w)}(\xvec)$ is symmetric if and only if $(P,w)$
describes the weak and strict inequalities in a skew Young diagram,
required for generating skew semi-standard Young tableaux.

In other words, every symmetric $\pPartition_{(P,w)}(\xvec)$ is a \hyperref[schurSkew]{skew Schur function}.
\end{conjecture}
See \cite[Statement 3.11]{McNamara2006} for possible generalizations of this conjecture.


\subsection[pPartitionInGessel]{Fundamental quasisymmetric expansion}

Let $(P,w)$ be a labeled poset on $n$ elements. The \defin{Jordan--Hölder set} 
of a labeled poset is defined as
\begin{equation*}
\mathcal{L}(P,w)
\coloneqq\{\sigma\in\symS_n:\sigma^{-1}\circ w\text{ is order-preserving}\}.
\end{equation*}

The expansion of $\pPartition_{(P,w)}(\xvec)$ in the \hyperref[gessel]{fundamental quasisymmetric basis} is given by
\[
\pPartition_{(P,w)}(\xvec) = \sum_{\pi \in \mathcal{L}(P,w)} \gessel_{n,\DES(\pi)}(\xvec),
\]
where $\DES(\pi)$ is the \hyperref[notationPermutations]{descent set} of $\pi$.
For a reference of this result, see \cite[Eq. (7.95)]{StanleyEC2}.



\subsection[pPartitionPsiExp]{Quasisymmetric powersum expansion}


Alexandersson and Sulzgruber \cite{AlexanderssonSulzgruber2019} 
show that $\pPartition_P(x)$ is positive 
in the \hyperref[qPsi]{quasisymmetric powersum basis}. 

\begin{theorem}[Alexandersson and Sulzgruber (2019), \cite{AlexanderssonSulzgruber2019}]

Let $P$ be a poset on $n$ elements.
A surjection $f:P \to [k]$ has type $\alpha \vDash n$
if the cardinality of $f^{-1}(j)$ is $\alpha_j$, for $j=1,\dotsc,k$.
Let $\mathcal{O}_{\alpha}^{\ast}(P)$ be the set of 
order-preserving surjections $P \to [k]$
of type $\alpha$, such that each subposet $f^{-1}(j) \subseteq P$
has a unique minimal element.
Then
\[
\pPartition_P(\xvec)
=
% \sum_{\alpha\vDash n}
% \frac{\qPsi_{\alpha}(\xvec)}{z_{\alpha}}\left| \mathcal{L}_{\alpha}^{\ast}(P,w) \right|
% =
\sum_{\alpha\vDash n}
\frac{\qPsi_{\alpha}(\xvec)}{z_{\alpha}}\left| \opsurj_{\alpha}^{\ast}(P) \right|.
\]
\end{theorem}

As a consequence, whenever the symmetric function $f$ can be expressed as 
a non-negative linear combination of $\pPartition_P(x)$,
it is necessarily positive in the \hyperref[powerSum]{powersum basis}.
In hindsight, it is rather remarkable that this property was not discovered earlier.

\todo{Link families.}


\subsection[pPartitionMNrule]{Murnaghan--Nakayama rule}

The result by Alexandersson and Sulzgruber is 
later generalized to all $(P,w)$-partition 
by R. Liu and M. Weselcouch \cite[Thm. 6.9]{LiuWeselcouch2020}, 
where a signed rule is proved. 
This rule coincides with the 
\hyperref[skewSchurMurnaghanNakaygama]{classical Murnaghan--Nakayama rule} 
for (skew) Schur functions whenever $(P,w)$ is a skew diagram.



\begin{question}[See \cite{LiuWeselcouch2020}]
Suppose $P_1$ and $P_2$ are posets such that $\pPartition_{(P_1,w_1)}(x) = \pPartition_{(P_2,w_1)}(x)$,
and $P_1$ is series-parallel. 
Does it follow that $P_2$ is series-parallel?
\end{question}


\subsection[pPartitionLorentzian]{Not Lorentzian}

The $P$-partition generating functions $\pPartition_{(P,w)}(\xvec)$
are in general is not \hyperref[lorentzianPolynomial]{Lorentzian (or normalized Lorentzian)},
not even for naturally labeled $P$.
For example, the poset on three vertices with relations $2\leq 1 \geq 3$ (forming a $V$),
is not normalized Lorentzian.


\section[pPartitionsFlagged]{Flagged $(P,w)$-partitions}

S. Assaf and N. Bergeron \cite{AssafBergeron2019} consider a flagged version of $(P,w)$-partitions.
They show that these are positive in the fundamental slide basis.
\todo{Add ref to fund. slide basis.}
