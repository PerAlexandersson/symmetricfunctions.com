\metatitle{Crystals}
\metadescription{Background and properties of crystals}


A nice \href{http://sporadic.stanford.edu/crystals/index.html}{online introduction is given here.}

\section[crystals]{Crystals on words and semi-standard tableaux}

The notion of \defin{crystals} (in type $A$) refer to a set of \emph{raising}- and \emph{lowering operators} that give rise to a 
certain graph structure on combinatorial objects (words, tableaux) with weights.
If the operators satisfy certain axioms, each connected component in the graph is a singe Schur function.

The operators also provide an explicit $\symS_n$-action on the set of objects, 
thus also providing a representation-theoretic proof of Schur positivity.

A nice introduction to crystals in type $A$ is given in \cite{Shimozono2005}.
See also \cite{BumpSchilling2017} for a thorough introduction to crystals.

In this video, \href{https://www.youtube.com/watch?v=qr7BcV3swMk}{J. Blasiak gives a nice overview}
of crystals in type A, starting at 25:00.

\subsection[crystal-operators]{Operators on words}

We define two operators, $\cryse_i$, $\crysf_i : \setN^k \to \setN^k \cup \{ \emptyset \}$ as follows.
Given a word $w$ consider the subword $w_i$ consisting only of the letters $i$ and $i+1$.
Replace each instance of $i$ with a right bracket and each $i+1$ with a left bracket.
Remove all pairs of matching brackets and consider the remaining unmatched brackets,
which is now consists of $a$ right-brackets and $b$ left-brackets.
These brackets correspond to a subword $w'$ of the form $i^a (i+1)^b$ in $w$.

The operator $\cryse_i$ acting on $w$ turns the leftmost $i+1$ of $w'$ into $i$,
if such an entry exists, otherwise, $\cryse_i(w)=\emptyset$.
Similarly, $\crysf_i$ acting on $w'$ turns the rightmost $i$ in $w'$ into $i+1$,
if such an entry exists, otherwise, $\crysf_i(w)=\emptyset$.
The operator $\cryse_i$ is a \defin{crystal raising operator}
while $\crysf_i$ is a \defin{crystal lowering operator}.


We also define \defin{crystal reflections} $\cryss_i(w)$ by 
replacing the subword $i^a (i+1)^b$ above with $i^b (i+1)^a$.
Such a reflection can be realized by applying a number of $\cryse_i$ or $\crysf_i$ to the word.
The \defin{crystal reflection operators} $\cryss_1,\dotsc,\cryss_{n-1}$ generate an $\symS_n$ action on words.
These crystal reflection operators are also called \defin{Lascoux--Schützenberger involutions}.


\begin{example}
Let us compute $\crysf_2$ and $\cryse_2$ of the word $213313212131$.
First find the subword consisting of the letters $2$ and $3$,
replace with brackets and remove paired brackets.
\[
\begin{matrix}
2&1&3&3&1&3&2&1&2&1&3&1 \\
2& &3&3& &3&2& &2& &3&  \\
]& &[&[& &[&]& &]& &[&  \\
]& &[& & & & & & & &[& 
\end{matrix}
\]
We can now see that
\[
 \crysf_2(213313212131) = \underline{3}13313212131, \qquad \cryse_2(213313212131) = 31\underline{2}313212131.
\]
\end{example}



\subsection[crystal-graphs]{Strings and graphs}

Given a word $w$, consider the sequence $\dotsc, \cryse^2_i(w), \cryse_i(w), w, \crysf_i(w), \crysf^2_i(w),\dotsc$.
This is referred to an \defin{$i$-string}.
For example, 
\[
\emptyset \overset{1}{\rightarrow}  
12112111 \overset{1}{\rightarrow}  
12112112 \overset{1}{\rightarrow}
12112122 \overset{1}{\rightarrow} 
12122122 \overset{1}{\rightarrow} 
22122122 \overset{1}{\rightarrow} \emptyset
\]
is a $1$-string. 


Consider the connected graph consisting of words connected with edges given by $\cryse_i$ and $\crysf_i$,
for all $i$. 
This is referred to as a \defin{crystal}.
Each such crystal contains a unique word $w$ of the form $n^{\lambda_n} \dotsm 2^{\lambda_2} 1^{\lambda_1}$,
where $\lambda_1 \geq \lambda_2 \geq \dots \geq \lambda_n$ is a partition.
Note that  for this particular word, $\cryse_i(w)=\emptyset$ for all $i$. 
This word is called the \defin{highest weight} in the crystal 
and $\lambda$ is the \emph{highest weight vector} of the crystal.


\subsection[crystal-ssyt]{Crystals on semi-standard tableaux}

One can show that all words in a crystal with highest weight vector $\lambda$ are in bijection 
with semi-standard Young tableaux of shape $\lambda$.
In fact, the set of reading words of tableaux of shape $\lambda$ is 
closed under $\cryse_i$ and $\crysf_i$.
This can be proved by realizing that an $i+1$ on 
top of an $i$ in the tableau will always become paired brackets.


In conclusion, if a set of combinatorial objects is closed under the crystal operators,
the sum over the weights of these objects is Schur-positive.
One way to do this is to exhibit a crystal-preserving bijection with words or SSYTs 
--- a bijection that commutes with the raising and lowering operators.



Note that the crystal operators $\cryse_i$, $\crysf_i$ and $\cryss_i$
are also defined on \emph{skew semi-standard Young tableaux}, by acting on the reading word.
The crystal graph on $\SSYT(\lambda/\mu)$ is no longer connected --- the connected components 
correspond to the right hand side in the Schur expansion
\[
\schurS_{\lambda/\mu} = \sum_{\nu} c^{\lambda}_{\mu \nu} \schurS_\nu.
\]
Here, $c^{\lambda}_{\mu \nu}$ are the \hyperref[schurLittlewoodRichardson]{Littlewood--Richardson coefficients},
and the crystal graph on $\SSYT(\lambda/\mu)$ contains  $c^{\lambda}_{\mu \nu}$
connected components isomorphic to the (irreducible) crystal graph on $\SSYT(\nu)$.
In other words, crystals can be used to prove the Littlewood--Richardson rule.

One important property of the crystal operators acting on skew shapes is that they are \defin{coplactic},
meaning that they commute with \hyperref[jeu-de-taquin]{jeu-de-taqin slides}.


\begin{example*}[Crystal graphs on words and SSYT]
In the following figures, the solid lines are the $\crysf_1$ edges,
and the dashed lines are $\crysf_2$.
The two graphs are isomorphic, and they have to be since the highest weight is $\lambda =31$
in both cases.
\begin{figure}
	\svgimg[width=0.3\textwidth]{svg-images/wordCrystal.svg}{Crystal on words.}  
	\svgimg[width=0.3\textwidth]{svg-images/ssytCrystal.svg}{Crystal on SSYT.}  
\end{figure}
\end{example*}



\section[kashiwaraCrystals]{Kashiwara crystals}

The following definition is taken from \cite{GillespieLevinson2019}.

A finite \defin{$\GL_n$ Kashiwara crystal} is a set $B$ together with 
raising and lowering operators $\cryse_i$, $\crysf_i$ on $B$ and \defin{length functions}
$\epsilon_i$, $\phi_i$ from $B$ to $\setZ$, and a \defin{weight function} $w$,
satisfying the following axioms (where $1 \leq i \leq n-1$);
\begin{enumerate}
 \item The operators $\cryse_i$, $\crysf_i$ are partial inverses, and if $Y = \cryse_i(X)$, then 
 \[
		\left(\epsilon_i(Y), \phi_i(Y)   \right) = \left( \epsilon_i(X)-1, \phi_i(X)+1 \right) \quad \text{and} \quad
		w(Y) = w(X)+ \alpha_i,
 \]
 where $\alpha_i = \evec_i - \evec_{i+1}$, the vector with coordinate $i$ set to $1$,
 and coordinate $i+1$ set to $-1$.
 \item For any $i \in [n-1]$ and any $X \in B$,  $\phi_i(X) = \langle w(X), \alpha_i \rangle + \epsilon_i(X)$.
\end{enumerate}
The inner product used here is from the \hyperref[root-system]{root system}.

Moreover, a Kashiwara crystal is a type $A$ \defin{Stembrige crystal} if 

\begin{enumerate}
\item If $|i-j| \gt 1$ and $\cryse_i(X)$, $\cryse_j(X)$ are defined, then their compositions are defined and equal, i.e., 
$\cryse_i \cryse_j(X) = \cryse_j \cryse_i(X)$. Same statement is true for $\crysf_i$, $\crysf_j$.

\item If $\crysf_{i \pm 1}(Y) = X$, then 
\[
	 \left(\epsilon_i(Y) - \epsilon_i(X), \phi_i(Y) - \phi_i(X) \right) \in \{ (0,-1), (1,0) \}.
\]

\item Suppose $|i-j|=1$ and $\crysf_i(Z)=X$, $\crysf_j(Z)=X$ are both defined. Set 
\[
 \Delta \coloneqq \left(\epsilon_i(Z) - \epsilon_i(X), \epsilon_i(Z) - \epsilon_i(Y) \right).
\]
(By previous axioms, $\Delta \in \{ (1,1), (1,0), (0,1), (0,0) \}$.) 
If $\Delta \neq (0,0)$ then $\crysf_i \crysf_j(Z) = \crysf_j \crysf_i(Z) \neq \emptyset$. 
Otherwise, 
$\crysf_i {\crysf}^{\;2}_j \crysf_i(Z) = \crysf_j {\crysf}^{\;2}_i \crysf_j(Z) \neq \emptyset$.

\item We have the dual axiom, where the $\crysf_i$ above are replaced with $\cryse_i$,
and  the $\epsilon_i$ are replaced with $\phi_i$.
\end{enumerate}
These are reworded versions of \name{John Stembridge}'s local axioms given in \cite{Stembridge2003}.


\section[demazure-crystals]{Demazure crystals}

Demazure crystals (in type $A$) are truncated versions of the classical type $A$
crystals. Connected components are now \hyperref[key]{Demazure polynomials}.

Some papers using these crystal structures are
\cite{Wang2020x}, \cite{AssafSchilling2018} and \cite{AssafGonzalez2020x}.



\section[quasi-crystals]{Quasi-crystals}


Quasi-crystals are used to find the fundamental quasisymmetric expansion,
see \url{https://arxiv.org/pdf/2309.14898}, \url{https://arxiv.org/pdf/2309.14887}.




A similar notion to crystals are \hyperref[schurPositivityDualEquivalence]{dual equivalence graphs}.




\section[crystals-type-B]{Crystals for type B}

In \cite{GillespieHawkesPohSchilling2020,AssafOguz2018,AssafOguz2020},
the authors define crystal operators are defined on skew shifted SSYT.
This gives a crystal structure where connected components are 
\hyperref[schurP]{Schur P functions}.



In \cite{GillespieLevinson2019}, a Stembridge-type set of local axioms are used to define a type $B$ crystal structure.
The using raising and lowering operators act on shifted tableaux.
These operators commute with jeu-de-taquin slides (as in type $A$), making them \defin{coplactic}.
This gives a crystal structure where connected components are 
\hyperref[schurQ]{Schur Q functions}.

In the follow-up paper, \name{Maria Gillespie}, \name{Jake Levinston} and \name{Kevin Purbhoo} \cite{GillespieLevinsonPurbhoo2020}
further studies this crystal structure on skew shifted tableaux,
and show how one can act on the reading-word of the tableaux.
Furthermore, they identify the highest weight elements in the crystals, which turn out to
be shifted Littlewood--Richardson tableaux.
With this machinery, they obtain a new proof of the Littlewood--Richardson rule for the 
\hyperref[schurQ]{Schur Q functions}.




\section[crystals-see-also]{See also}

There is a lot of more background and references 
in the \href{http://doc.sagemath.org/html/en/thematic_tutorials/lie/crystals.html}{Sage manual for crystals}.

There are some results on the interaction with crystals and dual RSK here, \cite{Azenhas2006}.
Also, \url{http://www.mat.uc.pt/~oazenhas/azenhasmamede.pdf}


