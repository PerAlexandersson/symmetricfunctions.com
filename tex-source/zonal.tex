\metatitle{Zonal symmetric functions}
\metadescription{An introduction to zonal symmetric functions, their definition as Jack symmetric functions at alpha=2, and their power sum expansion.}
\metakeywords{Zonal symmetric functions,Jack symmetric functions,Gelfand pair,Hyperoctahedral group,Power sum expansion}


\section[zonal]{Zonal symmetric functions}

\begin{polydata}{zonal}
  Name     & Zonal symmetric functions \\
  Space    & Sym \\
  Basis    & True \\
  Rating   & 3 \\
  Bib      & Hua1963 \\
  Year     & 1963 \\
  Symbol   & $\zonal_\lambda(\xvec)$ \\
  Category & Schur \\
\end{polydata}

The zonal symmetric functions were introduced by \name{Hua} in \cite{Hua1963}.
For an introduction, see \cite[Chapter 7]{Macdonald1995}.

The zonal symmetric functions $\zonal_\lambda(\xvec)$ 
associated with the Gelfand pair $(\symS_{2n},H_n)$
are given by the specialization $\alpha=2$
in the \hyperref[jackJ]{Jack symmetric functions}.
That is, $\zonal_\lambda(\xvec) = \jackJ_\lambda(\xvec;2)$.

Here, $H_n$ is the hyperoctahedral group of degree $n$,
given by the centralizer of the simple transpositions $(12),(34),\dotsc,(2n-1,2n)$.
We have that $|H_n| = 2^n n!$.


\subsection[zonalPowerSumExpansion]{Power sum expansion}

In \cite{FeraySniady2011}, the authors present a formula for $\zonal_\lambda(\xvec)$
as a signed sum over $T$-admissible pair-partitions.
\[
\zonal_\lambda(\xvec) = \sum_{(S_1,S_2) \text{ T-admissible}} (-1)^{L(S_1,S_2)} \powerSum_{L(S_1,S_2)}(\xvec).
\]
Here, $T$ is the standard Young tableau of shape $2\lambda$,
with $1,2,\dotsc,2\lambda_1$ in the first row, and so on.
The expression $L(S_1,S_2)$ is the sizes of the components of certain bipartite graphs,
where every vertex has degree $2$, there are $2n$ edges labeled $1,\dotsc,2n$
and $\{i,j\} \in S_c$ if and only if edges $i$ and $j$ are share a vertex
with color $c \in \{1,2\}$.

Pair-partitions can be interpreted as a disjoint product of transpositions,
and thus as elements in $\symS_{2n}$.
Let $S = \{\{1,2\},\{3,4\},\dotsc,\{2n-1,2n\}\}$.
Let $T$ be a standard Young tableau. 
The pair $(S_1,S_2)$ is $T$-admissible if $S \circ S_1$
preserves the rows of $T$, and $S_2$ preserves the columns.
