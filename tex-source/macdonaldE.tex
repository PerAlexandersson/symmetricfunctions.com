

\section[macdonaldE]{Macdonald E polynomials}

\begin{polydata}{macdonaldE}
  Name     & Macdonald E polynomials \\
  Space    & All \\
  Basis    & True \\
  Rating   & 2 \\
  Bib      & Opdam1995\\
  Year     & 1995\\
  Symbol   & $\macdonaldE_\mu(\xvec;q,t)$ \\
  Keywords & fillings \\
  Category & Schur \\
\end{polydata}


The non-symmetric Macdonald polynomials were introduced in \cite{Opdam1995},
\cite{Macdonald1996} and \cite{Cherednik1995nonsymmetric}.
The non-symmetric Macdonald polynomials are closely related with \emph{affine root systems},
and the \emph{double affine Hecke algebra}. Their definition is rather indirect,
and does not give an efficient way of computing these non-symmetric polynomials.

In \cite{HaglundHaimanLoehr2008}, J. Haglund, M. Haiman and N. Loehr 
found an explicit formula for the non-symmetric Macdonald polynomials in type $A$,
as a sum over \emph{non-attacking fillings}.
This model is the basis for the \hyperref[macdonaldEperm]{permuted basement Macdonald polynomials},
and we refer to that page for definitions.

Later, an alternative combinatorial formula using alcove walks was
proved by A. Ram and M. Yip \cite{RamYip2011}, which works for all Lie types.
At $q=t=0$, their model reduces to the \emph{Littelmann path model}.

In \cite{BorodinWheeler2019}, an integrable vertex model is used to 
give an alternative formula for non-symmetric Macdonald polynomials.
\todo{Explain this model!}

See the book \cite{qtCatalanBook} for more background on the type $A$
non-symmetric Macdonald polynomials.
My personal research with non-symmetric Macdonald polynomials and further generalizations 
can be found in 
\cite{Alexandersson2015gbMacdonald,AlexanderssonSawhney2017,AlexanderssonSawhney2019}.

A model using multiline queues is introduced in \cite{CorteelMandelshtamWilliams2018},
and further explored in \cite{CorteelMandelshtamMasonWilliams2019x}.
\todo{Explain this model also!}


\subsection[macdonaldEMonks]{Monk's rule}

W. Baratta \cite{Baratta2009,Baratta2011} give Monk type 
rules (terminology from \hyperref[schubertMonksRule]{Schubert calculus}) for products of the form 
\[
x_j \macdonaldE_\mu(x_1,\dotsc,x_n;q,t), \quad \elementaryE_1(\xvec) \cdot \macdonaldE_\mu(x_1,\dotsc,x_n;q,t), \quad 
 \elementaryE_{r}(\xvec) \cdot\macdonaldE_\mu(x_1,\dotsc,x_n;q,t),
\]
expanded again in the $\{\macdonaldE_\alpha\}$ basis. 
Baratta uses interpolation Macdonald polynomials in the proof.

In \cite{HalversonRam2022x}, the authors prove Monk type formulas for 
\[
x_j \macdonaldE_\mu, \qquad (x_1+\dotsb + x_j) \macdonaldE_\mu, \qquad 
x_j^{-1} \macdonaldE_\mu, \qquad (x_j^{-1}+\dotsb + x^{-1}_j) \macdonaldE_\mu.
\]
Halverson--Ram uses a different method based on intertwiners.

\todo{Write about recursion relations}


\section[macdonaldEspec]{Non-symmetric q-Whittaker polynomials}

\begin{polydata}{macdonaldEspec}
  Name     & Non-symmetric q-Whittaker polynomials \\
  Space    & All \\
  Basis    & True \\
  Rating   & 2 \\
  Bib      & Opdam1995\\
  Year     & 1995\\
  Symbol   & $\macdonaldE_\mu(\xvec;q,0)$ \\
  Keywords & fillings, key-positive \\
  Category & Schur \\
\end{polydata}

The \defin{non-symmetric $q$-Whittaker polynomials} are obtained by letting $t=0$ in $\macdonaldE_\mu(\xvec;q,t)$.
The result can be though of as a non-symmetric analogue of the \hyperref[hallLittlewoodT]{transformed Hall--Littlewood polynomials},
and generalize the \hyperref[qWhittaker]{$q$-Whittaker functions}.

The polynomials $\macdonaldE_\mu(\xvec;q,0)$ expand positively in key polynomials,
and the coefficients are \hyperref[kostkaFoulkes]{Kostka--Foulkes polynomials},
see \cite{AlexanderssonSawhney2017,Assaf2018Kostka,AssafGonzalez2018}.

Models using \emph{quantum alcove walks} and \emph{quantum Lakshmibai--Seshadri path} 
are considered in \cite{LenartNaitoSagakiSchillingShimozono2017}.

