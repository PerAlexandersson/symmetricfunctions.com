\metatitle{Touchard--Riordan polynomials}
\metadescription{Touchard--Riordan polynomials and Catalan objects.}


\section[unitIntervalAO]{A weighted Dyck path enumeration}

In the \hyperref[chromaticQuasisymmetricGesselExpansion]{expansion of chromatic quasisymmetric functions}
in the \hyperref[gessel]{Gessel quasisymmetric fundamental basis}, the following expression shows up,
where $\avec$ is the area sequence of a \hyperref[chromaticQuasisymmetricUnitIntervalGraph]{unit interval graph}.
The number of area sequences of length $n$ is given by the $n$th Catalan number, and
there is a natural correspondence between area sequences and Dyck paths of size $\DP(n)$.

Let $\avec$ be an area sequence of length $n$ and let $X_{\avec}(q) \coloneqq \prod_{i=1}^n [a_i+1]_q$,
where we use the $q$-analogue $[n]_q = 1+q+q^2+\dotsb+q^{n-1}$.
Notice that $X_{\avec}(q)$ defines a weight on the corresponding Dyck path.
This weight is quite remarkable.

By using \cite[Thm. 1]{Flajolet2006}, we immediately get the following continued fraction:
\[
\sum_{n\geq 0} z^n \sum_{\avec \in \DP(n) } X_{\avec}(q) = 
\cfrac{1}{
1 - \cfrac{z[1]_q}{
1 - \cfrac{z^2[2]_q}{
1 - \cfrac{z^3[3]_q}{
\cdots
}}}}
\]

\subsection[aoPerfectMatchings]{Perfect matchings}

Let $\setPM(n)$ be the set of perfect matchings on $[2n]$, placed in a circle.
Let $cr(M)$ denote the number of crossings. Let the \defin{Touchard--Riordan polynomials},
see \oeis{A067311} be defined as
\[
T_n(q) = \sum_{M \in \setPM(n)} q^{cr(M)}.
\]
Then
\[
T_n(q) = \sum_{\avec \in \catalan_n} X_{\avec}(q).
\]
This follows from \cite{Read1979} who shows that $\sum_{n \geq 0} z^n T_n(q)$
is given by the continued fraction expression above.


Another formula is 
\[
 (q-1)^m T_m(q) = \sum_{j=0}^{2m} (-1)^j \binom{2m}{j} q^{\binom{m-j+1}{2}},
\]
see \cite{PrasadRam2022x} for a connection with certain sets of subspaces of $F_q^{2m}$.

\begin{example*}[Table of $T_n(q)$]

Here is a table of the first few Touchard--Riordan polynomials.
See \oeis{A067311} for more information.

\begin{array}{ll}
\toprule
n & T_n(q) \\
\midrule
 1 & 1 \\
 2 & q+2 \\
 3 & q^3+3 q^2+6 q+5 \\
 4 & q^6+4 q^5+10 q^4+20 q^3+28 q^2+28 q+14 \\
 5 & q^{10}+5 q^9+15 q^8+35 q^7+70 q^6+117 q^5+165 q^4+195 q^3+180 q^2+120 q+42 \\
\bottomrule
\end{array}
\end{example*}


A perfect matching $M \in \setPM(n)$ may be described as a list of $n$ tuples, 
which we represent as an array.
\[
M=
\begin{bmatrix}
m_1 & m_3 &  m_5 & \dotsc & m_{2n-1} \\
m_2 & m_4 &  m_6 &\dotsc & m_{2n},
\end{bmatrix}
\]
where $m_{2i-1} \lt m_{2i}$ for $i\in [n]$ and $m_{2i-1} \lt m_{2i+1}$
for all $i \in [n-1]$. That is, the top row is strictly increasing, and the smallest entry in each column
appear at the top.
We write $(i,j) \in M$ whenever $\binom{i}{j}$ appear as a column in $M$.


\subsection[pfaffian]{The pfaffian}

Perfect matchings are closely related to pfaffians.
Let $A$ be a $2n \times 2n$ skew-symmetric matrix.
The \defin{pfaffian} of $A$ is defined as 
\[
\pfaff(A) \coloneqq \sum_{M \in \setPM(n)} (-1)^{cr(M)} \prod_{(i,j) \in M} a_{ij}.
\]
We have that $\pfaff(A)^2 = |A|$. 
See \cite{Stembridge1990} for combinatorial appliactions of the pfaffian.



\subsection[aoMaps]{Projection to Dyck paths}

There is a natural projection $\psi$ from perfect matchings to Dyck paths.

Let $a_i \coloneqq (2i-1) - m_{2i-1}$ be the \defin{area sequence} of the matching $M$.
This defines a map from $\setPM(n)$ to $\DP(n)$.
One can show $\psi$ is surjective and that $\psi: \setPM(n) \to \DP(n)$ 
restricted to \emph{non-crossing} perfect matchings is a bijection.

\begin{example}

The perfect matching
\[
\{ \{1, 6\}, \{2, 4\}, \{3, 11\}, \{5, 7\}, \{8, 9\}, \{10, 12\}\}
\]
is mapped to the area sequence $0 1 2 2 1 1$
which is the Dyck path $0 0 0 1 0 1 1 0 1 0 1 1$.

\todo{Ensure this looks fine}
\begin{figure}
\svgimg[width=0.25\textwidth]{svg-images/perfectMatch.svg}{Perfect matching on 12 vertices.}
\begin{ytableau}
\none \emph{1} & *(lightgray) & *(lightgray) & *(lightgray) & *(lightgray)&   &*(yellow) 10 \\
\none \emph{1} & *(lightgray) & *(lightgray) & *(lightgray) & &*(yellow) 8 \\
\none \emph{2} & *(lightgray) & &  &*(yellow) 5 \\
\none \emph{2} &   	 &   &*(yellow) 3 \\
\none \emph{1} &    &*(yellow) 2 \\
\none \emph{0} & *(yellow) 1 \\
\end{ytableau}
\end{figure}

\end{example}

\subsection[aoOnUnitIntervalGraphs]{Acyclic orientations of unit interval graphs}

Let $AO(\avec)$ be the set of acyclic orientations on the unit interval graph with area sequence $\avec$.
Furthermore, let $\asc(\theta)$ be the number of edges oriented from smaller to larger vertex label.
Then
\begin{theorem}[See \cite[Section 9]{AlexanderssonPanova2018}]
\[
 X_\avec(q) = \sum_{\theta \in AO(\avec)} q^{\asc(\theta)}.
\]
\end{theorem}


\subsection[aoRooks]{Rook placements}

Given an area sequence $\avec$ of length $n$, we can complete the Dyck path to a Ferrers board.
Let $RP(\avec)$ be the set of non-attacking rook placements on this board, using $n$ rooks.
An \defin{inversion} of a rook placement $\pi$ is a square on the board with no rook above it,
and no rook to its left. Let $\inv(\pi)$ denote the number of such inversions.

\begin{theorem}[See e.g. \cite[Section 9]{AlexanderssonPanova2018}]
\[
X_\avec(q) = \sum_{\pi \in RP(\avec)} q^{\inv(\pi)}.
\]
\end{theorem}

The fact that the number or rook placements factors 
nicely was first proved in \cite{GoldmanJoichiWhite975}.
Another $q$-analog of the above identity is due to Garsia and Remmel, \cite{GarsiaRemmel1986},


\subsection[aoMacdonald]{Macdonald polynomials}

In \cite[Prop. 5.9]{AlexanderssonUhlin2020} J. Uhlin and I prove that
\[
[\monomial_{1^{2n}}] \macdonaldE_{(n,n)}(\xvec;q,0) = [n]_q! T_n(q).
\]
This formula appear as a conjecture in \cite{Uhlin2019}.

\todo{Add refinement and relation with modified Macdonald polys.}


\todo{Add the stuff below?}
% 
% \subsection[aoPBWBasis]{A PBW-basis for the positive part of $U_q(\widehat{\mathfrak{sl}}_2)$}
% 
% The following is from P. Terwilliger \cite{Terwilliger2019}.
% 
% In 1993 I. Damiani obtained a PBW basis for $U_q^+$, the positive part of $U_q(\hat{\mathfrak{sl}}_2)$.
% This basis is defined recursively. 
% Rosso gave an injective algebra homomorphism $\phi$ from $U_q^+$ to the \defin{shuffle algebra}.
% 
% Terwilliger shows that the image of the PBW-basis under $\phi$ is given by
% expressions that seem closely related to $X_\avec(q)$ (although the $q$-analogues used are slightly different).
% 
% \[
% q^{-2n}(q-q^{-1})^{2n}x C_n, \qquad 
% q^{-2n}(q-q^{-1})^{2n}  C_n y, \qquad 
% q^{-2n}(q-q^{-1})^{2n-1}  C_n
% \]
% 

\section[aoOther]{Other references}

See \cite{CiglerZeng2011} for connection with Hermite polynomials and $q$-Fibonacci numbers.

