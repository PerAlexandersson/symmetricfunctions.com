\metatitle{Quasisymmetric Schur functions and variants}
\metadescription{An introduction to quasisymmetric Schur functions and their variants, including definitions, properties, and connections to tableaux and 0-Hecke algebras.}  


\todo{In AssafSearles2018 there is also the quasi-Schur functions}

\section[qSchur]{Quasisymmetric Schur functions}

\begin{polydata}{qSchur}
  Name     & Quasisymmetric Schur functions \\
  Space    & QSym \\
  Basis    & True \\
  Rating   & 1 \\
  Bib      & HaglundLuotoMasonWilligenburg2011 \\
  Year     & 2011 \\
  Symbol   & $\qSchur_\alpha(\xvec)$ \\
  Category & Schur \\
\end{polydata}

The \defin{quasisymmetric Schur functions}, $\{\qSchur_\alpha \}$, were introduced 
by J. Haglund, K. Luoto, S. Mason and S. van Willigenburg \cite{HaglundLuotoMasonWilligenburg2011}.
These refine the classical \hyperref[schurS]{Schur functions} in the sense that
\[
 \schurS_\lambda(\xvec) = \sum_{\alpha \sim \lambda} \qSchur_\alpha(\xvec),
\]
where the sum is taken over all compositions $\alpha$ whose parts rearrange to the parts of $\lambda$.
The $\{\qSchur_\alpha \}$ constitute a basis for the 
space of \hyperref[quasiSymmetricFunctions]{quasisymmetric functions}.

A skew version is introduced in \cite{BessenrodtLuotoWilligenburg2011},
and they show that the \defin{skew quasisymmetric Schur functions} 
expand positively into quasisymmetric Schur functions.

The quasisymmetric Schur functions can be described 
as certain characters of the 0-Hecke algebra, see \cite{TewariWilligenburg2015}.


The original definition of the quasisymmetric Schur 
functions is in terms of \hyperref[atom]{Demazure atoms}:
\begin{definition}
Let $\alpha$ be a composition. Then 
\[
\qSchur_\alpha(\xvec) \coloneqq \sum_{\gamma : \mathrm{comp}(\gamma) = \alpha} \atom_{\gamma}(\xvec)
\]
where the sum ranges over all \emph{weak compositions} $\gamma$ which can be obtained by 
inserting $0$s between parts of $\alpha$.
\end{definition}



We can compute the expansion of $\qSchur_\alpha$ in terms of \hyperref[gessel]{fundamental quasisymmetric functions}
by using standard reverse composition tableaux.
A \defin{semistandard reverse composition tableaux}, (SSRCT), of shape $\alpha \vDash n$ 
is a filling of the diagram $\alpha$ \hyperref[prelimPartitions]{(English notation)}
with positive integers such that 
\begin{enumerate}
\item rows are weakly decreasing;
\item the first column is strictly increasing with the row index;
\item for every triple $a$, $b$ $c$ arranged as 

\begin{ytableau}
b & c \\
\none  &\none  \vdots \\
\none  &  a
\end{ytableau}

we have that $a \leq b$ implies that the box $c$ is present and $a\lt c$.
\end{enumerate}
\todo{Fix above figure}

A \defin{standard reverse composition tableaux} (SRCT) uses each integer $1,2,\dotsc,n$
exactly once in the filling.
The \defin{descent set} $\des(T)$ of a SRCT is defined as 
\[
 \des(T) := \{ i : i+1 \text{ appears weakly to the right of } i \}.
\]

\begin{example*}[Example of a SRCT and its descent set]
(From \cite{TewariWilligenburg2015})
The tableau below is an element in $SRCT(3,4,3,2)$ and has descent set 
$\{1,2,5,8,9,11\}$.
\begin{figure}
\begin{ytableau}
5&4&2 \\
8&7&6&3\\
11 & 10 & 1 \\
12 & 9 
\end{ytableau}
\end{figure}
\end{example*}

\begin{theorem}
We have that the fundamental quasisymmetric expansion of $\qSchur_\alpha$ is given by
\[
\qSchur_\alpha(\xvec) = \sum_{T \in \mathrm{SRTC}(\alpha)} \gessel_{\DES(T)}(\xvec).
\]
\end{theorem}

Skew quasisymmetric Schur functions have a similar expansion.


\begin{example*}[Fundamental expansion of $\qSchur_{(2,1,3)}$]
(From \cite[Ex. 2.7]{TewariWilligenburg2015})
We have that $\qSchur_{(2,1,3)} = \gessel_{(2,1,3)} + \gessel_{(2,2,2)}+\gessel_{(1,2,1,2)}$,
corresponding to the three SRTCs
\begin{figure}
\begin{ytableau}
\mathbf{2}&1 \\
\mathbf{3}\\
6& 5 & 4 \\
\end{ytableau}

\begin{ytableau}
\mathbf{2}&1 \\
\mathbf{4}\\
6& 5 & 3 \\
\end{ytableau}

\begin{ytableau}
\mathbf{3}&\mathbf{1} \\
\mathbf{4}\\
6& 5 & 2 \\
\end{ytableau}
\end{figure}
where the descents are the bold entries.
\end{example*}



\section[yqSchur]{Young quasisymmetric Schur functions}


\begin{polydata}{yqSchur}
  Name     & Young quasisymmetric Schur functions \\
  Space    & QSym \\
  Basis    & True \\
  Rating   & 1 \\
  Bib      & LuotoMykytiukWilligenburg2013 \\
  Year     & 2013 \\
  Symbol   & $\yqSchur_\alpha(\xvec)$ \\
  Category & Schur \\
\end{polydata}


The \defin{Young quasisymmetric Schur functions}, introduced in \cite{LuotoMykytiukWilligenburg2013},
are obtained from the quasisymmetric Schur functions via the relation
$\yqSchur_\alpha = \rho(\qSchur_{\rev(\alpha)})$
where $\rho$ is a certain \hyperref[standardQuasiInvolution]{involution on quasisymmetric functions}.



\section[rsqSchur]{Row-strict quasisymmetric Schur functions}


\begin{polydata}{rsqSchur}
  Name     & Row-strict quasisymmetric Schur functions \\
  Space    & QSym \\
  Basis    & True \\
  Rating   & 1 \\
  Bib      & MasonRemmel2013 \\
  Year     & 2013 \\
  Symbol   & $\rsqSchur_\alpha(\xvec)$ \\
  Category & Schur \\
\end{polydata}

The \defin{row-strict quasisymmetric Schur functions} were first 
studied by S. Mason and J. Remmel \cite{MasonRemmel2013} in 2013.
These quasisymmetric functions are indexed by compositions, and refine the Schur functions 
as $\schurS_\lambda(\xvec) = \sum_{\alpha \sim \lambda'} \rsqSchur_\alpha(\xvec)$.

The relation with the quasisymmetric Schur functions is straightforward;
we have that $qSchur_\alpha = \omega(\rsqSchur_\alpha)$, where $\omega$
is a certain \hyperref[standardQuasiInvolution]{involution on quasisymmetric functions}.



A product rule for multiplying a row-strict quasisymmetric Schur function with a 
regular Schur function, and expanding into row-strict Schur functions,
is given in \cite[Thm 13]{Ferreira2011b}. 
See also \cite[Thm. 10, Thm. 11]{MasonNiese2015}.





\section[rsyqSchur]{Row-strict Young quasisymmetric Schur functions}


\begin{polydata}{rsyqSchur}
  Name     & Row-strict Young quasisymmetric Schur functions \\
  Space    & QSym \\
  Basis    & True \\
  Rating   & 1 \\
  Bib      & MasonNiese2015 \\
  Year     & 2013 \\
  Symbol   & $\rsyqSchur_\alpha(\xvec)$ \\
  Category & Schur \\
\end{polydata}

\defin{Row-strict Young quasisymmetric Schur functions} and their skew versions 
are introduced in \cite{MasonNiese2015}.
They can be defined via the relation with Young quasisymmetric Schur functions:
$\rsyqSchur_\alpha = \omega(\yqSchur_\alpha)$, see \cite[Thm. 12]{MasonNiese2015}.

A 0-Hecke algebra module whose charateristic is given by $\rsyqSchur_\alpha$
is provided in \url{https://arxiv.org/pdf/2012.12568.pdf}.
\todo{Fix proper bib}


\section[diSchur]{Dual immaculate Schur functions}

\begin{polydata}{diSchur}
  Name     & Dual immaculate Schur functions \\
  Space    & QSym \\
  Basis    & True \\
  Rating   & 1 \\
  Bib      & BergBergeronSaliolaSerranoZabrocki2014 \\
  Year     & 2014 \\
  Symbol   & $\diSchur_\alpha(\xvec)$ \\
  Category & Schur \\
\end{polydata}

The \defin{dual immaculate Schur functions} (and their skew version), 
$\{\diSchur_\alpha\}$, were introduced in \cite{BergBergeronSaliolaSerranoZabrocki2014}.
They constitute a basis for the space of quasisymmetric functions.
The original definition is that they are dual to the \emph{immaculate Schur functions},
which are non-commutative.

The definition is as follows:
\[
 \diSchur_\alpha = \sum_{T \in IT(\alpha)} \xvec^T
\]
where we sum over all \emph{immaculate tableaux} of shape $\alpha$.
Fillings of the diagram $\alpha$, such that the rows are weakly increasing,
and the first column is \emph{strictly increasing} with respect to the row indexing.
The weight $\xvec^T$ is computed in the same manner as for semistandard Young tableaux.


We have \cite[Prop. 3.37]{BergBergeronSaliolaSerranoZabrocki2014} that
\[
 \diSchur_\alpha = \sum_{\beta \leq_\ell \alpha} L_{\alpha,\beta} \gessel_\beta
\]
where $L_{\alpha,\beta}$ are certain non-negative coefficients, and $\leq_\ell$
denotes lexicographic ordering.

\emph{Note:} The Schur functions are not positive in the dual immaculate basis!


\begin{theorem}
From \cite{BergBergeronSaliolaSerranoZabrocki2014}.
The number of standard immaculate tableaux of shape $\alpha = (\alpha_1,\dotsc,\alpha_\ell)$,
with $|\alpha|=n$, is equal to
\[
  \frac{ (n-1)! }{ \prod_{k=1}^{\ell-1}(n-\alpha_1-\alpha_2-\dotsb - \alpha_k) \prod_{k=1}^\ell (\alpha_k - 1)! }.
\]
\end{theorem}
Alexandersson 2022: Can one find a q-analog of this and some nice instance of cyclic sieving?

\todo{Product rule?}


\section[rsdiSchur]{Row-strict dual immaculate Schur functions}


\begin{polydata}{rsdiSchur}
  Name     & Row-strict dual immaculate Schur functions \\
  Space    & QSym \\
  Basis    & True \\
  Rating   & 1 \\
  Bib      & NieseSundaramWilligenburgVegaWang2022x \\
  Year     & 2022 \\
  Symbol   & $\rsdiSchur_\alpha(\xvec)$ \\
  Category & Schur \\
\end{polydata}

The \defin{row-strict dual immaculate Schur functions}
$\{\rsdiSchur_\alpha\}$, were introduced in \cite{NieseSundaramWilligenburgVegaWang2022x},
and are defined in a very similar manner to the \hyperref[diSchur]{dual immaculate Schur functions}.
They constitute a basis for the space of quasisymmetric functions.
For an introduction, see \href{https://www.youtube.com/watch?v=6O-AuCLZUlE}{Sheila Sundaram's AlCoVE 2022 video}.


The definition is as follows:
\[
 \rsdiSchur_\alpha = \sum_{T \in IT(\alpha)} \xvec^T
\]
where we sum over all \emph{row-strict immaculate tableaux} of shape $\alpha$.
These are fillings of the diagram $\alpha$, such that the rows are strictly increasing,
and the first column is \emph{weakly increasing} with respect to the row indexing.

One can show that the $\rsdiSchur_\alpha$
are positive in the \hyperref[gessel]{fundamental quasisymmetric basis}.


\section[extSchur]{Extended Schur functions}


\begin{polydata}{extSchur}
  Name     & Extended Schur functions \\
  Space    & QSym \\
  Basis    & True \\
  Rating   & 1 \\
  Bib      & AssafSearles2019 \\
  Year     & 2019 \\
  Symbol   & $\extSchur_\alpha(\xvec)$ \\
  Category & Schur \\
\end{polydata}

The \defin{extended Schur functions} were introduced by \name{Sami Assaf} and \name{Dominic Searles} in \cite{AssafSearles2019}.
They are obtained by taking the stable limit of \hyperref[lock]{lock polynomials}.
Alternatively, the extended Schur functions are dual 
to the \emph{shin polynomials} (see \cite{CampbellFeldmanLightShuldinerXu2014}) 
which constitute a basis for the ring of non-commutative symmetric functions.

For a connection with 0-Hecke algebra, see \cite{Searles2020}, where these are shown to be characters of 
certain 0-Hecke modules.
