\metatitle{Stanley symmetric functions}
\metadescription{Definitions of Stanley symmetric functions and various generalizations, for example affine Stanley symmetric functions and affine Schur functions.}


\section[stanleySym]{Stanley symmetric functions}


\begin{polydata}{stanleySym}
  Name   & Stanley symmetric functions \\
  Space    & Sym \\
  Basis    & False \\
  Rating   & 5 \\
  Bib      & Stanley1984 \\
  Year     & 1984 \\
  Keywords & schur-positive, operators, frobenius \\
  Symbol   & $\stanleySym_\omega(\xvec)$ \\
  Category & Schubert \\
\end{polydata}

\todo{
Thomas Lam: Nice overview
https://pdfs.semanticscholar.org/d6fc/62f1d0edaf0a7c79ffceee1211da9e18d2ce.pdf
There are Affine Stanley Sym functions, which expand positively into affine Schur functions.
}

\todo{
Affine in type B/C/D
https://arxiv.org/pdf/1111.3312.pdf 
}

\todo{
Bialgebra connection, type A and other types
https://arxiv.org/pdf/1809.09857.pdf
}

The Stanley symmetric functions were introduced by \name{Richard Stanley} in \cite{Stanley1984}.
They are used for studying the number of reduced words of permutations.
It was later proved that these symmetric 
functions are Schur positive --- and now there are many different proofs of this result.


\subsection[stanleySymReducedWordDefinition]{Reduced word definition}

Let $\omega \in \symS_n$, and let $Red(\omega)$ be the set of reduced words of $\omega$.
Given a reduced word $a$, let $I(a)$ be the set of integer sequences $1 \leq i_1 \leq i_2 \leq \dotsb \leq i_{\ell(\omega)}$ 
such that $a_j \lt a_{j+1}$ implies $i_j \lt i_{j+1}$. 
Then \defin{the Stanley symmetric functions} $\stanleySym_\omega(\xvec)$ are defined as
\begin{equation*}
 \stanleySym_\omega(\xvec) \coloneqq \sum_{a \in Red(\omega)} \sum_{i \in I(a)} x_{i_1}\dotsm x_{i_{\ell(\omega)}}.
\end{equation*}

From this definition, it is fairly easy to obtain the expansion in the \hyperref[gessel]{Gessel fundamental basis}.
\begin{equation*}
 \stanleySym_\omega(\xvec) \coloneqq \sum_{a \in Red(\omega)} \gessel_{n,\DES(i_1i_2 \dotsc i_\ell))}
\end{equation*}
where $a = s_{i_\ell} s_{i_{\ell-1}}\dotsb s_{i_1}$.
Using the \hyperref[gesselSlinkyRule]{slinky rule}, this formula provides a 
way to compute the Schur expansion of these symmetric functions.
\todo{
https://arxiv.org/pdf/1603.09744.pdf
}

\subsection[stanleyDecreasingFactorizationDefinition]{Decreasing factorization definition}

A permutation is decreasing if it admits a reduced word $a_1 \dotsc a_\ell$
with $a_1 \gt \dotsb \gt a_{\ell}$. If such a word exists, it is unique.
A decreasing factorization of a permutation $\omega \in \symS_n$ is an expression of the form 
$\omega = v_1 \dotsm v_r$ where each $v_i$ is a decreasing permutation.

In \cite{Stanley1984}, the following formula was proved:
\begin{equation*}
 \stanleySym_\omega(x) = \sum_{\omega = v_1 \dotsm v_r } x^{\ell(v_1)} \dotsm x^{\ell(v_r)}.
\end{equation*}



\subsection[stanleySymLimitDefinition]{Stable limit definition}

The Stanley symmetric functions can also be defined via the stable 
limit of the \hyperref[schubert]{Schubert polynomials}.
Given a permutation $\omega \in \symS_n$, we let 
\begin{equation*}
1^k \times \omega \coloneqq 1,2,3,\dotsc,k,\omega_1+k,\omega_2+k,\dotsc,\omega_n+k.
\end{equation*}

We have
\begin{equation*}
 \stanleySym_\omega(x) = \lim_{k\to \infty}  \schubert_{1^k \times \omega}(x).
\end{equation*}

The function $\stanleySym_\omega(x)$ is homogeneous with degree given by the number of inverions of $\omega$.
There is also a recursive definition. 



\subsection[stanleySymFrobenius]{Frobenius image of Specht modules}

\todo{
https://arxiv.org/pdf/1304.7870.pdf 
}

There is a generalization of Specht modules, indexed by diagrams $D$, denoted $S^D$, see \cite{BilleyPawlowski2014}.
The Frobenius image of $S^D$ is denoted $\schurS_D$, where we obtain the irreducible Specht modules 
whenever $D$ is a Ferrers diagram. Given a permutation $\omega$, $D(\omega)$ is the associated Rothe diagram:
\begin{equation*}
D(\omega) \coloneqq \{ (i, \omega_j) : 1\leq i \lt j \leq n, \omega_i > \omega_j \}
\end{equation*}
For a general diagram $D$ and a filling $T$ of $D$, let 
\[
y_T = \sum_{\substack{\sigma \in R(D) \\ \tau \in C(D)}} \sign(\tau) \tau \sigma \qquad \in \setC[\symS_n]
\]
where $R(D)$ is the set of permutations in $\symS_n$ permuting entries within rows of $D$, and $C(D)$ is similar but for columns.
The \defin{Specht module} of $D$ is then $\setC[\symS_n]y_T$, and we let $\schurS_D$ be its Frobenius image.

Then $\stanleySym_{\omega}(x) = \schurS_{D(\omega)}(x)$, which implies that $\stanleySym_{\omega}$ is Schur positive.

\begin{problem}[See \cite{LiuThesis}]
Find a combinatorial description of the 
decomposition of $\schurS_D$ into Schur polynomials.
\end{problem}


Liu's conjecture, \cite{LiuThesis} states that the coefficients of the 
Schur expansion of $\schurS_D$ are the same as certain coefficients appearing when studying 
cohomology classes of Schubert varieties defined by $D$.


\todo{
This should be related to Foulkes characters. 
Yes, adding up a bunch of skew shapes (whose Frobenius img is skew Schur), gives the Foulkes characters.
https://arxiv.org/pdf/1102.5159.pdf
}


\subsection[stanleySymSchurExpansion]{Schur expansion}

\todo{https://arxiv.org/pdf/1304.7870.pdf}
Let $EG(\omega)$ be the set of semi-standard tableaux whose column reading word (reading columns left to right, bottom to top)
is a reduced word for $\omega$. Then 
\begin{equation*}
 \stanleySym_\omega = \sum_{T \in EG(\omega)} \schurS_{sh(T)^t}.
\end{equation*}
This is proved using the Edelman--Greene correspondence, see \cite{EdelmanGreene1987}.

There is also a crystal structure, see \cite{MorseSchilling2015} proving Schur positivity.


\section[stanleySymBC]{Type B/C Stanley symmetric functions}


\begin{polydata}{stanleySymBC}
  Name   & Type B/C Stanley symmetric functions \\
  Space    & Sym \\
  Basis    & False \\
  Rating   & 2 \\
  Bib      & FominKirillov1996 \\
  Year     & 1996 \\
  Keywords & reduced-words \\
  Symbol   & $\stanleySym^B_\omega(\xvec)$ \\
  Category & Schubert \\
\end{polydata}

The type $B$ and $C$ Stanley symmetric functions were introduced in \cite{FominKirillov1996},
where the authors also introduce type $B$ and $C$ analogs of Schubert polynomials.

Let $W_C$ be the type $B_n/C_n$ Coxeter group, consisting of signed permutations.
The group $W_C$ is generated by $s_0,\dotsc,s_{n-1}$ subject to 
\[
s_i s_j =s_j s_i \text{ if } |i-j|>1, \quad
s_i s_{i+1} s_i = s_{i+1} s_i s_{i+1} \text{ if } i>1, \text{ and }
s_0 s_{1} s_0 s_1 =  s_1 s_0 s_{1} s_0.
\]
We have the notion of reduced words of generators.
A reduced word $w_1,w_2,\dotsc,w_k$ is unimodal if $w_1 \lt w_2 \lt \dotsb \lt w_j \gt \dotsb \gt w_k$
for some $j$. A \defin{unimodal factorization} of a reduced word $w$ is a factorization 
\[
\omega = (w_1,\dotsc w_{\ell_1})(w_{\ell_1+1},\dotsc w_{\ell_2}) \dotsm (w_{\ell_{m-1}+1},\dotsc w_{\ell_m})
\]
where each factor is unimodal. Factors can be empty. Let $UF(\omega)$ be the set of unimodal factorizations of $\omega$.
Given such a factorization $F$, let $w(F)$
be the vector of the number of elements in each factor, and let $nz(F)$ be the number of non-zero factors.

Given $\omega \in  W_C$, the \defin{type $C$ Stanley symmetric function} is defined as
\[
\stanleySym^C_\omega(\xvec) = \sum_{F \in UF(\omega)} 2^{nz(F)}\xvec^{w(F)}.
\]
and the \defin{type $B$ Stanley symmetric function} is defined as
\[
\stanleySym^B_\omega(\xvec) = 2^{-zero(\omega)} \stanleySym^C_\omega(\xvec)
\]
where $zero(\omega)$ count the number of zeros in a reduced word for $\omega$.


\subsection[stanleySymBCSchurExpansion]{Schur expansion}

The type $B$ and $C$ Stanley symmetric functions are Schur positive,
a crystal proof is given in \cite{HawkesParamonovSchilling2017}, 
where a combinatorial interpretation of the coefficients are given.


T. Lam shows that the type $B$ Stanley symmetric functions are Schur's $P$-function positive (see \cite{Lam1996}),
which is a stronger statement.


\section[stanleySymDouble]{Double Stanley symmetric functions}


\begin{polydata}{stanleySymDouble}
  Name   & Double Stanley symmetric functions \\
  Space    & Sym \\
  Basis    & False \\
  Rating   & 1 \\
  Bib      & Hawkes2020 \\
  Year     & 2018 \\
  Symbol   & $\stanleySym_\omega(\xvec;\yvec)$ \\
  Category & Schubert \\
\end{polydata}



In \cite{Hawkes2020}, an interpolaton of the type $A$ and type $C$ Stanley symmetric functions is introduced.
Hawkes show that his symmetric functions, $\stanleySym_\omega(\xvec;\yvec)$
are Schur-positive for all $\omega \in A_n$, by a version of the Edelman--Greene insertion algorithm.


\section[stanleySymAffine]{Affine Stanley symmetric functions}


\begin{polydata}{stanleySymAffine}
  Name   & Affine Stanley symmetric functions \\
  Space    & Sym \\
  Basis    & False \\
  Rating   & 2 \\
  Bib      & Lam2006 \\
  Year     & 2006 \\
  Keywords & affine \\
  Symbol   & $\stanleySymAffine_{w/v}(\xvec)$ \\
  Category & Schubert \\
\end{polydata}



The affine Stanley symmetric functions were introduced by \name{Thomas Lam} in \cite{Lam2006}.
The family $\stanleySymAffine_\omega(\xvec)$ are indexed by affine permutations $\omega \in \asymS_n$.
Whenevere $\omega \in \symS_n$, they agree with the classical 
Stanley symmetric function $\stanleySym_\omega(\xvec)$.

There is a "skew" version of affine Stanley symmetric functions, where
\[
\stanleySymAffine_{w/v}(\xvec) = \stanleySymAffine_{wv^{-1}}(\xvec).
\]
In other words, the family of affine Stanley symmetric functions contain these skew versions.

The family of affine Stanley symmetric functions contains the \hyperref[schurCylindric]{cylindrical Schur functions}.


\subsection[stanleySymAffineCoproduct]{Coproduct}

It is proved in \cite{Lam2006} that
\[
\stanleySymAffine_w(x_1,y_1,x_2,y_2,\dotsc) = \sum_{uv=w} \stanleySymAffine_u(\xvec)\stanleySymAffine_v(\yvec).
\]

\subsection[stanleySymAffineSchurExpansion]{Schur expansion}

In \cite{MorseSchilling2015}, 
the authors give a crystal proof that the (some?) affine Stanley symmetric functions are Schur positive.

\subsection[stanleySymAffineSchurAffineExpansion]{Affine Schur expansion}


\begin{conjecture}[See \cite{Lam2006}]
For $w\in \asymS_n$, the expansion
in the \hyperref[schurAffine]{affine Schur functions}
\[
\stanleySymAffine_w(\xvec) = \sum_{\lambda} a_{w\lambda} \stanleySymAffine_\lambda(\xvec)
\]
is non-negative.
\end{conjecture}

This result be analogous to the Schur 
expansion of \hyperref[stanleySymSchurExpansion]{Stanley symmetric functions}.



\section[schurAffine]{Affine Schur functions}


\begin{polydata}{schurAffine}
  Name   & Affine Schur functions \\
  Space    & Sym \\
  Basis    & False \\
  Rating   & 2 \\
  Bib      & Lam2006 \\
  Year     & 2006 \\
  Keywords & affine \\
  Symbol   & $\stanleySymAffine_{\lambda}(\xvec)$ \\
  Category & Schur \\
\end{polydata}



The \defin{affine Schur functions} are obtained as a special case of the 
\hyperref[stanleySymAffine]{affine Stanley symmetric functions}.

Let $\omega$ be an affine Grassmann permutation, corresponding to $\lambda = \lambda(\omega)$,
we let $\stanleySymAffine_\lambda(\xvec) \coloneqq \stanleySymAffine_\omega(\xvec).$
These are called the affine Schur functions.
These are dual to the \hyperref[kSchur]{$k$-Schur functions}, see \cite{MorseSchilling2015,LapointeMorse2008},
and are thus sometimes referred to as \defin{dual $k$-Schur functions}.

Let $Par^n$ denote the set of partitions with $\lambda_1 \leq n-1$.
\[
\{ \stanleySymAffine_{\lambda}(\xvec) : \lambda \in Par^n \}
\]
form a basis for the space $\spaceSym^{(n)}$ where
\[
\spaceSym^{(n)} \coloneqq\{ \monomial_{\lambda}(\xvec) : \lambda \in Par^n \}, \qquad 
\spaceSym_{(n)} \coloneqq\{ \completeH_{\lambda}(\xvec) : \lambda \in Par^n \}.
\]


\section[schurAffineSkew]{Skew affine Schur functions}

\begin{polydata}{schurAffineSkew}
  Name   & Skew affine Schur functions \\
  Space    & Sym \\
  Basis    & False \\
  Rating   & 2 \\
  Bib      & Lam2006 \\
  Year     & 2006 \\
  Keywords & affine \\
  Symbol   & $\stanleySymAffine_{\lambda/\mu}(\xvec)$ \\
  Category & Schur \\
\end{polydata}


\todo{
\[
\stanleySymAffine_{\lambda/\mu}(\xvec) = \sum_{a=(a_1,\dotsc,a_r)} 
\langle \completeHomogeneous_{a_1} \dotsm \completeHomogeneous_{a_r} \cdot \mu , \lambda \rangle \xvec^a 
\]
}

Let $\mu \subseteq \lambda$ be two $n$-cores such that there is some $w \in \asymS_n$
such that $u_w \cdot \mu = \lambda$. This means that $\lambda$ can be obtained from $\mu$
under a certain $\asymS_n$-action given by $u_w$.
Then \cite{Lam2006}, a formula of the form
\[
\stanleySymAffine_{\lambda/\mu}(\xvec) = \sum_{T} \xvec^{w(T)}
\]
is presented, where the sum is over certain $k$-tableaux of shape $\lambda/\mu$,
and it is shown that whenever $\lambda \subseteq ((n-m)^m)$ for some $1\leq m\leq n-1$, then
$\stanleySymAffine_{\lambda}(\xvec) = \schurS_\lambda(\xvec)$.

The family of skew affine Schur functions contains the cylindrical Schur functions.

\begin{remark}
Note that some skew affine Schur functions are not obtained as 
(skew) affine Stanley symmetric functions.
\end{remark}


\subsection[schurSkewAffineInschurAffine]{Expansion in affine Schur functions}

See \cite[Theorem 6.9]{LapointeMorse2008}, which expands the skew affine Schur functions 
in terms of affine Schur functions.


\section[stanleySymInvolution]{Involution Stanley symmetric functions}



\begin{polydata}{stanleySymInvolution}
  Name   & Involution Stanley symmetric functions \\
  Space    & Sym \\
  Basis    & False \\
  Rating   & 1 \\
  Bib      & HamakerMarbergPawlowski2017 \\
  Year     & 2017 \\
  Keywords & schur-positive \\
  Symbol   & $\stanleySymInvolution_{y}(\xvec)$ \\
  Category & Schubert \\
\end{polydata}


The involution Stanley symmetric functions were introduced in \cite{HamakerMarbergPawlowski2017},
as the stable limit of the \hyperref[schubertInvolution]{involution Schubert polynomials}.

The involution Stanley symmetric functions are by definition 
a positive linear combination of the usual Stanley symmetric functions,
and are therefore Schur positive.

\begin{theorem}[See \cite[Corollary 4.37]{HamakerMarbergPawlowski2017}]
The involution Stanley symmetric functions are \hyperref[schurP]{$P$-Schur positive}.
\end{theorem}


\section[stanleySymAffineInvolution]{Affine involution Stanley symmetric functions}

\begin{polydata}{stanleySymAffineInvolution}
  Name   & Affine involution Stanley symmetric functions \\
  Space    & Sym \\
  Basis    & False \\
  Rating   & 1 \\
  Bib      & MarbergZhang2018 \\
  Year     & 2018 \\
  Symbol   & $\stanleySymAffineInvolution_{y}(\xvec)$ \\
  Category & Schubert \\
\end{polydata}

There is a unification of the involution Stanley symmetric functions and the affine Stanley symmetric functions,
introduced by \name{Eric Marberg} and \name{Yan Zhang} 2018 in \cite{MarbergZhang2018}. 
It is expected that these are related to geometry of affine analogs of certain symmetric varieties.


\section[stanleySymAffineFPF]{Affine fixed-point free Stanley symmetric functions}

\begin{polydata}{stanleySymAffineFPF}
  Name   & Affine fixed-point free Stanley symmetric functions \\
  Space    & Sym \\
  Basis    & False \\
  Rating   & 1 \\
  Bib      & Zhang2019 \\
  Year     & 2019 \\
  Symbol   & $\stanleySymAffineFPF_{y}(\xvec)$ \\
  Category & Schubert \\
\end{polydata}

In \cite{MarbergZhang2018}, the authors also introduced the \defin{fixed-point free 
Stanley symmetric functions} which are indexed by fixed-point-free permutations.
An affine extension is introduced by \name{Yan Zhang} \cite{Zhang2019}.
These are indexed by self-inverse permutations without fixed-points.

