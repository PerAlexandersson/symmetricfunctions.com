\metatitle{Modified Macdonald polynomials}
\metadescription{Modified Macdonald polynomials and related symmetric functions.}


\section[macdonaldH]{Modified Macdonald polynomials}

\begin{polydata}{macdonaldH}
  Name     & Modified Macdonald polynomials \\
  Space    & Sym \\
  Basis    & True \\
  Rating   & 5 \\
  Bib      & Macdonald1987 \\
  Year     & 1987 \\
  Symbol   & $\macdonaldH_\lambda(\xvec;q,t)$ \\
  Category & Schur \\
\end{polydata}


\todo{
http://www.sciencedirect.com/science/article/pii/0012365X94001345
}

\todo{
This paper defines an interpolation between J and H, and also a "better" formula for the monomial expansion.
https://arxiv.org/pdf/1810.12905.pdf
}



The modified (or transformed) Macdonald polynomials $\macdonaldH_{\lambda}(\xvec;q,t)$ appear as a combinatorial 
version of the \hyperref[macdonaldP]{Macdonald $P$ polynomials}.
The family is indexed by partitions $\lambda$ and they are symmetric in $\xvec$.
It was proved by \name{M. Haiman} in \cite{Haiman2001} that the modified Macdonald polynomials 
are \hyperref[frobeniusCharacteristic]{bigraded Frobenius characteristics} of certain $\symS_n$-modules
that appear in diagonal harmonics.
For details on this story, see \href{https://www.jpswanson.org/notes/n_factorial.pdf}{J. Swanson's notes from the n! conjecture seminar}.

A combinatorial formula for the modified Macdonald polynomials was proved in \cite{HaglundHaimanLoehr2005},
where a close connection with \hyperref[LLT]{LLT polynomials} is made apparent.
The canonical reference on modified Macdonald polynomials is the book by \name{Jim Haglund}, \cite{qtCatalanBook}.
It is a major open problem in algebraic combinatorics to give a combinatorial 
proof (\hyperref[schurPositivity]{by using RSK, dual equivalence or crystals})
that the $\macdonaldH_{\lambda}(\xvec;q,t)$ are Schur-positive.


\subsection[macdonaldHDefinitionViaMacdonaldJ]{Definition}

The \defin{modified Macdonald polynomials} were originally defined via \hyperref[macdonaldJ]{the Macdonald J polynomials},
as
\[
\macdonaldH_{\mu}(\xvec;q,t) = t^{n(\mu)} \macdonaldJ[ X/(1-t^{-1}) ;q,1/t].
\]
Note that we use \hyperref[plethysm]{plethystic notation} here.


\subsection[macdonaldHUniqueDefinition]{Inner product characterization}

The modified Macdonald polynomials $\{\macdonaldH_{\mu} \}_{\mu \vdash n}$
is the unique family of symmetric functions with coefficients in $\setQ(q,t)$,
such that
\begin{itemize}
\item $ \macdonaldH_{\mu}[X(1-q);q,t] \in \setQ(q,t)\{ \schurS_\lambda : \lambda \trianglerighteq \mu \} $  
\item $ \macdonaldH_{\mu}[X(1-t);q,t] \in \setQ(q,t)\{ \schurS_\lambda : \lambda \trianglerighteq \mu' \} $ 
\item $\langle \macdonaldH_{\mu} , \schurS_\mu \rangle = 1$. 
\end{itemize}

Alternatively, they are the the unique family of polynomials that fulfills
\begin{itemize}
\item $ \langle \macdonaldH_{\mu}[X;q,t], \schurS_\lambda[X/(t-1)] \rangle =0 $ whenever $\lambda \triangleright \mu $,
\item $ \langle \macdonaldH_{\mu}[X;q,t], \schurS_\lambda[X/(1-q)] \rangle =0 $ whenever $\lambda \triangleleft \mu $,
\item $\langle \macdonaldH_{\mu} , \schurS_{(n)} \rangle = 1$.
\end{itemize}



\subsection[macdonaldHCombinatorialFormula]{Haglund's combinatorial formula}

A combinatorial formula for the modified Macdonald 
polynomials was obtained in \cite{HaglundHaimanLoehr2005}. 
It is given by
\[
\macdonaldH_{\lambda}(\xvec;q,t) = \sum_{T:\lambda \to \setN} q^{\inv_\lambda(T)} t^{\maj_\lambda(T)} \xvec^T,
\]
where $\inv_\lambda(T)$ and $\maj_\lambda(T)$ are certain combinatorial statistics.

I highly recommend reading \href{https://www.emis.de/journals/SLC/wpapers/s54Ahaglund.pdf}{The genesis of the Macdonald polynomial statistic} by \name{Jim Haglund}.

\todo{Describe the macdonald statistics.}


\begin{example*}[Macdonald polynomials for $|\lambda|=4$.]

The modified Macdonald polynomial $\macdonaldH_{\lambda}(\xvec;q,t)$ 
have the following Schur expansions:

\begin{array}{llll}
\toprule
 \; & \textbf{4} & \textbf{31} & \textbf{22} & \textbf{211} & \textbf{1111} \\
\midrule
 \textbf{4} & 1 & t^3+t^2+t & t^4+t^2 & t^5+t^4+t^3 & t^6 \\
 \textbf{31} & 1 & q+t^2+t & q t+t^2 & q t^2+q t+t^3 & q t^3 \\
 \textbf{22} & 1 & q t+q+t & q^2+t^2 & q^2 t+q t^2+q t & q^2 t^2 \\
 \textbf{211} & 1 & q^2+q+t & q^2+q t & q^3+q^2 t+q t & q^3 t \\
 \textbf{1111} & 1 & q^3+q^2+q & q^4+q^2 & q^5+q^4+q^3 & q^6 \\
 \bottomrule
 \end{array}
 
 For example, $\macdonaldH_{31}(\xvec;q,t)$ is given by
 \[
 \schurS_{4}(\xvec) + (q+t+t^2) \schurS_{31}(\xvec)+(q t+t^2) \schurS_{22}(\xvec)
 +(q t+q t^2+t^3) \schurS_{211}(\xvec)+ q t^3 \schurS_{1111}(\xvec)
 \]
\end{example*}

% Table[
%    Coefficient[
%     ToSchurBasis[MacdonaldHSymmetric[lam, q, t], ss], ss[mu]]
%    , {lam, IntegerPartitions[4]}, {mu, IntegerPartitions[4]}] // TableForm // TeXForm




\subsection[macdonaldHCombinatorialFormula2]{Alternative combinatorial formulas}

A second combinatorial formula is given by R. Kaliszewski and J. Morse in \cite{KaliszewskiMorse2019}.
We have
\[
\macdonaldH_{\lambda}(\xvec;q,t) = \sum_{T} q^{betrayal(T)} t^{\cocharge(T)} \xvec^T,
\]
where the sum is over all tabloids with content $\mu$.


A more recent formula is given in \cite{CorteelMandelshtamMasonWilliams2022},
where certain terms are grouped together:
\[
\macdonaldH_{\lambda'}(\xvec;q,t) = \sum_{\sigma \in \mathrm{ST}(\lambda)}
\xvec^\sigma t^{\inv_\lambda(\sigma)} q^{\maj_\lambda(\sigma)} \mathrm{perm}_t(\sigma,\lambda)
\]
where $\mathrm{perm}_t(\sigma,\lambda)$ is a certain $t$-multinomial coefficient,
and $\mathrm{ST}(\lambda)$ is the set of \emph{sorted tableaux} of shape $\lambda$.
In some sense, certain monomial terms in the original formula have been
grouped together in order to produce multinomial coefficients.


\subsection[macdonaldHProperties]{Properties}

We have the following specializations:
\[
\macdonaldH_{\lambda}(\xvec;0,0) = \completeH_{n}(\xvec), \qquad
\macdonaldH_{\lambda}(\xvec;1,0) = \prod_i \completeH_{\lambda_i}(\xvec), \qquad
\macdonaldH_{\lambda}(\xvec;1,1) = (\elementaryE_1(\xvec))^{|\lambda|}, \qquad
\]

We have the following symmetries:
\[
\macdonaldH_{\lambda}(\xvec;q,t) = \macdonaldH_{\lambda'}(\xvec;t,q) 
= q^{n(\lambda)} t^{n(\lambda')}\omega \macdonaldH_{\lambda}(\xvec;q^{-1},t^{-1})
\]
where $n(\lambda) = \sum_i (n-i)\lambda(i)$ when $\lambda$ has size $n$.
The first identity follows immediately from the fact that the 
modified Macdonald polynomials is a bigraded Frobenius series with this symmetry.
However, a bijective proof is not known. 
Some partial results have been found by M. Gillespie, \cite{Gillespie2016}.

The modified Macdonald polynomials specialize to the \hyperref[hallLittlewoodH]{modified Hall--Littlewood polynomials}
at $q=0$, and these are in turn closely related to the \hyperref[hallLittlewoodT]{transformed Hall--Littlewood polynomials}.



\subsection[macdonaldHSchurPositivity]{Schur expansion}

M. Haiman defines a family of bigraded $\symS_n$-modules, 
which has the property that the \hyperref[frobeniusCharacteristic]{Frobenius image}
of this family gives the modified Macdonald polynomials, see \cite{Haiman2001}.
It follows that the modified Macdonald polynomials are Schur positive, although no combinatorial formula is known.


\begin{problem}[Macdonald, \cite{Macdonald1995}]
Find pairs of statistics $\maj_\mu(\cdot)$ and $\inv_\mu(\cdot)$ 
on standard Young tableaux, such that
\[
\macdonaldH_{\mu}(\xvec;q,t) = \sum_{T \in \SYT(n)} q^{\inv_\mu(T)} t^{\maj_\mu(T)} \schurS_{sh(T)}(\xvec).
\]

This would give a combinatorial proof that the \defin{modified Macdonald--Kostka} polynomials $\tilde{K}_{\lambda\mu}(q,t)$ ---
also called \defin{$qt$-Kostka polynomials} --- are elements in $\setN[q,t]$. 
These polynomials are defined via the relation
\[
\macdonaldH_{\mu}(\xvec;q,t) = \sum_{\lambda} \tilde{K}_{\lambda\mu}(q,t) \schurS_{\lambda}(\xvec).
\]
Note that $\tilde{K}_{\lambda\mu}(1,1) = f^\lambda$, the number of standard Young tableaux.
\end{problem}


\subsection[macdonaldHqtKostka]{$qt$-Kostka polynomials}

The following properties can be found in \cite[p.32]{qtCatalanBook}.

We have that the \defin{modified Kostka polynomials} are 
$\tilde{K}_{\lambda\mu}(q,t) = t^{n(\mu)}K_{\lambda \mu}(q,1/t)$,
where the $K_{\lambda \mu}$ are the \hyperref[macdonaldJqtKostka]{$qt$-Kostka polynomials}.

Recall that the $qt$-Kostka polynomials $K_{\lambda \mu}(q,t)$ have the following properties:
\[
K_{\lambda \mu}(0,t) = K_{\lambda \mu}(t) = \sum_{T \in \SSYT(\lambda,\mu)} t^{\charge(T)}.
\]
and thus $K_{\lambda \mu}(0,1) = K_{\lambda \mu}$.
We also define the \hyperref[kostkaFoulkes]{cocharge} polynomial, 
\[
\tilde{K}_{\lambda \mu}(t) = t^{n(\mu)}K_{\lambda \mu}(1/t) \sum_{T \in \SSYT(\lambda,\mu)} t^{\cocharge(T)}.
\]
Also, $\tilde{K}_{\lambda\mu}(1,1) = K_{\lambda\mu}(1,1) = f^{\lambda}$, the number of
standard Young tableaux of shape $\lambda$.

Then there are the following symmetries:
\begin{align}
K_{\lambda,\mu}(q,t) &=  t^{n(\mu)} q^{n(\mu')}  K_{\lambda',\mu}(1/q,1/t) \\
K_{\lambda,\mu}(q,t) &=   K_{\lambda',\mu'}(t,q)\\
%
\tilde{K}_{\lambda',\mu}(q,t) &=  t^{n(\mu)} q^{n(\mu')}  \tilde{K}_{\lambda',\mu}(1/q,1/t) \\
\tilde{K}_{\lambda,\mu}(q,t) &=   \tilde{K}_{\lambda,\mu'}(t,q)\\
\end{align}



\subsection[macdonaldHstretching]{Stretching symmetry property}

The following property is stated as a conjecture in \cite{LeeOhRhoades2022x},
but it was later revealed that this is a known property, following from
a result by Garsia--Tesler\cite[Thm I.1]{GarsiaTesler1996}.

%
% To be precise, they showed that the (q,t)-Kostka polynomial
% K_{lambda,mu}(q,t) can be realized as a certain symmetric
% function f_lambda(X;q,t) specialized at B_mu(q,t), where B_mu(q,t)
% is the sum of monomials q^(i-1) t^(j-1) for (i,j)  a box in the diagram mu.
% Their theorem implies that if we have B_mu(q^a,q^b)=B_lambda(q^a,q^b), then we have H_mu(X;q^a,q^b)=H_lambda(X;q^a,q^b), where H_mu(X;q,t) is the modified Macdonald polynomials.


\begin{proposition}
Let $k$ be a positive integer. Then
\[
\macdonaldH_{k \mu}(\xvec;q,q^k) = \macdonaldH_{k \mu'}(\xvec;q,q^k).
\]
That is, we have a type of symmetry under conjugation and stretching.
\end{proposition}



\subsection[macdonaldHSkew]{General diagrams}

In \cite{Bandlow2007}, J. Bandlow studies more general shapes, in particular skew shapes.


\subsection[macdonaldHCumulants]{Macdonald cumulants}

A generalization of the $qt$-Kostka coefficients are defined in \cite{Dolega2019},
which are related to a certain Schur-positivity problem, 
generalizing the Schur positivity of the modified Macdonald polynomials.


\subsection[macdonaldHNonCommutative]{Non-commutative Macdonald polynomials}

In \cite{BergeronZabrocki2005}, the authors introduce a non-commutative family of symmetric functions
with properties similar to those of the modified Macdonald polynomials.

\todo{cite preprint properly.}
There is also an analog in defined in \href{https://www.newton.ac.uk/files/preprints/ni01035.pdf}{this preprint}.

