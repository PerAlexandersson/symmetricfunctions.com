\section[schurLoop]{Loop Schur polynomials}

\begin{polydata}{schurLoop}
  Name   & Loop Schur polynomials \\
  Space    & LSym \\
  Basis    & False \\
  Rating   & 4 \\
  Bib      & LamPylyavskyy2012 \\
  Year     & 2012 \\
  Symbol   & $\schurS^{(r)}_{\lambda/\mu}(\xvec)$ \\
  Keywords & murnaghan-nakayama, fillings, jacobi-trudi\\
  Category & Schur \\
\end{polydata}


The loop Schur functions were introduced in \cite{Lam2010,LamPylyavskyy2012}.
To define them, we first need to define loop symmetric functions.

\subsection[loopSymmetricFunctions]{Loop symmetric functions}

The following definitions are from \cite{LamPylyavskyy2013}.
We fix $n$ and $m$ and consider the array of variables
\[
x_i^{(r)} \quad 1\leq i \leq m \qquad r \in \setZ/n\setZ.
\]

The \defin{loop elementary symmetric functions}
and the \defin{loop homogeneous symmetric functions}
are defined as
\[
\elementaryE_{k}^{(r)}(\xvec) \coloneqq \sum_{1\leq i_1 \lt i_2 \lt \dotsc \lt i_k \leq m}
x_{i_1}^{(r)} x_{i_2}^{(r+1)} \dotsb x_{i_r}^{(r+k-1)}
\]
\[
\completeH_{k}^{(r)}(\xvec) \coloneqq \sum_{1\leq i_1 \leq i_2 \leq \dotsc \leq i_k \leq m}
x_{i_1}^{(r)} x_{i_2}^{(r-1)} \dotsb x_{i_r}^{(r-k+1)}
\]
where the upper indices is taken modulo $n$.
The superscript is commonly referred to as the \defin{color}.
Note that for $n=1$, we recover the usual symmetric functions.

The loop elementary symmetric functions generate a ring ---
this is the ring of loop symmetric functions, $LSym_n$.

\emph{Remark:} There are two versions defined in \cite{LamPylyavskyy2012},
"whirl" and "curl". This is the "whirl" notation.
The difference is in how the color indices progress.


\subsection[schurLoopDefinition]{Definition}

Given a square $(i,j)$, (row, column) in a (skew) Young diagram,
define its content as $c(i,j)\coloneqq i-j$ mod $n$.
If $T \in \SSYT(\lambda/\mu)$, let 
\[
wr_r(T) \coloneqq \prod_{(i,j)\in \lambda/\mu} x_{T(i,j)}^{(r+c(i,j))}.
\]
The \defin{loop Schur polynomial} is then defined as
\[
\schurS^{(r)}_{\lambda/\mu}(x_1,\dotsc,x_m) \coloneqq 
\sum_{T \in \SSYT(\lambda/\mu)} wt_r(T).
\]

\begin{example}
For $\lambda=(3,2)$, $n=3$ and $m=2$, we have 
that
\[
\schurS^{(2)}_{(3,2)}(x_1,x_2) = x_1^{(2)}x_1^{(1)}x_1^{(3)}x_2^{(3)}x_2^{(2)} + 
x_1^{(2)}x_1^{(1)}x_2^{(3)}x_2^{(3)}x_2^{(2)}.
\]
The two tableaux are
\begin{figure}
\begin{ytableau}
1^{(2)} & 1^{(1)} & 1^{(3)} \\
2^{(3)} & 2^{(2)}
\end{ytableau}
\begin{ytableau}
1^{(2)} & 1^{(1)} & 2^{(3)} \\
2^{(3)} & 2^{(2)}
\end{ytableau}
\end{figure}
where the superscript indicate the color.
\end{example}

\begin{problem}
The loop Schur functions do not span the ring of loop symmetric functions.
Also, they are not linearly independent! 
It is an open problem, see \cite{Lam2010}, to find an appropriate Schur-like basis (or a monomial-like basis) for $LSym_n$.
\end{problem}


\subsection[schurLoopWeylFormula]{Alternant quotient identity}

As with the Schur polynomials, we can express loop Schur functions
as a ratio of alternants. 
Define the following $m\times m$-matrix:
\[
(A_\alpha^{(r)})_{ij} \coloneqq t_{j,m} (x_m^{(r+m-1)} x_m^{(r+m-2)}
\dotsb x_m^{(r+m-\alpha_i)})
\]
where $t_{j,m}$ is a certain transposition under a birational $S_m$-action
on $\setQ( x_i^{(j)})$, see \cite{Frieden2020} for details.

Then
\[
\schurS^{(r)}_{\lambda} = \frac{\left| A_{\lambda+\delta}^{(r)}  \right|}{ \left| A_{\delta}^{(r)}  \right| }
\]
where $\delta = (m-1,m-2,\dotsc,1,0)$.

This also appears in \cite{Lam2010}, but as pointed out in 
\cite{Frieden2020}, there is a typo in Lam's notes.
The idea of the proof is attributed to Greg Anderson.


\subsection[schurLoopJacobiTrudi]{Jacobi--Trudi identity}

In \cite{LamPylyavskyy2012}, it is proved that
\[
\schurS^{(r)}_{\lambda/\mu} = 
\det\left( \completeH_{\lambda_i - \mu_j - i+j}^{(r-\mu_j+j-1)} \right)
=
\det\left( \elementaryE_{\lambda'_i - \mu'_j - i+j}^{(r-\mu'_j-j+1)} \right).
\]
These identities generalize the classical 
\hyperref[schurJacobiTrudi]{Jacobi-Trudi identities}
for Schur polynomials.


\subsection[schurLoopMurnaghanNakayama]{Murnaghan--Nakayama rule}

A Murnaghan--Nakayama rule was proved in \cite{Ross2013}.
An alternative proof can be found in \cite{Frieden2020}.

Define the loop power-sum symmetric functions as
\[
\powerSum_k(x_1,\dotsc,x_m) \coloneqq \sum_{i=1}^m 
\left( x_i^{(1)}x_i^{(2)}\dotsb x_i^{(n)} \right)^k.
\]

\begin{theorem}[See \cite{Ross2013}]
Let $\lambda$ be a partition with at most $m$ parts and $k\geq 1$.
Then
\[
\powerSum_k(x_1,\dotsc,x_m) \schurS^{(r)}_{\lambda}(x_1,\dotsc,x_m)
=
\sum (-1)^{\mathrm{ht}(\mu/\lambda)}
\schurS^{(r)}_{\lambda}(x_1,\dotsc,x_m)
\]
where the sum is over all partitions $\mu$ with at most $n$
parts, such that $\mu/\lambda$ is a ribbon of size $kn$.
\end{theorem}
For notation and terminology, see 
\hyperref[schurMurnaghanNakaygama]{the Murnaghan--Nakayama rule for Schur functions}.

