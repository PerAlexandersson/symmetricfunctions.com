\section[gessel]{Gessel quasisymmetric functions}

\begin{polydata}{gessel}
  Name   & Gessel quasisymmetric functions \\
  Space    & QSym \\
  Basis    & True \\
  Rating   & 4 \\
  Bib      & Gessel1984 \\
  Year     & 1984 \\
  Symbol   & $\gessel_\alpha(x)$ \\
  Category & QuasiElementary \\
\end{polydata}


The \defin{Gessel quasisymmetric functions} $\gessel_\alpha(x)$, 
also called  the \defin{fundamental quasisymmetric functions},
were introduced by I. Gessel in \cite{Gessel1984}.
They are indexed by integer compositions and constitute 
a basis for the space of \hyperref[quasiSymmetricFunctions]{quasisymmetric functions}. 


\subsection[gesselDefinitionCompositions]{Definition (compositions)}

For $\alpha \vDash n$, we define 
\begin{equation*}
 \gessel_\alpha(\xvec) = \sum_{\beta \leq \alpha} M_\beta(\xvec) =
 \sum_{\beta \leq \alpha} \sum_{i_1 \lt \dotsb \lt i_n } x_{i_1}^{\beta_1} \dotsm x_{i_n}^{\beta_n}
\end{equation*}
where $\leq$ is refinement partial order.
There is a second way to index the Gessel quasisymmetric functions, by using subsets.

\subsection[gesselDefinitionSubsets]{Definition (subsets)}

Let $S \subseteq [n-1]$. Then we define $\gessel_{n,S}(\xvec)$ as
\begin{equation*}
 \gessel_{n,S}(\xvec) = \sum_{\substack{1 \leq b_1 \leq \dotsb \leq b_n \\  i \in S \Rightarrow b_i \lt b_{i+1} }}
 x_{b_1} \dotsm x_{b_n}
\end{equation*}

To translate between the two definitions, note that $\gessel_\alpha(\xvec) = \gessel_{n,S_\alpha}(\xvec)$
where $S_\alpha = \{\alpha_1, \alpha_1+\alpha_2,\dotsc, \alpha_1+\dotsb + \alpha_{n-1}\}$.
Conversely, $\gessel_{n,S}(\xvec) = \gessel_{comp(S)}$, where
$comp(S) = (s_1,s_2-s_1,s_3-s_2,\dotsc,s_{\ell} - s_{\ell-1},n-s_{\ell})$.

In terms of the \hyperref[qmonom]{monomial quasisymmetric functions},
we have $\gessel_{\alpha}(\xvec) = \sum_{\beta \preceq \alpha} \qmonom_\beta(\xvec)$,
where $\preceq$ denotes \emph{composition refinement}.

\section[gesselProperties]{Properties}

See \hyperref[standardQuasiInvolution]{the page on quasisymmetric functions} on how 
the \hyperref[standardInvolution]{standard involution on symmetric functions} 
that sends $\schurS_\lambda$ to $\schurS_{\lambda'}$ 
can be extended to quasisymmetric functions.


\subsection[gesselProduct]{Product rule}

The product $\gessel_{m,S} \gessel_{n,T}$ has a positive rule as follows.
 Choose $\sigma \in \symS_m$ and $\tau \in \symS_n$ 
such that the descent sets of the permutations satisfy
$S = \DES(\sigma)$ and $T = \DES(\tau)$.
A \defin{shuffle} of $\sigma$ and $\tau$ is a permutation in $\symS_{m+n}$
such that entries $1,\dotsc,m$ appear in the same relative order as in $\sigma$,
and the entries $m+1,\dotsc,m+n$ appear in the same relative order as in $\tau$, when $m$ is subtracted from these entries.
For example, the shuffles of the permutations $21$ and $12$ are given by
\[
2134, 2314, 3214, 2341, 3241, 3421.
\]
There are of course $\binom{m+n}{m}$ shuffles in total.
We then have that
\[
\gessel_{m,S}(\xvec) \gessel_{n,T}(\xvec) = 
\sum_{\pi \text{ shuffle of } \sigma,\tau } F_{m+n,\DES(\pi)}(\xvec).
\]


\subsection[gesselSpecializations]{Specializations}

We have that for $\length(S)\lt m$,
\begin{equation*}
 \gessel_{n,S}(1^m) = \binom{n+m-\length(S)-1}{n}
\end{equation*}
and for $\length(S)\lt m$,
\begin{equation*}
 \gessel_{n,S}(1,q,q^2,\dotsc,q^m) = q^{m \length(S)+m-|S|-1}\qbinom{n+m-1-\length(S)}{m}_q
\end{equation*}
where $|S|$ denotes the sum of the elements.


\subsection[gesselSlinkyRule]{Egge--Loehr--Warrington and the slinky rule}

In \cite{EggeLoehrWarrington2010}, the following lifting of 
the fundamental quasisymmetric functions to the Schur functions was given.
An alternative interpretation of this rule is proved in \cite{GarsiaRemmel2018}.
A short proof by Ira Gessel is given in \cite{Gessel2018OnTheSchurFunction}.

Suppose $X(\xvec)$ is a symmetric function, with fundamental quasisymmetric expansion
\[
 X(\xvec) = \sum_\alpha c_\alpha \gessel_\alpha(\xvec)
\]
Then
\[
 X(\xvec) = \sum_\alpha c_\alpha \schurS_\alpha(\xvec).
\]
The composition indices in $\schurS_\alpha(\xvec)$ should be 
computed via the \hyperref[schurJacobiTrudi]{Jacobi--Trudi identity}
using the complete homogeneous symmetric functions.
By using determinant formulas, the compositions can be "straighened" into partitions, 
which can be done using the \defin{slinky rule} as it can be described 
pictorially as the parts of the compositions falling down
when $\alpha$ is illustrated in French notation, see \cite{EggeLoehrWarrington2010}.

\begin{example*}[The slinky rule]

We use the slinky rule on the composition $(4,3,8)$.

\begin{figure}
\begin{ytableau}
*(lightblue) & *(lightblue) & *(lightblue) & *(lightblue) & *(lightblue) & *(lightblue) & *(lightblue) & *(lightblue) \\
*(gray) & *(gray) & *(gray) \\
*(lightgreen) & *(lightgreen) & *(lightgreen) & *(lightgreen)
\end{ytableau}

\begin{ytableau}
*(lightblue) & *(lightblue) & *(lightblue) & *(lightblue) \\
*(gray) & *(gray) & *(gray) & *(lightblue) & *(lightblue) & *(lightblue) & *(lightblue) \\
*(lightgreen) & *(lightgreen) & *(lightgreen) & *(lightgreen)
\end{ytableau}

\begin{ytableau}
*(lightblue) & *(lightblue) & *(lightblue) & *(lightblue) \\
*(gray) & *(gray) & *(gray) & *(lightblue) & *(lightblue)  \\
*(lightgreen) & *(lightgreen) & *(lightgreen) & *(lightgreen) & *(lightblue) & *(lightblue) 
\end{ytableau}
\end{figure}

The resulting shape is $(6,5,4)$ and the sign is $(-1)^2$ as the height 
of the ribbon of size $8$ ends two levels below its starting row.
That is, the height of a ribbon is computed in the same 
manner as in the \hyperref[schurMurnaghanNakaygama]{Murnaghan--Nakayama rule}.
Hence, $\schurS_{438}=\schurS_{654}$.


As a second example, $\schurS_{25115} = -\schurS_{43322}$ according to the slinky rule.

\begin{figure}
\begin{ytableau}
*(lightblue) & *(lightblue) & *(lightblue) & *(lightblue) & *(lightblue) \\
\; \\
\; \\
*(gray) & *(gray) & *(gray) & *(gray) & *(gray) \\
*(lightgreen) & *(lightgreen)
\end{ytableau}

\begin{ytableau}
*(lightblue) & *(lightblue) \\
\; & *(lightblue)  \\
\; & *(lightblue)  & *(lightblue) \\
*(gray) & *(gray) & *(gray) \\
*(lightgreen) & *(lightgreen)  & *(gray) & *(gray)
\end{ytableau}
\end{figure}

Note that the sign is $(-1)^{2}(-1)^{1} = -1$.

Finally, $\schurS_{12} = 0$, since there are not enough boxes to perform a slinky move.
Similarly, $\schurS_{2115} = 0$, since the slinky rule does not give a partition.
\begin{figure}
\begin{ytableau}
*(lightblue) & *(lightblue) & *(lightblue) & *(lightblue) & *(lightblue) \\
\; & \none \\
\; & \none \\
*(lightgreen) & *(lightgreen) 
\end{ytableau}
\begin{ytableau}
*(lightblue) & *(lightblue) \\
\; & *(lightblue)  \\
\; & *(lightblue) & *(lightblue)  \\
*(lightgreen) & *(lightgreen) 
\end{ytableau}
\end{figure}

\end{example*}


\subsection[gesselPowerSum]{Relationship with the powersum basis}


Athanasiadis \cite{Athanasiadis2015}, (also implicitly in \cite{AdinRoichman2015}) 
prove the following: Suppose $X(x)$ is a symmetric function, such that
\begin{equation*}
X(\xvec) = \sum_{S \subseteq [n-1]} a_S \gessel_{n,S}(\xvec).
\end{equation*}
Then
\begin{equation*}
X(\xvec) = \sum_{\lambda \vdash n} z_{\lambda}^{-1} \powerSum_{\lambda}(\xvec) \sum_{S\in U_\lambda} (-1)^{|S\setminus S_\lambda|} a_S,
\end{equation*}
where $U_\lambda$ is the set of $\lambda$-unimodal subsets of $[n-1]$
and $S_\lambda \coloneqq \{ s_1,s_2,\dotsc,s_{\ell-1} \}$.

Here, $s_i$ is defined as $\lambda_1+\dotsb + \lambda_{i}$
and a set $A \subseteq [n-1]$ is $\lambda$-unimodal if for $0\leq i \lt \ell$
the intersection of $A$ with the intervals $\{s_i+1,s_i+2,\dotsc,s_{i+1}-1\}$
is a prefix of the latter.


\subsection[gesselQuasiPowerSum]{Relationship with a quasisymmetric powersum basis}

In \cite{BallantineDaughertyHicksMason2020}, two versions of quasisymmetric powersum bases are given.
They give the fundamental quasisymmetric expansion of these two quasisymmetric powersum bases.

In \cite{AlexanderssonSulzgruber2019}, we provide the \hyperref[qPsi]{quasisymmetric powersum} expansion of $\gessel_\alpha(\xvec)$.
We have
\begin{equation*}
\gessel_{n,S}(\xvec) = \sum_{\alpha} \frac{\qPsi_\alpha(\xvec)}{z_\alpha} (-1)^{|S\setminus S_\alpha|}
\end{equation*}
where the sum is taken over all compositions $\alpha$ such that the set $S$ is $\alpha$-unimodal.
This formula implies Athanasiadis formula above.


\todo{
Lifting F-positivity to Schur-positivity via equivalence classes
https://arxiv.org/pdf/1508.07052.pdf
}


\todo{
See Haglund and http://users.wfu.edu/masonsk/hook.pdf
where a super-version of Gessels are used.
}


\section[zeroHeckeAlgebra]{The 0-Hecke algebra}

\todo{Write more about 0-Hecke algebra}

The \defin{0-Hecke algebra} $H_n(0)$ in type $A$ is the $\setC$-algebra generated by $T_i$, $i=1,\dotsc,n-1$,
subject to the relations 
\[
T_i^2 = T_i \qquad T_iT_{i+1}T_i = T_{i+1}T_iT_{i+1}\quad 
T_{j}T_i = T_{i}T_j \text{ whenever } |i-j|\geq 2.
\]
Note that $T_{s_1}\dotsc T_{s_{\ell}}$ is independent for 
any choice of reduced word $w = s_1\dotsm s_{\ell}$
with $w \in \symS_n$. It follows that the dimension of $H_n(0)$ is $n!$.

As an example of such operators, the divided difference operators 
generating the \hyperref[keyDefinitionOperators]{key polynomials} are such operators.
This also highlights the connection with the \hyperref[macdonaldE]{non-symmetric Macdonald polynomials}.

Similar to how $\symS_n$-modules decompose into irreducible modules indexed by partitions,
we can completely characterize indecomposable $H(0)$-modules, see \cite{KrobThibon1997}.
There are $2^{n-1}$ irreducible representations $F^\alpha$, indexed by integer compositions of $n$.

The map $\frobChar$ on a $H_n(0)$-module $M$, sending $F^\alpha \to \gessel_\alpha$ 
is the \defin{quasisymmetric characteristic} of $M$. 
This is an analogue of the \hyperref[frobeniusCharacteristic]{Frobenius characteristic}.

For example, in \cite{TewariWilligenburg2015}, the authors show that
the quasisymmetric Schur functions are given as the quasisymmetric characteristic of certain
$H_n(0)$-modules. 
\todo{ add https://arxiv.org/pdf/2003.11225.pdf }

See also \cite{Huang2013} for some examples of graded vector spaces.
