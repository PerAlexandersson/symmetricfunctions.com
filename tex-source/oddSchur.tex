\metatitle{Odd Schur functions}
\metadescription{Definition of the odd Schur functions and related concepts.}


\section[oddSchur]{Odd Schur functions}


\begin{polydata}{oddSchur}
  Name   & Odd Schur functions \\
  Space    & OSym \\
  Basis    & False \\
  Rating   & 1 \\
  Bib      & EllisKhovanov2012 \\
  Year     & 2012 \\
  Symbol   & $\oddSchur_\lambda$ \\
  Category & Schur \\
\end{polydata}

The odd Schur functions are introduced in \cite{EllisKhovanovLauda2012,EllisKhovanov2012}.
They constitute a basis for the space of \defin{odd symmetric functions}.

\todo{define odd symmetric functions}


There are several definitions of the \defin{odd Schur functions},  $\{\oddSchur_\lambda\}$,
using divided difference operators or plactic relations. In \cite{Ellis2012},
it was shown that all the previous definitions coincide, and that we have the following tableau formula.
\[
 \oddCompleteH_\mu = \sum_{T \in \SSYT(\lambda,\mu)} \sign(T_\lambda) \sign(T) \oddSchur_\lambda.
\]
Here, $\sign(T)$ is the sign of the shortest permutation that sorts
the reading word of $T$ in an increasing fashion, and $T_\lambda$ is the unique SSYT in 
$\SSYT(\lambda,\lambda)$.


In \cite{Ellis2012}, a Littlewood--Richardson rule is proved for the odd Schur functions.
