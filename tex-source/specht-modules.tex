\metatitle{Specht modules}
\metadescription{An introduction to Specht modules, their construction using tableaux, and their role in representing the symmetric group.}
\metakeywords{Specht modules,symmetric group,tableaux,row-stabilizer,column-stabilizer,polytabloid}

\section[specht-modules]{Specht modules}


\todo{https://arxiv.org/pdf/1007.2946}

\todo{https://arxiv.org/pdf/2511.03702}

\subsection[specht-modules-intro]{Introduction to Specht modules}

The purpose of \defin{Specht modules} is to construct $\symS_n$-modules, one for each partition of $n$,
such that the modules are irreducible.


In order to construct an $\symS_n$-module, we need a vector space on which $\symS_n$ acts.
A \defin{tableau} of shape $\lambda$ is a filling of the Young diagram $\lambda$ with $\{1,2,3,\dotsc,n\}$.
The \defin{row-stabilizer} $R_T$ of a tableau $T$ is the set of permutations that keep entries within rows.
Similarly, \defin{column-stabilizer} $C_T$ of a tableau $T$ is the set of permutations
that preserves the column entries.

For example, the following tableaux are in the same orbit under $R_T$, and there are $3! \cdot 2!$
permutations total in $R_T$.
\begin{figure}
\begin{ytableau}
1 & 2 & 4 \\
3 & 5
\end{ytableau}
\begin{ytableau}
4 & 1 & 2 \\
3 & 5
\end{ytableau}
\end{figure}

Similarly, all tableaux which can be obtained from the first one by
some element in $C_T$ are the following.
\begin{figure}
\begin{ytableau}
1 & 2 & 4 \\
3 & 5
\end{ytableau}
\begin{ytableau}
3 & 2 & 4 \\
1 & 5
\end{ytableau}
\begin{ytableau}
1 & 5 & 4 \\
3 & 2
\end{ytableau}
\begin{ytableau}
3 & 5 & 4 \\
1 & 2
\end{ytableau}
\end{figure}

Two tableaux are considered \emph{equivalent} if they are in the same orbit under $R_T$,
and that $\symS_n$ acts on such equivalence classes.
(Equivalently, we can think of $\symS_n$ acting on tableaux, by first permuting
the entries, then rearranging each of the rows in increasing order.)


A \defin{polytabloid}

