\metatitle{Boolean product polynomials}
\metadescription{An introduction to boolean product polynomials, their definition, Schur positivity, and bivariate extension.}

\section[booleanProduct]{Boolean product polynomials}

\begin{polydata}{booleanProduct}
  Name     & Boolean product polynomials \\
  Space    & Sym \\
  Basis    & False \\
  Rating   & 1 \\
  Bib      & BilleraBilleyTewari2018 \\
  Year     & 2018 \\
  Category & Other \\
\end{polydata}


The \defin{boolean product polynomials} is a family of polynomials defined as follows:
\[
\booleanProduct_{k}(\xvec_n) \coloneqq 
\prod_{ 1 \leq i_1 \lt i_2 \lt \dotsb \lt i_k \leq n } (x_{i_1} + x_{i_2}+\dotsb + x_{i_k}).
\]
These were introduced by \name[Lukas Billera]{L. Billera}, \name[Sara Billey]{S. Billey}, and \name[Vasu Tewari]{V. Tewari} in an FPSAC abstract,
and inspired by certain applications in hyperplane arrangement, among other things.


\subsection[booleanProductSchurPositivity]{Schur positivity}

In \cite{BilleyRhoadesTewari2019} it is proved that the 
functions $\booleanProduct_{k}(\xvec_n)$ are Schur-positive.
The authors use a new method for proving Schur positivity, called \emph{Chern plethysm}.

\begin{problem}
Find a combinatorial interpretation for the coefficients in the expansion
\[
\booleanProduct_{k}(\xvec_n) = \sum_{\lambda} \delta^{n,k}_{\lambda} \schurS_\lambda(\xvec_n).
\]
Find natural $\symS_n$-modules whose \hyperref[frobeniusCharacteristic]{Frobenius characteristic}
is $\booleanProduct_{n,k}(\xvec)$.
\end{problem}

\subsection[booleanProductBivariate]{Bivariate extension}

There is a bivariate extension,
\[
\booleanProduct_{k,\ell}(\xvec_n;\yvec_m) 
 \coloneqq 
\prod_{ \substack{ 1 \leq i_1 \lt i_2 \lt \dotsb \lt i_k \leq n 
\\
1 \leq j_1 \lt j_2 \lt \dotsb \lt j_\ell \leq m
}}
(x_{i_1} + x_{i_2}+\dotsb + x_{i_k} + 
 y_{j_1} + y_{j_2}+\dotsb + y_{j_\ell}
).
\]
One can show using representation theory (see \cite{BilleyRhoadesTewari2019}) 
that the $a_{\lambda,\mu}$ in 
\[
\booleanProduct_{k,\ell}(\xvec_n;\yvec_m) = \sum_{\lambda,\mu} a_{\lambda,\mu} \schurS_\lambda(\xvec_n)\schurS_\mu(\yvec_m).
\]
are non-negative integers. When $k=\ell = 1$, this reduces 
to the \hyperref[schurCauchyFormula]{dual Cauchy identity}.

