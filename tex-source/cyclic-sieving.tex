\metatitle{The Cyclic sieving phenomenon}
\metadescription{Index over topics concerning the cyclic sieving phenomenon.}



\todo{
Mention \cite[Sec. 2.5]{BessisReiner2011}, where a \emph{non-faithful action} is described
}

\section[cyclicSieving]{The cyclic sieving phenomenon}

See below for the \hyperref[cspDefinition]{definition of the cyclic sieving phenomenon}, (CSP).
There are many instances of the CSP, which I have put into different categories
based roughly on the type of combinatorial object it concerns.

<div class="csp-icon-container">
<a href="cyclic-sieving-list-word.htm">
<img src="./svg-images/csp-icon-word.svg" 
alt="CSP Icon Word" 
class="cspIcon" />
<span>Words &amp; permutations</span>
</a></div>

<div class="csp-icon-container">
<a href="cyclic-sieving-list-path.htm">
<img src="./svg-images/csp-icon-path.svg" 
alt="CSP Icon Path" 
class="cspIcon" />
<span>Paths</span>
</a></div>

<div class="csp-icon-container">
<a href="cyclic-sieving-list-catalan.htm">
<img src="./svg-images/csp-icon-catalan.svg" 
alt="CSP Icon Catalan" 
class="cspIcon" />
<span>Catalan objects</span>
</a></div>

<div class="csp-icon-container">
<a href="cyclic-sieving-list-match.htm">
<img src="./svg-images/csp-icon-match.svg" 
alt="CSP Icon Match" 
class="cspIcon" />
<span>Matchings &amp; crossings</span>
</a></div>

<div class="csp-icon-container">
<a href="cyclic-sieving-list-young.htm">
<img src="./svg-images/csp-icon-young.svg" 
alt="CSP Icon Young" 
class="cspIcon" />
<span>Tableaux &amp; fillings</span>
</a></div>

<div class="csp-icon-container">
<a href="cyclic-sieving-list-misc.htm">
<img src="./svg-images/csp-icon-misc.svg" 
alt="CSP Icon Misc" 
class="cspIcon" />
<span>Miscellaneous</span>
</a></div>


\subsection[cspDefinition]{Definition}

The \defin{cyclic sieving phenomenon} (CSP) was introduced by V. Reiner, D. Stanton and 
D. White in \cite{ReinerStantonWhite2004}. A nice survey by B. Sagan is given in \cite{Sagan2011}.

Let $X$ be a set of combinatorial objects, $\grpc_n = \langle g \rangle$ be a finite cyclic group of size $n$, and $f(q) \in \setN[q]$.
Then the triple $(X,\grpc_n,f(q))$ is said to \emph{exhibit the cyclic sieving phenomenon} if for all $d \in \setN$, we have
\[
|\{x \in X : g^d \circ x =x \}| = f\left(\exp\left(2\pi i  \frac{d}{n}\right)\right).
\]
In words, $f(q)$ evaluated at certain roots of unity gives the number of elements in $X$ fixed by powers of $g$.
Note that $f(1) = |X|$, so in many instances, $f(q)$
is a \hyperref[q-analogs]{$q$-analog}(or $q$-enumeration) of the set $X$.



\begin{proposition}
Let $\xi$ is a primitive $n^\thsup$ root of unity,
and suppose that $f \in \setN[q]$ such that $f(\xi^j) \in \setZ$ for all $j \in \setZ$.
Then for all $j \in \setZ$, $f(\xi^j) = f( \xi^{\gcd(j,n)})$.
\end{proposition}
\begin{proof*}
In \cite{Desarmenien1989} and \cite[Lem. 2.2]{AlexanderssonAmini2018}, 
it is proved that $f$ (up to mod $q^n-1$) is a linear combination of 
\[
 h_d(q) \coloneqq \sum_{j=0}^{n/d-1} q^{dj}  = \frac{[n]_q}{[d]_q} \qquad \text{ where } d \mid n.
\]
It then suffices to verify that 
\[
h_d(\xi^j) = h_d( \xi^{\gcd(j,n)})
=
\begin{cases}
\frac{n}{d} &\text{ if } n \mid j d \\
0 &\text{ otherwise}
\end{cases}
\]
for all $d \mid n$, $j\in \setZ$, which is straightforward.
\end{proof*}
This proposition is very handy: Suppose we know that $f$ is an integer at $n^\thsup$ roots of unity.
Then it suffices to verify that for all $d \mid n$, we have that
\[
|\{x \in X : g^d \circ x =x \}| = f\left(\exp\left(2\pi i  \frac{d}{n}\right)\right)
\]
in order to prove CSP.



\begin{theorem}[From \cite{ReinerStantonWhite2004}]

Given $X$ and $\grpc_n$ acting on $X$ and $f(q) \in \setN[q]$,
then $(X,\grpc_n,f(q))$ exhibits the CSP if and only if 
\[
f(q) \equiv \sum_{O \in Orb_{\grpc_n}(X)} \frac{q^n-1}{ q^{n/|O|} - 1} \mod (q^n-1),
\]
where $Orb_\grpc(X)$ is the set of orbits of $X$ under $\grpc$.

In other words, in a CSP triple $(X,\grpc_n,f(q))$ 
the polynomial $f(q)$ is essentially uniquely determined by $\grpc_n$ acting on $X$.
\end{theorem}

There is a converse to the above theorem.
\begin{theorem}[From \cite{AlexanderssonAmini2018}]

Let $\xi$ be a primitive $n$th root of unity,
and suppose $f(q) \in \setN[q]$ has the property that $f(\xi^j) \in \setN$ for all $j\in \setN$.
Let $S_k$ be defined as
\[
S_k \coloneqq \sum_{j|k} \mu(k/j) f(\xi^j).
\]
Then there is a group action $\grpc_n$ acting on some $X$ with cardinality $f(1)$
and $(X,\grpc_n,f(q))$ exhibits the CSP if and only if all $S_k$ are non-negative.

In other words, these are necessary and sufficient conditions on $f(q) \in \setN[q]$
for there to be a CSP involving $f(q)$.
\end{theorem}

Research situation: You have your favorite set $X$ and a $q$-analog $f(q)$.
The previous theorem allows you to check (by computer, preferably) if there is 
some CSP phenomenon on $X$, involving $f(q)$.
Note that it is not sufficient for $f(q)$ to evaluate to non-negative integers at roots of unity!

The number of orbits of size $k$ is determined by $\frac{1}{k}\sum_{j|k} \mu(k/j) f(\xi^j)$,
and it follows that the total number of orbits is 
\[
\sum_{k|n} \frac{1}{k}\sum_{j|k} \mu(k/j) f(\xi^j).
\]

\begin{example*}[Example 2.10 from \cite{AlexanderssonAmini2018}]
 Take $f(q)=q^5+3q^3+q+10$.
 At any $6^\thsup$ roots of unity, $f(\xi^j)$ is a positive integer.
 However, for $k=3$, the sum $\sum_{d|k} \mu(k/d) f(\xi^d)$ is $-3$.
 This sum represents the number of elements in an orbit of size $3$ under the cyclic group,
 and this number is not allowed to be negative.
 Hence, there is no set $X$ of size $15$ with a cyclic group action $C_6$ of order $6$,
 such that $(X,C_6,f(q))$ is a CSP-triple.
\end{example*}


\subsection[cspAndRepTheory]{Connection with representation theory}

We can prove cyclic sieving using \hyperref[general-representation-theory]{representation theory}.

Suppose $G$ (in this case the cyclic group $C_n$) act on the set $X$. 
We can form the $|X|$-dimensional vector space 
\[
V = \setC X = \{a_1 x_1 + a_2x_2+\dotsb + a_M x_M : a_1,\dotsc,a_m \in \setC \},
\]
and note that $G$ act on $V$. 
We can \hyperref[computingSnCharacters]{compute} the character $\chi(g)$ for $g\in G$,
as a certain type of trace. Since $G$ in our case act via 
permutations, $\chi(g)$ is simply the number of elements in $X$ fixed by $g$.


One then wish to find a different basis $B$ for $V$ (perhaps starting with the 
diagonal matrix with entries $(1,\zeta,\zeta^2,\dotsc,\zeta^{n-1})$ representing a generator of $C_n$)
such that the action of $G$ on $B$ is diagonal. 
It is then easy to compute the trace, expressed in terms of roots of unity.
If this trace of $g^d$ is exactly $f(\zeta^d)$ for our polynomial $f$(q), we have proved an instance of CSP.

Several examples are given in \cite{Sagan2011}.
One can also construct new CSP from existing ones by using representation theory, see \cite{BergetEuReiner2011,ReinerStantonWhite2004}.



\subsection[cspUniversal]{Universal CSP statistics}

The noition of universal statistics was introduced in \cite{AhlbachSwanson2018}.

Suppose $(X,\grpc_n,f(q))$ exhibits the CSP, where $f(q) = \sum_{x\in X} q^{\tau(x)}$
for some combinatorial statistic $\tau:X \to \setN$.
Then $\tau$ is said to be \emph{universal} if for every $\grpc_n$-orbit $O \subseteq X$,
we have that $(O,\grpc_n,\sum_{x\in O} q^{\tau(x)})$ exhibits the CSP.

The usual \hyperref[cspWords]{major index on words} is not universal.
The authors of \cite{AhlbachSwanson2018} introduce a new (universal) statistic on words
equidistributed with $\maj$ on words of length $n$ with fixed content $\alpha$.

\subsection[cspProper]{Proper statistic}

\emph{The following definition has not been defined in any journal to my knowledge.
This is possibly related/equivalent with the notion of a universal CSP statistic.
}

Suppose $X$ is a combinatorial family with a statistic $\sigma:X \to \setN$,
that defines the $q$-analog $f(q)$ of $X$, and that $(X, \grpc_n, f(q))$ is a CSP-triple.

Then the statistic is called \defin{proper}
if whenever $d|n$, we have that
\[
 g^{d}(x) = x \implies   \sigma(x) \equiv_{n/d} 0   \text{ for all } x \in X. 
\]

\begin{example}
Let $n=2$ and suppose $g$ generate a $\grpc_2$-action on $X$.
If $\sigma$ is a proper statistic on $X$, and $X^g \subseteq X$ are the elements 
fixed under $g$, then there should (in principle) 
exist an involution $\psi : X\setminus X^g \to X\setminus X^g$,
such that $(-1)^{\sigma(x)}+(-1)^{\sigma(\psi(x))} = 0$,
which would provide a proof of CSP.
\end{example}



\section[subsetCyclicSieving]{Subset cyclic sieving}

There is a natural generalization of the cyclic sieving phenomenon described in 
\cite{AlexanderssonLinussonPotka2019}.
Let $X$ be a set of combinatorial objects and $Y \subseteq X$. 
Let $\grpc_n = \langle g \rangle$ be a finite cyclic group of size $n$, and $f(q) \in \setN[q]$.
Then the triple $(Y\subseteq X,\grpc,f(q))$ is said to 
\emph{exhibit the subset cyclic sieving phenomenon} if for all $d \in \setN$, we have
\[
|\{y \in Y : g^d \circ y = y \}| = f\left(\exp\left(2\pi i  \frac{d}{n}\right)\right).
\]
Note that in particular, $f(1)=|Y|$.

The subset CSP can be thought of as a CSP on $Y$, but where the group action
is allowed to leave $Y$. By using the main theorem in \cite{AlexanderssonAmini2018},
one can show that if $Y\subset X$ admits a subset CSP,
then $Y$ also admits a proper CSP, with some cyclic group action that does not leave $Y$.


\begin{example}[Taken from \cite{AlexanderssonLinussonPotka2019}]

Let $X_n$ be the set of binary words, and let $Y_n \subseteq X_n$ be the set of words ending with a $1$.
Furthermore, let $\eta$ act on $X_n$ by a \emph{twisted two-step cyclic shift}
\[
\eta(b_1,b_2,\dotsc,b_{n-1},b_n) = (1-b_{n-1},1-b_n,b_1,b_2,\dotsc,b_{n-2}).
\]
It is easy to show that $\langle \eta \rangle$ is a cyclic group of order $n$.
The polynomial we shall use is $f_n(q) = \prod_{j=1}^{n-1}(1+q^j)$.
Then
\[
(Y_n \subseteq X_n, \langle \eta \rangle, f_n(q))
\]
exhibits the subset cyclic sieving phenomenon.
Note that $(X_n, \langle \eta \rangle, 2f_n(q))$ is an instance of the classical CSP.
\end{example}

\section[lyndonCSP]{Lyndon-like cyclic sieving}

In \cite{AlexanderssonLinussonPotka2019}, we consider a special type of 
cyclic sieving phenomenon, on a family $\{X_n\}_{n=1}^\infty$ of combinatorial objects.
This idea captures the case where fixed points in $X_n$ under the group action
are in bijection with smaller members of the family.

\begin{definition}
Let $\{(X_n, C_n, f_n(q))\}_{n=1}^\infty$ be a family of instances of CSP.
The family is \defin{Lyndon-like} if for all $n\geq 1$,
\[
 f_{n/m}(1) = f_n\left( e^{\tfrac{2 \pi i}{m}} \right), \text{ whenever } m|n.
\]
By the definition of CSP, we have that
\[
 f_n\left( e^{\tfrac{2 \pi i}{m}} \right) = |\{ x \in X_n : g^{n/m}(x) = x \}|,
\]
where $\langle g \rangle = C_n$. 
The family is therefore Lyndon-like if and only if
for every $d|n$ the number of elements in $X_n$ fixed under $g^{d}$ is equal to $|X_{d}|$.
\end{definition}

The connection with counting fixed-points under iterations of a map is also 
discussed in the earlier work \cite{Zarelua2008}, 
where many related notions are discussed.


There are several examples of Lyndon-like families of CSP.
For example, \hyperref[cspWords]{Cyclic sieving on words}, \hyperref[cspAreaSequences]{CSP on circular Dyck paths}
and \hyperref[cspMacdonaldE]{(conjecturally) CSP on non-attacking filling}.


Notice that it is easy to gather computer evidence for 
a sequence of $q$-analogs $\{f_n(q)\}_{n=1}^\infty$ to be Lyndon-like.
Such evidence should give strong hints about 
possible corresponding cyclic group actions, 
as it should in principle behave as a type of cyclic shift on words.

\section[lyndonCSPII]{$q$-Gauß congruences}

\todo{Add the CSP-phenomena that appear in Gorodetsky2019}

In \cite{Gorodetsky2019}, O. Gorodetsky introduces the notion of $q$-Gauß congruences.
A sequence of polynomials $\{ a_n(q) \}_{n=1}^\infty$ is said to 
satisfy \defin{$q$-Gauß congruences} if for every $n \in \setP$, we have 
\[
   \sum_{d|n} \mu(d) a_{n/d}(q^d) \equiv 0 \mod [n]_q.
\]
It can be verified that this condition on the family is equivalent with being \hyperref[lyndonCSP]{Lyndon-like},
that is $a_{n/m}(1) = a_n(\xi)$ where $\xi$ is an $n$th root of unity with order $m$,
see \cite[Cor. 2.5]{Gorodetsky2019}.





\section[orbitalHarmonics]{Orbital harmonics}

In \href{https://www.mat.univie.ac.at/~slc/wpapers/FPSAC2021/70Oh.pdf}{this FPSAC abstract}, Jaeseong Oh
gives an introduction to a representation-theoretical approach to proving 
cyclic sieving results.


\section[dilateralSieving]{Dilateral Sieving and $G$-sieving}

It is possible to extend the cyclic sieving phenomenon to other groups, see \cite{RaoSuk2020}.
The next natural example is the dilateral group, $D_n$. This group
is presented as $\langle r,s : r^n=s^2=e, rs=sr^{-1} \rangle$.
We can interpret this as $r$ being a one-step rotation of an $n$-gon and $s$ being a reflection.
Several examples of dilateral sieving is given in \cite{RaoSuk2020}.

Rao and Suk has introduced the notion of \defin{$G$-sieving} for arbitrary groups.



\section[homomesy]{Homomesy}


The term \defin{homomesy} introduced by J. Propp and T. Roby in \cite{ProppRoby2013x} is defined as follows.
Let $X$ be some combinatorial set and let $\langle g \rangle$ act on $X$ such that every orbit is finite,
and let $\sigma : X \to \setZ$ be statistic on $X$.
Then the triple $(X,\langle g \rangle,\sigma)$ is said to exhibit homomesy if for every orbit $O \subseteq X$, we have
\[
\frac{1}{|O|} \sum_{x \in O} \sigma(x) = \frac{1}{|X|} \sum_{x \in X} \sigma(x).
\]
In other words, the average value of $\sigma$ is the same on every orbit.

The first instance of the homomesy phenomenon was observed by D. Panyushev in \cite{Panyushev2009},
when studying an action on antichains of positive roots. 
Since then, many more instances have been found.


For homomesy on miniscule posets, see \cite{Okada2021},
and for the Foata map, see \cite{LaCroixRoby2020x}.
Spectral theory approach to homomesy, see \cite{Propp2021x}.


An action is \defin{homometric} if the total sum of the statistic on orbits of the same size,
is the same. For example, for all orbits of size 4, the sum of the statisic on each orbit must be the same.
This notion was introduced by Elizalde, Roby, Plante and Sagan.

Homomesy and homometry for promotion on rooted trees is considered in \cite{DangwalKimbleLiangLouSaganStewart2023}.
