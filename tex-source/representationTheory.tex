\metatitle{Representation theory}
\metadescription{An introduction to representation theory, including general definitions, irreducible representations, 
tensor products, characters, and specific focus on the representation theory of the symmetric group $\symS_n$ and the general linear group $\GL_n$.}

\todo{Add https://files.core.ac.uk/download/pdf/636347084.pdf}

\section[representationTheory]{Representation theory}

\subsection[representation-definition]{Representation of a group}


Let $G$ be a group and $V$ a vector space. A \defin{representation} 
is a map $\rho$ that sends elements in $G$ to linear maps on $V$, 
with the condition that it is compatible with the group structure.
To be precise, let $\psi_g : V \to V$ be the linear map on $V$ we associate with $g \in G$. 
Then
\[
\psi_{g_1 g_2} = \psi_{g_1} \circ \psi_{g_2} \text{ for all } g_1,g_2 \in G.
\]
Thus, a representation is a group homomorphism $\rho : G \to GL(V)$.

\begin{example}
Let $G = \symS_n$ and $V=\setC[x_1,\dotsc,x_n]$. Note that $V$ is not finite-dimensional.
Then we let $G$ acts on $V$ by permuting the indices of the variables. 
By picking a basis of monomials of $V$, then $\rho$ sends a permutation $g \in G$ to a permutation matrix in $GL(V)$.
\end{example}


\subsection[representation-irreducible]{Irreducible representation}

Let $W \subset V$ be a subspace and $\rho:G \to GL(V)$ be a representation.
If $\rho(g)w \in W$ for all $w \in W$, then we say that $W$ is $G$-invariant. 

If the only $G$-invariant subspaces of $V$ is $V$ itself and $\{0\}$, then we say that $\rho$
is an \defin{irreducible representation}.

Every representation can be decomposed as a direct sum irreducible representations.
That is, if $\rho: G \to V$ is a representation, we can write $V = V_1 \oplus \dotsb \oplus V_k$
such that for every $i$, we have
\[
\rho(g) v \in V_i \; \text{ for all }\; v \in V_i
\]
and $\rho:G \to V_i$ is an irreducible representation.
Here, one needs to be a bit more formal and restrict the maps in an appropriate fashion.


\subsection[representation-tensor-product]{Tensor products}

If $\rho_1$ and $\rho_2$ are two representations of $G$, sending group elements to elements in $GL(V)$ and $GL(W)$,
then we define the tensor product of representations as the map sending $g$ to the tensor product of the individual images:
\[
(\rho_1 \otimes \rho_2)(g) \coloneqq (\rho_1(g) \otimes \rho_2(g))
\]

\subsection[representation-character]{Character}

Let $\rho: G \to GL(V)$ be a representation of $G$. 
Then the \defin{character} is the map $\chi_{\rho}:G \to \setC$, defined as $\chi_{\rho}(g) \coloneqq tr( \rho(g))$.

A character is irreducible if it is the character of an irreducible representation.


\section[Sn-representation-theory]{Representation theory of $\symS_n$}

It is very common to consider $\symS_n$ acting on some vector space.
This is then an $\symS_n$-module, and a representation of $\symS_n$.

As a typical example of $\symS_n$-modules, we have the \hyperref[stanleySymFrobenius]{Specht modules}.


\subsection[frobeniusCharacteristic]{Frobenius characteristic}

Let $\chi$ be a class function on $\symS_n$. 
We then define the \defin{Frobenius characteristic} (or \defin{Frobenius image})
as the map
\[
\frobChar(\chi) \coloneqq \frac{1}{n!} \sum_{\sigma \in \symS_n} \chi(\sigma) \powerSum_\sigma(\xvec)
= \sum_{\mu} \chi_\mu \frac{\powerSum_\mu(\xvec)}{z_\mu}.
\]
This map has the property that for the irreducible characters $\chi^\lambda$ of $\symS_n$,
which are class functions, we have 
\[
\frobChar(\chi^\lambda) = \schurS_\lambda(\xvec).
\]

Note that the \defin{Hilbert series} can be recovered from the Frobenius characteristic.
If $M = \bigoplus_k M^k$ is a graded $\symS_n$-module, and
\[
\frobChar(M) = \sum_{k \geq 0} \sum_{\lambda \vdash k} a(\lambda,k) \schurS_\mu(\xvec)
\]
then the Hilbert series is given by
\[
\sum_{k \geq 0} t^k \sum_{\lambda \vdash k} a(\lambda,k) f^{\mu}
\]
where $f^\mu$ is the number of standard Young tableaux of shape $\mu$.
The Hilbert series can alternatively described simply as $\langle \frobChar(M), \completeH_{1^n} \rangle$.

\subsection[computingSnCharacters]{Computing characters of $\symS_n$}

You have an $\symS_n$-module $M$ and wish to compute its Frobenius characteristic $\frobChar_M(\xvec)$.
This is the step-by-step guide (inspired by lecture notes by M. Zabrocki).

The same method can be adapted to compute characters for arbitrary finite groups.

\begin{enumerate}
\item 
We assume that $M$ is the span of some set of polynomials in $\setC[x_1,\dotsc,x_n]$.

\item 
Use linear algebra and find a basis for $M = \langle \bvec_1,\bvec_2,\dotsc,\bvec_d \rangle$.

\item 
For each $\sigma \in \symS_n$, we compute its character $\chi_M(\sigma)$ as follows.
 For each $i \in [d]$, solve the linear system 
 \[
 \sigma(\bvec_i) = c_{i1} \bvec_1 + c_{i2} \bvec_2 + \dotsb +  c_{id} \bvec_d.
 \]
 Then $\chi_M(\sigma) \coloneqq c_{11} + c_{22} + \dotsb + c_{dd}$.
 This is a type of trace and independent of the choice of basis.
 
 It is in fact sufficient to compute $\chi_M(\sigma)$ for each type $\mu$,
 as $\chi_M(\sigma)=\chi_M(\tau)$ whenever $\sigma$ and $\tau$ have the same cycle type.


\item 
 
Now that we have computed $\chi_M(\sigma)$, we use the above formula and get that
\[
\frobChar_M(\xvec) = \frac{1}{n!} \sum_{\sigma \in \symS_n} \chi_M(\sigma) \powerSum_\sigma(\xvec).
\]

Using the fact that $\frac{n!}{z_\mu}$ is the number of permutations of type $\mu$
and the above observations, we see that
\[
\frobChar_M(\xvec) = \sum_{\mu \vdash n} \chi_M(\mu) \frac{\powerSum_\mu(\xvec)}{z_\mu},
\]
where $\chi_M(\mu)$ is the character of some permutation of type $\mu$.

\item 

We then have the following relation:
 \[
  M = \bigoplus_{\lambda \vdash n} \left(S^{\lambda} \right)^{\oplus c_\lambda} \qquad 
  \Longleftrightarrow 
  \qquad  
  \frobChar_M(\xvec) = \sum_{\lambda \vdash n} c_\lambda \schurS_\lambda(\xvec).
 \]
 Here, we have expressed $M$ as a sum of irreducible $\symS_n$-modules, denoted $S^{\lambda}$. 
 In particular, we must have that the coefficients $c_\lambda$ are non-negative integers.
 Note that $\dim(S^{\lambda}) = f^\lambda$, the number of standard Young tableaux of shape $\lambda$.

\end{enumerate}

In all examples below, $\symS_n$ act on the space by permuting the variable indices.

\begin{example}
Let $M = \langle x_1 + x_2 + x_3 \rangle$. This is a one-dimensional vector space,
and all elements in $\symS_3$ fixes $M$.
Thus $\chi_M(\sigma)=1$ for all $\sigma \in \symS_3$ and 
\[
\frobChar_M(\xvec) = \frac{1}{3!} \sum_{\sigma \in \symS_3} \powerSum_\sigma(\xvec) = \schurS_{3}(\xvec).
\]
The same thing happens for $M = \langle x_1 x_2 x_3 \rangle$.
\end{example}

\begin{example}
Let $M = \langle (x_1-x_2)(x_1-x_3)(x_2-x_3) \rangle$. 
This is also a one-dimensional vector space,
but odd permutations change the sign of the basis vector.
Thus $\chi_M(\sigma)=\sign(\sigma)$ and
\[
\frobChar_M(\xvec) = \frac{1}{3!} \sum_{\sigma \in \symS_3} \sign(\sigma) \powerSum_\sigma(\xvec) = \schurS_{111}(\xvec).
\]
\end{example}

\begin{example}
Let $M = \langle x_1 , x_2 , x_3 \rangle$. 
Note that $\langle x_1 + x_2 + x_3 \rangle$ is an invariant subspace.

We have that $\frobChar_M(\xvec) = \schurS_{3}(\xvec) + \schurS_{21}(\xvec)$.
\end{example}


\begin{example}
Let $\Delta = \prod_{1\leq i < j \leq n} (x_j-x_i)$.
Let $M$ be the module spanned by $\Delta$ and all partial derivatives of all orders of $\Delta$.
Then the graded Frobenius characteristic is given by
\[
 \frobChar_M(\xvec) = \sum_{\lambda \vdash n} \schurS_\lambda(\xvec) \sum_{T \in SYT(\lambda)} q^{\maj(T)}.
\]
In particular, $M$ has dimension $n!$ --- this is a result by M. Haiman.
\end{example}

\begin{example}
From \cite{ArmstrongLoehrWarrington2015} we have the following result.
The symmetric group $\symS_n$ acts on \emph{parking functions} by relabeling.
We can generalize this to view parking functions as labeled $m$-Dyck paths
in an $n \times mn$-rectangle (counted by Fuss-Catalan numbers).
The Frobenius characteristic for $(m,n)$ is then
\[ 
   \sum_{\lambda \vdash n} K_{\lambda,m} \completeH_\lambda
\]
where $K_{\lambda,m}$ is the number of $m$-Dyck-paths of type $\lambda$.
Orbits of the action above are indexed by $m$-Dyck paths.
\end{example}


\begin{example}
Symmetric group action on set-partitions and non-crossing set partitions: \url{https://www.youtube.com/watch?v=IJ3tTGJTofQ}.
See also $SL_3$-webs.
\end{example}


\begin{example}[Garsia--Procesi modules]
Garsia--Procesi modules, see \cite{CarlssonChou2024x}.
\end{example}

\begin{example}[Garsia--Haiman modules]
Garsia--Haiman modules, see \cite{Armon2022}.
\end{example}





\section[representation-Gln]{Representation theory of $\GL_n$}

We can also study representation theory of infinite groups, 
such as the group of invertible $n\times n$-matrices with complex entries.
This is the \defin{general linear group}, $\GL_n(\setC)$.
For a brief introduction to representation 
theory of $\GL_n(\setC)$, see for example \cite[Chap. 7, App. 2]{StanleyEC2}.


A \defin{representation} of $\GL_n(\setC)$ is 
a group homomorphism $\phi : \GL_n(\setC) \to \GL_m(\setC)$ for some $m$.
We will only study the cases when $\phi$ is \defin{homogeneous} and \defin{rational}, 
meaning that $\phi(\alpha A) = \alpha^k \phi(A)$
for some $k \in \setZ$ and all $\alpha \in \setC \setminus \{0\}$.
The integer $k$ is called the \defin{degree} of $\phi$, and if $k\geq 0$,
we say that $\phi$ is a \defin{polynomial representation}.

As a concrete example, $\phi : \GL_n(\setC) \to \GL_1(\setC)$ defined as
$\phi(A) \to \det(A)$ is a homogeneous polynomial representation of degree $n$.


Let $\phi$ be a rational representation $\phi$ of $\GL_n(\setC)$.
If $A$ has eigenvalues $x_1,\dotsc,x_n$, there is a Laurent polynomial
\[
\glChar(\phi)(\xvec) = \sum_{x^\alpha \in \setZ^n} m_\alpha x^\alpha
\]
called the \defin{character} of $\phi$,
such that $\phi(A)$ has eigenvalue $x^\alpha$ with multiplicity $m_\alpha$.
Moreover, if $\phi$ is polynomial, then $\glChar(\phi)(\xvec)$ is a polynomial in $x_1,\dotsc,x_n$.



\begin{theorem}
Every irreducible homogeneous polynomial representation $\phi$ of $\GL_n(\setC)$
is given as
\[
\glChar(\phi)(\xvec) = \schurS_\lambda(x_1,\dotsc,x_n)
\]
for some $\lambda \vdash n$, where $\schurS_\lambda$ is a \hyperref[schurS]{Schur polynomial}.
\end{theorem}

Given two characters, $\phi$, $\varphi$, we can define the tensor product
$\phi \otimes \varphi$. We then have
\[
\glChar(\phi \otimes \varphi)(\xvec) = \glChar(\phi)(\xvec) \cdot \glChar(\varphi)(\xvec).
\]
In particular, if $\phi$ and $\varphi$ are irreducible, then 
the character $\glChar(\phi \otimes \varphi)(\xvec)$ decompose into irreducible 
representations via the \hyperref[schurLittlewoodRichardson]{Littlewood--Richardson rule}.


If $\phi : \GL(U) \to \GL(V)$ and $\varphi : \GL(V) \to \GL(W)$ are representations,
then the composition $\phi\varphi$ is a representation.
The character of $\phi\varphi$  is related to the individual characters as
\[
\glChar(\phi \varphi) = \glChar(\phi)[ \glChar(\varphi) ],
\]
where we use the \hyperref[plethysm]{pletystm} operation.


\section[repTheoryMore]{Additional topics}

For the interaction between representation theory and 0-Hecke algebras, see \cite{HicksMillerBrown2024x}.


\todo{
Generalized flagged Schur modules, and Key modules
https://core.ac.uk/download/pdf/82753652.pdf
}

\todo{

\url{https://arxiv.org/pdf/0912.0569.pdf}

\url{https://sbseminar.wordpress.com/2008/07/08/how-to-write-down-the-representations-of-gl_n/}

\url{https://www.math.upenn.edu/~zvihr/courses/RTNotes.pdf}

\url{http://math.cmu.edu/~cnewstea/notes/reptheory.pdf}

\url{https://arxiv.org/pdf/2201.04006.pdf} Explicit Frobenius characteristic of an ideal.

}
