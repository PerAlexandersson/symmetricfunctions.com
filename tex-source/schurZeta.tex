\metatitle{Schur multiple zeta functions}
\metadescription{An introduction to Schur multiple zeta functions, their definition via semistandard Young tableaux, and related determinant formulas such as Jacobi--Trudi and Giambelli.}
\metakeywords{Schur multiple zeta functions,jacobi-trudi,giambelli,cauchy-identity}


\section[schurZeta]{Schur multiple zeta functions}


\begin{polydata}{schurZeta}
  Name   & Schur multiple zeta functions \\
  Space    &  NA   \\
  Basis    &  False \\
  Rating   &  1     \\
  Bib      &  NakasujiPhuksuwanYamasaki2018 \\
  Year     &  2018 \\
  Keywords & jacobi-trudi, giambelli, cauchy-identity\\
  Symbol   & $\zeta_{\lambda}(\svec)$ \\
  Category & Schur \\
\end{polydata}


The \defin{Schur multiple zeta functions} where introduced in \cite{NakasujiPhuksuwanYamasaki2018}
in order to interpolate between previous generalizations of the Riemann zeta function.
We have that $\zeta_\lambda(\svec)$ is defined as
\[
 \zeta_\lambda(\svec) \coloneqq \sum_{T \in \SSYT(\lambda)} \prod_{(i,j) \in \lambda} T_{ij}^{s_{ij}}.
\]
The product is taken over all boxes in the diagram of $\lambda$.
Note that this is not a symmetric function and not a polynomial in $\svec$.
In fact, when $\lambda = (1)$, $\zeta_\lambda(\svec)$ coincides 
with the classical Riemann zeta function.

When $\lambda$ is a single row or a single column, we recover the functions previously introduced
by Hoffman \cite{Hoffman1992} and Zaiger \cite{Zagier1994}. 

If we set all variables equal, then 
\[
\zeta_\lambda(s,s,s,\dotsc,s) = \schurS_\lambda(1^{-s},2^{-s},3^{-s},\dotsc).
\]

Jacobi--Trudi identities and Giambelli formulas are proved in \cite{NakasujiPhuksuwanYamasaki2018}.
More general determinant formulas of the same type 
as Lascoux--Pragacz and \hyperref[schurStripDecompFormula]{Hamel--Goulden}
are proved in \cite{BachmannCharlton2020}.

Sum formulas, see \cite{BachmannKadotaSuzukiYamamotoYamasaki2023x}.

Pieri formulas and a Littlewood--Richardson type rule is proved in \cite{Nakaoka2023x}.
For connection with quasisymmetric functions, see \cite{Hoffman2008}.
For connections with Chern numbers and hyper-Kähler and Calabi-Yau manifolds, see \cite{Li2021x}.

\subsection[schurZetaP]{Schur multiple zeta P and Q}

In \cite{NakasujiTakeda2022x} the authors introduce the multiple zeta 
analogs of \hyperref[schurP]{Schur P} and \hyperref[schurQ]{Schur Q} functions.

They also define multiple zeta functions analogous to the 
\hyperref[schurOrthogonal]{Orthogonal Schur polynomials} and the 
\hyperref[schurSymplectic]{Symplectic Schur polynomials}.
