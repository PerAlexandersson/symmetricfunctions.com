\metatitle{Schur-S polynomials}
\metadescription{Definition and properties of Schur-S polynomials, including their relation to standard Young tableaux and differentiation.  }

\section[schurSS]{Schur-S polynomials}


\begin{polydata}{schurSS}
  Name   & Schur-S polynomials \\
  Space    &  All   \\
  Basis    &  True   \\
  Rating   &  1      \\
  Bib      &  BehzadGatto2020x\\
  Year     &  ? \\
  Keywords & jacobi-trudi,tableaux, cauchy-identity,skew \\
  Symbol   & $S_{\lambda/\mu}(\xvec)$ \\
  Category & Schur \\
\end{polydata}





Let us first introduce $S_j(\xvec)$ via the expansion
\[
\sum_{j \geq 0} S_j z^j = \exp\left( \sum_{j \geq 1} x_j z^j\right).
\]
Then define 
\[
 S_\lambda(\xvec) \coloneqq \det[ S_{\lambda_j-j+i} ].
\]
Note that this is the image of $\schurS_\lambda$ under 
the algebra homomorphism $\completeH_j \mapsto S_j$.

As an example,
\[
S_{32}(\xvec)= \frac{x_1^5}{24}+\frac{1}{6} x_2 x_1^3-\frac{1}{2}
   x_3 x_1^2+\frac{1}{2} x_2^2 x_1-x_4 x_1+x_2 x_3.
\]

After this map, we have that $\{ S_\lambda \}_\lambda$
is a basis for $\setQ[\xvec]$, which is a bit unusual compared 
to other families of Schur polynomials.

In \cite{BehzadGatto2020x}, the main result is that when $\lambda \vdash d$, 
\[
   \frac{\partial^d}{\partial x_1^d} S_\lambda(x) =  f^\lambda.
\]

