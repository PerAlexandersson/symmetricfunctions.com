\metatitle{Key polynomials and Demazure atoms}
\metadescription{An introduction to key polynomials (Demazure characters) and Demazure atoms, including their definitions via divided difference operators, combinatorial models using tableaux and fillings, representation-theoretical interpretations, and properties such as products and Pieri rules.} 

\section[key]{Key polynomials}

\begin{polydata}{key}
  Name     & Key polynomials \\
  Space    & All \\
  Basis    & True \\
  Rating   & 5 \\
  Bib      & Demazure1974nouvelle \\
  Year     & 1974 \\
  Symbol   & $\key_{\lambda,\sigma}(z)$ \\
  Keywords & atom-positive, divided-difference, fillings, polytopes \\
  Category & Schur \\
\end{polydata}

The \defin{key polynomials},  also known as  \defin{Demazure characters},
is a family of non-symmetric polynomials, indexed by weak compositions, or alternatively, an integer partition and a 
permutation. The set $\{ \key_\alpha(x_1,\dotsc,x_n) \}_\alpha$ where $\alpha$
ranges over all compositions of length $n$, is a basis for $\setC[x_1,\dotsc,x_n]$.

The key polynomials generalize the \hyperref[schurS]{Schur polynomials}.

Demazure characters were introduced in \cite{Demazure1974nouvelle,Demazure1974}.
Later, a combinatorial formula using so called \defin{key tableaux} was discovered in
\cite{Lascoux1990Keys}.
Key polynomials are also given as a specialization of \hyperref[macdonaldE]{non-symmetric Macdonald polynomials},
see \cite{HaglundHaimanLoehr2008}, and this gives several combinatorial 
models for the key polynomials, see the works of S. Mason \cite{Mason2009},
and also \cite{Kurland2016,AlexanderssonSawhney2019}.

Key polynomials are closely related to \hyperref[atom]{Demazure atoms}.


The K-theoretic version of key polynomials are the \hyperref[lascoux]{Lascoux polynomials}.

\subsection[keyDefinitionOperators]{Definition (Operators)}

Define the \emph{divided difference operator} $\partial_i$ as
\[
 \partial_i(f) = \frac{f - s_i(f)}{z_i - z_{i+1}},
\]
and let $\pi_i(f) \coloneqq \partial_i(z_i f)$ for $i=1,\dotsc,n-1$ whenever $f \in \setC[z_1,\dotsc,z_n]$.
One can show that $\pi_i^2 = \pi_i$ and that they satisfiy the braid relations.

Let $\sigma = s_{i_1} s_{i_2} \dotsc s_{i_\ell}$ be a \emph{reduced word} of a permutation $\sigma \in \symS_n$,
and let
\[
 \pi_\sigma \coloneqq \pi_{i_1} \circ \pi_{i_2} \circ \dotsb \circ \pi_{i_\ell}.
\]
This is well-defined, as the action of $\pi_\sigma$ is independent of the choice of reduced word.


Let $\lambda$ be a partition with at most $n$ parts, and let $\sigma \in \symS_n$.
In fact, we can choose $\sigma$ to be the \emph{minimal coset representation} for 
the parabolic subgroup $W_J \subseteq \symS_n$ that fixes $\lambda$.
The \defin{key polynomial} $\key_{\lambda,\sigma}(z)$ is defined as
\begin{equation*}
 \key_{\lambda,\sigma}(z) \coloneqq \pi_\sigma\left( z_1^{\lambda_1} \dotsm z_n^{\lambda_n} \right).
\end{equation*}
Note that if $\sigma^1(\lambda)=\sigma^2(\lambda)$, then 
$\key_{\lambda,\sigma^1}(z) = \key_{\lambda,\sigma^2}(z)$.


This definition shows the close connection between \hyperref[schubert]{Schubert polynomials}
and key polynomials.
Notice that if we take $w_0$ to be the longest permutation in $\symS_n$, then
\[
 \schurS_\lambda(z_1,\dotsc,z_n) = \key_{\lambda,w_0}(z_1,\dotsc,z_n),
\]
so the key polynomials are a superfamily of the \hyperref[schurS]{Schur polynomials}.



\begin{example}[Taken from \cite{AlexanderssonAlhajjar2018}]

Let $\lambda = (2,1,0,0)$ and $\sigma = [2,4,3,1] \in \symS_4$ in one-line notation.
The permutation can be expressed as a reduced word as 
$\sigma = s_2 s_3 s_2 s_1$.
We compute the key polynomial as follows:
\begin{align*}
 \key_{\lambda,\sigma}(z) &= \pi_2 \pi_3 \pi_2 \pi_1 (z_1^2 z_2) = 
 \pi_2 \pi_3 \pi_2 \partial_1 (z_1^3 z_2 ) = \pi_2 \pi_3 \pi_2 \left( \frac{z_1^3z_2 - z_1 z^3_2}{z_1-z_2}\right) \\
 &= \pi_2 \pi_3 \pi_2  (z_1^2z_2 + z_1 z_2^2).
\end{align*}
We continue the calculation by applying $\pi_2$ and get
\begin{align*}
\key_{\lambda,\sigma}(z) &=
 \pi_2 \pi_3 (z_2 z_1^2+z_3 z_1^2+z_2^2 z_1+z_3^2 z_1+z_2 z_3 z_1).
 \end{align*}
 Applying $\pi_2 \pi_3$ then finally gives
\begin{equation*}
\begin{split}
\key_{\lambda,\sigma}(z) &=
z_1^2 z_2  + 
z_1^2 z_3  + 
z_1^2 z_4  + 
z_1 z_2^2  + 
z_1 z_3^2  \\
&\phantom{=}+
z_1 z_4^2  +
z_1 z_2 z_3  + 
z_1 z_2 z_4  + 
z_1 z_3 z_4 .
 \end{split}
\end{equation*}
In general, some monomials may appear multiple times.
\end{example}


Alternatively, key polynomials are also commonly indexed by weak compositions. 
If $\alpha = (\alpha_1,\dotsc,\alpha_n)$ is a weak composition,
we let $\key_{\alpha}(z)$ be the key polynomial $\key_{\lambda,\sigma}(z)$,
where $\lambda$ is the partition obtained from $\alpha$ by sorting the elements in decreasing order,
and $\sigma \in \symS_n$ is a permutation with the property that it sorts $\alpha$
into a partition; $\lambda = (\alpha_{\sigma(1)},\alpha_{\sigma(2)},\dotsc,\alpha_{\sigma(n)})$.

For example, $\alpha = (1,0,3,2)$ corresponds to $\lambda = (3,2,1,0)$ and $\sigma = [3, 4, 1, 2]$.
With these conventions, $\key_{\lambda}(z)=\schurS_{\lambda}(z)$.

\emph{Note:} This indexing convention is not uniform in the literature;
in some places, one needs to reverse the indexing composition $\alpha$.



\subsection[keyDefinitionGT]{Definition (GT-patterns)}

A formula for the key polynomials 
as a sum over lattice points in a union of faces 
of Gelfand--Tsetlin polytopes was proved in \cite{KiritchenkoSmirnovTimorin2010}.

\begin{theorem}[See \cite{KiritchenkoSmirnovTimorin2010}]
Let $\GT(\lambda,\sigma)$ be defined as the polytopal complex
\[
 \GT(\lambda,\sigma) \coloneqq  \bigcup_{\substack{\mathcal{F} \in \GT(\lambda) \\ \type(\mathcal{F}) = w_0\sigma}} \mathcal{F}.
\]
That is, $\GT(\lambda,\sigma)$ is the union of all reduced Kogan faces of type $w_0\sigma$
in the polytope $\GT(\lambda)$.

The key polynomial $\key_{\lambda,\sigma}(z)$ can be computed as
\begin{align}
 \key_{\lambda,\sigma}(z_1,\dotsc,z_n) = \sum_{G \in \GT(\lambda,\sigma) \cap \setZ^{\tfrac{n(n+1)}{2}} } z_1^{w_1(G)} \dotsm z_n^{w_n(G)}
\end{align}
\end{theorem}


Fix $\lambda$, $\sigma \in \symS_n$. 
With the above theorem, it follows that for the map 
\[
k \mapsto \key_{k \lambda,\sigma}(1^n)
\]
is a polynomial in $k$. 
This is in fact an Ehrhart polynomial of a union of faces in a GT-polytope.
Let $P_{\sigma}(\lambda_1,\dotsc,\lambda_n;k)$ denote this polynomial.
From the definition via divided difference operators,
one can see that $P_{\sigma}(\lambda_1,\dotsc,\lambda_n;k)$
is a polynomial in $\setQ[\lambda_1,\dotsc,\lambda_n,k]$.

See \cite{AlexanderssonAlhajjar2018} for the following conjecture.
\begin{conjecture}[Alexandersson--Alhajjar (2018)]
The polynomial $P_{\sigma}(\lambda_1,\dotsc,\lambda_n;k)$ has non-negative coefficients.

Moreover, after setting $a_j = \lambda_1+\lambda_2+\dotsb+\lambda_j$, the function
$P_{\sigma}(a_1,\dotsc,a_n;k)$ is a polynomial in $\setQ[a_1,\dotsc,a_n,k]$
with non-negative coefficients.
\end{conjecture}

See also \hyperref[keyRealRootedHstar]{my conjecture} regarding the corresponding $h^*$-polynomial.


\subsection[keyDefinitionKeyFillings]{Definition (Skyline fillings)}

In \cite{HaglundHaimanLoehr2008}, the authors presented a combinatorial model for the
\hyperref[macdonaldE]{non-symmetric Macdonald polynomials}.
It is relatively straightforward to see that setting $q=t=0$ in a non-symmetric Macdonald polynomial
$\macdonaldE_{\alpha}(\xvec,q,t)$, we recover the key polynomial $\key_\alpha(\xvec)$.


From the work of \name{Sarah Mason} \cite{Mason2009}, 
we in fact get two combinatorial models for key polynomials 
as special cases of the \hyperref[macdonaldEperm]{permuted basement Macdonald polynomials}.
In the below definition, we assume that the reader is familiar with 
the combinatorial model for the permuted basement Macdonald polynomials.


\begin{definition}
A \defin{semi-standard augmented filling} of shape $\alpha = (\alpha_1,\dotsc,\alpha_n)$
is a non-attacking filling of the diagram with shape $\alpha$,
basement $\sigma = [n,n-1,\dotsc,2,1]$ such that all rows (including the basement)
are weakly decreasing, and there are no coinversion triples.
Let $\mathrm{SSAF}(\alpha)$ be the set of semi-standard augmented fillings
of shape $\alpha$.

The key polynomial $\key_\alpha(\xvec)$ is defined as 
\[
\key_\alpha(\xvec) = \sum_{T \in \mathrm{SSAF}(\alpha)} \xvec^T.
\]
Note that when $\alpha$ is a partition, we sum over fillings with 
weakly decreasing rows, and strictly decreasing columns. 
It is then easy to see that this sum is the Schur polynomial $\schurS_{\lambda}(x_1,\dotsc,x_n)$.

Alternatively, by choosing a different basement,
we can realize $\key_\alpha(\xvec)$ as a sum over augmented fillings 
with partition shape.
Let $\mathrm{SSAF}_{\sigma}(\lambda)$ be the set of semi-standard augmented fillings 
with basement $\sigma$ and shape $\lambda$.
Then 
\[
\key_\alpha(\xvec) = \sum_{T \in \mathrm{SSAF}_{\sigma}(\lambda)} \xvec^T,
\quad
\text{where} \quad
\alpha = (\lambda_{\sigma^{-1}(n)}, \lambda_{\sigma^{-1}(n-1)},\dotsc,\lambda_{\sigma^{-1}(1)}).
\]
\end{definition}

In fact, one can find a series of bijections 
proving the equivalence of the two above formulas,
see \cite[Prop. 5.5]{AlexanderssonSawhney2019} (with $t=0$).
It amounts to prove that interchanging two adjacent rows (including the basement) in 
an augmented diagram in such a way that a longer row is moved upwards,
preserves the weighted sum of SSAF's with corresponding shape and basement.



\todo{
Reference Proctor's papers, and Postnikov-Stanley paper.
}


\subsection[keyDefinitionSSYT]{Definition (SSYT)}

A \defin{key} is a semistandard tableau, with the property
that the entries in column $j+1$ is a subset of the entries in column $j$, for all $j$.

Below is the unique key tableau with weight $(1,3,0,1,2,4)$:
\begin{figure}
\begin{ytableau}
1 & 2 & 2 & 6\\
2 & 5 & 6 \\
4 & 6 \\
5 \\
6 \\
\end{ytableau}
\end{figure}
One can prove that the map that sends a key to its weight is a bijection between the set of key tableaux
and the set of compositions. Given a composition $\alpha = (\alpha_1,\dotsc,\alpha_\ell)$
we simply place a cell $j$ in the first $\alpha_j$ columns.
This produces a unique key tableau, $\mathrm{key}(\alpha)$, of shape $\sort(\alpha)$.


Given a semi-standard tableau $T$, we can compute its \defin{right key} $K_+(T)$,
via a rather involved process. A simpler method is proved in \cite{Mason2009}.
The key polynomial can then be computed by
\[
  \key_\alpha(\xvec) = \sum_{\substack{T \in \SSYT(\sort(\alpha) \\ K_+(T) \leq \mathrm{key}(\alpha)}} \xvec^T
\]
where we sum over all semi-standard tableaux $T$ whose right key, $K_+(T)$, are entry-wise weakly smaller than $\mathrm{key}(\alpha)$.

Similarly, the Demazure atoms are given as the sum over semi-standard tableaux with a given right key:
\[
  \atom_\alpha(\xvec) = \sum_{\substack{T \in \SSYT(\sort(\alpha) \\ K_+(T) = \mathrm{key}(\alpha)}} \xvec^T
\]




\subsection[keyRepresentation]{Representation theory of $GL_n$}

See introduction in e.g. \cite{FanGuoPengSun2020}.





\subsection[keyProduct]{Products of key polynomials}

A product of key polynomials is not in general key positive. 
However, in some instances such a product is key positive, see \cite{Kouno2018}.


\todo{survey of models for key polys: https://guilhermezeus.com/maththesis.pdf}



\begin{conjecture}[Attributed to unpublished work by Reiner--Shimozono]
A product of two key polynomials expands positively into Demazure atoms.
\end{conjecture}
See \cite{Pun2016Thesis} for special cases of this result (she verifies the conjecture in $\setZ[x_1,x_2,x_3]$).



\subsection[keyPieri]{Pieri rule for key polynomials}

A Pieri rule for key polynomials is described via 
an \hyperref[rsk]{Robinson--Schensted--Knuth}-type algorithm is given by \name{Sami Assaf} and \name{Danjoseph Quijada} in \cite{AssafQuijada2019x}.


\todo{Write more about key polynomial pieri rule.}

\todo{Add skew key family}

\subsection[keySkew]{Skew key polynomials}

In \cite{AssafvanWilligenburg2019}, the authors introduce skew key polynomials,
and show that these are positive in the key basis. They use weak \hyperref[dualEquivalence]{dual equivalence} 
to obtain their result. 
This generalizes an earlier result by \name{Vic Reiner} and \name{Mark Shimozono}, \cite{ReinerShimozono1995}.



\section[atom]{Demazure atoms}

\begin{polydata}{atom}
  Name   & Demazure atoms \\
  Space    & All \\
  Basis    & True \\
  Rating   & 4 \\
  Bib      & Lascoux1990Keys \\
  Year     & 1990 \\
  Symbol   & $\atom_{\lambda,\sigma}(z)$ \\
  Keywords & divided-difference, fillings, polytopes, vertex model \\
  Category & Schur \\
\end{polydata}


\todo{
5color vertex model  : 
http://de.arxiv.org/pdf/1902.01795.pdf
}

The Demazure atoms (originally called \defin{standard bases})
may be defined in a manner similar to \hyperref[keyDefinitionOperators]{key polynomials} using operators.
They were introduced by Lascoux and Schützenberger in \cite{Lascoux1990Keys}.
Using the notation for divided difference operators, let $\theta_i \coloneqq \pi_i -1$.
These satisfy the braid relations and we have $\theta_i \theta_j = \theta_j \theta_i$
whenever $|i-j| \geq 2$. However, note that $\theta_i^2 = 0$,
so we must now pay extra attention to choices of permutations.

Let $\lambda$ be a partition with at most $n$ parts, and let $\sigma \in \symS_n$
be a minimal coset representatative in $\symS_n / Stab_n(\lambda)$.
The \defin{Demazure atom} $\atom_{\lambda,\sigma}(z)$ is defined as
\begin{equation*}
 \atom_{\lambda,\sigma}(z) \coloneqq \theta_\sigma\left( z_1^{\lambda_1} \dotsm z_n^{\lambda_n} \right).
\end{equation*}
Again, instead of using $(\lambda,\sigma)$ as index, we simply use $\sigma(\lambda)$
as index. 
Here, we must must choose $\sigma$ to be the \emph{shortest permutation} $\sigma$
which sends $\lambda$ to $\alpha = \sigma(\lambda)$.

For example, if $\alpha = 1113$, then $\lambda=3111$, 
and we must pick $\sigma = 4123$, and 
\[
\atom_{1113}(z) = \atom_{3111,4123}(z) = 
z_1 z_2 z_3 z_4^3 + 
z_1 z_2 z_3^2 z_4^2 + 
z_1 z_2^2 z_3z_4^2 + 
z_1^2 z_2 z_3 z_4^2.
\]
Similarly,
\[
 \atom_{103}(z) = = \atom_{310,312}(z)  z_1^3 z_2  + z_1^3 z_3  + z_1^2 z_2^2 +  z_1^2 z_3^2 +  z_1^2 z_2 z_3 + 
  z_1 z_2^3 + z_1 z_3^3 + z_1 z_2 z_3^2 + z_1 z_2^2 z_3.   
\]

Since $\pi_i = \theta_i+1$, one can easily show that key polynomials are 
positive in the Demazure atom basis. 
In fact we have that (see \cite[Thm. 2.8]{Pun2016Thesis})
\[
	\key_{\alpha}(z) = \sum_{\beta \leq \alpha} \atom_{\beta}(z),
\]
where we have $\beta \leq \alpha$ if one can obtain 
$\beta$ from $\rev(\alpha)$ by successively permuting entries which are
in increasing order.
For example, 
\[
 \key_{301} = \atom_{103} + \atom_{130}+\atom_{301}+\atom_{310}
\]
since $103$, $130$, $301$ and $310$ can be obtained from $\rev(301)=103$
by successive swaps of elements (see also \hyperref[bruhatOrder]{Bruhat order}).
Similarly,
\[
\key_{002} = \atom_{200} + \atom_{020} + \atom_{002}.
\]
\emph{Again, note that the notation used here might differ from yours.}

\begin{example*}[Key polynomials into atoms]
We have that
\[
 \key_{202} = \atom_{202}+\atom_{220}
\]
and 
\[
 \key_{301} = \atom_{103} + \atom_{130}+\atom_{301}+\atom_{310}
\]
\end{example*}

See \name{AnnaPun}s PhD thesis for a good overview of Demazure atoms and key polynomials \cite{Pun2016Thesis}.


\subsection[atomCauchyKernel]{Cauchy kernel}

A bijective proof of the Cauchy-type identity below is given in \cite[Thm. 6]{Lascoux2003}.
The proof uses the \hyperref[rsk]{Robinson--Schensted--Knuth correspondence}.

\begin{theorem}[See \cite[Thm. 6]{Lascoux2003}]
We have that
\[
\prod_{i+j \leq n+1} (1-x_iy_j)^{-1} = 
\sum_{\alpha \in \setN^n} \atom_{w_0\alpha}(\xvec) \key_{\alpha}(\yvec).
\]
\end{theorem}

In \cite{FuLascoux2009}, the authors prove this formula and several other Cauchy-type 
identities involving key polynomials and Demazure atoms in other types.

A generalization to \emph{truncated staircase shapes} is provided 
by \name{Olga Azenhas} and \name{Aram Emami} in \cite{AzenhasEmami2015}. 
They use a version of RSK introduced by \name{Sarah Mason} in \cite{Mason2008}.

An alternative proof also appear in \cite{ChoiKwon2017} where the 
authors use $U_q(\mathfrak{g})$ \hyperref[crystals]{crystals}.

See also \cite{AzenhasGobetLecouvey2022x} for the connection between the Cauchy kernel, 
\hyperref[RSK]{RSK}, \hyperref[crystals]{crystals} and last passage percolation.


\begin{example}[Personal communication, Vasu Tewari 2019]
Recall the Cauchy identity for 
\hyperref[schubert]{Schubert polynomials} (see e.g. \cite[p. 30]{PostnikovStanley2008}).
\[
 \sum_{w \in \symS_n} \schubert_w(\xvec) \schubert_{w w_0}(\yvec) = \prod_{i+j \leq n} (1 + x_iy_j) =
 \prod_{k=1}^{n-1}\sum_{j=0}^k y_{n-k}^{k-j} \elementaryE_{j}(x_1,\dotsc,x_k).
\]
Since the Schubert polynomials are positive in the key basis,
it follows that the \hyperref[schubertAsSumOfElementary]{standard elementary monomials}
are key-positive.
\end{example}




\section[keysAndAtomsTypeBCD]{Demazure keys and Demazure atoms in type $B$, $C$ and $D$}


In \cite{FuLascoux2009}, \name{Amy Fu} and \name{Alain Lascoux} define divided difference operators
that allow us to define key polynomials and Demazure atoms in types $B$, $C$ and $D$.

\todo{Write down the operators.}

See \href{http://www.mat.uc.pt/~oazenhas/KC.pdf}{these slides} for a nice overview.
In particular, the type $C$ key polynomials contain the \hyperref[schurSymplectic]{symplectic Schur functions}.
There is an operator definition, crystal graph definition and a key-tableau definition of these.

See also \cite{RamYip2011}, as key polynomials are specializations of non-symmetric Macdonald polynomials.


\section[keyTypeBC]{Type $B/C$ key polynomials}

\begin{polydata}{keyTypeBC}
  Name   & Type $B/C$ key polynomials \\
  Space    & All \\
  Basis    & True \\
  Rating   & 1 \\
  Bib      & FuLascoux2009\\
  Year     & 2009\\
  Category & Schur \\
\end{polydata}

In \cite{Santos2021}, \name[J. M. Santos]{João Miguel Santos} gives 
combinatorial formulas for the type $C$ key polynomials and Demazure atom polynomials.
The type $C$ key polynomials expand positively into type $C$ Demazure atoms,
by summing over an ideal in the Bruhat poset, see \cite[Prop. 53]{Santos2021}.
This is analogous to the type $A$ setting.
A crystal characterization is also given, as well as analogs of key tableaux.



\section[atomsTypeBC]{Type $B/C$ Demazure atoms}

\begin{polydata}{atomsTypeBC}
  Name   & Type $B/C$ Demazure atoms \\
  Space    & All \\
  Basis    & True \\
  Rating   & 1 \\
  Bib      & FuLascoux2009\\
  Year     & 2009\\
  Category & Schur \\
\end{polydata}


See \cite{FuLascoux2009} for definitions of key polynomials in other types.
These are related to the \hyperref[schurOrthogonal]{orthogonal} 
and \hyperref[schurSymplectic]{symplectic} Schur functions.



