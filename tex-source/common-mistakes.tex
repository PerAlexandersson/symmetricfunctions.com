
\metatitle{Per Alexandersson Common mistakes when writing mathematics}
\metadescription{Common mistakes when writing mathematics, and how to fix them}

\section[sec-mistakes-intro]{Introduction}

First and foremost, a mathematical text with equations, expressions and symbols is still a text.
It should still be built with complete sentences, ending with a period.
The text should have a natural flow when read.
\emph{It is a good idea to read the text out loud, even all the symbols and equations!}



\subsection[sec-mistakes-symbols]{Symbols}

It is common to overuse symbols or being redundant.
\begin{example}
\emph{This implies that}
$
\implies x^2=2+x.
$
The implication arrow ($\implies$) is redundant.
\end{example}

Symbols should usually not replace words in inline text.
\begin{example}
\emph{We have that $x=2 \wedge x=3$ are solutions.}

It is better to just write \emph{We have that $x=2$ and $x=3$ are solutions.}
\end{example}


Symbols are mainly used when making definitions, where it is important to avoid ambiguity.
\begin{example}
\emph{An injective function is a function such that whenever $x$ and $y$ are different values in its domain, the function values 
at $x$ and $y$ are different.}

Notice that this definition is rather difficult to parse. 
The following definition is easier to understand.

\emph{
A function $f$ is injective if $x\neq y \implies f(x) \neq f(y)$ whenever $x,y \in D_f$.
}

\end{example}

\subsection[sec-mistakes-sentece-start]{Inline math expressions}

Do not start sentences with a mathematical expression.
This usually looks strange and more often than not, sound strange when read out loud.

\begin{example}
\emph{$x=2$ is the only solution.}

Better write 
\emph{The only solution is $x=2$.}
\end{example}




\subsection[sec-mistakes-equation-mode]{Equation mode expressions}

A centered equation should also be part of a sentence.

\begin{example}
\emph{We add two to both sides.}
\[
(x-2) + 2 = (x^2-4)+2
\]
\emph{Both sides are now simplified.}
\[
x = x^2-2
\]

Notice that there is no real sentence structure here. 
Rewrite as follows instead
and notice the period and comma in the equations.

\emph{We add $2$ to both sides and get
\[
(x-2) + 2 = (x^2-4)+2,
\]
which then simplifies to the equation
\[
x = x^2-2.
\]
}
\end{example}


\subsection[sec-mistakes-colon]{Usage of colon and semicolon}

There is usually no need to use a colon, or a semicolon before an equation.

\begin{example}
\emph{The identity:}
\[
\sin^2(x) + \cos^2(x)=1
\]
\emph{is called the "Pythagorean trigonometric identity".}

Here, the colon makes no sense, as the identity is part of the statement.
However, for other sentence structures, it makes sense to use a colon or semicolon.

\emph{Consider the following identity, which is called the Pythagorean trigonometric identity:}
\[
\sin^2(x) + \cos^2(x)=1.
\]

In this case, the formula does not fit into the sentence as it is --- we need the semicolon to start a new phrase.
\end{example}





\section[sec-mistakes-type]{Problems with types}

\subsection[sec-mistakes-types-intro]{Confusing notation}

We are exposed to different \emph{types} (as in programming) in mathematics.
Think of these as units in physics.
Some common types are \emph{numbers}, \emph{statements}, \emph{functions}, \emph{matrices}, etc.
Note that for example, $3$ can be seen as both a number and a function.
We have different notation for dealing with equality between objects of different types.

Make sure that equivalence arrow ($\iff$) is only between statements (things that can be true or false).
Arrows $(\to)$ are used for various purposes, for example when talking about limits.
Equality $(=)$ is for things which have the same (numerical, usually) value.
It is unfortunate that we use $=$ both for \emph{identically equal to} (i.e, equal for all values of parameters),
and when talking about \emph{equations}, where we mainly are interested in for which values equality holds.
For example, the \emph{identity} $\sin^2(x)+\cos^2(x)=1$ is an equality between \emph{functions},
while $\sin(x)+\cos(x)=1$ is interpreted as an \emph{equation} and thus $=$ is now equality between \emph{numbers}.


Equality is also used for introducing new notation, or making substitutions.
For example, in the middle of
\[
\lim_{x\to 1} \frac{(\sqrt{x}-1)^2}{x(\sqrt{x}-1)} = 
\left[
\begin{smallmatrix}
t = \sqrt{x} \\
t^2 = x
\end{smallmatrix}
\right] = 
\lim_{t\to 1} \frac{(t-1)^2}{t^2(t-1)} 
\]
we make a substitution by introducing new notation. In this context, $t = \sqrt{x}$ 
can be seen as introducing the function $t(x) = \sqrt{x}$.

When introducing new functions or notation, mathematicians sometimes write $\coloneqq$.
This is read as \emph{defined as}. For example,
\[
|x| \coloneqq 
\begin{cases}
x \text{ if } x\geq 0 \\
-x \text{ otherwise}.
\end{cases}
\]
This is read out loud as 
\begin{blockquote}
Let the absolute value of x be defined as x, if x is greater-than-or-equal-to zero,
and minus x otherwise.
\end{blockquote}


\subsection[sec-mistakes-types-equal]{Example with equality signs}

A common source of confusion is the mixing of equality-as-expression with equation-equality.
\begin{example}
Consider the following fragment from a solution:

\emph{
In order to find extremal points of $f(x)=x^3+2x-\cos(x)$, we set the derivative to 0.
That is, $f'(x) = 3x^2+2+\sin(x)=0$.
}

The problem here is that the first equality sign is the \emph{identity} $f'(x) = 3x^2+2+\sin(x)$. This is true for all value of $x$.
The second equality sign $3x^2+2+\sin(x)=0$ is an \emph{equation} and only true for some particular values of $x$.
What is \emph{not} true, is that $f'(x)=0$ for all $x$.

We can instead write as follows in order to avoid mixing types of equality.

\emph{
In order to find extremal points of $f(x)=x^3+2x-\cos(x)$, we want to find zeros of the derivative.
The derivative is given by $f'(x) = 3x^2+2+\sin(x)$,
so we need to solve the equation $3x^2+2+\sin(x) = 0$.
}

This is a bit more verbose, but it has the advantage of being correct.

\end{example}



\subsection[sec-mistakes-types-limits]{Examples with limits}


We shall now see a few common mistakes in the context of limits.

\begin{example}
\[
\text{Incorrect: } 
\lim_{t\to 0} \frac{t^3-1}{t-1} \to 3
\qquad 
\text{Correct: } 
\lim_{t\to 0} \frac{t^3-1}{t-1} = 3.
\]
\emph{Explanation:} A limit is either a number, (or $\pm \infty$ or undefined), so equality sign should be used.
\end{example}

\begin{example}
\[
\text{Incorrect: } 
\lim_{t\to 0} \frac{t^3-1}{t-1}  = \frac{t^2+2t+1}{1}
\qquad 
\text{Correct: } 
\lim_{t\to 0} \frac{t^3-1}{t-1} = \lim_{t\to 0} \frac{t^2+2t+1}{1}.
\]
\emph{Explanation:} The first one is incorrect, as it states that the number $3$ is equal to the expression $\frac{t^2+2t+1}{1}$.

Perhaps more true to the writers intention, one could alternatively express the identity as
\[
\lim_{t\to 0} \frac{t^3-1}{t-1}  = \left. \frac{t^2+2t+1}{1} \right\vert_{t=1}.
\]
Here, $\left. \frac{t^2+2t+1}{1} \right\vert_{t=1}$ is a commonly established shorthand for 
\[
f(1) \text{ where } f(t) = \frac{t^2+2t+1}{1}.
\]
In computer science lingo, this notation allows us to evaluate an anonymous function with a particular argument.
We read $t^2+4\vert_{t=1}$ out loud as 
\begin{blockquote}
Tee-squared-plus-four evaluated at tee-equals-one.
\end{blockquote}

\end{example}


\begin{example}
\[
\text{Incorrect: } 
 \frac{t^3-1}{t-1}  = 3 \text{ when $t=1$}.
\qquad 
\text{Correct: } 
\frac{t^3-1}{t-1}  \to 3 \text{ as $t\to1$}.
\]
\emph{Explanation:} The expression is not defined at $t=1$. However, it is true that the limit is $3$,
and we are justified to use the arrows to express this. The latter statement is read out loud as
\begin{blockquote}
Tee-cubed minus one, over tee-minus-one approaches three, as tee approaches one.
\end{blockquote}

\end{example}


\begin{example}
The following is incorrect:
\[
 \lim_{x\to 0} \frac{e^x-1}{x} \frac{\sin(x)}{x} \implies  
 \lim_{x\to 0} \frac{e^x-1}{x} \lim_{x\to 0} \frac{\sin(x)}{x}.
\]
We do not use implication between values. To make it correct, put an equality sign instead.
What is meant is perhaps that the equality is a consequence of the following implication,
correctly stated as:
\[
 \lim_{x\to 0} \frac{e^x-1}{x} = A \text{ and } \lim_{x\to 0} \frac{\sin(x)}{x} \implies 
 \lim_{x\to 0} \frac{e^x-1}{x} \frac{\sin(x)}{x} = AB.
\]
Note that this implication in general has some restrictions on $A$ and $B$.
\end{example}


\begin{example}
The following is incorrect, for the same reason as above.
\[
 \lim_{x\to 0} \sqrt{ \frac{e^{2x}-1}{x} } \implies  
 \sqrt{\lim_{x\to 0} \frac{e^{2x}-1}{x} }.
\]
Here, it would be clearer to give the limit a name:
\[
\text{Let } A \coloneqq \lim_{x\to 0} \sqrt{ \frac{e^{2x}-1}{x} }, \text{ then } \lim_{x\to 0} \frac{e^{2x}-1}{x} = \sqrt{A}.
\]
Since $\sqrt{A}=2$, we find that $A=4$.
\end{example}


