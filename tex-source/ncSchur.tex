\metatitle{Symmetric functions in non-commutative variables}
\metadescription{Symmetric functions in non-commutative variables}




\section[schurInNonCom]{Symmetric functions in noncommuting variables}


The content on this page is mainly based on \cite{AliniaeifardLiWilligenburg2022},
and has been contributed by \href{https://sites.google.com/view/aliniaeifard/home}{Farid Aliniaeifard}.



The graded \defin{Hopf algebra of symmetric functions in noncommuting variables}, $\mathrm{NCSym}$,
\[
\mathrm{NCSym}  = \mathrm{NCSym}^0 \oplus \mathrm{NCSym}^1 \oplus \cdots \subset 
\setQ \langle \langle x_1, x_2, \dotsc \rangle\rangle
\]
where $\langle \langle\cdot \rangle\rangle$ means that the variables do not 
commute, $\mathrm{NCSym}^0 = \mathrm{span} \{1\}$ and the $n$th graded piece 
for $n\geq 1$ has the following bases \cite{RosasBruce2006}, known respectively as 
the $n$th graded piece of the \defin{$m$-, $p$-, $e$-, $h$-basis of $\mathrm{NCSym}$},
\[
\mathrm{NCSym}^n = \mathrm{span}\{ m_\pi : \pi\vDash [n]\} = \mathrm{span}\{ p_\pi : \pi\vDash [n]\}
= \mathrm{span}\{ e_\pi : \pi\vDash [n]\} = \mathrm{span}\{  {h_\pi} : \pi\vDash [n]\}
\]
where these functions are defined below, and $\pi$ is a set partition of $[n]$.

The symmetric functions in noncommuting variables (indexed by set partitions) is a 
superset of noncommutative symmetric functions (indexed by integer compositions).



The \defin{monomial symmetric function in $\mathrm{NCSym}$}, $m_\pi$, is given by 
\[
m_\pi = \sum _{(i_1, i_2, \ldots , i_n)} x_{i_1}x_{i_2} \cdots x_{i_n}
\]
summed over all tuples $(i_1, i_2, \ldots , i_n)$ with $i_j=i_k$ 
if and only if $j$ and $k$ are in the same block of $\pi$.

\begin{example*}[Expansion of $m_{13|2}$]
$m_{13|2}=x_1x_2x_1+x_2x_1x_2+x_1x_3x_1+x_3x_1x_3+x_2x_3x_2+x_3x_2x_3+\cdots$
\end{example*}


The \defin{power sum symmetric function in $\mathrm{NCSym}$}, $p_\pi$, is given by 
\[
p_\pi = \sum _{(i_1, i_2, \ldots , i_n)} x_{i_1}x_{i_2} \cdots x_{i_n}
\]
summed over all tuples $(i_1, i_2, \ldots , i_n)$ with $i_j=i_k$ if  $j$ and $k$ are in the same block of $\pi$.

\begin{example*}[Expansion of $p_{13|2}$]
$p_{13|2}=x_1x_2x_1+x_2x_1x_2+\cdots + x_1^3+x_2^3 +\cdots $
\end{example*}

The \defin{elementary symmetric function in $\mathrm{NCSym}$}, $e_\pi$, is given by 
\[
e_\pi = \sum _{(i_1, i_2, \ldots , i_n)} x_{i_1}x_{i_2} \cdots x_{i_n}
\]
summed over all tuples $(i_1, i_2, \ldots , i_n)$ with $i_j\neq i_k$ if  $j$ and $k$ are in the same block of $\pi$.

\begin{example*}[Expansion of $e_{13|2}$]
$e_{13|2}= {x_1x_1x_2+x_1x_2x_2+x_2x_2x_1+x_2x_1x_1}+\cdots + x_1x_2x_3+x_2x_3x_4 +\cdots$
\end{example*}

The \defin{complete homogeneous symmetric function} in $\mathrm{NCSym}$, $h_\pi$, is given
by
\begin{equation}
h_{\pi}=\sum_{\eta} \sum_{(i_1,i_2,\ldots,i_n)  } x_{i_{\eta(1)}}x_{i_{\eta(2)}}\cdots x_{i_{\eta(n)}}
\end{equation} 
where
the first sum is over all $\eta\in \symS_n$ that fixes the blocks of $\pi$,
and, the second sum is over all $(i_1,i_2,\ldots,i_n)\in \setN^n$ such that if $j$ and $k$ are in the same block of $\pi$ with $j \lt k$, then $i_j\leq i_k$.

\begin{example*}[Expansion of $h_{13|2}$]
$h_{13|2}= 2 m_{123} + m_{12|3} + m_{1|23} + 2 m_{13|2} + m_{1|2|3}$
\end{example*}


The \emph{permutation map} \cite[p. 219]{RosasBruce2006} and \cite[p. 230]{GebhardSagan2001}, 
which is an action on places (not variables), is defined as follows. 
Given $\delta \in \symS_n$ and a monomial of degree $n$ in noncommuting variables, define
\[
\delta \circ (x_{i_1}x_{i_2} \cdots x_{i_n}) = x_{i_{\delta^{-1}(1)}}x_{i_{\delta^{-1}(2)}} \cdots x_{i_{\delta^{-1}(n)}}
\]
and extend linearly. In \cite[p. 219]{RosasBruce2006} they also noted that if $\pi$ is a basis element of any of the above bases of $\mathrm{NCSym}$ and $\delta$ is a permutation, then

\begin{equation}
\delta \circ b_\pi = b_{\delta\pi}
\end{equation}
where $\delta$ acts on set partitions in the natural way.


The  \defin{noncommutative analogue of Leibniz' determinantal formula} 
for any matrix $A=(a_{ij}) _{1\leq i,j\leq n}$ with noncommuting entries $a_{ij}$ 
is defined to be
\begin{equation}
\mathbf{det}(A) = \sum_{\varepsilon \in \symS_n} \sign (\varepsilon) a_{1\varepsilon (1)}a_{2\varepsilon (2)}\cdots a_{n\varepsilon (n)}
\end{equation}
that takes the product of the entries from the top 
row to the bottom row, and $\sign (\varepsilon)$ is 
the sign of permutation $\varepsilon$.



\section[sourceSchur]{Source Schur functions}

Before the definition of Schur functions in noncommuting variables, 
let's define functions that will be their genesis.

\begin{definition}
Let $\lambda / \mu$ be a skew diagram. 
Then the \defin{source skew Schur function in noncommuting variables} 
$s_{[\lambda/\mu]}$ is defined to be
\begin{equation}
s_{[\lambda/\mu]} = \mathbf{det} \left( \frac{1}{(\lambda _i -\mu _j - i +j)! } 
h_{[\lambda _i -\mu_j - i + j]}\right) _{1\leq i,j \leq \ell(\lambda)}
\end{equation} 
where we set $\mu_j = 0$ for all $\ell(\mu) \lt j \leq \ell(\lambda)$, $h_{[0]}=h_\emptyset = 1$ 
and any function with a negative index equals 0. 
When $\mu = \emptyset$, we call $s_{[\lambda]}$ 
a \defin{source Schur function in noncommuting variables}.
\end{definition}


\begin{example}
The source  Schur function in noncommuting variables $s_{[21]}$ is
\begin{align*}s_{[21]} &  = {\mathbf{det} \begin{pmatrix} \frac{1}{2!} h_{[2]}& \frac{1}{3!} h_{[3]}\\
\frac{1}{0!} h_{[0]}& \frac{1}{1!} h_{[1]}
\end{pmatrix}} = \mathbf{det} \begin{pmatrix} \frac{1}{2!} h_{12}& \frac{1}{3!} h_{123}\\
\frac{1}{0!} h_{\emptyset}& \frac{1}{1!} h_{1}
\end{pmatrix}\\
&= \frac{1}{2!} h_{12} \frac{1}{1!} h_{1} - \frac{1}{3!} h_{123}\frac{1}{0!} h_{\emptyset} = \frac{1}{2} h_{12|3} - \frac{1}{6} h_{123}.
\end{align*}

Meanwhile, the source skew Schur function in noncommuting variables 
$s_{[22|1]}$ is
\begin{align*}s_{[22|1]} &=  {\mathbf{det} \begin{pmatrix} \frac{1}{1!} h_{[1]}& \frac{1}{3!} h_{[3]}\\
\frac{1}{0!} h_{[0]}& \frac{1}{2!} h_{[2]}
\end{pmatrix}} =  \mathbf{det} \begin{pmatrix} \frac{1}{1!} h_{1}& \frac{1}{3!} h_{123}\\
\frac{1}{0!} h_{\emptyset}& \frac{1}{2!} h_{12}
\end{pmatrix}\\
&= \frac{1}{1!} h_{1}\frac{1}{2!} h_{12}  - \frac{1}{3!} h_{123}\frac{1}{0!} h_{\emptyset} = \frac{1}{2} h_{1|23} - \frac{1}{6} h_{123}.
\end{align*}
\end{example}




\section[ncSchur]{Schur functions in non-commuting variables}

\subsection[NCSYMBases]{The standard  and permuted bases}

Recall the definition of \hyperref[skewShapes]{skew diagrams} and \hyperref[prelimTableaux]{standard Young tableaux}.
In particular, the \defin{permutation} $\delta_T \in \symS_n$ of a tableau $T$ 
is obtained by concatenating the rows of $T$, from first to last row.
This definition makes sense as long as entries from $[n]$ appears exactly once; 
the tableau is not required for this definition to make sense.
Observe that this is different from the \emph{reading word of $T$}, where rows are read in a different order!
Evidently, we have that $T \leftrightarrow (\delta_T, sh(T))$ is a bijection, where $sh(T)$ is the shape of $T$.

Now consider the set of all Young tableaux $T$ such that
\begin{enumerate}
\item $sh(T) = \lambda$ for some fixed integer partition $\lambda\vdash n$,
\item the entries in each row of $T$ increase from left to right,
\item if $\lambda = \lambda _1 \lambda _2\cdots \lambda _{\ell(\lambda)}$ 
and $\lambda _i = \lambda _j$ with $i\lt j$, then in $T$
\[
\text{(the first entry of row $i$)}  \lt \text{(the first entry of row $j$)}.
\]
\end{enumerate}

Observe that this set is in bijection with the set consisting
of all set partitions $\pi$ of $\vDash [n]$: 
the Young tableau $T_\pi$ corresponds to set partition $\pi$ if and only if the set 
of entries for each row of $T_\pi$ are precisely the blocks of $\pi$,
and that the integer partition determined by 
the block sizes of $\pi$ are given by $sh(T_\pi)$.
In this case, define the permutation $\delta_\pi \coloneqq \delta_{T_\pi}$. 
Informally, $\delta_\pi$ is obtained from the set-partition $\pi$
by sorting blocks by length decreasingly; blocks with smaller first entry are placed first.
Finally, bars separating blocks are erased.

\begin{example*}[Obtaining $\delta_T$ from tableaux]
If $T$ is
\begin{figure}
\begin{ytableau}
\none & \none & 3 & 8 & 7 \\
\none &\none &2\\
1 & 9 & 6 \\
5&4
\end{ytableau}
\end{figure}
then $\delta_T = 387219654$.
If $T_\pi$ is
\begin{figure}
\begin{ytableau}
1&6&9 \\
3&7&8 \\
4 & 5 \\
2
\end{ytableau}
\end{figure}
then $\pi = 169|378|45|2 = 169|2|378|45$ and $\delta_\pi = \delta_{T_\pi} = 169378452$.
\end{example*}





\begin{polydata}{ncSchur}
  Name     & Schur functions in non-commuting variables \\
  Space    & NCSym \\
  Basis    & Yes \\
  Rating   & 2 \\
  Bib      & AliniaeifardLiWilligenburg2022 \\
  Keywords & non-commutative \\
  Year     & 1995 \\
  Category & Schur \\
\end{polydata}


\begin{definition}
Let $\lambda/\mu$ be a skew diagram of size $n$ and $\delta \in \symS_n$. 
Then the \defin{skew Schur function in noncommuting variables} 
$s_{(\delta, \lambda /\mu)}$ is defined to be
\begin{equation}\label{eq:skewNCSchur}
s_{(\delta, \lambda /\mu)} = \delta \circ s_{[\lambda/\mu]} = \delta \circ \mathbf{det} \left( \frac{1}{(\lambda _i -\mu _j - i +j)! } h_{[\lambda _i -\mu_j - i + j]}\right) _{1\leq i,j \leq \ell(\lambda)}.
\end{equation}
Moreover, if $\mu = \emptyset$, then we call $s_{(\delta, \lambda)}$ a \defin{Schur function in noncommuting variables}. 

Furthermore, if $\pi \vDash [n]$ and $\lambda (\pi) = \lambda _1 \lambda _2 \cdots \lambda _{\ell(\pi)}$, 
then the \defin{standard Schur function in noncommuting variables} $s_\pi$ is defined to be
\begin{equation}
s_{\pi} = s_{(\delta _\pi, \lambda (\pi))}= \delta _\pi \circ s_{[\lambda (\pi)]} = 
\delta_\pi \circ \mathbf{det} 
\left( \frac{1}{(\lambda   _i  - i +j)! } h_{[\lambda   _i  - i + j]}\right) _{1\leq i,j \leq \ell(\lambda(\pi))}.
\end{equation}
\end{definition}



\begin{example*}[Schur functions in non-commuting variables]
If $\pi = 12|3$, then $\delta _\pi = 123 = \mathrm{id}$. 

Hence, the standard Schur function in noncommuting variables $s_{12|3}$ is
\begin{align*}
s_{12|3} &= \mathrm{id} \circ s_{[21]} = \mathrm{id} \circ \mathbf{det} 
\begin{pmatrix} \frac{1}{2!} h_{12}& \frac{1}{3!} h_{123}\\
\frac{1}{0!} h_{\emptyset}& \frac{1}{1!} h_{1}
\end{pmatrix}\\
&= \frac{1}{2!} h_{12} \frac{1}{1!} h_{1} - \frac{1}{3!} h_{123}\frac{1}{0!} h_{\emptyset} = \frac{1}{2} h_{12|3} - \frac{1}{6} h_{123}. 
\end{align*}
If $\pi = 13|2$, then $\delta _\pi = 132$. 
Hence, the standard Schur function in noncommuting variables $s_{13|2}$ is
\[
s_{13|2} = 132 \circ s_{[21]} = 132\circ \left(\frac{1}{2} h_{12|3} - \frac{1}{6} h_{123}\right) = \frac{1}{2} h_{13|2} - \frac{1}{6} h_{123}.
\]
\end{example*}

\begin{theorem}
The set $\{ s_\pi \} _{\pi \vDash [n], {n\geq 0}}$ is a basis for $\mathrm{NCSym}$. 
\end{theorem}

\todo{Link these other non-commutative families.}
See \cite{AliniaeifardLiWilligenburg2022} for connections with the immaculate Schur functions and ribbon Schur functions.
