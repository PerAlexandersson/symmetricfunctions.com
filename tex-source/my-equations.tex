\metatitle{Equations from my research}

\metadescription{Equations which I proved, or was part of proving}

The term equation is not really accurate, as these are really \emph{identities}.
However, the general public seem to use the term equation to
mean \emph{a bunch of math symbols with an equality sign somewhere}.

I take no responsibility for regrets if you decide 
to get a tattoo of any of these --- but if you get one, send me a picture!


\section[my-equations-catalan]{$q$-Catalan refinement}

We have the identity
\begin{equation*}
\catalan(n-2;q) = 
\sum_{k \geq 0}
q^{k(k-2)} \frac{[n]_q}{[k]_q} \qbinom{n-4}{2k-4}_q\catalan(k-2;q)
\left( \sum_{j=0}^{n-2k} q^{j(n-2)}\qbinom{n-2k}{j}_q\right).
\end{equation*}

This is a refinement of the \hyperref[prelimQanalogsCatalan]{$q$-Catalan numbers}.


\section[my-equations-llt-e]{Unicellular LLT $e$-expansion}

I conjectured, and a few years later together with R. Sulzgruber
\cite{AlexanderssonSulzgruber2020x}, proved the following 
\hyperref[unicellularLLTEExpansion]{formula for unicellular LLT polynomials}
in the elementary symmetric functions basis.

For any unit-interval graph $\avec$ with $n$ vertices
we have the expansion
\begin{equation*}
\LLT_\avec(\xvec;q+1) =
\sum_{\theta \in O(\avec)} q^{\asc(\theta)}\elementaryE_{\pi(\theta)}(\xvec).
\end{equation*}

\section[my-equations-BSD]{Border-strip decompositions of a rectangle}

Let $a(n)$ be the number of ways to tile a $2n \times n$ rectangle with 
border-strips of size $n$. Then with $a(0)=1$, we have 
\[
 a(n) = 
 \frac{1}{2} \sum_{i=1}^n \frac{ i(n - i + 1)}{ (n + 2)} \binom{n - 1}{i - 1} 
\binom{n + 3}{i + 1} a(i - 1) a(n - i).
\]
This is proved in \cite{AlexanderssonJordan2018}. See also \oeis{A115047}.


\section[my-equations-CircularDyck]{Circular Dyck diagrams}

In \cite{AlexanderssonLinussonPotka2019}, we
find the major-index generating function for \hyperref[chromaticQuasisymmetricUnitIntervalGraph]{circular Dyck diagrams}
of size $n$ and width at most $w$.
\begin{equation*}
|\CDP(n,w)|_q =
\sum_{s \in \setZ}
\sum_{j=1}^w
q^{s^2\delta + s(j+1)}
\left(
\qbinom{2n-1}{n-1-\delta s}_q
-
\qbinom{2n-1}{n+j+\delta s}_q
\right),
\end{equation*}
where $\delta = w+2$.

See also \oeis{A194460}.


\section[my-equations-JackKostka]{Normalized Jack--Koskta coefficients}

This is from \cite[Thm. 5.12]{AlexanderssonFeray2017}.
For $\lambda \vdash n$, let
\[
\mathfrak{K}_{(k)}^{(\alpha)}(\lambda) \coloneqq 
\frac{1}{(n-k)!} [\monomial_{k,1^{n-k}}] \jackJ_\lambda(\xvec;\alpha).
\]
which is a normalized hook coefficient in the monomial basis of 
\hyperref[jackJ]{the Jack J symmetric function}.

Then
\[
\mathfrak{K}_{(k)}^{(\alpha)}(\lambda) =  
\sum_{\substack{ A \subseteq \lambda, \ |A|=k \\ \text{column-distinct}}}
\prod_{\substack{R \text{ row} \\ \text{of }\lambda}} P_{|R \cap A|}(\alpha)
,
\]
where  $P_i(\alpha) \coloneqq \prod_{j=0}^{i-1} (1 + j \, \alpha)$.

This equation allow us to see that $\mathfrak{K}_{(k)}^{(\alpha)}(\mathbf{r}^{\mathbf{p}})$
is non-negative in the falling factorial basis (where we use multi-rectangular coordinates).
