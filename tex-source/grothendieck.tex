\metatitle{Grothendieck polynomials}
\metadescription{An introduction to Grothendieck polynomials, including their definitions via divided difference operators and pipe dreams, stable Grothendieck polynomials and their properties such as Jacobi--Trudi identities, Pieri rules, Murnaghan--Nakayama rules, Schur expansions, and skew versions, as well as double Grothendieck polynomials and canonical stable Grothendieck polynomials.}  

\todo{ Set-valued Rothe diagram formula for other classes of permutations
https://arxiv.org/pdf/1908.04164.pdf

Gustavsson--Milne formula for Factorial Grothendieck:
http://de.arxiv.org/pdf/1812.04390.pdf

Crystal for Lascoux polynomials and Grothendiecks
https://arxiv.org/pdf/1807.03294.pdf

Affine Grothendieck
https://arxiv.org/pdf/math/0501335.pdf

Info about double grothendieck:
http://de.arxiv.org/pdf/1711.00147.pdf
http://de.arxiv.org/pdf/math/0502144.pdf (Knuthson)


Skew Grothendieck and Cauchy identity
https://doi.org/10.1016/j.jcta.2018.09.006

Mod-HL in dual Stable grothendieck (signed)
https://hal.inria.fr/hal-01186288/document

Dual stable Grothendieck sums:
http://de.arxiv.org/pdf/1806.06369.pdf
}



\section[grothendieck]{Grothendieck polynomials}


\begin{polydata}{grothendieck}
  Name   & Grothendieck polynomials \\
  Space    & All \\
  Basis    & True \\
  Rating   & 2 \\
  Bib      & Lascoux1990Grothendieck \\
  Year     & 1990 \\
  Symbol   & $\grothendieck_\omega(\xvec)$ \\
  Category & Schubert \\
\end{polydata}


The Grothendieck polynomials are closely related to the \hyperref[schubertOperatorformula]{Schubert polynomials}.
They were introduced by A. Lascoux, 1990 in \cite{Lascoux1990Grothendieck}.



In \cite{BrubakerFrechetteHardtTiborWeber2020x}, the authors 
introduce biaxial Grothendieck polynomials, which generalize the double Grothendieck polynomials as well as 
the double dual Grothendieck polynomials.
They prove several results regarding Grothendieck polynomials, such as branching rules 
and Cauchy identities. They also provide a solvable lattice model whose partition function
is given by the Grothendieck polynomials, and duals.



\subsection[grothendieckOperatorformula]{Operator definition}

Define $\partial_i(f)$ as $\frac{f-s_i(f)}{x_i - x_{i+1}}$ for $i=1,\dotsc,n-1$,
and $\pi_i \coloneqq \partial_i(1-x_{i+1})$.

The Grothendieck polynomials are then defined as
\begin{equation*}
 \grothendieck_\omega(x_1,\dotsc,x_n) \coloneqq \pi_{\omega^{-1}\omega_0}(x_1^{n-1}x_2^{n-2}\dotsm x_{n-1}).
\end{equation*}

Note that the lowest degree homogenous part of $\grothendieck_\omega(\xvec)$ is given by the Schubert polynomial
$\schubert_\omega(\xvec)$.


There is a connection with polytopes just as for Schubert polynomials, see \cite{MeszarosDizier2017}.


\subsection[grothendieckPipedreams]{Pipedreams}


As with Schubert polynomials, there is a way to describe 
Grothendieck polynomials using pipe dreams, see \cite{FominKirillov1994grothendieck,KnutsonMiller2004}.
The double Grothendieck polynomials are given as
\[
\grothendieck^{(\beta)}_\omega(\xvec,\yvec) = 
\sum_{P \in PD(\omega)} 
\beta^{-\length(\omega)}
\prod_{(i,j)\in S(P)} \beta x_i \oplus y_j
\]
where we sum over \emph{pipe dreams}
and $a \oplus b = a+b+\beta ab$.
Here, $S(P)$ is the set of crosses the pipe dream (this particular version of the formula is from
\cite[Thm.6.1]{Weigandt2021}).



A more recent formula expresses the the double Grothendieck polynomials 
as a sum over \emph{bumpless pipe dreams} \cite[Thm.1.1]{Weigandt2021}:
\[
\grothendieck^{(\beta)}_\omega(\xvec,\yvec) = 
\beta^{-\length(\omega)} 
\sum_{P \in BPD(\omega)} 
\prod_{(i,j)\in D(P)} \beta (x_i \oplus y_j )
\prod_{(i,j)\in NW(P)} 1+ \beta (x_i \oplus y_j )
\]
where $BPD(\omega)$ is the set of \defin{bumpless pipe dreams} of type $\omega$
and $NW(P)$ is the set of West-to-North pipe bends.

However, this formula does not immediately give the same pipe dream expression, 
as the Fomin--Kirillov formula, and it is an open problem 
to find a bijection between the two models.

A lattice model is also given in \cite{BuciumasScrimshaw2020x}, and a connection with bumpless pipe dreams is given.



\subsection[grothendieckASM]{Alternating sign matrices}


There is an expression for Grothendieck polynomials using alternating sign matrices,
due to A. Lascoux.


\section[grothendieckStable]{Stable Grothendieck polynomials}


\begin{polydata}{grothendieckStable}
  Name   & Stable Grothendieck polynomials \\
  Space    & Sym \\
  Basis    & False \\
  Rating   & 2 \\
  Bib      & FominKirillov1994grothendieck \\
  Year     & 1994 \\
  Symbol   & $\grothendieckStable_\omega(\xvec)$ \\
  Category & Schubert \\
\end{polydata}

The \defin{stable Grothendieck polynomials} were introduced by S. Fomin and A. Kirillov in \cite{FominKirillov1994grothendieck,FominKirillov1996YangBaxter}.
Similar to \hyperref[stanleySymLimitDefinition]{Stanley symmetric functions},
the stable Grothendieck polynomials are defined as the limit 
\begin{equation*}
 \grothendieckStable_\omega(\xvec) \coloneqq \lim_{k\to \infty}  \grothendieck_{1^k \times \omega}(\xvec).
\end{equation*}


\section[grothendieckGrassman]{Grassman Grothendieck polynomials}


\begin{polydata}{grothendieckGrassman}
  Name   & Grassman Grothendieck polynomials \\
  Space    & Sym \\
  Basis    & True \\
  Rating   & 5 \\
  Bib      & FominKirillov1994grothendieck \\
  Year     & 1994 \\
  Symbol   & $\grothendieckStable_\lambda(\xvec)$ \\
  Keywords & jacobi-trudi, weyl, operators, cauchy-identity \\
  Category & Schubert \\
\end{polydata}


See \hyperref[schubertGrassmanianPermutation]{this page} for the definition of 
a Grassman permutation, and how to associate a partition with such a permutation.
For a nice recent paper with lots of references and results, see \cite{MonicalPechenikScrimshaw2018}.


Stable Grothendieck polynomials indexed by Grassman permutations are symmetric polynomials.
They serve as a K-theoretic analog of the \hyperref[stanleySym]{Stanley symmetric functions}.
These can be computed using a combinatorial formula, using set-valued tableaux:
\begin{align}
\grothendieckStable_\lambda(\xvec) = \sum_{T \in \mathrm{SVT}(\lambda)} (-1)^{|T|-|\lambda|} \xvec^T.
\end{align}
The set $\mathrm{SVT}(\lambda)$ consists of all set-valued tableaux of shape lambda,
where each box is filled with a non-empty set of natural numbers.
Rows must be weakly increasing and column must be strictly increasing.
The sum of the sizes of the sets in $T$ is denoted $|T|$.

A \hyperref[gtpatterns]{Gelfand--Tsetlin pattern} type combinatorial formula is described in \cite{MotegiSakai2014},
where marked GT-pattern are used to give the monomial expansion of $\grothendieckStable_\lambda$.
See also \cite{MonicalPechenikScrimshaw2018}, where they give an alternative proof of this.


\subsection[grothendieckStableWeyl]{Weyl identity}

A Weyl-type bialternant formula is given by the following quotient,
see \cite{IkedaNaruse2013,Kirillov2016,Iwao2019}.

\begin{align}
\grothendieckStable_\lambda(x_1,\dotsc,x_n) = 
\frac{ \det \left( x_i^{\lambda_j+n-j}(1+\beta x_i)^{j-1} \right)_{1\leq i,j \leq n} }{
\prod_{1 \leq i \lt j \leq n} (x_i - x_j)
} \qquad (\beta = -1).
\end{align}


\subsection[grothendieckStableJacobiTrudi]{Jacobi--Trudi identity}

There is a Jacobi--Trudi type identity, \cite[Thm. 1.10]{Kirillov2016} given by
\begin{align}
\grothendieckStable_\lambda(\xvec) = 
\det\left(
\sum_{m=0}^\infty \binom{i-1}{m} \beta^m \completeH_{\lambda_i-i+j+m}(\xvec)
\right)_{1\leq i,j \leq n}.
\end{align}


\subsection[grothendieckStablePieri]{Pieri rule}

C. Lenart \cite{Lenart2000} proved the following Pieri rule for stable Grassman Grothendieck polynomials:
\[
\grothendieckStable_{(a)}\grothendieckStable_{\lambda} = 
\sum_{\mu / \lambda \text{ horizontal strip}} 
(-1)^{|\mu/\lambda|-a} \binom{ \textrm{rows}(\mu/\lambda)-1 }{ |\mu/\lambda|-a } \grothendieckStable_{\mu}
\]
where $\textrm{rows}(\mu/\lambda)$ is the number of rows in the horizontal strip.



\subsection[grothendieckStableMN]{Murnaghan--Nakayama rule}

In \cite{NguyenHiepSonThuy2023}, the following Murnaghan--Nakayama type rule is the main result.
\begin{align}
\grothendieckStable_\lambda(\xvec) \powerSum_k(\xvec) = 
 \sum_{\mu} (\beta)^{|\mu|-|\lambda|-k} (-1)^{k-c(\mu/\lambda)} \binom{r(\mu/\lambda)-1}{k-c(\mu/\lambda)} 
 \grothendieckStable_\mu(\xvec)
 \end{align}
where the sum runs over all connected skew shapes $\mu/\lambda$,
containing at least one $k$-ribbon along its northwestern border.
Here, $c(\mu/\lambda)$ is the number of columns spanned by the shape. Likewise,
$r(\mu/\lambda)$ is the number of rows spanned by the shape.

Note that at $\beta=0$, we recover the classical 
\hyperref[murnaghanNakayamaRule]{Murnaghan--Nakayama rule for Schur polynomials.}


\subsection[grothendieckStableSchurExpansion]{Schur expansion}

C. Lenart \cite[Thm. 2.8]{Lenart2000} prove that whenever $k\geq \length(\lambda)$,
\[
\grothendieckStable_{\lambda}(x_1,\dotsc,x_k)
= \sum_{\lambda \subseteq \mu \subseteq \hat{\lambda}} (-1)^{|\mu|-|\lambda|} a_{\lambda\mu} \schurS_{\mu}(\xvec)
\]
where the sum is taken over partitions $\mu$, and $\hat{\lambda}$ is the unique maximal partition of length $k$
obtained from $\lambda$ by adding at most $j-1$ boxes to the $j$-th row of the Young diagram $\lambda$.

The coefficients $a_{\lambda\mu}$ are non-negative, and 
count the number of \emph{increasing tableaux} of shape $\mu/\lambda$.
These are tableaux which are strictly increasing along rows and columns,
and for each $i\geq 1$, row $i$ only contain entries $\leq i-1$.


Alternatively, in \cite[Cor. 3.11]{MonicalPechenikScrimshaw2018}, it is proved that 
$a_{\lambda\mu}$ count the number of set-valued tableaux of shape $\lambda$ and weight $\mu$,
which are also Yamanouchi. The authors archieve this by 
defining a type $A$ \hyperref[crystals]{crystal} graph on set valued fillings.


\subsection[grothendieckStableLittlewoodRichardson]{Littlewood--Richardson rule}

There is a Littlewood--Richardson rule for the product of stable Grothendieck polynomials:
\[
\grothendieckStable_{\lambda}(\xvec;\yvec) \grothendieckStable_{\mu}(\xvec;\yvec) = \sum_{\nu} 
(-1)^{|\lambda|+|\mu|-|\nu|} c_{\lambda\mu}^{\nu} \grothendieckStable_{\nu}(\xvec;\yvec)
\]
where $c_{\lambda\mu}^{\nu}$ is the number of set-valued tableaux of shape $\lambda * \mu$ with content $\nu$.



\subsection[grothendieckSkewStable]{Skew stable Grothendieck}


In \cite{Buch2002,LiMorseShields2016}, 
the skew version of the stable Grothendieck polynomials are defined as
\begin{align}
\grothendieckStable_{\nu/\lambda}(\xvec) = \sum_{\mu} 
\sum_{T \in \mathrm{SVT}(\nu/\lambda,\mu)} (-1)^{|\lambda|-|\nu|-|\mu|} \monomial_\mu(\xvec).
\end{align}

From a result in \cite{BilleyJockuschStanley1993}, it follows that the skew stable Grothendieck polynomials 
are precisely the stable limit of Grothendieck polynomials indexed by 321-avoiding permutations.


A recent result in \cite{ChanPflueger2019} gives an expansion of the skew stable Grothendieck polynomials
in terms of skew Schur polynomials. Here, $\sigma$ and $\mu$ are skew shapes.
\begin{align}
\grothendieckStable_{\sigma}(\xvec) = 
\sum_{\mu \supseteq \sigma} (-1)^{|B(\mu/\sigma)|} a_{\sigma,\mu} \schurS_\mu(\xvec).
\end{align}
The motivation behind this formula is the connection with Euler characteristics of certain varieties, see \cite{ChanPflueger2017}.

The coefficients $a_{\sigma,\mu}$ are non-negative integers with an explicit combinatorial interpretation.
M. Chan and N. Pflueger also prove a similar \emph{inverse} formula, for connected skew shapes $\sigma$:
\begin{align}
\schurS_{\sigma}(\xvec) = 
\sum_{\mu \supseteq \sigma} (-1)^{|A(\mu/\sigma)|} b_{\sigma,\mu} \grothendieckStable_\mu(\xvec).
\end{align}



\subsection[grothendieckStableMoreInfo]{More information}


See \href{https://arxiv.org/pdf/math/0601514.pdf}{this paper} for 
the expansion of stable Grothendieck polynomials into Grassman Grothendieck polynomials.
In \url{https://arxiv.org/pdf/1701.03561.pdf}, a flagged version is considered.
See \url{https://arxiv.org/pdf/1711.09544.pdf} for several Cauchy identities,
and skew versions.

\todo{
flagged version 
https://arxiv.org/pdf/1701.03561.pdf
}


\todo{
http://reu.dimacs.rutgers.edu/~fsalazar/paper.pdf
}

\todo{ 
https://www.degruyter.com/downloadpdf/j/crll.ahead-of-print/crelle-2017-0033/crelle-2017-0033.pdf
}

\begin{problem}[See \cite{MonicalPechenikScrimshaw2018}]
Describe a \hyperref[crystals]{crystal structure}, whose connected components give stable Grothendieck polynomials.
This would be analogous to the type $A$ crystals that produce Schur polynomials.
\end{problem}



\section[grothendieckDouble]{Double Grothendieck polynomials}

\begin{polydata}{grothendieckDouble}
  Name     & Double Grothendieck polynomials \\
  Space    & All \\
  Basis    & True \\
  Rating   & 1 \\
  Bib      & FominKirillov1994grothendieck \\
  Year     & 1994 \\
  Symbol   & $\grothendieck_{\omega}(x|b)$ \\
  Category & Schubert \\
\end{polydata}



The \defin{double Grothendieck polynomials} are defined via divided difference operators, $\pi_i$,
defined as
\[
\pi_i \coloneqq \frac{ (1+\beta x_{i+1})f-(1+\beta x_i)s_i(f)}{x_i - x_{i+1}},
\]
where $s_i$ permutes $x_i$ and $x_{i+1}$. We then have the base case
\[
\grothendieck_{\omega_0}(x|b) = \prod_{i+j \leq n}(x_i + b_j + \beta x_i b_j),
\]
and whenever $\length(\omega s_i) = 1 + \length(\omega)$ the recursive definition
\[
\grothendieck_{\omega}(x|b) \coloneqq \pi_i \left( \grothendieck_{\omega s_i}(x|b) \right).
\]



\section[grothendieckCanonical]{Canonical stable Grothendieck polynomials}

\begin{polydata}{grothendieckCanonical}
  Name     & Canonical stable Grothendieck polynomials \\
  Space    & Sym \\
  Basis    & True \\
  Rating   & 2 \\
  Bib      & Yeliussizov2017 \\
  Year     & 2017 \\
  Symbol   & $\grothendieckStable^{(a,b)}_\lambda(\xvec)$ \\
  Keywords & jacobi-trudi, weyl, cauchy-identity, tableau, operator \\
  Category & Schubert \\
\end{polydata}

In \cite{Yeliussizov2017}, the author introduce a two-parameter deformation of the stable Grothendieck polynomials,
$\grothendieckStable^{(a,b)}_\lambda(\xvec)$ and the dual stable Grothendieck polynomials,
$\grothendieckDual^{(a,b)}_\lambda(\xvec)$ with the following properties.

\begin{enumerate}

\item Specializes to Schur polynomials,
$\schurS_\lambda(\xvec) = \grothendieckStable^{(0,0)}_\lambda(\xvec)$,

\item Specializes to classical Grothendieck polynomials,
$\grothendieckStable(\xvec) = \grothendieckStable^{(0,-1)}_\lambda(\xvec)$ and 
$\grothendieckDual(\xvec) = \grothendieckDual^{(0,1)}_\lambda(\xvec)$,

\item Fulfill
$\omega(\grothendieckStable(\xvec)) = \grothendieckStable^{(-1,0)}_{\lambda'}(\xvec)$ and 
$\omega(\grothendieckDual(\xvec)) = \grothendieckDual^{(1,0)}_{\lambda'}(\xvec)$,

\item Highest (lowest) degree term is a Schur polynomial,
\[
\grothendieckStable^{(a,b)}_\lambda(\xvec) = \schurS_\lambda(\xvec) + \text{higher terms} \quad
\grothendieckDual^{(a,b)}_\lambda(\xvec) = \schurS_\lambda(\xvec) + \text{lower terms}
\]

\item They are dual with respect to the Hall inner product,
$
\langle \grothendieckStable^{(-a,-b)}_\lambda, \grothendieckDual^{(a,b)}_\mu \rangle = \delta_{\lambda\mu}
$

\item Nice involution properties,
\[
\omega( \grothendieckStable^{(a,b)}_\lambda ) = \grothendieckStable^{(b,a)}_{\lambda'} \text{ and }
\omega( \grothendieckDual^{(a,b)}_\lambda ) = \grothendieckDual^{(b,a)}_{\lambda'}
\]

\item 
They have the same structure constants $c^{\nu}_{\lambda\mu}$ as the usual stable Grothendieck polynomials,
\[
\grothendieckStable^{(a,b)}_\lambda 
\grothendieckStable^{(a,b)}_\mu 
=
\sum_{\nu} (a+b)^{|\nu|-|\lambda|-|\mu|} c^{\nu}_{\lambda\mu}\grothendieckStable^{(a,b)}_\nu. 
\]
\end{enumerate}

There is a skew version of these, a Weyl-type alternant formula, and many other nice properties, such as tableau descriptions,
operator descriptions etc.

The $\grothendieckDual^{(a,b)}_\mu $ are Schur-positive, and a \hyperref[crystals]{crystal} structure on 
the canonical Stable Grothendieck polynomials, as well as their 
duals is given by G. Hawkes and T. Scrimshaw \cite{HawkesScrimshaw2019}.



In \cite{GunnaZinnJustin2020}, solvable vertex models are used to 
prove Cauchy identities for the canonical Grothendieck polynomials:
\[
  \sum_{\lambda}  \grothendieckStable^{(-a,-b)}_\lambda(x_1,\dotsc,x_m) \grothendieckDual^{(a,b)}_\lambda(y_1,\dotsc,y_n) 
  = \prod_{\substack{ 1 \leq i \leq m \\ 1 \leq j \leq n }} \frac{1}{1 - x_i y_j}
\]

\[
  \sum_{\lambda}  \grothendieckStable^{(-a,-b)}_{\lambda'}(x_1,\dotsc,x_m) \grothendieckDual^{(a,b)}_\lambda(y_1,\dotsc,y_n) 
  = \prod_{\substack{ 1 \leq i \leq m \\ 1 \leq j \leq n }} (1 + x_i y_j ).
\]



\section[grothendieckQuantumDouble]{Quantum double Grothendieck polynomials}

\begin{polydata}{grothendieckQuantumDouble}
  Name     & Quantum double Grothendieck polynomials \\
  Space    & All \\
  Basis    & True \\
  Rating   & 1 \\
  Bib      & LenartMaeno2006 \\
  Year     & 2006 \\
  Category & Schubert \\
\end{polydata}


In \cite{LenartMaeno2006}, the authors introduce the \defin{quantum double Grothendieck polynomials}.


\section[grothendieckFactorial]{Flagged factorial Grothendieck polynomials}


\begin{polydata}{grothendieckFactorial}
  Name     & Flagged factorial Grothendieck polynomials \\
  Space    & All \\
  Basis    & False \\
  Rating   & 1 \\
  Bib      & MatsumuraSugimoto2019 \\
  Year     & 2019 \\
  Symbol   & $\grothendieckStable_{\lambda,f}(\xvec|b)$ \\
  Category & Schubert \\
\end{polydata}


In \cite{MatsumuraSugimoto2019}, the authors consider a 
generalization of the \hyperref[grothendieckGrassman]{Grassman Grothendieck polynomials}.
They introduce the \defin{flagged factorial Grothendieck polynomials},
indexed by a partition $\lambda$ with $r$ parts and a flag $f = (0 \lt f_1 \leq f_2 \leq \dotsb \leq f_r)$.
We then let
\[
\grothendieckStable_{\lambda,f}(\xvec|b) \coloneqq \sum_{T \in SVT(\lambda,f)} \beta^{|T|-|\lambda|}[x|b]^T
\]
where the sum runs over all flagged set-valued tableaux of shape $\lambda$
and flag $f$. This means that the sets in row $i$ may only contain elements from $\{f_1,\dotsc,f_i\}$.
Furthermore,
\[
[x|b]^T \coloneqq \prod_{e\in T} x_{val(e)} \oplus b_{val(e)-r(e)+c(e)}.
\]
Here, $a\oplus b \coloneqq a+b+\beta ab$.

In\cite{MatsumuraSugimoto2019}, the authors give a Jacobi--Trudi type identity 
for flagged factorial Grothendieck polynomials, and 
show that these are equal to \hyperref[grothendieckDouble]{double Grothendieck polynomials}
for vexillary ($2143$-avoiding) permutations.


\section[grothendieckDualStable]{Dual stable Grothendieck polynomials}


\begin{polydata}{grothendieckDualStable}
  Name   & Dual stable Grothendieck polynomials \\
  Space    & Sym \\
  Basis    & True \\
  Rating   & 2 \\
  Bib      & LamPylyavskyy2007combinatorialhopf \\
  Year     & 2007 \\
  Symbol   & $\grothendieckDual_\lambda(\xvec)$ \\
  Category & Schur \\
  Keywords & schur-positive
\end{polydata}


The dual stable Grothendieck polynomials were first studied in \cite{LamPylyavskyy2007combinatorialhopf}.
They are defined through the \hyperref[hallInnerProduct]{Hall inner product}
\[
\langle \grothendieckDual_\lambda, \grothendieckStable_\mu \rangle = \delta_{\lambda \mu}.
\]

Let $\mathrm{RPP}(\lambda)$ be the set of fillings of $\lambda$ with weakly increasing rows and columns.
Then
\[
\grothendieckDual_\lambda(\xvec) \coloneqq \sum_{T \in \mathrm{RPP}(\lambda)} \xvec^{ev(T)}
\]
where $ev(T)_i$is the number of columns of $T$ where $i$ appears.
Note that
\[
\grothendieckDual_\lambda(\xvec) = \schurS_\lambda(\xvec) + \text{lower order terms}.
\]
The skew version $\grothendieckDual_{\lambda/\mu}(\xvec)$ is defined in a similar manner.
One can prove that $\grothendieckDual_{\lambda/\mu}(\xvec)$ is symmetric by using a  version
of the Bender--Knuth involutions, see \cite{GalashinGrinbergLiu2016}.


In \cite{LiMorseShields2016}, it is shown that the constants appearing in the expansion
\[
\grothendieckDual_{\nu/\mu}(\xvec) = \sum_{\nu} 
(-1)^{|\lambda|+|\mu|-|\nu|}  c^{\nu}_{\mu\lambda} \grothendieckDual_{\lambda}(\xvec),
\]
agree with the coefficients $c_{\lambda\mu}^{\nu}$ in 
the \hyperref[grothendieckStableLittlewoodRichardson]{Littlewood--Richardson rule for stable Grothendieck polynomials}.


\subsection[grothendieckDualStableBialternant]{Bialternant formula}


A \defin{bialternant formula} for the dual stable Grothendieck polynomials 
is proved in \cite[Thm. 17]{Yeliussizov2019}.
It states that
\begin{align}
\grothendieckDual_{\lambda}(\xvec) = \frac{1}{ \prod_{i \lt j} (x_i-x_j)}
\det\left[
\phi^{i-1}x_j^{\lambda_i+n-i}
\right]_{1\leq i,j \leq n}
\end{align}
where $\phi^k x^n = \sum_{j=0}^n \binom{k+j-1}{j}x^{n-j}$.


\subsection[grothendieckDualStableLittlewood]{Littlewood-type identities}


Analogous to the \hyperref[schurSums]{Schur function Littlewood identities}, we 
have 
\[
\sum_{\lambda : \length(\lambda)\leq n} \grothendieckDual_\lambda(x_1,\dotsc,x_m) = \prod_{i=1}^m \frac{1}{(1-z_i)^n}.
\]



\subsection[grothendieckDualStableJacobiTrudi]{Jacobi--Trudi identity}

There is a Jacobi--Trudi type identity, \cite[Prop. 4.4]{Iwao2019} given by
\begin{align}
\grothendieckDual_\lambda(\xvec) = 
\det\left[
\sum_{m=0}^\infty \binom{1-i}{m} \beta^m \completeH_{\lambda_i-i+j-m}(\xvec)
\right]_{1\leq i,j \leq n}.
\end{align}

Alternatively, \cite{Yeliussizov2017,AmanovYeliussizov2020}, we have 
\begin{align}
\grothendieckDual_\lambda(\xvec) = 
\det\left[
\completeH_{\lambda_i-i+j}(1^{\lambda_i-1},\xvec)
\right]_{1\leq i,j \leq \length(\lambda)}
=
\det\left[
\elementaryE_{\lambda'_i-i+j}(1^{\lambda'_i-1},\xvec)
\right]_{1\leq i,j \leq \lambda_1}.
\end{align}

\todo{More dual Grothendieck stuff here + applications https://arxiv.org/pdf/1910.13378.pdf}

\todo{ Add ref to refined dual stable JT-identity.}

There is also a Jacobi--Trudi identity for the \defin{skew dual stable Grothendieck polynomials},
originally conjectured by \name[Darij Grindberg]{D. Grindberg} \href{http://www.cip.ifi.lmu.de/~grinberg/algebra/chicago2015.pdf}{(see these slides)}. 
A proof was later presented in \name{Y.-S Kim} \cite{Kim2020}.
A different proof was obtained independently by \name{A. Amanov} and \name{D. Yeliussizov} \cite{AmanovYeliussizov2020}.
Both approaches use quite novel techniques with \hyperref[lgvLemma]{non-intersecting lattice paths}.

The skew Jacobi--Trudi identity states that
\begin{align}
\grothendieckDual_{\lambda/\mu}(\xvec) = 
\det\left[
\elementaryE_{\lambda'_i-i-\mu'_j+j}(1^{\lambda'_i-\mu'_j-1},\xvec)
\right]_{1\leq i,j \leq \lambda_1}.
\end{align}


The skew Jacobi--Trudi identity extends to the family of \defin{refined dual stable Grothendieck polynomials},
introduced in \cite{GalashinGrinbergLiu2016}, see \cite{Kim2020,AmanovYeliussizov2020} for proofs.



\subsection[dualGrothendieckSchurExpansion]{Schur expansion}

In \cite[Thm. 9.8]{LamPylyavskyy2007combinatorialhopf}, the authors show that 
\[
\grothendieckDual_\lambda(\xvec) = \sum_\mu f^\mu_\lambda \schurS_\mu(\xvec)
\]
where $f^\mu_\lambda $ is the number of \defin{elegant fillings} of shape $\lambda/\mu$.
A filling is elegant if it is semi-standard, and all entries in row $i$ is in $1,\dotsc,i-1$.
See also \cite{Galashin2017} for a crystal proof.


A flagged version of the skew dual Grothendieck polynomials are 
shown to be \hyperref[key]{key positive} in \cite{Kundu2023x}.


\subsection[dualGrothendieckPercolation]{Last passage percolation}

\name{D. Yeliussizov} \cite{Yeliussizov2020} show that the dual Grothendieck polynomials
can be realized as a column distribution of a directed last-passage percolation model.


\section[grothendieckRefinedDualStable]{Generalized and refined dual stable Grothendieck polynomials}


\begin{polydata}{grothendieckRefinedDualStable}
  Name   & Generalized and refined dual stable Grothendieck polynomials \\
  Space    & Sym \\
  Basis    & True \\
  Rating   & 1 \\
  Bib      & GalashinGrinbergLiu2016 \\
  Year     & 2016 \\
  Symbol   & $\grothendieckRefinedDualStable_\lambda(\xvec)$ \\
  Category & Schur \\
  Keywords & schur-positive
\end{polydata}


The \defin{refined dual stable Grothendieck polynomial} were introduced in \cite{GalashinGrinbergLiu2016}.
They define these as 
\[
\grothendieckRefinedDualStable_\lambda(\xvec;\tvec) \coloneqq 
\sum_{T \in RPP(\lambda)}  \xvec^{ev(T)} \tvec^{b(T)}
\]
where (as usual), $ev(T)_i$ is the number of columns containing $i$,
and $b(T)_j$ is the number of boxes in row $j$ whose entry is equal to the entry in 
box immediately below it. 


In \cite{Yeliussizov2019}, the \defin{generalized dual stable Grothendieck polynomials}
are defined:
\[
\grothendieckDual_\lambda(\xvec;\zvec)
\coloneqq \sum_{\pi \in PP(\lambda)} \prod_{(i,j) \in \DES(\pi)} x_i z_{\pi(i,j)}.
\]
Note that $\grothendieckDual_\lambda(1^n;\zvec) =  \grothendieckDual_\lambda(\zvec)$.
It is observed that $\grothendieckDual_\lambda(\xvec;\zvec) = \xvec^{\lambda}\grothendieckRefinedDualStable_\lambda(\xvec^{-1},\zvec)$,
where $\grothendieckRefinedDualStable_\lambda$ is a refined dual stable Grothendieck polynomial.


D. Yeliussizov (\cite[Eq. (11)]{Yeliussizov2019}) prove that
\[
\sum_{\lambda : \length(\lambda)\leq n} \grothendieckDual_\lambda(x_1,\dotsc,x_m; z_1,\dotsc,z_n) = 
\prod_{i=1}^m \prod_{j=1}^n  \frac{1}{1-x_iz_j}.
\]
D. Yeliussizov proves several new $q$-identites involving 
plane partitions and dual Grothendieck polynomials in \cite{Yeliussizov2019}.



In \cite{MotegiScrimshaw2020x}, the authors introduce a vertex model whose partition function
is the refined dual stable Grothendieck polynomial. They give numerous new identities,
and prove Jacobi--Trudi identities for the skew refined dual stable Grothendieck polynomials.
Moreover, a new Cauchy identity is given as well.
This article has a lot of interesting new identities and connections.




