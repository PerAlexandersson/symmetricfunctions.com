\metatitle{Partitions, permutations and tableaux}
\metadescription{Partitions, permutations, combinatorial statistics and tableaux.}


\section[preliminaries]{Preliminaries}

We use standard notation, $\setN$ is the set of natural numbers $\{0,1,2,\dotsc,\}$
and $[n]\coloneqq \{1,2,\dotsc,n\}$.


\section[prelimPartitions]{Partitions}

A \defin{partition} is a weakly decreasing sequence of positive integers, referred to as \emph{parts}.
A partition $\lambda$ can be described as $1^{m_1},2^{m_2},\dotsc,k^{m_k}$
where $m_i$ is the number of parts of $\lambda$ equal to $i$.
The \defin{type} of $\lambda$ is the vector $(m_1,\dotsc,m_k)$.
For example, the partition $444332111111$ can in a more compact form be described as $1^6,2^1,3^2,4^3$,
and it has type $(6,1,2,3)$.

Partitions may be illustrated as \defin{Young diagrams}. 
\begin{example}
Consider $\lambda=(7,5,5,4,3)$. There are two common ways to draw $\lambda$ as Young diagram,
the \defin{English} way --- which is what is mainly used here, and the \defin{French way}. 
The English way uses matrix coordinates while the French uses Cartesian coordinates in the first quadrant.
The lengths of the rows are determined by the parts of $\lambda$.

\begin{figure}
\begin{ytableau}
 \circ & & & & & & \\
 & \circ & & & \\
 & & \circ & & \\
 & & & \circ \\
 & &
\end{ytableau}
\begin{ytableau}
 & & \\
 & & & \\
 & & & & \\
 & & & & \\
 & & & & & &
\end{ytableau}
\end{figure}

There is a third way to describe a partition, using \defin{Frobenius coordinates}.
Let $d$ be the number of boxes on the main diagonal in the Young diagram of $\lambda$. 
This is the \defin{Durfee size} of $\lambda$.
The \defin{Frobenius coordinates} of $\lambda$ is given by $(a_1,a_2,\dotsc,a_d|b_1,b_2,\dotsc,b_d)$
where $a_i = \lambda_i-i$ is the number of boxes to the right of the diagonal in row $i$,
and $b_i = \lambda'_i-i$ is the number of boxes below the diagonal in column $i$.
In the figure above, $d=4$ and $\lambda=(6,3,2,0|4,3,2,0)$.
\end{example}


The \defin{conjugate} of a partition $\lambda$ is denoted $\lambda'$, 
and correspond to transposing the (English) Young diagram.
Thus, the conjugate of $(7,5,5,4,3)$ is $(5,5,5,4,3,1,1)$, which is simply the sizes of the columns.

\subsection[compositions]{Compositions}

A \defin{composition} of $n$ is a vector with positive integer entries summing to $n$.
Hence, every partition is a composition. These are also sometimes called \emph{strong compositions}.
It is a classical exercise to show that there are $2^{n-1}$ compositions of $n$.

A \defin{weak composition} of $n$ is a vector with non-negative integer entries summing to $n$.


\subsection[partialOrderingsOnPartitions]{Partial orderings on partitions}

We write $\lambda \supseteq \mu$ if $\lambda_i \geq \mu_i$ for all $i$.
The \defin{dominance order} is defined as
\[
\lambda \trianglerighteq \mu \text{ whenever }
\lambda_1+\dotsb + \lambda_i \geq \mu_1 + \dotsb + \mu_i \text{ for all $i$}.
\]
Dominance order is defined in the same manner for compositions and weak compositions.

\subsection[skewShapes]{Skew shapes and ribbons}

If $\lambda \supseteq \mu$, we let $\lambda/\mu$ denote the set of boxes 
in the Young diagram of $\lambda$ which do not appear in $\mu$.
We say that $\lambda/\mu$ is a \defin{skew shape}.

A skew shape is a \defin{ribbon} if it is connected and does not 
contain a $2\times 2$-arrangement of boxes.
The following figure illustrates a $12$-ribbon, as it has $12$ boxes.
\begin{figure}
\begin{ytableau}
\none & \none & & & & & & h\\
\none & \none & \\
\none & \none & \\
\none &  & \\
\none &  \\
t &  \\
\end{ytableau}
\end{figure}
Ribbons are sometimes called \defin{border strips} or \defin{rim-hooks}.
Ribbons play a special role in the definition of \hyperref[borderStripTableaux]{border-strip tableaux}.
The \defin{head} of a ribbon is the rightmost box in the first row of the ribbon,
while the \defin{tail} is the leftmost box in the last row of the ribbon.





\subsection[prelimPartitionStatistics]{Statistics on partitions}

Define $\partitionN(\lambda) \coloneqq \sum_i (i-1) \lambda_i = \sum_{i \geq 2} \binom{\lambda'_i}{2}$ and 
let $z_\lambda \coloneqq \prod_j \left( m_j! j^{m_j} \right)$
where $m_i$ is the number of parts of $\lambda$ of size $i$.

Note that $z_\lambda$ is the number of elements in $\symS_n$, which fixes $g_\lambda$
under conjugation, where $g_\lambda$ is some fixed permutation of type $\lambda$.
That is,
\[
z_\lambda = |\{ \sigma \in \symS_n : \sigma g_\lambda \sigma^{-1} = g_\lambda \}|.
\]
The $z_\lambda$ often shows up as a normalizing factor when dealing with power-sum symmetric functions,
and in particular, in the \hyperref[hallInnerProduct]{Hall inner product}.


\subsection[partitionCores]{Hook lengths and cores}

Given a box $(i,j)$ in the Ferrers diagram of $\lambda$, 
the \defin{hook-length} of the box is given by $1 + (\lambda_i-i) + (\lambda'_j-j)$.
For example, the hook-length in the diagram below has been written in each box.
\begin{figure}
\begin{ytableau}
 11& 10& 9 & 7 & 5 & 2 & 1 \\
 8 & 7 & 6 & 4 & 2 \\
 7 & 6 & 5 & 3 & 1 \\
 5 & 4 & 3 & 1 \\
 3 & 2 & 1
\end{ytableau}
\end{figure}

The \defin{content} of a box $(i,j)$ is $j-i$, where $j$ is the column index, and $i$ is the row index.

A \defin{$p$-core} is a partition where none of the hook-lengths are divisible by $p$.
A closely related concept is the notion of \hyperref[partitionQuotient]{partition quotients}.


The \defin{hook formula} states that 
\[
 |\SYT(\lambda)| = \frac{n!}{\prod_{\square \in \lambda} h(\square)},
\]
where $h(\square)$ denotes the hook length of the box. 
There is a \hyperref[prelimQHook]{$q$-analog} of this identity.

There is a more complicated formula for skew diagrams, due to H. Naruse \cite{Naruse2014slides}.
For a proof, see Morales--Pak--Panova \cite{MoralesPakPanova2018}.
An extension to cylindric diagrams is conjectured in \cite{SuzukiToyosawa2021x}.


\section[weakCompositions]{Compositions and weak compositions}

Similar to partitions, a composition of $n$ is an integer vector of positive 
integers summing to $n$. For example, 
\begin{example*}[Compositions of 4]
There are $2^{n-1}$ compositions of $n$. For example,
\[
(4), 
(3,1), 
(1,3), 
(2,2), 
(2,1,1),
(1,2,1),
(1,1,2),
(1,1,1,1)
\]
are the compositions of $4$.
\end{example*}

A \defin{weak composition} allows for parts to be equal to $0$.


\section[notationPermutations]{Permutations and words}

The set of permutations of $n$ elements is denoted $\symS_n$.
Let $\pi \in \symS_n$, or more generally, a word of length $n$.
The \defin{descent set} $\DES(\pi)\subseteq [n-1]$
is defined as 
\[
\DES(\pi) \coloneqq \{i \in [n-1] : \pi_i \gt \pi_{i+1} \}.
\]
The set of \defin{inversions} is the set of pairs,
\[
\INV(\pi) \coloneqq \{(i,j) \in [n]\times [n] : \pi_i \gt \pi_{j} \}.
\]
We define the \defin{major index} on permutations and words:
\[
\inv(\pi) \coloneqq |\INV(\pi)| \text{ and } \maj(\pi) \coloneqq \sum_{i \in \DES(\pi)} i.
\]
For permutations, we also define \defin{charge} as
\[
\charge(\pi) \coloneqq \maj(\rev(\pi^{-1})) \text{ and } \cocharge(\pi) \coloneqq \binom{n}{2} - \charge(\pi).
\]

\begin{lemma}
We have that 
\[
\charge(\pi) = \inv \circ \mathrm{foata} \circ \rev \circ (\cdot)^{-1} \circ \pi,
\]
where $\mathrm{foata}$ denotes the map sending major index to inversions.
\end{lemma}



The \defin{weight} or \defin{type} of a word $w$ is the composition $\alpha$
such that $\alpha_i$ is the number of entries in $w$ equal to $i$.

A word $w_1,\dotsc,w_n$ is called \defin{Yamanouchi} if the number of occurrences of $i$ is at least as many 
as the number of occurrences of $i+1$ in every prefix $w_1,\dotsc,w_k$.
Such words are also known as \defin{Ballot words}, see \oeis{A000085}.


\subsection[bruhatOrder]{Bruhat order and weak order}

The \defin{Bruhat order} (or \defin{strong order}) on 
permutations is a partial order $\leq$ on $\symS_n$.
We have that $\sigma \leq \tau$ if 
$\tau$ can obtained from $\sigma$ via a sequence of transpositions, 
which increase the number of inversions in each step.
Alternatively, we have that $\sigma \leq \tau$ if and only if 
there are reduced words $w_\sigma$, $w_\tau$ for $\sigma$ and $\tau$ 
such that $w_\sigma$ is a subword of $w_\tau$.

The Bruhat order is a ranked poset, with rank given by the number of inversions.

The (right) \defin{weak order} is a subposet of the Bruhat order,
where only \emph{simple transpositions} are allowed, when multiplying on the right.
This corresponds to swapping adjacent entries.
Let us write $\leq_R$ for this relation.
Alternatively, we have that $\sigma \leq_R \tau$ if and only if 
there are reduced words $w_\sigma$, $w_\tau$ for $\sigma$ and $\tau$ 
such that $w_\sigma$ is a \emph{prefix} of $w_\tau$.

 
\begin{figure}
  \svgimg[width=0.8\textwidth]{svg-images/bruhatGraph.svg}{Bruhat poset for $n=4$.}
\end{figure}

In this figure, we show the Hasse diagram of the Bruhat order for $\symS_4$.
The solid edges indicate covering relations in the (right) weak order.
See \cite[Ch. 10.5]{Fulton1997}, \cite[Ex. 183, 185]{StanleyEC1} 
and \cite[Ch.2-Ch.3]{Bjorner2005} for more properties
and alternative definitions.


\subsection[notationAffinePermutations]{Affine permutations}

The set of \defin{affine permutations} $\symAff_n$ is the set of bijections $\pi: \setZ \to \setZ$
satisfying $\pi(i+n)=\pi(i)+n$ for all $i\in \setZ$ and $\pi(1)+\pi(2)+\dotsb+\pi(n)=1+2+\dotsb+n$.


\section[prelimTableaux]{Tableaux}

A tableau is a filling of the boxes determined by a partition, or sometimes a more general configuration of boxes.

The \defin{reading word} of a tableau is the word obtained by concatenating the rows,
starting from the bottom row in English notation.

The set of \defin{standard Young tableaux} of shape $\lambda$, denoted $\SYT(\lambda)$, 
is the set of bijective fillings $\lambda \to [n]$, such that rows and columns are increasing with row and column index.
The set of \defin{semistandard Young tableaux}, $\SSYT(\lambda)$,
is defined as the set of fillings with entries in $\setP$, with weakly increasing rows and strictly increasing columns.
We let $\SSYT(\lambda,\mu)$ denote the (finite) set of semi-standard Young tableaux with shape $\lambda$
and type $\mu$, that is, there are $\mu_i$ entries equal to $i$ in each such tableau.


\begin{example}
Consider the following tableau $T$.
\begin{figure}
\begin{ytableau}
1 & 2 & 5 & 7 \\
3 & 4 \\
6
\end{ytableau}
\end{figure}

The reading word of $T$ is $6341257$, denoted $\rw(T)$.
Note that if $T$ is a standard or semistandard Young tableau, then $T$
is uniquely determined by its reading word.

Whenever $T \in \SYT(\lambda)$, we define $\charge(T)$ and $\cocharge(T)$ as 
the charge and cocharge, respectively, of their reading words.
\end{example}

Given $T \in \SYT(\lambda)$ let 
\[
\DES(T) \coloneqq \{i \in [n-1] : i+1 \text{ appears in a lower row than } i \}.
\]
We then let $\maj(T)  \coloneqq \sum_{i \in \DES(T)} i$.
Thus, $\maj(T) = \maj(\rw(T)^{-1})$.





\subsection[prelimTableauxSmartMap]{SYT as Yamanouchi words}


The set $\SYT(\lambda)$ is in bijection with Yamanouchi words with weight $\lambda$.
The bijection $f(T)=w$ is given by $w_i = j$ if $i$ occur in row $j$ of $T$.
This map can be quite useful at times, and is usually called the \hyperref[companionMap]{companion map}.
Note that $\maj(T) = \binom{n}{2}-\maj(w)$.

There are 10 Yamanouchi words of length $4$, and these are in 
bijection with the 10 standard Young tableaux with $4$ boxes.
\begin{example}
We have the following $10$ Yamanouchi words:
\[
\mathtt{1111},\,
\mathtt{1112},\,
\mathtt{1121},\,
\mathtt{1211},\,
\mathtt{1122},\,
\mathtt{1212},\,
\mathtt{1123},\,
\mathtt{1213},\,
\mathtt{1231},\,
\mathtt{1234}.
\]
The standard Young tableaux are
\begin{figure}
\ytableaushort{1234}
\ytableaushort{123,4}
\ytableaushort{124,3}
\ytableaushort{134,2}
\ytableaushort{12,34}
\ytableaushort{13,24}
\ytableaushort{12,3,4}
\ytableaushort{13,2,4}
\ytableaushort{14,2,3}
\ytableaushort{1,2,3,4}
\end{figure}
\end{example}




\subsection[prelimTableauxAsLattice]{SSYT as lattice paths}

Semistandard Young tableaux can be put in correspondence with sets of non-intersecting lattice paths,
and this allows us to prove the two \hyperref[schurJacobiTrudi]{Jacobi--Trudi identities},
by using the Lindström--Gessel--Viennot lemma \cite{Lindstrom1973,GesselViennot1989}.

Let $T \in \SSYT(\lambda/\mu)$ where the entries are bounded by $m$.

\emph{Row bijection}: 
Each row $i$ of $T$ is mapped to a lattice path $P_i$, with steps in the set $\{ (0,1), (1,0)\}$.
The starting vertex of $P_i$ is $(-i + \mu_i,1)$,
the ending vertex is $(-i + \lambda_i, m)$,
and its number of horizontal steps on $y$-coordinate $j$ is given by the number of entries in
row $i$ equal to $j$. The length of path $P_i$ is $m-1 + \lambda_i-\mu_i$.
This bijection can be used to prove the first \hyperref[skewJacobiTrudi]{Jacobi--Trudi identity}.


\emph{Column bijection}: 
Each column of $T$ is mapped to a path $P'_i$,
with steps in the set $\{ (-1,1), (1,1)\}$.
The starting vertex of $P'_i$ is given by $(2i - 2\mu'_i, 0)$,
the ending vertex is $(2i - 2\lambda'_i + m, m)$
and the $j$th step of $P'_i$ is $(-1,1)$ if $j$ is present in column $i$,
otherwise it is $(1,1)$. Each path has length $m$.
This bijection can be used to prove the second Jacobi--Trudi identity.


\begin{example}[Bijection to lattice paths]

Consider the following skew semistandad Young tableau $T$.
\begin{figure}
\ytableaushort{{\none}{\none}{\none}1122346,{\none}{\none}1223455,{\none}{\none}2344566,113556,2346,456,6}
\end{figure}

The rows give rise to the following set of non-intersecting lattice paths,
where the topmost row in $T$ corresponds to the rightmost path.
\begin{figure}
<img src="./svg-images/ssytRowLatticePaths.svg" 
alt="SSYT rows as lattice paths" style="width:95%; margin:5px;"/>
\end{figure}

The columns in $T$ give rise to the following set of non-intersecting lattice paths,
where the leftmost column corresponds to the leftmost path.

\begin{figure}
<img src="./svg-images/ssytColumnLatticePaths.svg" 
alt="SSYT rows as lattice paths" style="width:95%; margin:5px;"/>
\end{figure}
\end{example}



The following appears in \url{https://arxiv.org/pdf/2003.04957.pdf},
and is proved using lattice paths.
\begin{theorem}
Let $A,B$ be subsets of $\{0,1,\dotsc,n\}$ of equal size and let $A^c$, $B^c$ be their complements.
Then
\[
\det \left[ \completeH_{b-a}( x_1,x_2,\dotsc,x_{a+1} ) \right]_{a \in A, b \in B} =
\det \left[ \elementaryE_{a'-b'}( x_1,x_2,\dotsc,x_{a'} ) \right]_{a' \in A^c, b' \in B^c}.
\]
Moreover, this is related to C. A. Aitken's identity;
\[
\det \left[ \completeH_{b-a}(\xvec ) \right]_{a \in A, b \in B} =
\det \left[ \elementaryE_{a'-b'}( \xvec ) \right]_{a' \in A^c, b' \in B^c}.
\]
\end{theorem}



See also \cite{Stembridge1990} for background on counting lattice paths with pfaffians.

One can also prove the Weyl alternant formula by LGV, see \cite{Xiong2020x}.

See \href{https://www.youtube.com/watch?v=H1wEiEIznaA}{this GOCC talk from 2023} on a generalization of the LGV lemma.




\subsection[prelimPPAsLattice]{Plane partitions as lattice paths}


See \cite[Thm. 10.7.1]{Krattenthaler2015} and \cite[Eq. (6)]{StanleyPitman2002} for the following
application of the Lindström--Gessel--Viennot lemma.

\begin{theorem}
Let $\lambda$ and $\mu$ be partitions with at most $\ell$ parts such that 
$\lambda/\mu$ defines a skew shape.
Then the number of lattice paths contained in $\lambda / \mu$ equals
\[
    \det \left[ \binom{\lambda_i - \mu_j+1}{i-j+1} \right]_{1\leq i,j \leq \ell.}
\]
\end{theorem}
For non-skew shapes, we can also interpret such partitions $\nu$ as lattice
points in Stanley--Pitman polytopes \cite{StanleyPitman2002}.
This is also the number of plane partitions of skew shape $\lambda/\mu$, with binary entries,
or the number of non-nesting rook placements on the Ferrers board of shape $\lambda / \mu$.
There is also an interpretation of these quantities using certain linear extensions, see \cite{AlexanderssonJal2024x}.



\subsection[prelimTableauxAsPolytope]{SSYT as lattice points in a polytope}

Semistandard Young tableaux are in bijection with \hyperref[gtpatternsAsSSYT]{Geltand--Tsetlin patterns},
which are lattice points in Gelfand--Tsetlin polytopes.



\subsection[companionMap]{The companion map}

In \cite[Def. 15]{KaliszewskiMorse2019}, the authors describes a \defin{companion map},
which interchanges the role of shape and weight of fillings.

