\metatitle{Schubert polynomials}
\metadescription{An introduction to Schubert polynomials, their definitions via divided difference operators, combinatorial formulas using reduced words and pipe dreams, and various properties including representation theory connections, GT polytopes, and multiplication rules.}


\section[schubert]{Schubert polynomials}

\begin{polydata}{schubert}
  Name   & Schubert polynomials \\
  Space  & All \\
  Basis  & Yes \\
  Rating & 5 \\
  Bib    & LascouxSchutzenberger1982 \\
  Year   & 1982 \\
  Category & Schubert \\
\end{polydata}


\todo{Add this on Specializations
http://de.arxiv.org/pdf/1811.11596.pdf
}

\todo{https://arxiv.org/pdf/1812.00321.pdf}





\todo{
When coeffs are 0 or 1. Permutation-patterns
https://arxiv.org/pdf/1903.10332.pdf
Also permutation patterns:
https://arxiv.org/pdf/1705.02065.pdf
Even more patterns:
https://arxiv.org/pdf/1909.07206.pdf (this has a nice background - incorporate!)
}

Schubert polynomials were introduced by A. Lascoux and M.-P. Schützenberger in \cite{LascouxSchutzenberger1982}.
The Schubert polynomials generalize the \hyperref[schurS]{Schur polynomials} and
represent Schubert cycles in flag varieties.
The Schubert polynomials are generalized by the \hyperref[grothendieck]{Grothendieck polynomials}.

The Schubert polynomials are indexed by permutations. The set 
\[
\{ \schubert_\omega(x_1,\dotsc,x_n) \}_{\omega \in \symS_n}
\]
is then a basis for the space $\setZ[x_1,\dotsc,x_n]/I_n$
where $I_n = \langle \elementaryE_1, \elementaryE_2,\dotsc, \elementaryE_n \rangle$,
i.e., the ideal generated by non-constant symmetric functions.
The Schubert polynomials $\schubert_\omega$ with $\omega \in \symS_\infty$
form an integral basis for $\setZ[x_1,x_2,\dotsc]$.


\subsection[schubertOperatorformula]{Operator definition}

Consider the ring $\setR[x_1,\dotsc,x_n]$ and let $s_i$ act by permuting $x_i$ and $x_{i+1}$.
Define $\partial_i(f)$ as $\frac{f-s_i(f)}{x_i - x_{i+1}}$ for $i=1,\dotsc,n-1$.
These divided difference operators satisfy the braid relations, 
and we let $\partial_\omega = \partial_{a_1} \partial_{a_2} \dotsm \partial_{a_k}$
where $(a_1,a_2,\dotsc,a_k)$ is a reduced word for $\omega$.

The \defin{Schubert polynomials} are then defined as
\begin{equation*}
 \schubert_\omega(x_1,\dotsc,x_n) \coloneqq \partial_{\omega^{-1}\omega_0}(x_1^{n-1}x_2^{n-2}\dotsm x_{n-1}).
\end{equation*}


\subsection[schubertReducedwordFormula]{Reduced word formula}
Let $RW(\omega)$ be the set of reduced words of $\omega$.

We say that a $k$-tuple $\alpha_1,\dotsc,\alpha_k$ is \defin{$a$-compatible}
if 
\begin{itemize}
\item
$\alpha_1 \leq \alpha_2 \leq \dotsb \leq \alpha_k$
\item
$\alpha_j \leq a_j$ for $1 \leq j \leq k$
\item
$\alpha_j \lt \alpha_{j+1}$ whenever $a_j \lt a_{j+1}$.
\end{itemize}
Let $C(a)$ denote the set of $a$-compatible sequences.
Then 
\begin{equation*}
 \schubert_\omega(x_1,\dotsc,x_n) = \sum_{a \in RW(\omega)} \sum_{\alpha \in C(a)}
 x^\alpha
\end{equation*}



\subsection[schubertPipeDreamFormula]{Pipe dream formula}

The set of reduced words and compatible sequences can be 
described in a more combinatorial manner using rc-graphs.
Suppose $(\alpha_1,\dotsc,\alpha_k)$ is an $a$-compatible sequence, for $a\in RW(\omega)$.
Let
\[
D(a,\alpha) = \{ (\alpha_j, a_j - \alpha_j + 1) \text{ for } j=1,\dotsc,k \}.
\]
This set of points is illustrated in an \defin{rc-graph}, also known as \defin{pipe-dreams}.

\begin{example}
Let $\omega = 314652$. Then $a = 521345$ is in $RW(\omega)$,
and $\alpha = 111235$ is $a$-compatible. 
We mark the corresponding entries in $D(a,\alpha)$ with a cross, $+$, in the rc-graph:

\begin{figure}
\begin{ytableau}
\none \; & \none 1 & \none  2 & \none 3 & \none  4 & \none  5 & \none  6 \\
\none 1 & + & + & \cdot  & \cdot  & + & \cdot  \\
\none 2 & \cdot  & + & \cdot  & \cdot  & \cdot   \\
\none 3 & \cdot  & + & \cdot  & \cdot  \\
\none 4 & \cdot  & \cdot  & \cdot  \\
\none 5 & + & \cdot  \\
\none 6 & \cdot 
\end{ytableau}
\end{figure}

\end{example}

Let $RC(\omega)$ denote the set of all such diagrams.
Let $\xvec_D = \prod_{(i,j) \in D} x_i$ then the Schubert polynomial is given as
\begin{equation*}
 \schubert_\omega(x_1,\dotsc,x_n) = \sum_{D \in RC(\omega)} \xvec_D.
\end{equation*}

See also \cite{FominKirillov1996YangBaxter}.


\subsection[schubertBumplessPipeDream]{Bumpless pipe dreams}

The notion of \defin{bumpless pipe dreams} was introduced by 
Lam, Lee and Shimozono in \cite{LamLeeShimozono2018x}.
They prove the following formula for the \hyperref[schubertDouble]{double Schubert polynomials}:
\[
\schubert_\omega(\xvec,\yvec) = \sum_{P \in RBPD(\omega)} \prod_{(i,j)\in D(P)} (x_i - y_j )
\]
where the sum runs over \emph{reduced bumpless pipe dreams}.
This formula can be lifted to a formula for \hyperref[grothendieck]{the doubble Grothendieck polynomials.}



\subsection[schubertRepresentationTheory]{Representation theory}

This subsection follows the introduction in \cite{FanGuo2019}.

Let $D=(D_1,\dotsc,D_n)$ be an $n$-tuple of subsets of $[n]$.
Then $D$ encodes a subset of $[n]\times [n]$, where $D_j$ is considered as a 
set of row indices (seen as a strictly increasing list).
For example, the following diagram encodes $(23,14,\emptyset,124)$
\begin{figure}
\begin{ytableau}
\; & *(lightblue) & \; & *(lightblue) \; \\
*(lightblue) &  & \; & *(lightblue) \; \\
*(lightblue) & \;  & \; &  \; \\
\; & *(lightblue) & \; &  *(lightblue) \; 
\end{ytableau}
\end{figure}
We define a partial order on such diagrams, where $C \leq D$ if for every $j \in [n]$,
we have $|C_j| = |D_j|$ and $C_{jk} \leq D_{jk}$ for all $k$.

Let $B \subset GL_n(\setC)$ be the set of invertible upper-triangular $n\times n$-matrices.
Let $Y$ denote the upper-triangular matrix with indeterminate entries $\{y_{ij}\}_{i\leq j}$,
and let $\setC[Y]$ be the polynomial ring with these indeterminates.
We have that a matrix $M \in GL_n(\setC)$ act on $f \in \setC[Y]$ on the right by $f \cdot M \coloneqq f(M^{-1} Y)$.

Given a diagram $D$, the \defin{flagged Weyl module} $\mathcal{M}_D$ is a $B$-module, defined as
\[
  \mathcal{M}_D \coloneqq \mathrm{Span} \left\{   \prod_{j=1}^n \det( Y_{D_j}^{C_j} ) : C \leq D  \right\} \subseteq \setC[Y],
\]
where $ Y_{D_j}^{C_j}$ denotes the submatrix of $Y$ with row-indices from $C_j$ and colum-indices from $D_j$.

The character, $char(M_D)(x_1,\dotsc,x_n)$ of $\mathcal{M}_D$ is defined as the trace of the map $X : \mathcal{M}_D \to \mathcal{M}_D$,
where $X = (x_1,\dotsc,x_n)$ is a diagonal matrix acting on $\setC[Y]$.
Note that for $C \leq D$,
\[
\left( \prod_{j=1}^n \det( Y_{D_j}^{C_j} ) \right) \cdot X = 
 \prod_{j=1}^n  \prod_{i \in C_j} x_i^{-1} \det( Y_{D_j}^{C_j} )
\]
so we see that all polynomials $\det( Y_{D_j}^{C_j} )$ with $C \leq D$ are eigenvectors, and do in fact span $\mathcal{M}_D$.
It was shown in \cite{KraskiewiczPragacz1987,KraskiewiczPragacz2004} 
that the Schubert polynomial indexed by $\omega$ is related to the character of $\mathcal{M}_{D(\omega)}$, via
\[
\schubert_\omega(x_1,\dotsc,x_n) = char(M_{D(\omega)})(x^{-1}_1,\dotsc,x^{-1}_n).
\]


\subsection[schubertGTFormula]{Gelfand--Tsetlin polytopes}

In \cite{LiuMeszarosDizier2019}, the authors express the Schubert polynomials
as a projection of a Minkowski sum of \hyperref[gtpolytopes]{Gelfand--Tsetlin polytopes}.
This generalizes the way Schur polynomials are expressible as 
an integer point transform of a single Gelfand--Tsetlin polytope.

See also a similar result in \cite{FinkMeszarosDizier2018}, 
using the integer point transforms of generalized permutahedra.



\subsection[schubertBlockDecomposition]{Block decomposition formula}

If $\sigma \in \symS_j$ and $\tau \in \symS_{k}$,
we can compute the \hyperref[directSumPermutations]{skew sum} $\sigma \times \tau \in \symS_{j+k}$.
We then have the factorization (see \cite{BilleyJockuschStanley1993})
\[
 \schubert_{\sigma \times \tau} = (x_1 \dotsm x_j)^k \schubert_{\sigma} \; \uparrow^j \! \schubert_{\tau}
\]
where $\uparrow^k x_i \coloneqq x_{j+k}$ is the operator which 
increases the variable index by $j$ on all variables.


\subsection[schubertGrassmanianPermutation]{Grassmanian permutations}


A permutation $\omega \in \symS_n$ is called \defin{Grassmanian} (of descent $k$) if $\omega_i \lt \omega_{i+1}$
for all $i\neq k$. All Grassmanian permutations are \defin{vexillary}, which means it is 2143-avoiding.

A Grassmanian permutation determines a partition:
\[
\lambda(\omega) = (\omega_k - k, \omega_{k-1} - k + 1, \dotsc, \omega_1 - 1 ).
\] 
We then have the identity 
\begin{equation*}
\schurS_{\lambda(\omega)}(x_1,\dotsc,x_k) = \schubert_{\omega}(\xvec).
\end{equation*}

Furthermore, the \hyperref[stanleySym]{Stanley symmetric functions} 
are obtained as a stable limit of the Schubert polynomial.


\subsection[schubertMurnaghanNakaygama]{Murnaghan--Nakaygama rule}

In \cite{MorrisonSotille2018} a Murnaghan--Nakaygama rule is proved:
\begin{equation*}
\powerSum_r(x_1,\dotsc,x_k) \schubert_\omega(\xvec) = \sum (-1)^{\epsilon_k(\sigma)} \schubert_{\omega \cdot \sigma}(\xvec)
\end{equation*} 






\subsection[schubertMonksRule]{Monks rule}

The geometric version of this result was given by D. Monk in \cite{Monk1959}.
(\href{http://www.wfnmc.org/erdmonk.html}{David Monk} is a prolific IMO problem constructor.)

Let $s_r$ be a simple transposition. Then
\[
\schubert_{s_r}(\xvec) \schubert_\omega(\xvec) = \sum \schubert_{\omega t_{ij}}(\xvec)
\]
where the sum is over all transpositions $t_{ij}$ such that $i\leq r \lt j$ and 
$\length(\omega t_{ij}) = \length(\omega)+1$.
Notice that 
\[
\schubert_{s_r}(\xvec) = x_1+x_2+\dotsb + x_r.
\]

A combinatorial proof is given in \cite{CokunTaskin2018}, and a bijective proof using bumpless pipe 
dreams can be found in \cite{Huang2020x}.
Huang also proves Monk's rule for the double Schubert polynomials with this model.

\subsection[schubertPieriRule]{Pieri rule}

F. Sotille \cite{Sottile1996} provides the following Pieri-type formulas for Schubert polynomials.
Let $\omega \in \symS_n$, then
\[
\completeH_{m}(x_1,\dotsc,x_k) \schubert_\omega(\xvec) = \sum_{\omega'} \schubert_{\omega'}(\xvec)
\]
where the sum run over all $w = w't_{a_1b_1}\dotsm t_{a_mb_m}$ such that $a_i \leq k \leq b_i$,
$\length(w't_{a_1b_1}\dotsm t_{a_ib_i}) = \length(w')+i$ and all $b_i$ distinct.

Similarly,
\[
\elementaryE_{m}(x_1,\dotsc,x_k) \schubert_\omega(\xvec) = \sum_{\omega'} \schubert_{\omega'}(\xvec)
\]
with the same sum as above, but now the $a_i$ are distinct.


\subsection[schubertSotillesRule]{Sotille rule}

There is a rule for multiplying $\schubert_\omega(\xvec)$ with a Schur polynomial indexed by a hook.



\subsection[schubertCauchyFormula]{Cauchy identity}

By letting $w=\omega_0$, in the \hyperref[schubertDoubleGiambelliFormula]{Giambelli formula} for double Schubert 
polynomials, one can prove the following \defin{Cauchy identity} for Schubert polynomials:
\[
\prod_{i+j \leq n} (x_i - y_j) = \sum_{\omega} \schubert_{\omega}(\xvec) \schubert_{\omega_0 \omega}(-\yvec).
\]


\subsection[schubertLittlewoodRichardson]{Littlewood--Richardson rule}


\begin{problem}
Give a combinatorial interpretation of the coefficients in the product
\[
\schubert_u(\xvec) \schubert_v(\xvec) = \sum_{w} c^w_{uv} \schubert_w(\xvec).
\]
\end{problem}
Due to the geometric interpretation of Schubert polynomials, 
it is known that the coefficients $c^w_{uv}$ are non-negative integers.

Note that the coefficients $c^w_{uv}$ generalize the classical Littlewood--Richardson 
coefficients in \hyperref[schurLittlewoodRichardson]{the Littlewood--Richardson rule for Schur polynomials}.
This is a major open problem in Schubert calculus and algebraic combinatorics.

This problem is closely related to \defin{the Fomin--Kirillov conjecture},
see \cite[Conjecture 8.1 and Problem 8.3]{FominKirillov1999}.
It concerns the evaluation of Schubert polynomials at Dunkl elements.


There are many special cases proved for this problem,
one particular case is treated in \cite{Huang2022},
regarding products of two permutations with separated descents.


Matthew J. Samuel has a python \href{https://pypi.org/project/schubmult/}{package for computing products of Schubert polynomials}.


\begin{problem}
Can one give an efficient algorithm to decide if $c^w_{uv} = 0$ or not?
\end{problem}
This problem is discussed in \cite{PakRobichaux2025xI}, where it is shown
that (assuming GRH) that this problem lies in $AM \cap coAM$,  
(see \href{https://en.wikipedia.org/wiki/Arthur%E2%80%93Merlin_protocol#AM}{Arthur--Merlin protocol} for complexity theory background).



\subsection[schubertKeyExpansion]{Expansion in key polynomials}

There are several proofs of the fact that Schubert polynomials expand positively into key polynomials,
the first proof appearing in \cite{LascouxSchutzenberger1990}.

One recent proof is by using crystal bases, see Assaf and Schilling \cite{AssafSchilling2018}.
A crystal structure on RC-graphs where the key expansion is obtained is given 
by Gold, Milićević and Sun in \cite{GoldMilicevicSun2024x}.

\subsection[schubertAsSumOfElementary]{Sum of elementary}

Let $\omega \in \symS_{n+1}$, and let $\elementaryE_{j}(m) = \elementaryE_{j}(x_1,\dotsc,x_m)$
denote the elementary symmetric functions of degree $j$ in $m$ variables.
Then there is a unique way to express $\schubert_\omega(\xvec)$ as
\[
\schubert_\omega(\xvec) = \sum a_{k_1 \dotsc k_n} \elementaryE_{k_1}(1) \elementaryE_{k_2}(2) \dotsm \elementaryE_{k_n}(n)
\]
where the sum ranges over integers such that $0 \leq k_i \leq i$ and $k_1+\dotsb + k_n = \length(\omega)$.
This expansion is the starting point for defining the quantum Schubert polynomials.
The coefficients $a_{k_1 \dotsc k_n}$ are in general not always positive.
These coefficients are studied in \cite{Winkel1998} but many questions are left unanswered.



\begin{problem}[See \cite{MerzonSmirnov2015}]
For each $n$, find a permutation $w\in\symS_n$
which maximizes $\schubert_w(1,1,\dotsc,1)$ and find the maximal value.
\end{problem}
The sequence is given as \oeis{A284661}.
See also \href{https://arxiv.org/pdf/1704.00851.pdf}{Richard Stanley's Some Schubert Shenanigans}.



\subsection[schurbertComplexity]{Complexity}

Deciding if a monomial coefficient, or if a Schubert structure constant coefficient $c^w_{uv}$
is non-zero or not, can be done in probabilistic polynomial time, see \cite{AdveRobichauxYong2021x,PakRobichaux2025xII}.
See \href{https://www.youtube.com/watch?v=41UCZFIxZpY}{Colleen Robichaux --- Deciding positivity of Schubert coefficients - IPAM at UCLA}
for a Youtube video on this topic.


\section[schubertBCD]{Type B/C/D Schubert polynomials}

\begin{polydata}{schubertBCD}
  Name     & Type B/C/D Schubert polynomials \\
  Space    & All \\
  Basis    & True \\
  Rating   & 2 \\
  Bib      & FominKirillov1996 \\
  Year     & 1996 \\
  Symbol   & $\schubert^B_\omega(\xvec)$ \\
  Keywords & divided-difference  \\
  Category & Schubert \\
\end{polydata}


In type $B$, there are combinatorial analogs of Schubert polynomials introduced in \cite{FominKirillov1996}.
They also introduce \hyperref[stanleySymBC]{type $B$ Stanely symmetric functions}.

Divided difference formulas for Schubert polynomials in types $B$, $C$ and $D$ were 
introduced by Billey and Haiman \cite{BilleyHaiman1995}.
See also Sara Billeys PhD thesis on Schubert polynomials for classical groups \cite{Billey1994Thesis}.

Type $B$, $C$ and $D$ double Schubert polynomials are defined in \cite{IkedaMihalceaNaruse2011}.



\section[schubertSkew]{Skew Schubert polynomials}


\begin{polydata}{schubertSkew}
  Name   & Skew Schubert polynomials \\
  Space  & All \\
  Basis  & No \\
  Rating & 1 \\
  Bib    & LenartSotille2003 \\
  Year   & 2003 \\
  Category & Schubert \\
\end{polydata}

In \cite{LenartSotille2003}, a skew version of Schubert polynomials are defined.
They expand positively into Schubert polynomials, and a formula in the monomial basis is given.


\section[schubertDouble]{Double Schubert polynomials}

\begin{polydata}{schubertDouble}
  Name     & Double Schubert polynomials \\
  Space    & All \\
  Basis    & True \\
  Rating   & 1 \\
  Bib      & LascouxSchutzenberger1982 \\
  Year     & 1982 \\
  Symbol   & $\schubert_\omega(\xvec;\yvec)$ \\
  Keywords & divided-difference  \\
  Category & Schubert \\
\end{polydata}


The double Schubert polynomials were introduced by I. G. Macdonald \cite{Macdonald1991}.
\todo{also introduced by Lascoux and Schützenberger in \cite{LascouxSchutzenberger1982}}

They are defined as
\begin{equation*}
 \schubert_\omega(x_1,\dotsc,x_n,y_1,\dotsc,y_n) = \partial_{\omega^{-1}\omega_0} \prod_{i + j \leq n} (x_i-y_j).
\end{equation*}
where the divided difference operator act on the $\xvec$-variables.
By letting $y_i=0$, we get the usual Schubert polynomials. 
The double Schubert polynomials generalize the double Schur polynomials.

There is a pipe-dream formula, expressing $\schubert_\omega(\xvec,\yvec) $
as a sum over \hyperref[schubertPipeDreamFormula]{pipe dreams}:
\begin{equation*}
 \schubert_\omega(\xvec,\yvec) = \sum_{D \in RC(\omega) } \prod_{(i,j) \in D} \left(x_{i} - y_{j} \right).
\end{equation*}
See also the survey by N. Bergeron and S. Billey \cite{BergeronBilley1993}.

In a recent paper, A. Knutson \cite{Knutson2019} provides new proofs that 
several combinatorial formulas for the double Schubert polynomials.


\subsection[schubertDoubleGiambelliFormula]{Giambelli formula}


\todo{Explain the giambelli formula https://projecteuclid.org/download/pdf_1/euclid.em/1048516036 }


The Giambelli formula for double Schubert polynomials state that
\begin{equation*}
 \schubert_w(\xvec;\yvec) = \sum_{\substack{ v^{-1}u=w \\ \length(u)+\length(v) = \length(w) }}
 (-1)^{\length(v)}\schubert_u(\xvec) \schubert_v(\yvec).
\end{equation*}



\section[schubertAffine]{Affine Schubert polynomials}

\begin{polydata}{schubertAffine}
  Name & Affine Schubert polynomials \\
  Space  & All \\
  Basis  & No \\
  Rating & 1 \\
  Bib    & Lam2006Schubert \\
  Keywords & affine, schubert \\
  Year   & 2006 \\
  Category & Schubert \\
\end{polydata}


In \cite{Lee2015}, an affine extension of Schubert polynomials (indexed by affine permutations) 
is defined via divided difference operators after being introduced by T. Lam in \cite{Lam2006Schubert}.

In the \href{https://www.youtube.com/watch?v=c7aY09b7XjI}{Youtube video, Seungjin Lee introduces the affine Schubert polynomials}.
See 47:16 for the definition.

\todo{ add definition }

In \cite[Thm. 6.1]{LamLeeShimozono2019}, the authors prove that the affine Schubert polynomials 
expand positively in the monomial basis.


\section[schubertInvolution]{Involution Schubert polynomials}

\begin{polydata}{schubertInvolution}
 Name & Involution Schubert polynomials \\
  Space  & All \\
  Basis  & No \\
  Rating & 1 \\
  Symbol & $\ischubert_y(x_1,\dotsc,x_n)$ \\
  Bib    & HamakerMarbergPawlowski2018 \\
  Keywords & divided-difference, schubert \\
  Year   & 2018 \\
  Category & Schubert \\
\end{polydata}

The involution Schubert polynomials were introduced in \cite{WyserYoung2016}.
Additional properties are proved in \cite{HamakerMarbergPawlowski2018}, 
where the authors introduce the name of the family of polynomials.
The following definition is from that paper. 

Let $y \in I_n$, the set of involutions in $\symS_n$, and let $A(y)$ denote the set of 
minimal-length permutations $w$ such that $y = w^{-1} \circ w$.
That is, we consider all minimal factorizations of $y$ into simple transpositions 
of the form $(s_{i_1}\dotsc s_{i_k})\circ (s_{i_k}\dotsc s_{i_1})$.
Then let
\[
\ischubert_y(x_1,\dotsc,x_n) \coloneqq \sum_{w \in A(y)} \schubert_w(x_1,\dotsc,x_n)
\]
There is also a recursive way to produce these polynomials via divided difference operators.

The stable limit of these yield the \hyperref[stanleySymInvolution]{involution Stanley symmetric functions}.

A combinatorial model using \emph{involution pipe dreams} for these polynomials
is given by Hamaker, Marberg and Pawlowski in \cite{HamakerMarbergPawlowski2019}.

\todo{Add more info}


\section[schubertQuantum]{Quantum Schubert polynomials}


\begin{polydata}{schubertQuantum}
 Name & Quantum Schubert polynomials \\
  Space    & All \\
  Basis    & Yes \\
  Rating   & 3 \\
  Symbol   & $\schubert^q_\omega(\xvec)$ \\
  Bib      & FominGelfandPostnikov1997 \\
  Keywords & schubert \\
  Year     & 1997 \\
  Category & Schubert \\
\end{polydata}


The quantum Schubert polynomials $\schubert^q_\omega(\xvec)$
is a deformation of the Schubert polynomials by a  vector $q=(q_1,\dotsc,q_{n-1})$.
These were introduced in \cite{FominGelfandPostnikov1997}.

Recall the \hyperref[schubertAsSumOfElementary]{formula}
that expresses the Schubert polynomials as sums of products of elementary symmetric functions:
\[
\schubert_\omega(\xvec) = \sum a_{k_1 \dotsc k_n} \elementaryE_{k_1}(1) \elementaryE_{k_2}(2) \dotsm \elementaryE_{k_n}(n)
\]
The quantum Schubert polynomials are then defined as
\[
\schubert_\omega(\xvec) = \sum a_{k_1 \dotsc k_n} \elementaryE^{q}_{k_1}(1) \elementaryE^{q}_{k_2}(2) \dotsm \elementaryE^{q}_{k_n}(n)
\]
where 
\[
\sum_{i=0}^k \elementaryE^{q}_{k}(i)t^i = \det(1+tG_k) \text{ and } G_k=
\begin{vmatrix}
x_1 & q_1 & 0 & \ldots & 0 \\
-1 & x_2 & q_2 & \ldots & 0 \\
0 & -1 & x_3 & \ldots & 0 \\
\vdots & \vdots &   & \ddots & \vdots \\
0 & 0 & 0 & \ldots & x_k \\
\end{vmatrix}
\]
By letting $q_i=0$, one recovers the classical Schubert polynomials.


There is a bumpless pipedream model for the quantum double Schubert polynomials, see \cite[Thm. 3.3]{LeOuyangTaoRestivoZhang2024x}.


\subsection[schubertQuantumMonksRule]{Quantum Monks rule}

The following refinement of Monk's rule is given in \cite[Theorem 7.1]{FominGelfandPostnikov1997}.
Let $s_r$ be a simple transposition. Then
\[
\schubert^q_{s_r}(\xvec) \schubert^q_\omega(\xvec) = 
(x_1+x_2+\dotsb + x_r) \schubert^q_\omega(\xvec) 
=
\sum \schubert^q_{\omega t_{ab}}(\xvec) +
\sum q_{cd} \schubert^q_{\omega t_{cd}}(\xvec)
\]
where the first sum is over all transpositions $t_{ab}$ such that $a\leq r \lt b$ and 
$\length(\omega t_{ab}) = \length(\omega)+1$,
and the second sum runs over all
transpositions $t_{cd}$ such that $c\leq r \lt d$ and 
$\length(\omega t_{cd}) = \length(\omega)-\length(t_{cd}) =
\length(\omega)-2(d-c)+1$.

Here, $q_{cd} = q_c q_{c+1} \dotsm q_d$.

\subsection[schubertQuantumPieriRule]{Quantum Pieri rule}


\subsection[schubertQuantumSotilleRule]{Quantum Sotille rule}

\todo{This is by BBCSS, 2017}

\subsection[schubertQuantumMurnaghanNakayama]{Quantum Murnaghan--Nakaygama}

This is work in progress (2018), by authors BBCSS.

\todo{BBCSS has a conjecture on this.}


\section[schubertQuantumDouble]{Quantum Double Schubert polynomials}


\begin{polydata}{schubertQuantumDouble}
 Name & Quantum Double Schubert polynomials \\
  Space    & All \\
  Basis    & Yes \\
  Rating   & 1 \\
  Symbol   & $\schubert^q_\omega(\xvec;\yvec)$ \\
  Bib      & KirillovMaeno2000 \\
  Keywords & schubert \\
  Year     & 2000 \\
  Category & Schubert \\
\end{polydata}


The double quantum Schubert polynomials were introduced by A. Kirillov and T. Maeno \cite{KirillovMaeno2000}.


\subsection[schubertQuantumDoubleCauchyIdentity]{Cauchy identity}

In \cite{KirillovMaeno2000}, the following Cauchy identity for
quantum double Schubert polynomials is given:
\[
\sum_{\omega \in \symS_n} \schubert^q_\omega(\xvec) \schubert^q_{\omega \omega_0}(\yvec)
= \schubert^q_{\omega_0}(\xvec;\yvec)
\]


\section[schubertUniversal]{Universal Schubert polynomials}


\begin{polydata}{schubertUniversal}
  Name & Universal Schubert polynomials \\
  Space    & All \\
  Basis    & Yes \\
  Rating   & 1 \\
  Symbol   & $\schubert_w(c;d)$ \\
  Bib      & Fulton1999 \\
  Keywords & schubert \\
  Year     & 1999 \\
  Category & Schubert \\
\end{polydata}


The universal Schubert polynomials were introduced by W. Fulton, \cite{Fulton1999} and 
generalize the (double) quantum Schubert polynomials.


The double universal Schubert polynomials are defined via
\begin{equation*}
 \schubert_w(c;d) = \sum_{u,v} (-1)^{\length(v)}\schubert_u(c) \schubert_v(d)
\end{equation*}
as for \hyperref[schubertDoubleGiambelliFormula]{double Schubert polynomials}.


\section[schubertTwisted]{Twisted Schubert polynomials}

\begin{polydata}{schubertTwisted}
 Name & Twisted Schubert polynomials \\
  Space    & All \\
  Basis    & Yes \\
  Rating   & 1 \\
  Symbol   & $\tilde{\schubert}_\omega$ \\
  Bib      & Liu2019  \\
  Keywords & schubert \\
  Year     & 2019 \\
  Category & Schubert \\
\end{polydata}

The \defin{twisted Schubert polynomials}
are defined via $\tilde{\schubert}_{\omega_0}=x_1^{n-1} x_2^{n-2}\dotsm x_{n-1}$,
and the recursion $\tilde{\schubert}_{\omega s_i}= (s_i + \partial_i)\tilde{\schubert}_{\omega}$.

It was recently proved in \cite{Liu2019} that these are positive in the monomial basis.
This is the earliest reference that I could find that explicitly define these polynomials.
They are implicitly studied in earlier works, and are related to
the Chern--Schwartz--MacPherson classes of Schubert cells in flag varieties. 
