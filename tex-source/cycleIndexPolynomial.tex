\metatitle{Cycle index polynomials}
\metadescription{An introduction to cycle index polynomials, their definition via powersum symmetric functions, their Schur-positivity, and the Foulkes conjecture.}
\metakeywords{cycle index polynomials,schur-positive}


\section[cycleIndexPolynomial]{Cycle index polynomials}

\begin{polydata}{cycleIndexPolynomial}
  Name     & Cycle index polynomials \\
  Space    & Sym   \\
  Basis    & False   \\
  Rating   & 2     \\
  Bib      & Polya1937\\
  Year     & 1937 \\
  Keywords & schur-positive \\
  Symbol   & $\cyc_G(\xvec)$ \\
  Category & Other \\
\end{polydata}

The cycle index polynomial (\emph{Zyklenzeiger}) of a group $G$ was 
introduced by \name{George Pólya} in \cite[Eq. (1,5)]{Polya1937}.
Let $G \subseteq \symS_n$ be a subgroup. The \defin{cycle index polynomial} is defined 
in terms of the \hyperref[powerSum]{powersum symmetric functions} as the average
\[
\cyc_G(\xvec) \coloneqq \frac{1}{|G|} \sum_{\pi \in G} \powerSum_{\type(\pi)}(\xvec).
\]

Let $\symS_n/G$ denote the set of let cosets of $G$, and let $\symS_n$ act on $\setC[\symS_n/G]$.
This makes $\setC[\symS_n/G]$ into an $\symS_n$-module, 
and its \hyperref[frobeniusCharacteristic]{Frobenius characteristic} is given by $\cyc_G(\xvec)$. 
In particular, $\cyc_G(\xvec)$ is Schur-positive.
For more info on the cycle index polynomial, we refer to \cite{LoehrWarrington2019}.


\begin{conjecture}[Foulkes Conjecture, \cite{Foulkes1950}]
Let $X_{a,b}$ be the set of set-partitions of $\{1,2,\dotsc,ab\}$ into 
$a$ blocks of size $b$. The symmetric group $S_{ab}$ act on $X_{a,b}$
in the obvious manner. Let $G_{a,b}$ be the stabilizer subgroup of some fixed element in 
$X_{a,b}$ under this action.
Let
\[
\cyc_{G_{a,b}}(\xvec) = \sum_\lambda c_{\lambda,a,b}\; \schurS_\lambda(\xvec).
\]
Foukes conjecture states that whenever $a\leq b$, then for all $\lambda$,
$c_{\lambda,a,b} \leq c_{\lambda,b,a}$.
This is equivalent with the statement that 
$
\completeH_b[\completeH_a]-\completeH_a[\completeH_b]
$
is Schur-positive whenever $a \leq b$. 
The notation here indicates \hyperref[plethysm]{plethysm}.
\end{conjecture}

See also \url{https://arxiv.org/pdf/2008.13070.pdf} and \url{https://arxiv.org/pdf/2507.06220}.
\todo{add this refs}




\section[lyndonSymmetric]{Lyndon symmetric functions}


\begin{polydata}{lyndonSymmetric}
  Name     & Lyndon symmetric functions \\
  Space    & Sym   \\
  Basis    & False   \\
  Rating   & 2  \\
  Bib      & GesselReutenauer1993\\
  Year     & 1993 \\
  Keywords & schur-positive \\
  Symbol   & $L_\lambda(\xvec)$ \\
  Category & Sym \\
\end{polydata}

The \defin{Lyndon symmetric functions}, also known as \defin{higher Lie characters} in \cite{AdinHegendusRoichman2019x},
are described using 
the \hyperref[gessel]{fundamental quasisymmetric functions}:
\[
L_{\lambda}(\xvec) \coloneqq \sum_{\pi \in K_\lambda} \gessel_{n,D(w)}(\xvec)
\]
where $K_\lambda$ is the set of permutations in $\symS_n$ with cycle type $\lambda$.
See also \cite[Def 1.5]{AdinHegendusRoichman2019x} for a representation-theoretical definition,


The Lyndon symmetric functions are also covered in \cite[p. 480]{StanleyEC2},
and it was shown by Gessel--Reutenauer \cite[Thm. 3.6]{GesselReutenauer1993}, that
we have
\begin{equation}
L_{n}(\xvec) = \frac{1}{n} \sum_{d \mid n} \mu(d)\powerSum_{d}^{n/d},
\quad 
L_{n^k}(\xvec) = \completeH_k[L_{n}],
\quad 
\text{ and }
\quad
L_{1^{m_1}2^{m_2}\dotsb}(\xvec) = L_{1^{m_1}} L_{2^{m_2}} \dotsm.
\end{equation}
These relations uniquely define the $L_{\lambda}(\xvec)$ where 
we write $\lambda = 1^{m_1}2^{m_2}\dotsb$.
Moreover, since $L_{n}(\xvec)$ can be shown to be Schur-positive,
it follows that the Lyndon symmetric functions are also Schur positive.
A monomial expansion for $L_{\lambda}(\xvec)$ is given by summing over all 
multisets of primitive necklaces whose sizes are given by $\lambda$.
Gesses then has a bijection which sends such multisets to
words of length $n$ whose \emph{standardization permutation} is $\lambda$.


See also \cite[Ex. 7.69c]{StanleyEC2} for some applications of Lyndon symmetric functions.
In particular, \name{Thomas Scharf} shows that a certain character is 
\[
  \sum_{n\geq 0} \completeH_{n}\left[ \sum_{d\mid k} L_d \right].
\]


\begin{problem}[Thrall's problem]
Find a combinatorial formula for the Schur expansion of $L_{\lambda}(\xvec)$.
\end{problem}
See the Youtube video \href{https://www.youtube.com/watch?v=bQc6D0aAu0M}{GOCC 4/12/23 "A crystal base approach to Thrall's problem"}
for background and a possible approach to solving this problem.



\begin{problem}[Yuval Roichman, FPSAC 2023]
Prove that for any $\lambda \vdash n$
\[
  \sum_{k=0}^n \langle L_{\lambda}, \schurS_{k,1^{n-k}} \rangle t^k
\]
is unimodal. This is \cite[Conj. 7.13]{AdinHegendusRoichman2019x}.
\end{problem}


\begin{problem}
It is an open problem (Stanley, 2022)
to find the (co)dimension of the span of the $L_\lambda$,
see 38:00 in \href{https://www.youtube.com/watch?v=NDAzHTP3vOs}{this youtube video}.
\end{problem}


The functions $L_{n}(\xvec)$ are Lie characters, see \cite[Thm. 8.3]{Reutenauer1993}.
\name{Richard Stanley} discusses these also in \href{https://math.mit.edu/~rstan/transparencies/whouse.pdf}{these slides}.
One can show that 
\[
L_{n}(\xvec) = \sum_{\lambda \vdash n} |\{T \in \SYT(n) : \maj(T) \equiv_n 1\}| \schurS_\lambda(\xvec).
\]

There is also the \defin{Whitehouse module}, see \cite{StanleyWhitehouse}.
Their Frobenius characteristic is 
\[
\powerSum_{1} L_{n-1}(\xvec) - L_{n}(\xvec)
\]
so this is Schur-positive.
Can we find a combinatorial proof that these are Schur-positive?


\todo{Add https://arxiv.org/pdf/2210.04478}

