\metatitle{Permutations and patterns}
\metadescription{Permutations, matrices and patterns}



\section[permutationMatrix]{Permutation matrices}


For permutations $\pi,\sigma \in \symS_n$, we think
of multiplication as composition of functions, so that
\[
  (\pi \circ \sigma)(k) = \pi(\sigma(k)) \text{ for $k \in [n]$.}
\]
Given $\pi$, we associate its \defin{permutation matrix},
\[
 P_\pi = \left( \delta_{i, \pi(j)} \right)_{1 \leq i, j \leq n}.
\]
The permutation matrix of the composition $\pi \circ \sigma$ is $P_{\pi} P_{\sigma}$.
We follow the convention in \cite{Kitaev2011}
and illustrate permutation matrices by letting row indices start at the \emph{bottom},
as in the figure below. 
From now on, this is the picture we have in mind when referring to permutation matrices.

\begin{example*}[The permutation matrix for 18537462]
The permutation matrix $P_\pi$ associated with $\pi = [1, 8, 5, 3, 7, 4, 6, 2]$
using conventional row/column coordinates, and the convention we use.
The rightmost figure is simply the graph of the function $i \mapsto \pi(i)$ 
(using Cartesian coordinates).
\begin{figure}
$
P_\pi = 
\begin{pmatrix}
  1 & 0 & 0 & 0 & 0 & 0 & 0 & 0 \\
 0 & 0 & 0 & 0 & 0 & 0 & 0 & 1 \\
 0 & 0 & 0 & 1 & 0 & 0 & 0 & 0 \\
 0 & 0 & 0 & 0 & 0 & 1 & 0 & 0 \\
 0 & 0 & 1 & 0 & 0 & 0 & 0 & 0 \\
 0 & 0 & 0 & 0 & 0 & 0 & 1 & 0 \\
 0 & 0 & 0 & 0 & 1 & 0 & 0 & 0 \\
 0 & 1 & 0 & 0 & 0 & 0 & 0 & 0 \\
\end{pmatrix}
$
\begin{ytableau}
\none 8& \, & \bullet & \, & \, & \, & \, & \, & \, \\
\none 7& \, & \, & \, & \, & \bullet & \, & \, & \, \\
\none 6&\, & \, & \, & \, & \, & \, & \bullet & \, \\
\none 5& \, & \, & \bullet & \, & \, & \, & \, & \, \\
\none 4 & \, & \, & \, & \, & \, & \bullet & \, & \, \\
\none 3 & \, & \, & \, & \bullet & \, & \, & \, & \, \\
\none 2 & \, & \, & \, & \, & \, & \, & \, & \bullet \\
\none 1 & \bullet & \, & \, & \, & \, & \, & \, & \, \\
\none   & \none 1 &\none 2 &\none 3 &\none 4 &\none 5 &\none 6 &\none 7 &\none 8 
\end{ytableau}
\end{figure}
\end{example*}

The \defin{Rothe diagram} (named after \name[H. A. Rothe]{Heinrich August Rothe}) of a permutation $\pi \in \symS_n$,
denoted $D(\pi)$, is defined as the set 
\[
D(\pi) \coloneqq \{(\pi_j,i) : i < j \text{ and } \pi_i > \pi_j \}.
\]
That is, we draw the permutation matrix for $\pi$ and strike out the boxes appearing somewhere to the right or above a dot (using our convention).
This is illustrated in the figure below.
\begin{figure}
  \svgimg[width=0.25\textwidth]{svg-images/rothe-diagram.svg}{The Rothe diagram for the permutation $\pi = [1, 8, 5, 3, 7, 4, 6, 2]$.}
\end{figure}



\subsection[permPatterns]{Permutation patterns}

A permutation $\pi \in \symS_n$ is said to \defin{contain} the pattern $\sigma \in \symS_k$,
if there is a subsequence of $\pi$ which is order-isomorphic to $\sigma$.
For example, $\pi = [1, 8, 5, 3, 7, 4, 6, 2]$ contains the pattern $[2,3,1]$
as the subsequence $3, 6, 2$ in $\pi$ have its elements in the same relative order.

\begin{example*}[A permutation in $\symS_8$ and the pattern 231]
The permutation matrices associated with $\pi = [1, 8, 5, 3, 7, 4, 6, 2]$ 
and $\sigma = [2,3,1]$. Here we see that $\pi$ contains the pattern $\sigma$
(there are other instances of this pattern in $\pi$).
\begin{figure}
\begin{ytableau}
 \, & \bullet & \, & \, & \, & \, & \, & \, \\
 \, & \, & \, & \, & \bullet & \, & \, & \, \\
 \, & \, & \, & \, & \, & \, & \odot & \, \\
 \, & \, & \bullet & \, & \, & \, & \, & \, \\
 \, & \, & \, & \, & \, & \bullet & \, & \, \\
 \, & \, & \, & \odot & \, & \, & \, & \, \\
 \, & \, & \, & \, & \, & \, & \, & \odot \\
 \bullet & \, & \, & \, & \, & \, & \, & \, \\
\end{ytableau}

\begin{ytableau}
 \, & \bullet & \, \\
 \bullet & \, & \, \\
 \, & \, & \bullet \\
\end{ytableau}
\end{figure}
\end{example*}


\begin{example*}[Structure of 132-avoiding permutations]
The number of permutations in $\symS_n$ avoiding $132$ is given by the Catalan number $\frac{1}{n+1}\binom{2n}{n}$.
The structure of a permutation matrix avoiding the pattern $132$ can be seen in the figure below.
The larger regions on the left and right are also $132$-avoiding.
\begin{figure}
  \svgimg[width=0.35\textwidth]{svg-images/132PermutationStructure.svg}{Structure of a 132-avoiding permutation.}
\end{figure}
This structure explains why 132-avoiding permutations satisfy the recursion for Catalan numbers.
\end{example*}


In \cite{Vatter2019}, the author studies possible growth rates for pattern-avoiding permutations.

\subsection[directSumPermutations]{Direct sums and skew sums of permutations}

Given $\pi \in \symS_m$ and $\sigma \in \symS_n$, we define the \defin{direct sum}
and the \defin{skew sum} of $\pi$ and $\sigma$ as
\begin{align}
 \pi \ast \sigma \coloneqq & [pi_1,\dotsc,\pi_m,\sigma_1+m,\dotsc,\sigma_n+m]  \in \symS_{m+n} \\
 \pi \times \sigma \coloneqq & [pi_1+n,\dotsc,\pi_m+n,\sigma_1,\dotsc,\sigma_n]  \in \symS_{m+n}
\end{align}
Note that these operations are not commutative, but they are associative.


\section[namedPermutationSets]{Named sets of permutation}

\todo{Organize this section better - or better yet, extract to separate wide page.}

\begin{itemize}

\item \defin{Up-down} ---  in one-line notation, we have $\pi_1 \lt \pi_2 \gt \pi_3 \lt \pi_4 \dotsb$.
\defin{Down-up} permutations are of the form $\pi_1 \gt \pi_2 \lt \pi_3 \gt \pi_4 \dotsb$.
Enumerated by the Euler numbers \oeis{A000111}.

\item \defin{André} --- Either up-down or down-up, \oeis{A001250}.

\item \defin{Derangements} --- permutations with no fixed point, \oeis{A000166}.

\item \defin{Dominant} --- 132-avoiding. Enumerated by the Catalan numbers, \oeis{A000108}.

\item \defin{Baxter} --- avoids $2-14-3$ and $3-14-2$, see  \url{https://doi.org/10.1016/S0012-365X(97)00112-X}.

\item \defin{Boolean} --- 123-avoiding and 3412-avoiding, see \cite{Tenner2007,GunawanPanRussellTenner2022x}.
A permutation is boolean iff the order ideal it defines in the Bruhat order, is a Boolean lattice, see \cite[Thm. 4.3]{Tenner2007}.
Enumerated by odd-indexed Fibonacci numbers, \oeis{A001519}.

\item \defin{Fireworks} --- Same as 3-12 avoiding, see \url{https://arxiv.org/pdf/2508.09107}. This means that the initial terms of decreasing runs, are increasing.

\item \defin{Fully commutative} --- 123-avoiding, see \cite{BilleyJockuschStanley1993,GunawanPanRussellTenner2022x}.
Enumerated by the Catalan numbers, \oeis{A000108}.

\item A \defin{simsun permutation} is a permutation such that restricting to entries $1,2,\dotsc,k$,
it does not have a double descent, i.e., $\pi(i)\gt \pi(i+1) \gt \pi(i+2)$, see \cite{MaYeh2016}. 
The number of simsun permutations in $\symS_n$ is given by the Euler number $E_{n+1}$, see \oeis{A000111}.

\item \defin{King permutations} are permutations where no two adjacent entries differ by $1$ in one-line notation.
This can be described via mesh-pattern avoidance, see \oeis{A002464}.

\item A permutation is \defin{vexillary} if it avoids the pattern 2143, see \oeis{A005802}.

\item A permutation is \defin{Grassmann} if it has at most one descent. These are special cases of vexillary permutations.
There are $2^n-n$ Grassman permutations in $\symS_n$, see \oeis{A000325}.

\item A permutation $\pi$ is \defin{bi-Grassmann} if both $\pi$ and $\pi^{-1}$ are Grassman. Equivalent with avoiding 2413.
This set has size $1+\binom{n+1}{3}$.

\item \defin{Separable} permutations are those which avoids $2413$ and $3142$, see \oeis{A006318}.
For reference, see \cite{West1995}.

\item \defin{Skew-merged} permutations avoid $2143$ and $3412$. These are exactly the ones 
created by a union of an increasing sequence, and a decreasing sequence, see \cite[Thm.2.9]{Stankova1994}, \oeis{A029759}.
See \cite{Atkinson1998} for exact enumeration and connection with RSK.

\item \defin{Flattened permutations}, where runs are in lex order. 
Enumerated by Bell numbers \oeis{A000110}, see \cite[Prop. 16]{NabawandaRakotondrajaoBamunoba2020}.

\item \defin{Parity-alternating permutations} are permutations which sends odd numbers to odd numbers,
and even numbers to even, see \cite{KebedeRakotondrajao2021}. 
This can be generalized to \defin{mod-k-alternating permutations},
where we demand that $\pi(i) \equiv i \mod k$. 
For pattern-avoidance in mod-k-alternating permutations, see \cite{AlexanderssonFufaGetachewQiu2022x}.


\item The \defin{Richardson permutation} with parameters $(k_1,k_2,\dotsc,k_m)$
is given by the following in two-line notation.
\[
\begin{pmatrix}
1 & 2 &       \dotsc & k_1   & k_1+1 & k_1+2 & \dotsc & k_1 + k2 & k_1+k_2+1 & \dotsc \\
k_1 & k_1-1 & \dotsc,   1		 & k_1+k_2 & k_1+k_2-1 & \dotsc & k_1+1 & k_1+k_2+k_3 & \dotsc \\
\end{pmatrix}
\]
Basically, we take the unique longest permutation in each of the sets $\symS_{k_1},\dotsc,\symS_{k_m}$,
and produce the \emph{direct sum} of these.

\item \defin{Wachs permutations} --- \oeis{A359039}. 
These are the permutations in $\symS_n$ defined as
\[
 W_n \coloneqq \{ \sigma \in \symS_n : |\sigma^{-1}(i)-\sigma^{-1}(i^*)|\leq 1 \text{ for all } i \in [n-1] \},
\]
where $i^*$ is defined as 
\[
 i^* \coloneqq \begin{cases}
 i-1 &\text{if $i$ is even}  \\
 i+1 &\text{if $i$ is odd and $i+1 \leq n$}  \\
 n & \text{otherwise.}
 \end{cases}
\]
These permutations are studied in \cite{BrentiSentinelli2022x}, and there are generalizations to
other types.

\end{itemize}




\section[permutationsTypeB]{Permutations of type $B$}

The set of permutations of type $B$, is defined as the permutations on $B_n \coloneqq \{\pm 1, \pm 2,\dotsc, \pm n\}$
with the property that $\pi(-i)=-\pi(i)$. This is also known as the \hyperref[weyl-lie-groups]{hyperoctahedral group}.


\todo{Add colored permutations. See Danai's thesis}
%\section[permutationsColored]{Colored permutations}


\section[permutationsStirling]{Stirling permutations}

In \cite{GesselStanley1978}, \name[Ira Gessel]{Gessel} and \name[Richard P. Stanley]{Stanley} 
introduced the set of \defin{Stirling permutations, $Q_n$}.
These are permutations of the multiset $\{1,1,2,2,3,3,\dotsc,n,n\}$ with the additional property
that for between the two entries equal to $i \in [n]$, only entries larger than $i$ may appear.
For example, $12234431$ is an element in $Q_n$.

We have that $|Q_n| = (2n-1)!!$.
