
\metatitle{Parking functions}
\metadescription{Properties of parking functions}


\section[parking-functions]{Parking functions}


\begin{definition}
Let $(a_1, a_2, \dots, a_n)$ be a sequence of positive integers.
We think of these as $n$ cars, where $a_i$ is the preferred parking spot for car $i$.
The parking spots are labeled $1, 2, \dotsc, n$ and the $n$ cars attempt to park according to the following
rules:
\begin{itemize}
  \item Car \( i \) attempts to park in spot \( a_i \).
  \item If that spot is taken, it proceeds to the next available empty spot to the right (higher index).
\end{itemize}
If all cars can park successfully, then the sequence \( (a_1, \dots, a_n) \) is a  \defin{parking function}.
\end{definition}

There are $(n+1)^{n-1}$ parking functions in total, Pyke (1956) and Konheim and Weiss (1966).
Pollak (1966) has a neat proof for this formula by adding an extra $(n+1)$th spot.


\begin{theorem}
The sequence $(a_1, \dots, a_n)$ of positive integers is a 
parking function if and only if, when rearranged 
in non-decreasing order $(b_1 \le b_2 \le \dots \le b_n)$, it satisfies
\[
b_i \le i \quad \text{for all } i = 1, \dots, n.
\]
\end{theorem}
\begin{proof}
\textbf{First direction:} 
If $(a_1, \dots, a_n)$ is a parking function, then $b_i \leq i$ for all $i$.

For any $k$, there must be at least $k$ cars preferring spots among $1,2,\dotsc,k$.
This immediately gives that $b_k \leq k$ for all $k$.

\textbf{Other direction:} If $b_i \le i$ for all $i =1,2,\dotsc,n$, 
then$(a_1, \dots, a_n)$ is a parking function.

Suppose this statement is not correct. We can then (by minimal counterexample) 
assume the first $n-1$ cars manage to park, but the last car fails to park.
Then there must be some (smallest) $j$ such that spots $\{j,j+1,\dotsc,n\}$
are occupied after the first $n-1$ cars, and $a_n \geq j$.

Since the spot $j-1$ is empty, we know that there are $(n-j+1)+1$
cars that want to park at spot at least $j$.

But then, only (the remaining) $j-2$ cars want to park at spots $\{1,2,\dotsc,j-1\}$.
This implies that $b_{j-1} > j-1$, violating our assumption.
\end{proof}


The number of \emph{weakly increasing} parking functions is given by the Catalan numbers.
See \url{https://arxiv.org/pdf/2511.20796} for a Pollak-style proof of this.
