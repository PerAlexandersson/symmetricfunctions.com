
\metatitle{Extended chromatic symmetric functions}
\metadescription{Extended chromatic symmetric functions and connections with Tutte symmetric functions.}


\section[extendedChromatic]{Extended chromatic symmetric functions}


\begin{polydata}{extendedChromatic}
  Name   & Extended chromatic symmetric functions \\
  Space  & Sym \\
  Basis  & False \\
  Rating & 2 \\
  Bib    & CrewSpirkl2019 \\
  Year   & 2019 \\
\end{polydata}


L. Crew and S. Spirkl \cite{CrewSpirkl2019x} introduce a vertex-weighted version of 
the \hyperref[chromaticQuasisymmetric]{chromatic symmetric functions}.
This has the advantage that it fulfills a deletion-contraction relation.
Furthermore, this family of polynomials is also $\omega \powerSum$-positive. 
This can easily be see from \cite{AlexanderssonSulzgruber2019}.

In \cite{CrewSpirkl2020x}, the authors examine a new basis, the \defin{complete multipartite basis},
which is closely related to the extended chromatic symmetric functions.

The deletion-contraction relation does not extend to the $q$-weighed version with ascends.

In \cite{AliniaeifardWangvanWilligenburg2021}, the authors study weighted paths with equal extended
chromatic symmetric functions.

In \cite{AliniaeifardWangvanWilligenburg2021}, the authors consider a Tutte-symmetric extension of the
vertex-weighted chromatic symmetric functions.
This generalizes both \hyperref[tutteSymmetric]{Tutte symmetric functions}
and the \hyperref[chromaticQuasisymmetric]{chromatic symmetric functions}. 
They provide a spanning-tree formula for these, which then provide a new
spanning-tree formula for the chromatic symmetric functions.

\todo{
Is there a quasi-symmetric chromatic version of this spanning-tree formula?
}
