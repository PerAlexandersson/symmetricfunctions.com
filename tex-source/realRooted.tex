\metatitle{Real-rooted and stable polynomials}

\metadescription{Polynomials with real roots, and their multivariate analogs. Also, Lorentzian polynomials.}


\section[polynomialRootIntro]{Introduction}

Whenever you encounter a family of polynomials, you
might want to study properties such as \emph{unimodality},
\emph{real-rootedness}, \emph{stability} or \emph{interlacing roots}.

I highly recommend \cite{Branden2015} and other works of \name{Petter Brändén}.
See also this \href{https://www.youtube.com/watch?v=wuQN0xaTkxE}{OPAC seminar}
about unimodality, real-rootedness and stable polynomials.


\section[polynomialUnimodality]{Unimodality, log-concavity and real roots}

Let $P = a_0 + a_1t+\dotsb + a_n t^n$ be a polynomial. It is called \emph{unimodal} if there is some $j$
such that $a_0 \leq a_1 \leq \dotsb \leq a_j \geq a_{j+1} \geq a_{j+2} \geq \dotsb \geq a_n$.

Suppose now all coefficients are positive. We say that $P$ is \defin{log-concave} if
\[
  a_{j-1}a_{j+1} \leq a_j^2
\]
for all relevant $j$. This property implies unimodality.

We get an even stronger property if the \defin{Newton inequalities} are satisfied:
\[
  \frac{a_{j-1}}{ \binom{n}{j-1} } \frac{a_{j+1}}{ \binom{n}{j+1} } \leq \frac{a_{j}^2}{ \binom{n}{j}^2 }.
\]

Finally, if $P$ has only real roots, it is called \defin{real-rooted}. 
Every real-rooted polynomial satisfies the Newton inequalities, see e.g. \cite{Branden2015}.


\begin{example*}[Ray through Pascal's triangle]
The following is proved in \cite{Yu2009}. 
Let $n \geq k \geq 0$ and $b \gt a \gt 0$, $k\lt b$.
Then the polynomial
\[
 \sum_{j \geq 0} \binom{n+ja}{k+jb} t^j
\]
is real-rooted. The proof uses an interpretation via lattice paths.
This solves \cite[Problem 12]{ForgcsTran2016}.
\end{example*}




\subsection[totallyNonnegative]{Totally non-negative matrices}

A matrix is \defin{totally non-negative} if every minor is non-negative.
One important class of such matrices arise from networks. Let $s_1,\dotsc,s_k$ 
and $e_1,\dotsc,e_k$ be vertices in a graph (or network),
and  let $a_{ij}$ count the number of paths from $s_i$ to $b_j$.
Then the matrix $(a_{ij})$ is totally non-negative.
Totally non-negative matrices are closed under taking products.

See \cite{BrandenLeite2024x} for some applications related to chains in posets.
This source also has several new methods for proving real-rootedness.

\begin{theorem}[Stanley, \cite{Stanley1998}]
A polynomial $Q(x)=1+a_1x+a_2x^2+\dotsb+a_n x^n$ has only 
negative real roots if and only if the product
\[
 \prod_{i \geq 1} Q(x_i)
\]
is Schur-positive.
\end{theorem}

\todo{Planar networks: https://arxiv.org/pdf/2512.08369}

\section[interlacingPolynomials]{Interlacing and interleaving roots}


The following definitions are from \cite{Wagner1992}.
See also \cite{Branden2015}

\begin{definition}[Interleaving polynomials]
Let $f$ and $g$ be polynomials with positive leading coefficients and 
with real roots $\{f_{i}\}$ and $\{g_{i}\}$, respectively.
We say that $f$ \defin{interlaces} $g$ if $\deg(f)+1=\deg(g)=d$ and
\[
g_1 \leq f_1 \leq g_2 \leq f_2 \leq \dotsm \leq  f_{d-1}\leq g_{d}.
\]
Moreover, we say that $f$ \defin{alternates left of} $g$ if $\deg(f)=\deg(g)=d$ and
\[
f_1 \leq g_1 \leq f_2 \leq \dotsm \leq f_{d}\leq g_{d}.
\]
We say that $f$ \defin{interleaves} $g$ if either $f$ interlaces $g$ or 
$f$ alternates left of $g$. We write this as $f \interl g$.
By convention, $0 \interl 0$, $0 \interl h$ and $h \interl 0$ whenever $h$ is a polynomial with 
a positive leading coefficient.
\end{definition}


\begin{lemma}[Folklore]
    Let $f, g$ be monic polynomials with $\deg(f) = \deg(g)+1$ and all real, non-positive roots.
    If $g \interl f$, then $f-tg$ is real-rooted and $g \interl f- tg \interl f$.
\end{lemma}


The following very useful lemma is from \cite[Sec. 3]{Wagner1992}.
\begin{lemma}[Wagner's lemma]
Let $f, g, h \in \setR[t]$ be real-rooted polynomials with
only real, non-positive roots and positive leading coefficients.
Then
\begin{itemize}
\item if $f \interl h$ and $g \interl h$ then $f+g \interl h$.
\item if $h \interl f$ and $h \interl g$ then $h \interl f+g$.
\item $g \interl f$ if and only if $f \interl tg$.
\end{itemize}
\end{lemma}

\name[D.G. Wagner]{David Wagner} also shows that $f \interl g$ implies 
that $f \interl \lambda f + \mu g \interl g$ for all $\lambda, \mu \geq 0$.


As an example, if $f$ is real-rooted, then $f' \interl f$.
Hence, recursions involving the derivative play well with interlacing roots.

\todo{Cauchy interlace theorem: https://arxiv.org/pdf/math/0502408
and this is same as  https://doi.org/10.1016/j.laa.2021.09.019
}
\todo{Binomial transform: https://doi.org/10.1112/jlms.70083 see also https://doi.org/10.1016/j.aam.2024.102776}



A more recent result is the following theorem by \name{Shi-Mei Ma} and \name{Yi Wang} \cite{MaWang2008}.
\begin{theorem}[Ma and Wang, 2008]
Suppose $f$ and $F$ are polynomials with positive leading coefficients,
such that
\[
 F(x) = u(x) f(x) + v(x) f'(x), \qquad  \deg(f) \leq \deg(F) \leq \deg(f) + 1
 \]
with $u(x),v(x)$ having real coefficients.
If $f$ is real-rooted and $v(r) \neq 0$ whenever $f(r)=0$,
then $F$ is real-rooted and $f \interl F$.
\end{theorem}
The theorem states more regarding root placements and multiplicities.


\subsection[sturmSequence]{Sturm sequences}


It is fairly common to see sequences of polynomials, $\{ P_n(t) \}_{n \geq 0}$
with the property that $P_j(t) \interl P_{j+1}(t)$ for all $j$.
If the leading coefficient is always positive and $\deg(P_n)=n$,
this is called a \defin{Sturm sequence}, see \cite{Liu2007}.
If we drop the degree condition, the sequence is called a \defin{generalized Sturm sequence}.

The notion of \defin{Sturm-unimodality} is defined in the analogous manner.


\todo{Add general thm from \cite{Liu2007} }

\subsection[limits]{Normality in the limit}

The recursions used to prove real-rootedness can sometimes be used to show
that one has convergence to a normal distribution, see \cite{Bender1973}.

\begin{theorem*}[Bender 1973, Harper 1967]
Suppose
\[
P_n(x) = \sum_{k} a_n(k) x^k = a_n \prod_j (x + r_j)
\]
where $r_j \geq 0$. Define
\[
\mu_n = \sum_j \frac{1}{1 + r_j}, \quad \sigma_n^2 = \sum_j \frac{r_j}{(1 + r_j)^2}.
\]
If $\sigma_n \to \infty$, then the \emph{normalized coefficients}
\[
p_n(k) = \frac{a_n(k)}{P_n(1)}
\]
are \emph{asymptotically normal}. That is, if we let $X_n$ be the random variable
with probability distribution given by $p_n(k)$, then
\[
\frac{X_n - \mu_n}{\sigma_n} \xrightarrow{d} \mathcal{N}(0,1).
\]
Moreover, if the $a_n(k)$ are log-concave (which follows from real-rootedness), a local limit theorem holds:
\[
p_n(k) \approx \frac{1}{\sigma_n \sqrt{2 \pi}} e^{ - (k - \mu_n)^2 / (2 \sigma_n^2) }
\]
uniformly in $k$.
\end{theorem*}

Bender provides several examples, including ordered set partitions,
Eulerian numbers, matchings on the $2\times n$ grid graph.

For Baxter permutations, see \cite{Zhao2024x}.
For P-recursively defined polynomials, see \cite{Li2025x}.

Several examples are also provided in \cite{Hitczenko2024x}.




\section[interlacingExamples]{Examples of Sturm sequences}


\begin{example*}[Eulerian polynomials, descents in permutations]

The \defin{Eulerian polynomials}, $A_n(t) \coloneqq \sum_{\sigma \in \symS_n} t^{\des(\sigma)}$.
These satisfy 
\[
A_n(t) = (1+t(n-1))A_{n-1}(t) +  t(1-t) A'_{n-1}(t),
\]
and from this recurrence, one can prove that $A_j(t) \interl A_{j+1}(t)$ for all $j$.
However, it is probably not possible to show this by only using Wagner's lemma above, 
as the size of the coefficient $(n-1)$ has an impact.

This result goes back to \name{Frobenius}, \cite{Frobenius1910}.

The proof proceeds as follows. Introduce $\tilde{A}_n(x) \coloneqq x A_n(x)$.
The recursion then becomes
\[
\tilde{A}_n(x) = x(1-x)\tilde{A}'_{n-1}(x) + nx \tilde{A}_{n-1}(x).
\]
Since $\tilde{A}'_{n-1} \interl \tilde{A}_{n-1}$, one can then see 
that $nx \tilde{A}_{n-1}(x) \interl x(1-x)\tilde{A}'_{n-1}(x)$.
Moreover, $x(1-x)\tilde{A}'_{n-1}(x)$ has a negative leading coefficient, so 
but the sum 
\[
x(1-x)\tilde{A}'_{n-1}(x) + nx \tilde{A}_{n-1}(x)
\]
has a positive leading coefficient due to $n \geq 2$ in the recursion.
By now drawing the right picture, the result follows.
\end{example*}


\begin{example*}[Multi-Eulerian polynomials]
Let $\symS_M$ be the permutations of some multiset $1^{\mu_1},2^{\mu_2},\dotsc,n^{\mu_n}$ with $N=|\mu|$.
Define 
\[
 A_M(x;q) \coloneqq  \sum_{\pi \in \symS_M } x^{\des(\pi)}q^{\maj(\pi)}.
\]
Then (MacMahon, \cite{MacMahon1960})
\[
 \frac{A_M(x;q)}{\prod_{j=0}^N (1-x q^j)} = \sum_{m \geq 0} \left(\prod_{j=1}^n \qbinom{\mu_j+m}{m}_q \right) x^m.
\]
The polynomials $A_M(x;1)$ are then real-rooted, see \cite[Section 2]{Simion1984}.
Ma and Pan \cite[Thm. 1.11]{MaPan2023} gives a proof of this using interlacing polynomials.
For stability in a multivariate version, see \cite[Eq. (4.7)]{BrandenHaglundVisontaiWagner2011}.

One can also consider a colored version, see \name{Danai Deligeorgaki}, \name{Bin Han}, \name{Liam Solus},
and obtain real-rooted results. 
MacMahon's formula above generalizes to their setting.
\end{example*}



\begin{example*}[Eulerian polynomials with descents from prescribed set]
In \cite{AthanasiadisEvangelinou2023xII} the authors show that for any $T \subseteq [n-1]$,
the polynomials
\[
 A_n^T(x) \coloneqq  \sum_{\substack{\pi \in \symS_n \\ \DES(\pi) \subseteq T}} x^{\des(\pi)}
\]
are real-rooted. The proof is based on refining the above polynomials 
by the value of $\pi(n)$, and then using a recursion which preserves an interlacing 
family of polynomials.
\end{example*}


\begin{example*}[Inversion-weighted Eulerian polynomials with fixed last entry]

Let us introduce the refinement $A_{n,j}(t,q) \coloneqq \sum_{\substack{\sigma \in \symS_n \\ \sigma(n)=j}} t^{\des(\sigma)} q^{\inv(\sigma)}$.
Then for any $q \gt 0$ we have that 
\[
 A_{n,n}(t,q), A_{n,n-1}(t,q), \dotsc, A_{n,1}(t,q)
\]
is an interlacing sequence, see \cite{SavageVisontai2015}.
A short proof is given in \cite{AthanasiadisEvangelinou2023x}.
\end{example*}


\begin{example*}[Type $B$ Eulerian polynomials]

We can instead consider \hyperref[permutationsTypeB]{permutations of type $B$}.
For $\pi \in B_n$, we set $\pi(0)\coloneqq 0$, and $\des(\pi)$ is the cardinality of
\[
 \{ i \in \{0,1,2,\dotsc,n-1\} : \pi(i) \gt \pi(i+1) \}.
\]
The polynomials $P_n(t) = \sum_{\sigma \in B_n} t^{\des(t)}$ then satisfy the recurrence
\begin{equation}
 P_n(t) = ((2n-1)t+1) P_{n-1}(t) + 2t(1-t) P'_{n-1}(t),
\end{equation}
from which one can prove the interlacing property, see \cite{Chow2022}.
They in fact show that the corresponding operator preserves
real-rootedness, using the method of \name{Julius Borcea} and \name{Petter Brändén} on stable polynomials.

The analogous polynomials for type $D$, are proved to be real-rooted, but it is only conjectured
to form interlacing pairs, see \cite[Section 5]{Chow2022}.
\end{example*}



\begin{example*}[Set partitions without singleton blocks]

Let us define $D_n(t) = \sum_{\pi \in SP_2(n)} t^{blocks(\pi)}$,
where $SP_2(n)$ are the set-partitions of $[n]$ where all blocks have size at least $2$.
Then 
\[
 D_n(t) = t\left( D'_{n-1}(t) + (n-1) D_{n-2}(t) \right)
\]
and $D_{n-1}(t)$ and $D_{n}$ have interlacing roots, see \cite[Thm. 1]{BonaHezo2016}.
Note that the degree of $D_n(t)$ is $\lfloor n/2 \rfloor$.


The proof is easy by using Wagner's lemma and induction. 
Prove the base case and assume $D_{n-2} \interl D_{n-1}$ for some $n \geq 2$.
We then know that 
\[
D'_{n-1} \interl D_{n-1} \text{ and } (n-1)D_{n-2} \interl D_{n-1}.
\]
Wagner's lemma now gives that 
\[
D'_{n-1} + (n-1)D_{n-2} \interl D_{n-1} \implies D_{n-1} \interl t \left( D'_{n-1} + (n-1)D_{n-2} \right),
\]
and this is precisely  $D_{n-1} \interl D_{n}$, which completes the induction step.

This result also follows from the general machinery in \cite[Subsection 3.4]{Hitczenko2024x}.
A recursion for set partitions into blocks of size at least $s$ is also mentioned 
there (originally due to \name{Louis Comtet})
but it does not immediately imply real-rootedness.

\end{example*}

\begin{example*}[Type 1 Stirling numbers (cycles, left to right minima)]

We have that
\[
t(t+1)(t+2)\dotsm (t+n-1) = \sum_{k=1}^n |s(n,k)|t^k
= \sum_{\pi \in \symS_n} t^{\mathrm{numberOfCycles}(\pi)} = \sum_{\pi \in \symS_n} t^{\mathrm{leftToRightMinima}(\pi)}
\]
where $|s(n,k)|$ are the unsigned Stirling numbers of the first kind, see \oeis{A132393}.
\end{example*}

\begin{example*}[Stirling permutations]

The descent of a \hyperref[permutationsStirling]{Stirling permutation} $\pi \in Q_n$
is defined as the cardinality of 
\[
 \{ i : \pi_i \gt \pi_{i+1} \} \cup \{2n\}.
\]
The descent-generating polynomials then satisfy the recurrence
\begin{equation}
 P_n(t) = (2n+1)t \cdot P_{n-1}(t) + t(1-t) P'_{n-1}(t),
\end{equation}
see \cite[Eq. 13]{Carlitz1965} and \cite[Thm. 2.1]{GesselStanley1978}.
The coefficients can be found in \oeis{A008517}, the first few values are presented below.
\begin{array}{rrrrrr}
\toprule
 1  \\
 1 & 2  \\
 1 & 8 & 6 \\
 1 & 22 & 58 & 24 \\
 1 & 52 & 328 & 444 & 120 \\
 1 & 114 & 1452 & 4400 & 3708 & 720 \\
\bottomrule
\end{array}
These numbers shows up when studying the Hilbert series of the Whitehouse complex,
see \cite{Readdy2005}. These polynomials are the $h^*$-polynomials of the
order polytope defined by the inequalities
\[
 y_1 \leq y_2 \leq \dotsb \leq y_n \qquad 0 \leq y_j \leq x_j \leq 1 \text{ for } j=1,\dotsc,n.
\]
It is worth pointing out that for $n=20$, the linear term of the Ehrhart polynomial (of degree 40)
has a negative coefficient: $-22354369153/6983776800$.

\name{Miklós Bóna} proved that these have only real zeros, \cite{Bona2009}.
We may also show real-rootedness and the interlacing property using the Ma--Wang theorem.

\todo{Add H.K Dey's result on interlacing, from Archiv der Mathematik}

\name{Jim Haglund} and \name{Mirko Visontai} \cite[Thm. 3.3]{HaglundVisontai2012} show 
that a multivariate generalization 
results in stable polynomials.
\end{example*}


\begin{example*}[Peak polynomials, peaks in permutations]

Peak-generating polynomials, $P_n(x) = \sum_{\sigma \in \symS_n} x^{\text{peaks}(\sigma)}$, see \oeis{A008303}.
Real-rootednesss was proved in \cite{WarrenSeneta1996}, by using the recursion
\begin{equation}
 P_n(x) = (2+x(n-2)) P_{n-1}(x) + 2x(1-x) P'_{n-1}(x).
\end{equation}

Introducing $Q_n = x\cdot P_n$, we have the recursion
\begin{equation}
 Q_n(x) = n x\cdot  Q_{n-1}(x) + 2(1-x)x Q'_{n-1}(x).
\end{equation}

The operator $T_n:f \mapsto nx f + 2(1-x)t \cdot f'$
satisfies $T_n[(x+y)^n] = n t (2+s-t)(s+t)^{n-1}$,
and since the right hand side vanish at $s=i$, $t = 2+i$,
we have that the operator does \emph{not} preserve real-rootedness.

However, we can apply the Ma--Wang theorem \cite{MaWang2008}
and deduce that all roots of $Q_n$ are real, and that  $Q_n \interl Q_{n+1}$
\end{example*}


\begin{example*}[Alternating runs]
In \cite{MaWang2008}, the authors considers the following polynomial (coefficients are \oeis{A059427}):
\[
  P_n(x) = x \cdot \sum_{\sigma \in \symS_n} x^{\text{peaks}(\sigma)+\text{valleys}(\sigma)}.
\]
The polynomials satisfy the recursion
\[
P_n = x(nx-2x+2)P_{n-1} + x(1-x^2) P'_{n-1}.
\]
By using Ma and Wang's general theorem, we obtain the
interlacing relation $P_n \interl P_{n+1}$ here as well.
However, note that there is a root at $x=-1$ with multiplicity $\lfloor n/2 \rfloor -1$.
See \cite{MaWang2008} for more references---real-rootedness was observed earlier.
\end{example*}

\begin{example*}[Descents and left peaks in simsun permutations]
Define the polynomials $P_n(t) = \sum_{\sigma \in \mathrm{RS}_n} t^{\des(\sigma)}$,
where we sum over all \hyperref[namedPermutationSets]{simsun permutations}.
It was shown in \cite{ChowShiu2011} that
\begin{equation}
 P_n(t) = ((n-1)t + 1) P_{n-1}(t) + t(1-2t) P'_{n-1}(t), \qquad P_0(t)=1.
\end{equation}
From this recursion, it follows that the $P_n(t)$ are real-rooted, and we have interlacing, see 
\cite[Thm. 2.1 (iii)]{ChowShiu2011}.
It is also noted that this is the same as counting left peaks in simsun permutations.

Later in \cite{MaYeh2016}, it is proved that interior peaks also give rise 
to real-rooted polynomials.
\end{example*}




\begin{example*}[Descents in run-sorted permutations]
In \cite{AlexanderssonNabawanda2021}, we show that the polynomials $R_n(t)$ 
given by $R_1(t)=R_2(t)=t$ and the recursion
\begin{equation}
R_n(t) = t\cdot R'_{n-1}(t) + t\cdot (n-2) R_{n-2}(t),
\end{equation}
satisfy $R_{n-1} \interl R_{n}$ for all $n \geq 1$.
The triangle of coefficients can be found in \oeis{A124324}.

The polynomial $R_n(t)$ is the generating function of the number of runs in \emph{run-sorted permutations}:
\[
 R_n(t) = \sum_{\pi \in \text{RSP}(n)} t^{\des(\pi)+1}.
\]
These are therefore fairly similar to Eulerian polynomials.
\end{example*}


\begin{example*}[Excedances and derangements, with proof]
Let $P_n(t)\coloneqq \sum_{\pi \in D(n)} t^{\exc(\pi)}$,
where $D(n)$ are the derangements (fixed-point-free permutations) of size $n$.
Then $P_{n-1} \interl P_n$.

The proof is as follows.
From \cite{MantaciRakotondrajao2003}, we can show that $P_1 =1$, $P_2 =t$ and for $n \geq 3$,
\[
 P_n = t \left( (n - 1) P_{n - 2} + (n-1)  P_{n - 1} + (1-t) P'_{n - 1} \right).
\]
Note that $P_n(t)$ is a polynomial of degree $n-1$, as the
permutation $\pi=2345\dotsc n 1$ contributes with $t^{n-1}$ in the definition.

We shall proceed by induction over $n$,
so assume that $P_{n-2} \interl P_{n-1}$, for some $n\geq 3$.

By the induction hypothesis, and using that $P'_{n-1} \interl P_{n-1}$, we can deduce that
\[
  P_{n-2} \interl P_{n-1}, \text{ and }
 (n-1)P_{n-1} + (1-t)P'_{n-1} \interl P_{n-1}.
\]
By using \cite{Wagner1992}, we have that
\begin{align*}
 (n-1)P_{n - 2} + (n-1)P_{n - 1} + (1-t)P'_{n-1} &\interl P_{n-1} \\
 t\left((n-1)P_{n - 2} + (n-1)P_{n - 1} + (1-t)P'_{n-1}\right) &\interl t P_{n-1} \\
 P_{n} &\interl t P_{n-1}.
\end{align*}
From here, we can conclude that $P_{n-1} \interl P_n$.

See also \cite{Liu2007} where this is proved using a different recursion.
\end{example*}



\begin{example*}[Narayana polynomials]

Let $P_n(t)\coloneqq \sum_{\pi \in Dyck(n)} t^{peaks(\pi)}$,
where $Dyck(n)$ Dyck paths of size $n$, and $peaks(\pi)$ count the number of peaks.
Equivalently, this is the same as counting descents in binary words with $n$ 0s and $n$ 1s,
where in each prefix of the word, there are at least as many $0$s as $1$s, see \oeis{A001263}.
A closed-form formula is 
\[
P_n(t) = \frac{1}{n} \sum_{k=1}^n \binom{n}{k} \binom{n}{k-1} t^k 
\]
These satisfy the recursion (see \cite[Eq. (2)]{Sulanke2002}, \cite[Eq. (3.3)]{Liu2007})
\[
(n+1) P_n(t) = (2n-1)(1+t)P_{n-1}(t) - (n-2)(1-t)^2 P_{n-2}(t),
\]
with initial conditions  $P_1(t)=t$, $P_2(t)=t(1+t)$, which is used to prove 
the interlacing property.

Real-rootedness and the interlacing property was first proved in \cite{Branden2006}.
See also \cite[Eq (31)]{Agapito2014}, where a different recursive formula (involving derivatives)
is listed:
\[
 (n+1)(n+2) P_{n+1} = t((1-t)^2 P''_{n} + 2(1+2n)(1-t)P'_{n} + (1+2n)(2+2n)P_n).
\]
Another approach is given in \cite{KostovFinkelshteinShapiro2009}.

\begin{lstlisting}

NN[1] := t;
NN[2] := t (1 + t);
NN[n_] := NN[n] = Expand[
	(
		(2 n - 1) (1 + t) NN[n - 1] -
		(n - 2) (1 - t)^2 NN[n - 2]
)/(n + 1)];
\end{lstlisting}

One can also prove real-rootedness by observing that the Narayana polynomials
are a simple transformation of Jacobi polynomials, or Gegenbauer polynomials
(which are orthogonal polynomials on an interval).
We have that the Narayana polynomials are given by the Hypergeometric series:
\[
P_n(t) = {}_2F_1(-n, -n-1;2;x)
\]
or, \texttt{HypergeometricPFQ[{-n, -n - 1}, {2}, x]}, see \cite[Sec. 12]{BarryHennessy2011}.


Yet another recurrence is
\[
(1 + n)P_n=(3nt + n - t - 1) P_{n-1} - 2 t(t-1) P'_{n-1}
\]
which can be proved using the first theorem in \cite{Dung2022x}.
\end{example*}


\begin{example*}[Type $B$ Narayana polynomials]
The type $B$ Narayana polynomials can be defined as
\[
P_n(t) = \sum_{k=0}^n \binom{n}{k}^2 t^k.
\]
Since these are produced by the Hadamard product of $(1+t)^n$ with itself, it
follows that the $P_n(t)$ are also real-rooted.
\end{example*}


\begin{example*}[Generalized Narayana polynomials]
A two-parameter family of Narayana-type polynomials
\[
P_{n,m}(t) = \sum_{k=0}^n \left(\binom{n}{k}\binom{m}{k} -\binom{n}{k+1}\binom{m}{k-1} \right)t^k.
\]
is considered in \cite[Thm. 1.4]{ChenYangZhang2018},
and they use a fairly general recurrence relation to obtain interlacing roots.
\end{example*}


\begin{example*}[NNE instances in Dyck paths with fixed number of peaks]

In \cite{WangZhang2024}, the authors consider the polynomials
\[
 W_{n,k}(t) \coloneqq [x^k] \sum_{\pi \in Dyck(n)} x^{\mathrm{peaks}(\pi)} t^{\mathrm{nne}(\pi)},
\] 
where $\mathrm{nne}(\pi)$ count the number of North-North-East instances in the path.
The authors show that $\{ W_{n,k}(t) \}_{n \geq k}$ for fixed $k$ form a \hyperref[sturmSequence]{Sturm sequence},
and that $\{ W_{n,k}(t) \}_{1\leq k \leq n}$ is \hyperref[sturmSequence]{Sturm-unimodal}.

Alexandersson (2023): The generalization to Fuss--Catalan paths (in an $n \times nm$ rectangle)
seem to be real-rooted as well.


The following appears in \cite{BonaDimitrovLabelleLiPappeVindasMelendezZhuang2025}.

Let $w_{n,k,m}$ be the number of Dyck paths of size $n$, with $k$
instances of $NE$ and $m$ instances of $NNE$. Then the polynomials defined via
\[
W_{n,k}(t) = \sum_{m\geq 0} w_{n,k,m}t^m = \frac{1}{k} \binom{n}{k-1} \sum_{m \geq 1}^{\min(k,n-k)}  \binom{n-k-1}{m-1}\binom{k}{m} t^m
\]
are real-rooted. The formula above is valid for $n \gt k$, and for $n \leq k$ the polynomials are constant.
The authors conjectured that for fixed $k$, $\{W_{n,k}\}_{n\geq k}$ form a \hyperref[sturmSequence]{Sturm sequence},
and that for fixed $n$, $\{W_{n,k}\}_{1 \leq k\leq n}$ is \hyperref[sturmSequence]{Sturm-unimodal}.
These conjectures were later proved in \cite{WangZhang2024}.
The authors there finds a two-term recurrence, expression $W_{n,k}(t)$ in terms of $W_{n-1,k}(t)$ and $W_{n,k}(t)$.
Alternatively, $W_{n,k}(t)$ can be expressed in terms of $W_{n,k-1}(t)$ and its derivative.
\end{example*}


\begin{example*}[Motzkin polynomials]

The non-crossing matching polynomial of the complete 
graph $K_n$ is the Motzkin polynomial, see \oeis{A055151}.
These polynomials satisfy the recursion 
\[
 (n+2)M_n(x) = (2n+1)M_{n-1}(x) - (n-1)(1-4x)M_{n-2}(x), \text{ with } M_1(x)=1, \;
 M_2(x)=x+1.
\]
By using the methods in \cite[Section 3]{Wagner1992}, 
it is now easy to use induction to show that the roots of $M_{n-1}(x)$ interlaces 
those of $M_{n}(x)$. In particular, these polynomials are real-rooted.

% M[1] = 1;
% M[2] = 1 + x;
% M[n_] := 
%   Together[
%     1/(n + 2) ((2 n + 1) M[n - 1] - (n - 1) (1 - 4 x) M[n - 2])] // 
%    Expand;

\end{example*}




\begin{example*}[Descents in standard Young tableaux]

Brenti \cite[Thm. 5.2.3]{Brenti1989} showed that the $(P,w)$-Eulerian polynomial associated with
a column-strict Ferrers diagram is real-rooted. This is a special case of the (now disproved in general)
Neggers--Stanley conjecture. The descent-generating polynomial ($i$ is a descent if $i+1$ appears in a lower row):
\[
   W_\lambda(t) \coloneqq \sum_{T \in \SYT(\lambda)} t^{\des(T)}
\]
is real-rooted. Note that \cite[Thm. 4.6.1]{Brenti1989} proves a more general statement.
It can be interpreted as being a Polya frequency sequence of the Ehrhart polynomial implies real-rootedness of the $h^*$-polynomial (of a polytope).
To be precise, \cite[Eq. (7.96)]{StanleyEC2} states that for a skew shape $\lambda/\mu$ with $n$ boxes,
\[
  \sum_{m \geq 0} \schurS_{\lambda/\mu}(1^m)z^m = \frac{\sum_{T \in \SYT(\lambda/\mu)} t^{\des(T)} }{(1-z)^{n+1}}.
\]
Now, the map $m \mapsto \schurS_{\lambda/\mu}(1^m)$ is an order polynomial which factors in linear factors.
This is due to the hook-content formula, so we get that $W_\lambda(t)$ is real-rooted by Brenti's result.

Later, \cite{Branden2004operators} shows that $W_\lambda(t)$ interlaces $W_{\lambda^+}(t)$ whenever
$\lambda^+$ is obtained from $\lambda$ by adding a box.

In particular, for $\lambda=(n,n)$ we have that $W_{\lambda}(t)$ is a Narayana polynomial,
so we have yet another proof of real-rootedness of these.
\end{example*}


\begin{example*}[Ehrhart polynomials from Schur functions]

The map
$
 n \mapsto |\SSYT(n \lambda, k)|
$
where we count the number of semi-standard Young tableaux of shape $n \lambda$ with maximal entry $k$,
is a polynomial in $n$. Explicitly, the Weyl character formula states that it is
\[
  P_\lambda(n) \coloneqq  \prod_{1 \leq i \lt j \leq k} \frac{n(\lambda_i - \lambda_j)+j-i}{j-i}.
\]
which is real-rooted. (Real-rootedness is attributed to Brenti \cite{Brenti1989}, but he uses 
a more convoluted proof). Brändén \cite{Branden2004operators} showed that $P_\lambda(n)$ 
interlaces $P_\mu(n)$ whenever $\mu$ covers $\lambda$ in Young's lattice.
The above map is the Ehrhart polynomial of the \hyperref[gtpatterns]{Gelfand--Tsetlin polytope} associated with $\lambda$ and $k$.


We note that the map 
\[
 n \mapsto |\SSYT(n \lambda/ n\mu, k)|
\]
for skew shapes is not real-rooted in general: $\lambda =(3,2)$, $\mu = (1)$ gives for $k=3$
the polynomial $\frac{1}{4} (n+1)^2 \left(7 n^2+10 n+4\right)$ which has non-real roots.

It does look like the corresponding $h^*$-polynomials are real-rooted even for skew shapes.
For non-skew shapes, this is proved by Brenti \cite{Brenti1989}.
\end{example*}



\begin{example*}[Type $B$ Set partitions]

In \cite{Wang2014}, a type $B$ version (and in general, a colored version)
of counting blocks in set partitions is considered.
The polynomials considered satisfy the (general) recursion
\[
  T_n(x) = (x+c) \cdot  T_{n-1}(x) + mx\cdot T'_{n-1}(x),
\]
where $m$ and $c$ are positive integers.

\end{example*}


\begin{example*}[Ordered partitions]
In \cite{Simion1984}, one takes any multiset $M$, and considers the number of compositions into exactly $k$
parts, $O(M,k)$. The polynomial $\sum_k O(M,k)x^k$ is shown to be real-rooted.

In particular, this implies the real-rootedness of the Touchard polynomials.
\end{example*}




\section[realRootedMatching]{Matching and independence polynomials}

Let $G = (V,E)$ be any graph.
The \defin{matching polynomial} $\mu_G(x)$ is defined as
\[
\mu_G(x) \coloneqq \sum_{\substack{M \subseteq E(G) \\ \text{$M$ matching}}} x^{|M|}
\]
where $|M|$ is the number of edges in the matching.

In \cite[Thm. 4.2]{HeilmannLieb1972}, it was shown that $\mu_G(x)$
is real-rooted. Moreover, $\mu_G(x)$ interlaces $\mu_{G'}(x)$, where $G'$
is any graph obtained from $G$ by removing a vertex.
This is a very powerful theorem, but it follows
from the more general Chudnovsky--Seymour theorem on independence polynomials below.

A hypergraph generalization (including multigraphs) 
is considered by \name{Nima Amini} in \cite{Amini2019x}.


\begin{example*}[Fibonacci tilings via matchings]
Recall that the number of ways to tile the  one-row Young diagram $(n)$
with tiles of shape $(1)$ and $(2)$ is given by the $(n+1)th$ Fibonacci number.
If we then define 
\[
 F_0(x) \coloneqq 1 \quad F_1(x) \coloneqq 1, \quad F_n(x) \coloneqq F_{n-1}(x) + x\cdot F_{n-2}(x)
\]
we see that $[x^k] F_n(x)$ is the number of ways to tile $(n)$ using exactly $k$ (2)-tiles.

By the Heilmann--Lieb theorem, we can see that for all $n$, $F_n(x)$ is real-rooted,
since we can see that $\mu_G(x) = F_n(x)$ for the path graph with $n$ vertices.
\end{example*}




\begin{example*}[Touchard polynomials (blocks in set partitions), with proof]

The \defin{Touchard polynomials} may be defined via
\[
T_n(x) \coloneqq \sum_{P \in SP(n)} x^{blocks(P)} = \sum_{k=1}^n S(n,k) x^k,
\]
where $SP(n)$ are the set partitions of $n$, and we count the number of blocks.
Note that the coefficients are Stirling numbers of the second type.

One can show (see \cite[p. 20, p. 139]{Wilf1994}) that 
\[
  T_n(x) = x \cdot  T_{n-1}(x) + x\cdot T'_{n-1}(x).
\]
We know $T_{n-1} \interl x T_{n-1}$. Moreover, $T_{n-1} \interl x T'_{n-1}$,
so by \cite{Wagner1992}, we have $T_{n-1} \interl x T_{n-1} +  x T'_{n-1}$.
But this is exactly $T_{n-1} \interl T_n$.


\emph{The matching approach:} \cite[Lem. 5.1]{Godsil2017}, states 
that the Stirling number $S(n,k)$ is equal to the number 
of size $n-k$ matchings in the bipartite graph $G_n$ on vertex set 
\[
 \{1,2,\dotsc,n\} \cup \{1',2',\dotsc,n'\}
\]
where $i$ has an edge with $j'$ if and only if $i \lt j$.

For example, the graph $G_4 = \{ \{1,2'\}, \{1,3'\}, \{1,4'\}, \{2,3'\}, \{2,4'\}, \{3,4'\} \}$
has $7=S(4,4-2)$ matchings of size $2$,
and $6=S(4,4-1)$ matchings of size $1$.
We then have that $x^n T_n(x^{-1}) = \mu_{G_n}(x)$ where
$\mu_{G_n}(x)$ is the matching polynomial of $G_n$.
Since removal of vertices gives interlacing, we have that 
$\mu_{G_{n-1}}(x) \interl \mu_{G_n}(x)$
from which interlacing for Touchard polynomials follows.


\emph{The rook placement approach:} 
The matching polynomial $\mu_{G_n}(x)$ for the bipartite graph above, is the rook polynomial 
of the staircase board of size $n-1$.

\end{example*}


\begin{example*}[r-Lah number polynomials and $r$-Touchard polynomials]

An interpretation of $r$-Lah polynomials and $r$-Stirling numbers as 
matching polynomials is given in \cite{NyulRacz2021}.
\end{example*}



\todo{
Independence polynomials sometimes have only real roots.
https://www.emis.de/journals/JACO/Volume19_3/pn5l8gh2q7025747.fulltext.pdf
https://ajc.maths.uq.edu.au/pdf/38/ajc_v38_p027.pdf (Caterpillars)
https://ajc.maths.uq.edu.au/pdf/71/ajc_v71_p104.pdf
}

\todo{Eigenvalues of Hermitian matrices and its principal minor}


\subsection[rookPolynomials]{Rook polynomials}

The \defin{rook polynomial} of a board $B$ is defined as
\[
  R_B(z) \coloneqq \sum_k r_k(B) z^k
\]
where $r_k(B)$ is the number of ways to place $k$ non-attacking rooks on the board.
Here, the board can be any subset of some $n {\times} n$-board.
Note that $R_B(x)$ is exactly a matching polynomial, $\mu_G(x)$ for a bipartite
graph $G$ on vertex set $\{v_1,\dotsc,v_n,v'_1,\dotsc,v'_n\}$
where  $(v_i,v'_j)$ is an edge if and only if $(i,j)$ is a cell of the board $B$.

Nijenhuis \cite{Nijenhuis1976} proved that the rook polynomials are always real-rooted.
In fact, Nijenhuis shows a weighted result involving permanents.
\begin{theorem}[Nijenhuis, 1976]
Let $A = (a_{ij})$ be a non-negative $m\times n$-matrix.
Then
\[
 R_{A}(z) \coloneqq \sum_{\rho} (-z)^{\length(\rho)} \prod_{j=1}^{\length(\rho)}
  a(\rho_j)
\]
where we sum over all ways to place rooks on the $m\times n$-board,
and $a(\rho_j)$ is the value of the matrix at position given by the $j$th rook.
\end{theorem}


The \defin{hit polynomial} of a board can also be defined in terms of the $r_k(B)$.
Let $B$ be a subset of the $n\times n$-board.
The hit polynomial is then defined as
\[
 T_B(z) \coloneqq \sum_{k =0}^n k! r_{n-k}(B)(z-1)^{n-k} = \sum_{j=0}^n h_j(B) z^j.
\]
Here, $h_j(B)$ is the number of ways to place $n$ rooks on the $n \times n$-board,
with exactly $j$ rooks in $B$, see \cite{KaplanskyRiordan1946,Haglund2000}.

A theorem by Haglund--Ono--Wagner \cite{HaglundOnoWagner1999}
states that if $B$ is a Ferrers board (same as a Young diagram),
then $T_B(z)$ is real-rooted.




\subsection[realRootedIndependence]{Independence polynomials for claw-free graphs}

Let $G = (V,E)$ be any graph. The \defin{independence polynomial} $I(G,x)$ is defined 
as
\[
I(G,x) \coloneqq \sum_{\substack{A \subseteq V(G) \\ \text{$A$ independent}}} x^{|A|}
\]
where $|A|$ is the number of edges in the matching.

A graph called \defin{claw-free} if no induced subgraph is isomorphic 
to the graph with edges $\{\{1,2\},\{1,3\},\{1,4\}\}$.

If $G$ is any simple graph, its \emph{line graph} $L(G)$ is also simple.
Moreover, $L(G)$  is claw-free. Finally, we have that $I(L(G),x) = \mu_G(x)$.

\begin{theorem}[Chudnovsky--Seymour 2007, \cite{ChudnovskySeymour2007},]
Suppose $G$ is claw-free. Then $I(G,x)$ is real-rooted.
As a corollary, we get the Heilmann--Lieb theorem on matching polynomials.
\end{theorem}
\begin{proof*}
This very briefly illustrates the ideas. Note that if $v \in V(G)$ and $nbh(v)$ 
is $v$ and all vertices adjacent to $v$, then
\[
 I(G,x) = I(G \setminus \{v\} ,x) + x \cdot  I(G \setminus nbh(v),x).
\]
Essentially, the above sum is broken down into the cases where $v \notin A$, and when $v \in A$.
Observe that if $G$ is claw-free, then any induced subgraph is also claw-free.
Chudnovsky--Seymour then use the notion of \emph{compatible polynomials}
to obtain their result.
\end{proof*}

This result conjectured by \name{Richard Stanley} in \cite{Stanley1998},
a vertex-weighted version is given by \name{Alexander Engström} \cite{Engstrom2007}
and an alternative (non-vertex-deletion-recursive) 
proof is given by \name{Bodo Lass} \cite{Lass2012}.
Finally, a multivariate refinement is given by \cite{LeakeRyder2019} where 
one obtains \hyperref[stablePolynomial]{stable polynomials}.

\name{Alexander Engström} proved the following weighted generalization of the Chudnovsky--Seymour theorem:
\begin{theorem*}[A. Engström, \cite[Thm. 2.5]{Engstrom2007}]
Let $G = (V,E)$ be a graph, and $w : V \to \setR$ be vertex weights.
The \defin{weighted independence polynomial} $I(G,w,x)$ is defined as
\[
I(G,w,x) \coloneqq \sum_{\substack{A \subseteq V(G) \\ \text{$A$ independent}}} x^{|A|} \prod_{v\in A} w(v).
\]
If $G$ is claw-free and all weights are non-negative, then $I(G,w,x)$ is real-rooted.
\end{theorem*}


\subsection[realRootsChow]{Chow polynomials}

In \cite{BrandenVecchi2025x}, the authors show various 
real-rootedness results regarding Chow polynomials.


\section[interlacingSequences]{Interlacing sequences}


The following definitions are from \cite{Branden2015}.

\begin{definition}[Interlacing sequences of polynomials]
A sequence $\{f_1,f_2,\dotsc,f_n\}$ of polynomials with positive leading coefficients
is said to be an \defin{interlacing sequence}
if $f_i \interl f_j$  for all $1\leq i \lt j \leq n$.

We let \defin{$\mathscr{F}_n^+$} denote the family of interlacing
sequences of length $n$ with \emph{non-negative} coefficients.
\end{definition}

In \cite{Branden2015}, there are nice examples of how interlacing
sequences can be used to prove real-rootedness.
In particular, rook polynomials associated with
Ferrers diagrams can be proved to be real-rooted using this method.

\subsection[realMatrixRecursion]{Matrices preserving interlacing sequences}


The following very useful theorem is proved in \cite[Thm. 7.8.5]{Branden2015}.

\begin{theorem}[Matrices preserving interlacing sequences]
Let $G$ be an $m \times n$ matrix of polynomials in $t$.
Then $G : \mathscr{F}_n^+ \to \mathscr{F}_m^+$ if and only if
\begin{enumerate}
\item All entries of $G$ have nonnegative coefficients, and
\item For every $2{\times}2$ sub-matrix
$\begin{bmatrix}
 a(t) & b(t) \\
 c(t) & d(t)
\end{bmatrix}$ of $G$ and real numbers
$\lambda, \mu > 0$, we have
\[
 (\lambda t + \mu) b(t) + d(t) \interl (\lambda x + \mu) a(t) + c(t).
\]
\end{enumerate}
\end{theorem}



\begin{example*}[Minors preserving interleaving sequences]

There are 81 matrices with entries in $\{0,1,t\}$.
Remove those where first and last row are equal and those where one of the rows is $(0,0)$.
There are 56 such matrices left, and among these, the following 21
satisfies the conditions in the theorem above.
\begin{equation*}
\begin{pmatrix}0 & 1 \\ 0 & t\end{pmatrix}, \quad
\begin{pmatrix}0 & 1 \\ t & 0\end{pmatrix}, \quad
\begin{pmatrix}0 & t \\ 0 & 1\end{pmatrix}, \quad
\begin{pmatrix}1 & 0 \\ 0 & 1\end{pmatrix}, \quad
\begin{pmatrix}1 & 0 \\ t & 0\end{pmatrix}, \quad
\begin{pmatrix}t & 0 \\ 0 & t\end{pmatrix}, \quad
\begin{pmatrix}t & 0 \\ 1 & 0\end{pmatrix}
\end{equation*}

\begin{equation*}
\begin{pmatrix}0 & 1 \\ t & 1\end{pmatrix}, \quad
\begin{pmatrix}0 & 1 \\ t & t\end{pmatrix}, \quad
\begin{pmatrix}1 & 0 \\ 1 & 1\end{pmatrix}, \quad
\begin{pmatrix}1 & 0 \\ t & 1\end{pmatrix}, \quad
\begin{pmatrix}1 & 1 \\ 0 & 1\end{pmatrix}, \quad
\begin{pmatrix}1 & 1 \\ t & 0\end{pmatrix}, \quad
\begin{pmatrix}t & 0 \\ t & t\end{pmatrix}
\end{equation*}

\begin{equation*}
\begin{pmatrix}t & 1 \\ 0 & t\end{pmatrix}, \quad
\begin{pmatrix}t & 1 \\ t & 0\end{pmatrix}, \quad
\begin{pmatrix}t & t \\ 0 & t\end{pmatrix}, \quad
\begin{pmatrix}1 & 1 \\ t & 1\end{pmatrix}, \quad
\begin{pmatrix}1 & 1 \\ t & t\end{pmatrix}, \quad
\begin{pmatrix}t & 1 \\ t & t\end{pmatrix}, \quad
\begin{pmatrix}t & t \\ 1 & 1\end{pmatrix}
\end{equation*}
\end{example*}


Below are some examples where the proofs uses interlacing sequences.

\begin{example*}[Binomial Eulerian polynomials]

The \defin{binomial Eulerian polynomials} can be defined as 
\[
 \tilde{A}_n(t) \coloneqq \sum_{\sigma \in \symS_n} (1+t)^{\fix(\sigma)} t^{\exc(\sigma)}.
\]
Athanasiadis conjectured that these are real-rooted, 
and it was later shown by \name{Jim Haglund} and \name[P.B. Zhang]{Philip B. Zhang} \cite{HaglundZhang2019} that this is indeed the case,
by means of interlacing polynomials.

It seems that $\tilde{A}_n(t) \interl \tilde{A}_{n+1}(t)$, but this does not follow immediately from 
the reference above. See also the binomial $\svec$-Eulerian polynomials.
\end{example*}


\begin{example*}[Ascents and descents in inversion sequences]
Fix $\svec = (s_1,\dotsc,s_n)$ and define 
\[
\mathcal{I}_n^\svec \coloneqq \{ (e_1,\dotsc,e_n) \in \setZ^n : 0 \leq e_i \lt s_i, \; 0 \leq i \leq n \}
\]
with the conventions that $e_0 = e_{n+1} =0$ and $s_0 = s_{n+1}=1$.
This is the set of \defin{inversion sequences} associated with $\svec$.

An index $0 \leq i \leq n$ is an \defin{ascent}, \defin{collision}, or \defin{descent}
of an inversion sequence $\evec$ if
\[
 \frac{e_i}{s_i} \lt \frac{e_{i+1}}{s_{i+1}}, \qquad 
 \frac{e_i}{s_i} = \frac{e_{i+1}}{s_{i+1}}, \qquad 
 \frac{e_i}{s_i} \gt \frac{e_{i+1}}{s_{i+1}},
\]
respectively. Moreover, let $\mathcal{D}_n^\svec \subset \mathcal{I}_n^\svec$ 
be the set of inversion sequences without collisions.

It was proved in \cite{SavageVisontai2015} and \cite{GustafssonSolus2020}, respectively, 
that
\[
E^\svec_n(t) = \sum_{\evec \in \mathcal{I}_n^\svec} t^{\asc(\evec)} \quad \text{and} \quad
D^\svec_n(t) = \sum_{\evec \in \mathcal{D}_n^\svec} t^{\asc(\evec)}
\]
are real-rooted. The methods use theory of interlacing sequences of polynomials.
In particular, if $\svec$ is a fixed \emph{sequence} of positive integers,
then $E^\svec_n(t) \interl E^\svec_{n+1}(t)$.

The proof by Savage and Visontai also shows that the $s$-Eulerian polynomials
are real-rooted. See also the proof in \cite[Sec. 8.2]{Branden2015}.


See \cite{HaglundZhang2019} for the binomial Eulerian polynomial analog of the above results.
\end{example*}



\subsection[compatiblePolynomials]{Compatible polynomials}

Chudnovsky and Seymour introduced a concept closely related to interleaving sequences
of polynomials. The polynomials $f_1,\dotsc,f_k \in \setR[z]$ have a \defin{common interlacing}
if there is some $h \in \setR[z]$ with $h \interl f_j$ for all $j$.

\begin{theorem}[Chudnovsky--Seymour, 2007, \cite{ChudnovskySeymour2007}]

Let $\{f_1,f_2,\dotsc,f_n\}$ be a set of real-rooted polynomials with \emph{positive} leading coefficients.
Such a set is said to be \defin{compatible} if for all
choices $\lambda_j$, the sum $\sum_{j=1}^n \lambda_j f_j$ is real-rooted.

Then the following are equivalent:
\begin{itemize}
\item $f_1,\dotsc,f_k$ are pairwise compatible;
\item $f_1,\dotsc,f_k$ are compatible;
\item $f_i,f_j$ have a common interlacing for all $1\leq i \lt j \leq n$;
\item $f_1,\dotsc,f_k$ have a common interlacing.
\end{itemize}
\end{theorem}
This theorem can be generalized to polynomials with arbitary real
coefficients (with negative leading coefficients), see \cite{LeakeRyder2024x}.


\todo{For polynomials arranged in a tree, with
interlacing relations along edges: https://cs.uwaterloo.ca/~lapchi/cs860-2022/notes/12-interlacing-family.pdf}




\section[stablePolynomial]{Stable polynomials}



Let $\mathcal{H} \subset \setC$ denote the 
upper half-plane $\{z \in \setC: \mathrm{im}(z) >0 \}$.

\begin{definition}[Stability]

A multivariate polynomial $P \in \setC[z_1,\dotsc,z_n]$ 
is called \defin{stable} if it does not vanish on $\mathcal{H}^n$.
That is, $P$ is stable if
\[
 \zvec^* \in \mathcal{H}^n \implies P(\zvec^*) \neq 0.
\]
One can easily show that $P \in \setR[z_1,\dotsc,z_n] $ is stable if and only if 
$P(\alpha+\lambda t)=0$ has only real zeros
for all $\alpha \in \setR^n$, $\lambda \in \setR_+^n$.
A univariate polynomial with real coefficients 
is stable if and only if all its roots are real.
\end{definition}


\begin{definition}[Strongly Rayleigh]
A measure $\mu$ on the set of subsets of $[n]$, is called 
\defin{strongly Rayleigh} if 
\[
 \sum_{S \subseteq [n]} \mu(S) \prod_{s\in S} x_s
\]
is real stable.
\end{definition}

\begin{theorem}[Stability preserving operators]
The following operations on polynomials preserve stability.
Let $p,q \in \setR[z_1,\dotsc,z_n]$ be stable polynomials.
Then the following are also stable.
\begin{itemize}
\item $p\cdot q$ (Product)
\item $\partial_{z_j} p$ (Differentiation)
\item $z_j \partial_{z_j} p$ (Degree-preserving differentiation)
\item $p(z_1,\dotsc,z_{j-1},r,z_{j+1},\dotsc,z_n)$ (Real specialization)
\item $p(z_1,\dotsc,z_{j-1},z_1,z_{j+1},\dotsc,z_n)$ (Projection)
\item $z_j^d p(z_1,\dotsc,-1/z_{j},z_1,z_{j+1},\dotsc,z_n)$ (Inversion)
\end{itemize}
Here, $d$ is the degree of $z_j$ in $p$.
\end{theorem}


A polynomial $P(z_1,\dotsc,z_n)$ is \defin{multi-affine} if the degree in each variable is at most $1$.


\begin{theorem}[See \cite{ChoeOxleySokalWagner2004}]
If $P$ is a homogeneous and multi-affine stable polynomial, 
then its support is the set of bases of a matroid.
See also  \cite[Cor. 3.4]{Branden2007}.
\end{theorem}
The converse is in general not true but some classes of matroids 
always have a basis generating polynomial which is stable.
For example, regular matroids (which include graphic matroids), uniform matroids and 
a subclass of transversal matroids (see \cite{ChoeOxleySokalWagner2004}).

Deletion and contraction on matroids preserve stability of the 
basis generating polynomial.


\begin{theorem}[See \cite{Branden2007}]
A homogeneous multiaffine (real) polynomial $f(x_1,\dotsc,x_n)$ is real stable if and only if
\[
 \frac{\partial f}{\partial x_i} \cdot \frac{\partial f}{\partial x_j} -
 \frac{\partial^2 f}{\partial x_i \partial x_j} \cdot f \geq 0
\]
whenever $x_1,x_2,\dotsc,x_n \in \setR$.
\end{theorem}
Tho satisfy this inequality is called to have the \defin{strongly Rayleigh property}.
The \defin{weak Rayleigh property} holds if the above inequality is true whenever
the variable are positive real numbers.


\begin{theorem}[See \cite{BorceaBranden2010}]
A polynomial in two variables, $P(z,w) \in \setR[s,t]$ is stable if and only if it can be expressed as
$\pm \det(zA+wB+C)$ where $A,B,C$ are real $n \times n$-matrices, where $A$ and $B$ are positive semidefinite
and $C$ is symmetric.
\end{theorem}


\begin{example*}[The monotone column permanent conjecture]

Let $A$ be a real $n\times n$-matrix in which entries along columns decrease.
Let $Z_n = \mathrm{diag}(z_1,\dotsc,z_n)$ be a diagonal matrix with indeterminates,
and let $J_n$ be the matrix with all entries equal to $1$.
Then
\[
 \mathrm{per}(J_nZ_n + A) = \mathrm{per}\left( (z_j + a_{ij})_{1 \leq i,j \leq n} \right)
\]
is stable, see \cite{BrandenHaglundVisontaiWagner2011}.
By noting that the Eulerian polynomials are generated by excedances,
they give a new proof that (a multivariate version of) Eulerian polynomials 
are stable.
\end{example*}


\begin{example*}[Multivariate Eulerian polynomials]
P. Bränden (see \cite[Thm. 2.5]{HaglundVisontai2012}) has proved that the 
following generalization of the Eulerian polynomial is a multivariate stable polynomial:
\[
 A_n(\xvec,\yvec) = \sum_{\pi \in \symS_n} \prod_{\pi_i \gt \pi_{i+1}} x_{\pi_i} \prod_{\pi_i \lt \pi_{i+1}} y_{\pi_i}.
\]
In fact, Haglund and Visontai show that keeping track of plateaus as well 
give rise to stable polynomials:
\[
 A_n(\xvec,\yvec,\zvec) = \sum_{\pi \in \symS_n} 
 \prod_{\pi_i \gt \pi_{i+1}} x_{\pi_i}
 \prod_{\pi_i \lt \pi_{i+1}} y_{\pi_i}
 \prod_{\pi_i = \pi_{i+1}} z_{\pi_i}.
\]
\end{example*}




\begin{example*}[Multivariate matching polynomials]
The multivariate matching polynomial is real stable, see \cite{BorceaBranden2009I,BorceaBranden2009II}.
See also In \cite{LeakeRyder2019} (there are two types of multivariate matching polynomials,
either by recording vertex labels or edge labels in the matchings, respectively;
\begin{align}
\mu_V(G;x_1,\dotsc,x_n) & \coloneqq \sum_{\substack{M \subseteq E(G) \\ \text{$M$ matching}}} \prod_{uv \in M} -x_u x_v, \\
\mu_E(G;x_1,\dotsc,x_n) & \coloneqq \sum_{\substack{M \subseteq E(G) \\ \text{$M$ matching}}} \prod_{e \in M} x_e.
\end{align}
\end{example*}

The \defin{multi-affine vertex matching polynomial}
$\mu_V(G;x_1,\dotsc,x_n)$ is real stable, see \cite{BorceaBranden2009I}.


\todo{
https://arxiv.org/pdf/0911.3569.pdf
}




\subsection[samePhaseStable]{Same phase stability}

The following notion was introduced in \cite{LeakeRyder2019}
and is a strictly weaker notion than stability.

\begin{definition}[Same-phase stability]
 A polynomial $P(z_1,\dotsc,z_n) \in \setR[z_1,\dotsc,z_n]$ is said 
 to be \defin{same-phase stable}  if for
 every $\lambda \in \setR_+^n$, we have that the univariate polynomial
 $P(\lambda_1 t,\lambda_2 t,\dotsc,\lambda_n t) \in \setR[t]$ is real-rooted.
\end{definition}

A set of polynomials $\{p_j(\xvec) \}_{j=1}^m$ are \defin{same-phase compatible}
if for every vector of positive numbers $\lambda$, we have that 
 $\{p_j(\lambda t) \}_{j=1}^m$ are compatible (as univariate polynomials).


Let $P, P_0, \dotsc, P_k$ be polynomials such that $P = P_0 + \sum_{j=1}^m z_{i_j} P_{j}$.
This is a \defin{proper splitting of $P$} if all $P_{j}$ do not depend on 
the variables $\{z_{i_1}, \dotsc, z_{i_m} \}$.


\begin{theorem}[See \cite{LeakeRyder2019}]
Suppose $P$ is a multiaffine polynomial with non-negative coefficients.
Then the following are equivalent:
\begin{itemize}
\item $P$ is same-phase stable;
\item For any proper splitting, $P = P_0 + \sum_{j=1}^m z_{i_j} P_{j}$,
we have that $P_0, z_{i_1} P_{1}, \dotsc, z_{i_m} P_{m}$
are same-phase compatible.

\item There is some proper splitting, $P = P_0 + \sum_{j=1}^m z_{i_j} P_{j}$,
such that $P_0, z_{i_1} P_{1}, \dotsc, z_{i_m} P_{m}$ are same-phase compatible.

\end{itemize}

\end{theorem}


The \defin{multiaffine edge matching polynomial}
$\mu_E(G;x_1,\dotsc,x_n)$ is same-phase stable \cite[Cor. 3.2]{LeakeRyder2019}.

Let $G$ be a graph on the vertex set $[n]$, and define the \defin{multivariate independence polynomial}
\[
I(G;x_1,\dotsc,x_n) \coloneqq \sum_{\substack{A \subseteq V(G) \\ A \text{ independent}}} \prod_{v \in A} x_v.
\]
In \cite{LeakeRyder2019}, it is proved that $I(G;x_1,\dotsc,x_n)$
is same-phase stable \emph{if and only if} $G$ is claw-free.
This is a reinterpretation of Engström's weighted result, \cite{Engstrom2007},
but Leake--Ryder gives a much sleeker proof, and the if-and-only-if condition explains 
that the claw-free requirement is sharp.


\begin{example*}[Multivariate Eulerian polynomials II]
In \cite{AlexanderssonNabawanda2021}, we show that the following generalization 
of the Eulerian polynomial is same-phase stable, but not stable:
\[
 \tilde{A}_n(\xvec) = \sum_{\pi \in \symS_n} \prod_{\pi_i \gt \pi_{i+1}} x_{i}.
\]
Note that we now keep track of the descent index, rather than the descent bottom value.
Moreover, we showed that 
\[
 \tilde{A}_{n-1}(\lambda_1 t, \dotsc, \lambda_{n-1} t) \interl \tilde{A}_n(\lambda_1 t, \dotsc, \lambda_n t)
\]
for all positive real numbers $\lambda_1,\dotsc,\lambda_n$.
\end{example*}



\section[lorentzianPolynomial]{Lorentzian polynomials}

Petter Brändén and June Huh introduced the notion of \defin{Lorentzian polynomials}.
All stable polynomials are Lorentzian, but the converse is not true for polynomials 
of degree three or more.
For an introduction to the topic, see \href{https://www2.math.upenn.edu/~jhaglund/IPAC/2022.html}{these lectures by Brändén}.


\begin{definition}[Lorentzian polynomial]

A homogeneous polynomial $h(x_1,\dotsc,x_n)$ of degree $d$ is \defin{strictly Lorentzian} if 
\begin{itemize}
 \item All coefficients of $h$ are positive, and
 \item The signature of (the degree $2$ polynomial)
 \[
   \frac{\partial}{\partial x_{i_1}}\frac{\partial}{\partial x_{i_2}} \dotsm \frac{\partial}{\partial x_{i_e}} h(x_1,\dotsc,x_n)
 \]
 is $(+,-,-,\dotsc,-)$ for any $i_1,\dotsc,i_e \in [n]$ where $e = d-2$. 
 
 To compute the signature of a homogeneous polynomial $P(x_1,\dotsc,x_n)$ of degree 2,
 consider the symmetric matrix 
 \[
  \left(  \frac{\partial}{\partial x_{i}}\frac{\partial}{\partial x_{j}} P \right)_{ij} \in \setR^{n \times n}.
 \]
 If it has only one positive eigenvalue, while the remaining are negative, then it has signature $(+,-,-,\dotsc,-)$.

\end{itemize}

A polynomial is called \defin{Lorentzian} if it is the limit of strictly Lorentzian polynomials.
There are two other equivalent conditions listed in \cite{HuhMatherneMezarosDizier2022}.

An equivalent definition for Lorentzian is:

\begin{itemize}
 \item All coefficients are non-negative.
 \item The support of the polynomial forms an \hyperref[MConvexity]{M-convex set}.
 \item All quadratic forms (as above) has at most one positive eigenvalue.
\end{itemize}

\end{definition}

The basis generating function of a matroid is Lorentzian.

Determining if polynomials are Lorentzian can be done efficiently, see \cite{Chin2024x}.
This is in contrast with checking for real stability which is coNP-complete, see \cite{Chin2024x}.


The following operations preserve the Lorentzian property:
\begin{itemize}
\item Taking the multi-affine part of a Lorentzian polynomial.
\item Products of Lorentzian polynomials are Lorentzian.
\item Specializing variables in a multiaffine Lorentzian polynomial.
\end{itemize}



In \cite{HuhMatherneMezarosDizier2022}, several generalizations of \hyperref[schurS]{Schur polynomials} 
are either proved or conjectured to be Lorentzian.
\begin{theorem}[See \cite{HuhMatherneMezarosDizier2022}]
For any integer partition $\lambda$, the normalized \hyperref[schurS]{Schur polynomial}
\[
   \sum_{ \mu } K_{\lambda \mu} \frac{\monomial_\mu}{\mu_1! \mu_2 ! \dotsm \mu_\ell !}
\]
is Lorentzian. Here, $K_{\lambda \mu}$ are the Kostka coefficients.
\end{theorem}
Another proof of this is given in \cite{RossToma2023}.


It is conjectured that the corresponding statement holds also for skew Schur polynomials, \hyperref[key]{key polynomials}
and \hyperref[schubert]{Schubert polynomials}.
In \cite{ChenFanYe2024x}, 0-1-Grothendieck polynomials are shown to be Lorentzian.


In \cite{ChinQin2025x}, the authors study the set of symmetric polynomials which are also Lorentzian.
This set is homeomorphic to a closed Euclidean ball.



\begin{example}[Chromatic symmetric polynomials]

For \emph{abelian Dyck paths}, the \hyperref[chromaticQuasisymmetricHistory]{chromatic symmetric polynomial}
(in any number of variables) is Lorentzian, see \cite{MatherneMoralesSelover2022x}.

In contrast, there is a (non-abelian) unit interval graph for which 
the chromatic symmetric polynomial is not Lorentzian, or normalized Lorentzian. 
The area sequence for when this fails is $01121121$ --- this example was found by 
Ricky Liu and Cynthia Vinzant.
\end{example}


In \cite{RossSussWannerer2023x}, the authors describe a way to construct operators preserving 
the Lorentzian property.

See \cite{BrandenLeake2023x,Ross2023x} for Lorentzian polynomials on fans.
See \cite{HuXiao2023x} for Lorentzian polynomials and intersection theory.


For Lorentzian polynomials from independence sets in graphs,
see \cite{BendjeddouHardiman2025}.



\section[symmetricDecomposition]{Symmetric decomposition}

See \cite{BrandenSolus2019} for the symmetric decomposition
and relationship with real-rootedness.





\section[gammaPositivity]{Gamma positivity}

See \name{Christos Athanasiadis} survey \cite{Athanasiadis2018}.

A polynomial $p(x)$ of degree $n$ is \defin{symmetric} or \defin{palindromic} if $x^n p(x^{-1}) = p(x)$.
Such a polynomial is said to be \defin{gamma-positive} if it can be expressed as
\[
 p(x) = \sum_{i=0}^{\lfloor n/2 \rfloor} \gamma_i x^i (1+x)^{n-2i}, \qquad \gamma_i \geq 0.
\]
The \defin{gamma-polynomial} of $p(x)$ is then the polynomial $\gamma_p(x) = \sum_j \gamma_j x^j$.


\todo{Add https://arxiv.org/pdf/2302.00754.pdf }

\todo{https://arxiv.org/pdf/2202.08984.pdf}

\todo{https://arxiv.org/pdf/2306.15785.pdf}




\begin{itemize}

\item $p(x)$ has all roots on the negative real axis if and only if $\gamma_p(x)$ does, \cite{Branden2004Poset,Gal2005}.

\item Suppose $p(x)$ is palindromic with center of symmetry $n/2$. 
If $\gamma_p(x)$ is ultra-log-concave of order $\lfloor n/2 \rfloor$ with no internal zeros, then
$p(x)$ is also ultra-log-concave of order $n$ with no internal zeros, \cite{BrandenFerroniJochemko2024x}.
The converse does not hold.

\item  Suppose $p(x)$ is palindromic. 
If $\gamma_p(x)$ is log-concave with no internal zeros, then
$p(x)$ is also log-concave with no internal zeros, \cite{FerroniPanovaVenturello2025x}.
The converse does not hold.

\end{itemize}



\subsection[gammaPositiveExamples]{Examples of gamma-positivity}


\begin{theorem}
Foata and Strehl proved that the Eulerian polynomials $A_n(t)$ are gamma-positive,
and the gamma-coefficients count orbits under the \defin{Foata--Strehl} action.

In \cite{Branden2008}, P. Brändén generalizes this result to sign-graded posets.
We also get gamma-positivity for the peak polynomials.
\end{theorem}


\begin{theorem}
Let $Z_m(t)$ be the $P$-Eulerian polynomial associated with the zig-zag poset on $m$ vertices.
Then $Z_m(t)$ is gamma-positive, see \cite{PetersenZhuang2024x}.
That is,
\[
 Z_m(t) = t \cdot \sum_{0 \leq 2j \leq m-2} \gamma_{m,j}t^j (1+t)^{m-2-2j}, \qquad \gamma_{m,j} \geq 0.
\]
We have that 
\[
 Z_m(t) = t \cdot \sum_{\pi \in U_m} t^{\ret_1(\pi)}
\]
where the sum is over \hyperref[namedPermutationSets]{up-down permutations},
and $\ret_1$ is the number of instances where $i+1$ appears to the left of $i$, but not adjacent to.

It is an open problem to explain (combinatorially) the symmetry and unimodality of these coefficients. 
\end{theorem}


\section[realrootedOpenProblems]{Open problems}


In \cite{CoulterDoMoskovsky2023x}, the authors consider polynomials related to Hurwitz numbers.
These are conjectured to be real-rooted and interlace in a certain fashion \cite[Conj. 3.14]{CoulterDoMoskovsky2023x}.
The authors prove this statement for one-point Hurwitz numbers.



