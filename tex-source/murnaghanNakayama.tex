\metatitle{Murnaghan--Nakayama rule}
\metadescription{An introduction to the Murnaghan--Nakayama rule, including its statement, various formulas for the characters of the symmetric group, and variants of the rule.}

\section[murnaghanNakayamaRule]{The Murnaghan--Nakayama rule}


The expansion of \hyperref[schurS]{Schur polynomials} in the 
\hyperref[powerSum]{power-sum} basis gives the
irreducible \hyperref[representation-character]{characters of the symmetric group} as coefficients:
\begin{equation*}
 \schurS_\lambda(\xvec) 
  = \sum_{\mu} \frac{\chi^{\lambda}_{\mu}}{z_\mu} \powerSum_{\mu}(\xvec)
  = \sum_{\mu} \frac{\powerSum_{\mu}(\xvec)}{z_\mu} \sum_{T \in \BST(\lambda,\mu)} (-1)^{\mathrm{ht}(T)}
\end{equation*} 
where the sum is taken over \hyperref[borderStripTableaux]{border strip tableaux} of shape $\lambda$ and weight $\mu$.

Another way to phrase the Murnaghan--Nakaygama rule is the following:
\begin{equation*}
\powerSum_r(\xvec) \schurS_\lambda(\xvec) = \sum_{\mu} (-1)^{\mathrm{ht}(\mu/\lambda)} \schurS_\mu(\xvec)
\end{equation*} 
where the sum ranges over all $\mu$ such that $\mu/\lambda$ is a border-strip with $r$ boxes.
The combinatorial rule was first stated in \cite{Murnaghan1937,Nakayama1940}.

A completely combinatorial proof of the Murnaghan--Nakayama rule is provided in \cite{Mendes2019},
which uses only a sign-reversing involution for the proof. 


We can also write the power-sum expansion as the average
\begin{equation*}
 \schurS_\lambda(\xvec) 
  = \frac{1}{n!} \sum_{\sigma \in \symS_n} \chi^{\lambda}(\sigma) \powerSum_{\text{type}(\sigma)}(\xvec).
\end{equation*}

The characters $\chi^{\lambda}_{\mu}$ also show up in the Schur expansion of power-sum symmetric functions.
\begin{equation*}
 \powerSum_\mu(\xvec) =  \sum_{\lambda} \chi^{\lambda}_{\mu} \schurS_{\lambda}(\xvec).
\end{equation*}
This is a consequence of the fact that the characters are orthogonal (so that inverting a transition matrix 
is more or less a transposition).


Yet another way to compute  $\chi^{\lambda}_{\mu}$ is by taking the coefficient of
$\prod_{i=1}^k x_i^{\lambda_i + \ell -i}$ in 
\[
 \prod_{i \lt j} (x_i - x_j ) \; \cdot \; \powerSum_{\mu}(x_1,\dotsc,x_\ell),
\]
where $\ell$ is at least the number of parts of $\lambda$.



\section[chiFormulas]{Formulas for $\chi^{\lambda}_{\mu}$}


\subsection[mnRecursion]{Murnaghan--Nakayama recursion}

One can reformulate the Murnaghan--Nakayama rule as a recursive rule,
see e.g. \cite[2.4.4]{JamesKerber1984}.

Let $\lambda \vdash n$ and $\mu \vdash n-m$.
\[
\chi^{\lambda}(\mu,m) = \sum_{\nu} (-1)^{\mathrm{ht}(\lambda /\nu)} \chi^{\nu}(\mu)
\]
where the sum ranges over all $\nu$ such that $\lambda/\nu$ is a border-strip of size $m$.


\subsection[roichmanChi]{Y. Roichman's formula}

Y. Roichman has an alternative way of expressing the coefficients $\chi^{\lambda}_{\mu}$.
This generalizes to so-called Kazhdan--Lusztig characters, see \cite{Roichman1997} and \cite{Roichman1999}.
\begin{theorem}[Y. Roichman, 1997]
We have that
\[
\chi^{\lambda}_{\mu} = \sum_{Q \in \SYT(\lambda)} \mathrm{weight}_\mu(Q)
\]
where
\[
\mathrm{weight}_\mu(Q) \coloneqq \prod_{\substack{1 \leq i \leq n \\ i \notin B(\mu)}} f_\mu(i,Q),
\quad 
B(\mu) \coloneqq \{ \mu_1 + \dotsb + \mu_r : 1\leq r \leq \length(\mu)\}
\]
and 
\[
f_\mu(i,Q) \coloneqq \begin{cases}
-1 &\text{ if $i+1$ is southwest of $i$} \\
0  &\text{ if $i+1$ is northeast of $i$, $i+2$ southwest of $i+1$ and $i+1 \notin B(\mu)$} \\
1  & \text{ otherwise}.
\end{cases}
\]
\end{theorem}

Another way to phrase this, is due to C. Athanasiadis \cite[Prop. 3.2]{Athanasiadis2015} and is as follows. 
\begin{theorem}
Let $\lambda$ be a partition of $n$.
Then the expansion of the Schur function $\schurS_{\lambda}$ into power sum symmetric functions is given by
\begin{align}\label{eq:schurRoichmanPexp}
\schurS_\lambda(\xvec)
=
\sum_{\mu \vdash n}
%
\frac{\powerSum_\mu(\xvec)}{z_\mu}
%
\sum_{\substack{ T \in \SYT(\lambda) \\  \DES(T) \in U_\mu }}
(-1)^{\DES(T) \setminus S_\mu}
\,,
\end{align}
where $S_\mu$ and $U_\mu$ are defined \hyperref[gesselPowerSum]{here}.
\end{theorem}



\subsection[holmesChi]{R. Holmes' recursion}

R. Holmes \cite{Holmes2017x} generalizes a results by James and Kerber \cite[2.4.3]{JamesKerber1984}.
This gives a recursive method of computing $\chi^{\lambda}_{\mu}$.
The interesting aspect of this recursion is that one computes characters of $\symS_n$
by only relying on characters values for $\symS_{n-1}$
(and not smaller groups as with the Murnaghan--Nakayama recursion).

The following set of relations are enough to compute all $\chi^{\lambda}_{\mu}$.
Here, $\lambda \vdash n$ and we always assume that the arguments to $\chi$
are partitions of the same size, and $\epsilon_i$ denotes the unit vector with $1$
at the $i^\thsup$ coordinate.

\emph{Special case of the Murnaghan--Nakayama rule:}
\[
	\chi^{\lambda}_{(n)} =
	\begin{cases}
	(-1)^{n-\lambda_1} & \text{ if $\lambda$ is a hook} \\
	0 &\text{ otherwise}.
	\end{cases}
\]

\emph{The James--Kerber branching rule:}
\[
	\chi^{\lambda}_{(\mu,1)} =
	\sum_{\substack{ 1 \leq i \leq \length(\lambda) \\ \lambda_i \gt \lambda_{i+1} }}
		\chi^{\lambda - \epsilon_i}_{\mu}.
\]

\emph{R. Holmes recursion}, $m \geq 2$:
\[
	\chi^{\lambda}_{(\mu,m)} = \frac{1}{m-1} \left[
	
	\sum_{\substack{ 1 \leq i \leq \length(\lambda) \\ \lambda_i \gt \lambda_{i+1} }}
		(\lambda_i - i) \chi^{\lambda - \epsilon_i}_{(\mu,m-1)}
	-
	\sum_{\substack{ 1 \leq j \leq \length(\mu) \\ \mu_{j-1} \gt \mu_{j} }}
		M_j \cdot \mu_j \cdot \chi^{\lambda}_{(\mu+\epsilon_j,m-1)}
	\right].
\]
In the second sum, we use the convention that $\mu_0 = \infty$
and $M_j$ denotes the number of parts equal to $\mu_j$
in the partition $(\mu,m-1)$.


To speed up computation, one can also add the following special case.

\emph{Hook formula case:}
\[
	\chi^{\lambda}_{(1^n)} = f^\lambda
\]
where $f^\lambda$ is the number of standard Young tableaux of shape $\lambda$.
This quantity can be computed using the \hyperref[partitionCores]{hook formula}.



\subsection[youngsConstroction]{Alfred Young's construction}

A construction due to Young is presented in \cite[Thm. 1.6]{GarsiaEgecioglu2020},
see also the discussion on \href{https://mathoverflow.net/questions/460613/an-identity-for-characters-of-the-symmetric-group}{this discussion on MathOverflow}.
This states
\[
\chi^{\lambda}_{\mu} =
z_\mu \frac{f^{\lambda}}{n!}
\sum_{\substack{\pi \in R(\mu) \\ \sigma \in C(\mu) \\ type(\pi \sigma) = \mu }} \sign(\sigma).
\]


\section[murnaghanNakayamaVariants]{Variants of the Murnaghan--Nakayama rule}

The following families have a Murnaghan--Nakayama rule.

\begin{itemize}

\item the \hyperref[grothendieckStableMN]{stable Grassmann Grothendieck polynomials}.

\item the \hyperref[kSchurMurnaghanNakayama]{$k$-Schur polynomials}.

\item $K\text{-}k$-Schur functions, see \cite{Nguyen2022x}.

\item Cylindric Schur functions (cylindric Hecke characters), see \cite[Lem. 5.3]{Korff2020}.

\end{itemize}


\subsection[quantumMNRule]{Quantum Murnaghan--Nakayama rule}

A \href{https://arxiv.org/pdf/1101.5250.pdf}{quantum version}
of the Murnaghan--Nakayama rule. Here, one multiplies the Schur function with a $q$-deformation
of the power-sum symmetric functions. These deformations are equal to
the Hall--Littlewood $P$-funtions, indexed by one-part partitions.
The paper has several conjectured generalizations of Murnaghan--Nakayama
rules for Hall--Littlewood $P$-functions.


\subsection[plethysticMNRule]{Plethystic Murnaghan--Nakayama rule}

A rule for computing the coefficients in the expansion
\[
\schurS_\mu \cdot (\powerSum_r[\completeH_m]) = \sum_{\lambda \vdash rm+|\mu|} (-1)^{\mathrm{ht}_r(\lambda/\mu)} \schurS_{\lambda}
\]
is given by M. Wildon \cite{Wildon2016}. See also \cite{Wildon2018} for a more general result.



\subsection[chernClassMNRule]{Murnaghan--Nakayama rule for Chern classes of Schubert cells}

In \cite{FanGuoXiong2022x}, the authors give a Pieri rule and a Murnaghan--Nakayama rule for 
Chern classes of Schubert cells. 

