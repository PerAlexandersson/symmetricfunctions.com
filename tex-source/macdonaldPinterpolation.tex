\metatitle{Interpolation Macdonald P polynomials}
\metadescription{Interpolation Macdonald P polynomials and related symmetric functions.}


\section[macdonaldPi]{Interpolation Macdonald P polynomials}

\begin{polydata}{macdonaldPi}
  Name     & Interpolation Macdonald P polynomials \\
  Space    & ShiftedSym \\
  Basis    & True \\
  Rating   & 2 \\
  Bib      & Okounkov1998shifted \\
  Year     & 1998 \\
  Symbol   & $\macdonaldP^*_\lambda(\xvec;q,t)$ \\
  Category & Schur \\
\end{polydata}

The interpolation Macdonald polynomials (in type $A$) is a family of non-homogeneous polynomials 
which generalize the \hyperref[schurShifted]{shifted Schur polynomials}.
They are also known under the name \defin{shifted Macdonald polynomials}
or \emph{inhomogeneous Macdonald polynomials}.

This definition is taken from \cite{Okounkov1998}.
Let $\lambda \vdash n$. The \defin{interpolation Macdonald polynomial} 
$\macdonaldP^*_\lambda(\xvec;q,t)$ is the unique (up to scalar) 
polynomial of of degree $n$ that is 
symmetric in $x_1t^{n-1},x_2 t^{n-2},\dotsc,x_n$, and fulfills
\[
\macdonaldP^*_\lambda(q^{\mu_1},q^{\mu_2},\dotsc,q^{\mu_n};q,t) = 0
\text{ whenever } \lambda \not\subset \mu.
\]

The shifted Schur functions obtained as the limit
\[
\schurS^*_\lambda(\xvec) = \lim_{q \to 1} \frac{\macdonaldP^*_\lambda(q^{x_1},\dotsc,q^{x_n} ;q,q)}{ (q-1)^n }.
\]
Moreover, the \hyperref[jackShifted]{shifted Jack polynomials} are obtained as the limit
\[
\jackShifted_\mu(\xvec;a) = \lim_{q \to 1} 
\frac{\macdonaldP^*_\lambda(q^{x_1},\dotsc,q^{x_n} ;q,q^a)}{ (q-1)^n }.
\]


\subsection[macdonaldPiRSSYTFormula]{RSSYT formula}

In \cite{Okounkov1998shifted} A. Okounkov proves the following combinatorial formula for 
the interpolation Macdonald polynomials:
\[
\macdonaldP^*_\mu(\xvec;q,t) = 
\sum_T \psi_{T}(q,t)  \prod_{\square \in \mu} t^{1-T(\square)} \left( \xvec_{T(\square)} - q^{\arm'(\square)} t^{-\leg'(\square)
} \right)
\]
Here, the sum is over all reverse-tableaux of shape $\mu$,
and $\psi_{T}(q,t)$ is the \hyperref[macdonaldPTableauFormula]{same weight} which appear in the formula for 
the classical Macdonald $P$ polynomials $\macdonaldP(\xvec;q,t)$.


\section[macdonaldPiBC]{Type BC interpolation Macdonald polynomials}

In \cite{Okounkov1998}, a type BC family is introduced.
These are polynomials in the variables $x_1^{\pm},\dotsc,x_n^{\pm}$,
with coefficients in $\setQ(q,t,s)$.
Moreover, as $s\to \infty$, the polynomials $\macdonaldP^*_\lambda(\xvec;q,t)$ are 
recovered.





