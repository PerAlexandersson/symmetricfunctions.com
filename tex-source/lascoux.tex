\metatitle{Lascoux polynomials}
\metadescription{An introduction to Lascoux polynomials, their definition via divided-difference operators, their combinatorial models using set-valued tableaux and skyline fillings, and properties including expansions into Grothendieck polynomials.}
\metakeywords{Lascoux polynomials,Lascoux atoms,divided-difference operators,set-valued tableaux,skyline fillings,Grothendieck polynomials}

\section[lascoux]{Lascoux polynomials}

\begin{polydata}{lascoux}
  Name   & Lascoux polynomials \\
  Space    & All \\
  Basis    & True \\
  Rating   & 2 \\
  Bib      & Lascoux2001  \\
  Year     & 2001 \\
  Symbol   & $\lascoux^{(\beta)}_{\alpha}(z)$ \\
  Keywords & divided-difference, fillings, k-theoretic \\
  Category & Schur \\
\end{polydata}

The \defin{Lascoux polynomials}, is a family of non-symmetric, 
non-homogeneous polynomials, first introduced by \name{Alain Lascoux} in \cite{Lascoux2001}.
The set $\{ \lascoux_\alpha(x_1,\dotsc,x_n) \}_\alpha$ where $\alpha$
ranges over all compositions of length $n$, is a basis for $\setC[x_1,\dotsc,x_n]$.

The Lascoux polynomials are the K-theoretical analog of the \hyperref[key]{key polynomials},
and they generalize the \hyperref[grothendieckGrassman]{Grothendieck polynomials}.
Similarly, there are Lascoux-atom polynomials, which are K-theoretical 
analogs of the \hyperref[atom]{Demazure atom polynomials}.

\emph{Note:} There seem to be other types of polynomials referred to as Lascoux polynomials,
e.g. \cite{BorziChenMotwaniVenturelloVodicka2021}.


\subsection[lascouxDefinition]{Definition}

We use the same notation as in the operator definition of \hyperref[key]{key polynomials}.
Define the $\beta$-versions of the divided difference operator $\partial_i$, and $\pi_i$:
\[
\partial_i^{(\beta)}(f) \coloneqq 
\partial_i(f + \beta x_{i+1} f)
\qquad 
\pi_i^{(\beta)}(f) \coloneqq \partial_i^{(\beta)}(x_i f).
\]
The \defin{Lascoux polynomial} is then defined as
\[
\lascoux^{(\beta)}_{\alpha}(\xvec) \coloneqq 
\begin{cases}
\xvec^{\alpha} & \text{ if $\alpha$ is a partition} \\
\pi_i^{(\beta)} 
\lascoux^{(\beta)}_{s_i \alpha}(\xvec)
& \text{ if $\alpha_i \lt a_{i+1}$}.
\end{cases}
\]
Note that $\key_{\alpha}(\xvec) = \lascoux^{(0)}_{\alpha}(\xvec)$,
that is, at $\beta=0$ we recover a key polynomial.

The first combinatorial model for Lascoux polynomials was conjectured by \name{Cara Monical} in \cite{Monical2017},
where these are given as a sum over certain set-valued tableaux. 
They also provide a combinatorial formula involving set-valued \emph{skyline fillings} 
(terminology from \hyperref[macdonaldE]{non-symmetric Macdonald polynomials}).
A proof of this conjecture was later given in \cite[Thm. 4.1]{BuciumasScrimshawWeber2020}.

Another set-valued tableau formula is proved in \cite[Thm. 1.1]{ReinerYong2021}.
A set-valued tableau formula is also given in \cite[Thm. 3.16]{Yu2023}.
Here, the tableaux have composition shape, and the author argues that his formula 
is simpler, as it does not use the Lusztig involution.

In \cite{MonicalPechenikScrimshaw2018}, the question is raised if there is some type of K-theoretical
crystal structure for Lascoux polynomials. This seems to be answered in \cite{Yu2023}.



\subsection[lascouxProperties]{Properties of Lascoux polynomials}


The \hyperref[schubert]{Schubert polynomials} expand positively into key polynomials.
In the same manner, in \cite{ShimozonoYu2021}, \name{Mark Shimozono} and \name{Tianyi Yu} give a formula
for the expansion of Grothendieck polynomials into Lascoux polynomials,
thus proving an earlier conjecture by \name{Vic Reiner} and \name{Alexander Yong} \cite{ReinerYong2021}.




\section[lascouxAtom]{Lascoux atom}

\begin{polydata}{lascouxAtom}
  Name   & Lascoux atom \\
  Space    & All \\
  Basis    & True \\
  Rating   & 2 \\
  Bib      & Lascoux2004  \\
  Year     & 2004 \\
  Symbol   & $\lascouxAtom^{(\beta)}_{\alpha}(z)$ \\
  Keywords & divided-difference, fillings, k-theoretic \\
  Category & Schur \\
\end{polydata}


\subsection[lascouxAtomVertexModel]{A vertex model for Lascoux atoms}

In \cite{BuciumasScrimshawWeber2020}, the authors construct a 5-vertex model 
whose partition function is the Lascoux-atoms. 
This is the first combinatorial model for the Lascoux-atoms.


