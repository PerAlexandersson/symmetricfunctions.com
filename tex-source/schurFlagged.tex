\metatitle{Schur polynomials (flagged)}
\metadescription{An introduction to flagged Schur polynomials, their definition via semistandard Young tableaux with row bounds, Jacobi--Trudi identity, Kostka coefficients, and connections to key polynomials and Schubert polynomials.}
\metakeywords{Schur polynomials (flagged),semistandard Young tableaux,row bounds,Jacobi-Trudi identity,Kostka coefficients,key polynomials,Schubert polynomials}

\section[schurFlagged]{Schur polynomials (flagged)}

\begin{polydata}{schurFlagged}
  Name   & Schur polynomials (flagged) \\
  Space    & All \\
  Basis    & False \\
  Rating   & 2 \\
  Bib      & LascouxSchutzenberger1982 \\
  Year     & 1982 \\
  Symbol   & $\schurS_{\lambda,b}(x_1,\dotsc,x_n)$ \\
  Category & Schur \\
  Keywords & key-positive \\
\end{polydata}


The flagged Schur polynomials generalize the classical
\hyperref[schurS]{Schur polynomials} and these are no longer symmetric in general.
They were first considered in \cite{LascouxSchutzenberger1982},
in the study of Schubert polynomials.

We follow the introduction given in \cite{MerzonSmirnov2015}.

Let $m \leq n$ be fixed integers and let $\lambda$ be a partition with
$m$ parts and let $1 \leq b_1 \leq b_2 \leq \dotsb \leq b_\ell = n$ be
a sequence of integers.
Let $T(\lambda,b)$ denote the set of semi-standard tableaux
of shape $\lambda$ such that the entries in row $i$ do not exceed $b_i$. 

Then the \defin{flagged Schur polynomial} is defined as
\[
\schurS_{\lambda,b}(x_1,\dotsc,x_n) \coloneqq \sum_{T \in T(\lambda,b)} \xvec^T.
\]
Note that if $b_1=\dotsb = b_m = n$, then $\schurS_{\lambda,b}(\xvec)$
is the usual Schur polynomial in $n$ variables.


Every flagged Schur polynomial $\schurS_{\lambda,b}(\xvec)$ is a key polynomial,
see \cite[Thm. 23]{ReinerShimozono1995}.
Moreover, a \hyperref[schubert]{Schubert polynomial} $\schubert_w(\xvec)$ is a flagged Schur function
if and only if $w$ is a \hyperref[namedPermutationSets]{vexillary permutation} (avoids 2143).


\subsection[hFlaggedSchurPolynomials]{$h$-flagged Schur polynomials}

Whenever $b = (h,h+1,\dotsc,h+m)$ for some $h$, we say that $\schurS_{\lambda,b}(\xvec)$
is \defin{$h$-flagged}, and use the shorthand notation $\schurS^{(h)}_{\lambda}(\xvec)$.

\subsection[flaggedSchurDividedDifferences]{Divided differences}

In \cite{Wachs1985}, it is proven that flagged Schur polynomials can 
be obtained via \hyperref[schubertOperatorformula]{divided difference operators}:
\begin{align*}
\schurS_{\lambda,b}(\xvec) = \partial_w( x_1^{a_1} \dotsm x_m^{a_m})
\end{align*}
where $a_i = \lambda_i + b_i -i$ and 
\[
w = (m,m+1,\dotsc, b_m-1,\;  m-1,m,\dotsc, b_{m-1} - 1, \dotsc 1,2,\dotsc, b_1 -1 )
\]
and the entries in the word are applied from left to right.


\subsection[flaggedSchurPolynomialJacobiTrudi]{Jacobi--Trudi identity}

Define the complete homogeneous polynomials in $k$ variables as 
\[
\completeH_d(\xvec_k) \coloneqq \sum_{i_1 \leq \dotsb \leq i_d}
x_{i_1}\dotsm x_{i_k}.
\]
Then, using the Lindström--Gessel--Viennot lemma, one can prove that
\[
\schurS_{\lambda,b}(\xvec) = \det(  \completeH_{\lambda_i -i+j}(\xvec_{b_i}) )_{1\leq i,j \leq m}.
\]
See also \cite{Wachs1985} or \cite[Cor. 2.6.3]{Manivel2001}.
This generalizes the classical \hyperref[schurJacobiTrudi]{Jacobi--Trudi identity}
for Schur polynomials.


The \defin{flagged skew Schur polynomials}, $\schurS_{\lambda/\mu,a,b}(\avec,\bvec)$
are now defined as the sum over all skew semistandard Young tableaux of shape $\lambda/\mu$
such that entries in row $i$ must be from the interval $\{a_i,a_i+1,\dotsb,b_i\}$.
We still require that the corresponding flags are weakly increasing.

With $\completeH_m(a,b) \coloneqq  \completeH_m(x_a,x_{a+1},\dotsc,x_b)$,
one can show the Jacobi--Trudi type formula
\begin{equation}
 \schurS_{\lambda/\mu}(\avec,\bvec) = 
 \left| h_{\lambda_i-\mu_j - i + j}(a_j,b_i) \right|_{1 \leq i,j \leq \ell}.
\end{equation}
This was proved by \name{I. Gessel} and \name{X. Viennot} (see \cite{GesselViennot1989}) and later in \cite{Wachs1985}.
An alternative proof is given in \cite{Chu1992}, which relies on a recursion.



Unlike the classical Jacobi--Trudi identities, exchanging $\completeH \leftrightarrow \elementaryE$ does
not in general yield the same determinant.
Rather, the corresponding determinant of elementary symmetric polynomials
is equal to a \enquote{column-flagged} Schur polynomial (where the tableaux in the generating
set have bounds on the columns rather than the rows); see \cite{Wachs1985}.
An identity between determinants of this type is given in \cite{McDowell2023},
thus giving a \enquote{duality} for row- and column-flagged Schur polynomials under certain conditions.

\subsection[flaggedSchurKostka]{Kostka coefficients}

The flagged Kostka coefficients are defined as the coefficients in the 
monomial expansion of the flagged Schur functions. Given a skew shape $\lambda/\mu$
and a vector $(\nu_1,\dotsc,\nu_\ell)$ with $\nu_i \geq 0$, we define 
the \defin{flagged Kostka coefficient} $K_{\lambda/\mu,\nu}(\avec,\bvec)$ via the expansion
\[
  \schurS_{\lambda/\mu}(\avec,\bvec) = \sum_{\nu}  K_{\lambda/\mu,\nu}(\avec,\bvec) \cdot \xvec^\nu.
\]

In \cite{AlexanderssonOguz2023x}, we prove the following results:
\begin{theorem}[Alexandersson--Kantarci-Oğuz (2023)]
We have that for any integer $k \geq 1$, 
\[
 K_{\lambda/\mu,\nu}(\avec,\bvec) \gt 0 \iff 
 K_{k\lambda/k\mu,k\nu}(\avec,\bvec) \gt 0
\]
where multiplication of a partition by $k$ is done elementwise. 
This is a generalization of \hyperref[gtKostkaSaturation]{saturation for Gelfand--Tsetlin polytopes}.

Moreover, the map
\[
 k \mapsto K_{k\lambda/k\mu,k\nu}(\avec,\bvec)
\]
is a polynomial in $k$. 
\end{theorem}

Note that it is relatively easy to see that the map 
\[
 k \mapsto K_{k\lambda/k\mu,k\nu}(\avec,\bvec)
\]
is a quasi-polynomial as this is the Ehrhart function for 
a face of a (rational) \hyperref[gtKoskta]{Gelfand--Tsetlin polytope}.

The following conjecture is then a flagged generalization of 
a \hyperref[gtKostkaEhrhart]{conjecture by King, Tollu and Toumazet}, \cite{KingTolluToumazet2004}.
\begin{conjecture}[Alexandersson--Kantarci-Oğuz (2023)]
The Ehrhart polynomial
\[
 k \mapsto K_{k\lambda/k\mu,k\nu}(\avec,\bvec)
\]
has non-negative coefficients.
\end{conjecture}




\subsection[flaggedSchurKeyExpansion]{Key expansion}


The flagged skew Schur polynomials expand positively into \hyperref[key]{key polynomials},
see \cite{ReinerShimozono1995}.



\subsection[flaggedSchurRootsOfUnity]{Twist by roots of unity}

In \cite{Kumar2023x} the author generalizes a previous result by Kumari \cite{Kumari2022x},
and \cite[Thm. 16]{LeeOh2022}.


