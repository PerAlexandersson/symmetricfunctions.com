\metatitle{Catalan cyclic sieving}
\metadescription{A list of instances of the cyclic sieving phenomenon on Catalan and Narayana-type objects}

See the \hyperref[cyclicSieving]{cyclic sieving phenomenon} 
page for the definition and related theorems.

The $q$-analog of the Catalan numbers here is defined as $\catalan(n,q) = \frac{1}{[n+1]_q}\qbinom{2n}{n}_q $.

\section[cspCatalan]{Catalan objects}

\subsection[cspCatalanTriangulations]{Triangulations of $(n+2)$-gons}

Let $\mathrm{TRI}(n)$ be the set of triangulations on an $n$-gon.
Note that $|\mathrm{TRI}(n)| = \catalan(n-2)$.
We let $\rot_{n}$ denote rotation by $2\pi/n$.
Then 
\[
\left( \mathrm{TRI}(n), \langle \rot_{n} \rangle,  \catalan(n-2,q) \right)
\]
exhibits the CSP, see \cite{ReinerStantonWhite2004}.




In \cite[Thm. 8.2]{AlexanderssonLinussonPotkaUhlin2021}, we refine this result.
We let $\mathrm{TRI}_{\mathrm{ear}}(n,k)$ denote the set of triangulations 
of an $n$-gon with $k$ ears. An ear is a triangle with at most one side in the interior.
\begin{theorem}
Let $2 \leq k \leq \frac{n}{2}$ and let
\begin{equation*}
\mathrm{Tri}_q(n,k) \coloneqq
q^{k(k-2)} \frac{[n]_q}{[k]_q} \qbinom{n-4}{2k-4}\catalan_q(k-2)
\left( \sum_{j=0}^{n-2k} q^{j(n-2)}\qbinom{n-2k}{j}\right).
\end{equation*}
Then $\sum_k \mathrm{Tri}_q(n,k) = \catalan_q(n-2),$ and
$
\left(
\mathrm{TRI}_{\mathrm{ear}}(n,k), \langle \rot_n \rangle, \mathrm{Tri}_q(n,k)
\right)
$
exhibits the cyclic sieving phenomenon.
\end{theorem}



\subsection[cspNonCrossingMatchingsI]{Non-crossing matchings}

Let $X_n$ be the set of non-crossing matchings, on $2n$ vertices. It is well-known
that $|X_n| =\catalan(n)$.
We let the generator of $\grpc_n$ act on $X_n$ by a $2\pi/n$ rotation.
Then 
\[
\left( X_{n}, \grpc_n,  \frac{1}{[n+1]_q}\qbinom{2n}{n}_q \right)
\]
exhibits the CSP.
This is a special case of promotion, $\partial$, on \hyperref[cspRectangularSYT]{rectangular SYT},
in the case of tableaux of shape $(n,n)$.
In fact, we can refine the above CSP, and let $\grpc_{2n}$ act on $X_n$ by a $\pi/n$ rotation.
Then 
\[
\left( X_{n}, \grpc_{2n},  \frac{1}{[n+1]_q}\qbinom{2n}{n}_q \right)
\]
exhibits the CSP, see \cite{PetersenPylyavskyyRhoades2008}.

\todo{
A direct combinatorial proof:
https://math.stackexchange.com/questions/2913242/computing-fixpoints-of-noncrossing-matchings-of-2n-points-under-rotation
}

\begin{example}
For $n=3$, there are five non-crossing matchings:
\begin{figure}
    \svgimg[width=0.17\textwidth]{./svg-images/nonCrossingMatching000.svg}{Non-crossing matching 1.}
    \svgimg[width=0.17\textwidth]{./svg-images/nonCrossingMatching010.svg}{Non-crossing matching 2.}
    \svgimg[width=0.17\textwidth]{./svg-images/nonCrossingMatching100.svg}{Non-crossing matching 3.}
    \svgimg[width=0.17\textwidth]{./svg-images/nonCrossingMatching110.svg}{Non-crossing matching 4.}
    \svgimg[width=0.17\textwidth]{./svg-images/nonCrossingMatching210.svg}{Non-crossing matching 5.}
\end{figure}

These are in bijection with north-east lattice paths from $(0,0)$ to $(n,n)$ which are never below the main diagonal.
The bijection is given by parenthesis matching, where a north-step is a left parenthesis, and an east step is a right parenthesis.
\begin{figure}
    \svgimg[width=0.17\textwidth]{./svg-images/dyckPath000.svg}{Dyck path 1.}
    \svgimg[width=0.17\textwidth]{./svg-images/dyckPath010.svg}{Dyck path 2.}
    \svgimg[width=0.17\textwidth]{./svg-images/dyckPath100.svg}{Dyck path 3.}
    \svgimg[width=0.17\textwidth]{./svg-images/dyckPath110.svg}{Dyck path 4.}
    \svgimg[width=0.17\textwidth]{./svg-images/dyckPath210.svg}{Dyck path 5.}
\end{figure}    

The lattice paths are now in bijection with $2\times n$ standard Young tableaux.
The bijection is given by making the number $k$ appear in the first row of the SYT if step $k$
in the path is a north-step.

\begin{figure}
\begin{ytableau}
1 & 3 & 5 \\
2 & 4 & 6
\end{ytableau}
\begin{ytableau}
1 & 3 & 4 \\
2 & 5 & 6
\end{ytableau}
\begin{ytableau}
1 & 2 & 5 \\
3 & 4 & 6
\end{ytableau}
\begin{ytableau}
1 & 2 & 4 \\
3 & 5 & 6
\end{ytableau}
\begin{ytableau}
1 & 2 & 3 \\
4 & 5 & 6
\end{ytableau}
\end{figure}
\end{example}



In \cite[Thm. 4.8]{AlexanderssonLinussonPotkaUhlin2021}, we refine the CSP on non-crossing matchings.
Let $\NCM_{sh}(n, k)$ be the set of non-crossing matchings with exactly $k$ short edges,
and $\SYT_{cdes}(n^2,k)$ be the set of $2 \times n$ standard Young tableaux with exactly $k$ cyclic descents.
\begin{theorem}
Let $k, n \geq 2$ be natural numbers and let
\[
 \SYT_q(n,k) \coloneqq \frac{q^{k(k-2)}(1+q^n)}{[n+1]_q} \qbinom{n+1}{k} \qbinom{n-2}{k-2}.
\]
Then \[
\sum_{k} \SYT_q(n,k) = \catalan_{n}(q),
\] and the triples
\[
(\SYT_{cdes}(n^2,k),\langle \partial_{2n} \rangle,\SYT_q(n,k) )\]
and
\[
(\NCM_{sh}(n, k),\langle \mathrm{rot}_{2n} \rangle, \SYT_q(n,k) )
\]
exhibit the cyclic sieving phenomenon.
\end{theorem}




\subsection[csp12]{Non-crossing (1,2)-configurations}

Let $\mathrm{NCC}_n$ be the set non-crossing $(1,2)$-configurations of $[n-1]$.
That is, elements are partitioned into sets of size $1$ or $2$
and sets of size $2$ are non-crossing.
We have that $|X_n| = \catalan(n)$, \cite[Family 60]{StanleyCatalan}.
The generator of $\grpc_{n}$ act on $\mathrm{NCC}_{n+1}$ by a $2\pi/n$ rotation.
Then 
\[
\left( \mathrm{NCC}_{n+1}, \grpc_{n}, \catalan_{n+1}(q) \right)
\]
exhibits the CSP, see \cite{Thiel2017}.


A representation-theoretical proof of this cyclic sieving phenomenon 
can be found in \cite{ZengZhang2024x}. The authors also 
give a dihedral group action with CSP. 


\subsection[csp12twist]{Non-crossing (1,2)-configurations, with twist}

There is an alternative group action, combining rotation with a type of twist,
see \cite[Thm. 5.4]{AlexanderssonLinussonPotkaUhlin2021}.
With a different $q$-analog of the Catalan numbers, we get CSP for non-crossing $(1,2)$-configurations.
\begin{theorem}
The triple
\[
\left(
\mathrm{NCC}(n+1), \langle \mathrm{twist}_{2n} \rangle, \qbinom{2n}{n} - q^2 \qbinom{2n}{n-2}
\right)
\]
exhibits the cyclic sieving phenomenon.
\end{theorem}

There is a type $B$ version of non-crossing $(1,2)$-configurations.
In \cite[Thm. 6.6]{AlexanderssonLinussonPotkaUhlin2021},
we show that
$
\left(\mathrm{NCC}^B(n+1), \langle \mathrm{twist}^2_{2n} \rangle, \qbinom{2n}{n}\right)
$
is a CSP-triple.


\subsection[cspNonCrossingPartitions]{Non-crossing partitions (Narayana)}

A non-crossing partition is a set-partition of the vertices in an $n$-gon,
so that the convex hulls of the sets do not intersect.
Let $X_{n,k}$ be the set of non-crossing partitions with $k$ blocks,
and let $\grpc_n$ act by rotation of the $n$-gon.
Then 
\[
\left( X_{n,k}, \grpc_n, q^{k(k+1)} \frac{1}{[n]_q} \qbinom{n}{k}_q  \qbinom{n}{k+1}_q \right)
\]
exhibits the CSP, see \cite{ReinerStantonWhite2004}.


\subsection[cspWCatalan]{Non-crossing partitions II}

D. Bessis and V. Reiner \cite{BessisReiner2011} study \emph{well-generated complex reflection groups}.
For such groups $W$, one can define a generalization of the $q$-Catalan numbers, $\catalan(W,q)$
as well as the notion of non-crossing partitions.
The set of non-crossing partitions, $NC(W)$ are identified with a certain set of reflections in $W$.
There is then a special element, $c\in W$ acting on $NC(W)$ via conjugation.

It is then proved that
\[
\left( 
NC(W), \langle c \rangle , \catalan(W,q)
\right)
\]
is a CSP-triple.

The authors provide a connection with Springer's theory, and end the article with 
several conjectures. Most of these conjectures are resolved later in \cite{ReinerSommers2018}.


\subsection[cspNarayana]{Non-crossing partitions III (Narayana, Kreweras)}

In \cite{ReinerSommers2018}, the authors find a cyclic sieving phenomenon 
on non-crossing partitions of $n$ with exactly $k$ parts.
The number of such non-crossing partitions is given by the \defin{Narayana numbers},
and we define a $q$-analogue of the \hyperref[prelimQanalogsCatalan]{Narayana numbers} as
\[
N_{n,k}(q) \coloneqq \frac{ q^{k(k-1)} }{ [n]_q } \qbinom{n}{k}_q \qbinom{n}{k-1}_q 
= \frac{ q^{k(k-1)} }{[k]_q} \qbinom{n-1}{k-1}_q \qbinom{n}{k-1}_q.
\]
Note that $\catalan_n(q)=\sum_k N_{n,k}(q)$.
The $\grpc_n$ group action on the non-crossing partitions is given by rotation.
For each fixed value of $n$ and $k$ with $1\leq k \leq n$, 
we have a cyclic sieving phenomenon.


The authors also prove CSP in the case of \defin{$q$-Kreweras numbers}.
Let $\lambda \vdash n$ and $\length(\lambda)=k$. The $q$-analogue 
of the Kreweras numbers is defined as 
\[
Krew(\lambda;q) \coloneqq \frac{q^{k(n-1)-c(\lambda)}}{[n+1]_q} \qbinom{n+1}{m(\lambda),n-k+1}
\]
where $c(\lambda) = \sum_{i} \lambda'_i \lambda'_{i+1}$
and $m(\lambda)$ is the type of $\lambda$.
Then $Krew(\lambda;1)$ is the number of non-crossing partitions where the part sizes 
are given by $\lambda$. 
The cyclic group action still gives a cyclic sieving phenomenon.

The authors generalize this to other classical types of 
Weyl groups ($A$,$B$,$C$,$D$).


\begin{example}[Non-crossing partitions]
Here is a list of the non-crossing partitions for $n=4$:

\begin{figure}
  \svgimg[width=0.95\textwidth]{./svg-images/nonCrossingPartitions.svg}{Non-crossing partitions for n=4.}
\end{figure}

\end{example}


\subsection[cspNonCrossingPartitionsFuss]{Non-crossing partitions IV (Fuß--Catalan)}

In \cite{KrattenthalerMuller2013}, the authors consider 
$m$-divisible non-crossing partitions.
Let $X_{n,m}$ be the number of non-crossing partitions of $[nm]$
where each part has a size which is a multiple of $m$.
Consider the $q$-analogue $q$-analogue of the Fuß--Catalan numbers:
\[
f_{n,m}(q) \coloneqq \frac{1}{[n]_q} \qbinom{(m+1)n}{n-1}_q =  \frac{1}{[(m+1)n+1]_q}\qbinom{(m+1)n+1}{n}_q.
\]
Then $(X_{n,m},\grpc_{mn},f_{n,m}(q))$ is a CSP-triple,
where $\grpc_n$ act by rotation of $2\pi/(mn)$.


Similarly, let $Y_{n,m}$ be the number of non-crossing partitions of $[nm]$
where each part has a size equal to $m$, and let
\[
g_{n,m}(q) \coloneqq \frac{1}{[n]_q} \qbinom{mn}{n-1}_q.
\]
Then $(Y_{n,m},\grpc_{mn},g_{n,m}(q))$ is a CSP-triple,
where $\grpc_n$ act by rotation of $2\pi/(mn)$.


The authors also consider \emph{irreducible well-generated complex reflection groups},
which reduce to the Fuß--Catalan numbers in type $A$.
See also \cite{ReinerSommers2018}.



\subsection[cspRationalCatalan]{Rational Catalan and Kreweras}

In \cite{BodnarRhoades2016} M. Bodnar and B. Rhoades prove that the 
rational $q$-Catalan numbers,
\[
\catalan_{a/b}(q) = \frac{1}{[a+b]_q} \qbinom{a+b}{a}_q,
\]
admits a cyclic sieving phenomenon. They require that $a\lt b$ and $a$, $b$ are coprime.
Then $\grpc_{b-1}$ act on the set of $(a,b)$-non-crossing 
partitions of $[b-1]$ by rotation, and this gives a CSP.
This settles a conjecture stated in \cite{Armstrong2012}.

They also prove the corresponding refined rational $q$-Narayana and $q$-Kreweras version of this.
The rational $q$-Narayana numbers are defined as
\[
N_{a/b,k}(q) = \frac{1}{[a]_q} \qbinom{a}{k}_q \qbinom{b-1}{k-1}_q,
\]
where the classical \hyperref[prelimQanalogsCatalan]{Narayana numbers} are recovered when $a=b=n$ and $q=1$.

