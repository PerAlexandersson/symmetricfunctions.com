\metatitle{Chromatic quasisymmetric functions}
\metadescription{An introduction to chromatic quasisymmetric functions, their definition via proper colorings of graphs, properties such as symmetry, reciprocity, and power-sum positivity, and connections to LLT polynomials and Hessenberg varieties.}  

\section[chromaticQuasisymmetric]{Chromatic quasisymmetric functions}

\begin{polydata}{chromaticQuasisymmetric}
  Name   & Chromatic quasisymmetric functions \\
  Space  & Sym \\
  Basis  & False \\
  Rating & 7 \\
  Bib    & ShareshianWachs2012 \\
  Year   & 2012 \\
\end{polydata}

\todo{
Some Lie algebra connection
https://arxiv.org/pdf/1908.08198.pdf
}

\todo{Hessenberg dot action: https://arxiv.org/pdf/2511.13913}

\todo{
 Chromatic quasisymmetric functions under group action:
\url{https://arxiv.org/pdf/2112.09109.pdf}
\url{https://arxiv.org/pdf/2106.02665.pdf}
}


\todo{
Relation with Macdonald Nabla operator, and Hopf algebra and e-positivity.
http://de.arxiv.org/pdf/1908.04841.pdf
}

\todo{
Non-symmetric quasisymmetric, and slide positivity
https://arxiv.org/pdf/1908.06598.pdf
}

\todo{
Dual-equivalence graphs, refinement and Schur expansion:
https://arxiv.org/pdf/2003.12123.pdf
}


\todo{
2-cycle chord graphs are e-positive (non-commutative)
https://arxiv.org/pdf/2112.06679.pdf
Same authors:
Spiders, brooms: 
https://arxiv.org/pdf/2112.06619.pdf
}


\subsection[chromaticQuasisymmetricHistory]{Historical overview}

R. Stanley introduced the \defin{chromatic symmetric function} of a graph $G$ in \cite{Stanley1995}.
This definition was later refined by J. Shareshian and M. Wachs in \cite{ShareshianWachs2012} and \cite{ShareshianWachs2016},
where the graph $G$ is labeled and a parameter $q$ is introduced.

In Spring 2016, \name{Brittney Ellzey} \cite{Ellzey2017} presented a generalization where labeled graphs are replaced with \emph{oriented graphs}.
This new model is equivalent to the Shareshian--Wachs model when the orientation of the edges has no directed cycle.
See also  \name{Greta Panova}, \name{Per Alexandersson} \cite{AlexanderssonPanova2018}, who considered the same model from a different perspective.

The main open problem in this area concerns the conjectured positive 
expansion of chromatic symmetric functions in the elementary symmetric function basis,
and later the refined conjecture by \name{John Shareshian} 
and \name{Michelle Wachs} in \cite{ShareshianWachs2012,ShareshianWachs2016}.
These conjectures are referred to 
the \hyperref[chromaticQuasisymmetricElementaryExpansion]{Stanley--Stembridge conjecture and Shareshian--Wachs conjecture}, respectively.
Shareshian and Wachs also conjectured that $\omega \chrom_G(\xvec;q)$ is 
\hyperref[chromaticQuasisymmetricHessenberg]{related to 
the equivariant cohomology of regular semisimple Hessenberg varieties}.



In \cite{Hwang2022x}, \name[B.-H. Hwang]{Byung-Hak Hwang} introduces an 
additional refinement on the chromatic quasisymmetric functions.
He defines a \enquote{flip} operator on colorings and show that connected 
components under such flips
are Schur-positive. The ascent parameter introduced by Shareshian--Wachs 
is preserved by such flips.
It is conjectured that each such component is positive in the elementary 
symmetric function basis.
See also \cite{Hwang2023x} for a connection with non-commutative functions.
A proof for Schur-positivity of the flip-components is also given in \cite{BlasiakErikssonPylyavskyySiegl2022x}.


For a representation-theoretical perspective, see \cite{Skandera2020},
where connections with immanents and characters is studied.


\subsection[chromaticQuasisymmetricDefinition]{Definition}

Let $G$ be an \emph{oriented graph} with $n$ vertices. 
Given a vertex-coloring $\kappa:G\to \setP$,
let $\asc(\kappa)$ denote the number of directed edges $(u,v)$ in $G$ such that $\kappa(u) \lt \kappa(v)$.

The \defin{chromatic quasisymmetric function} of $G$ is defined as 
\[
\chrom_G(\xvec;q) \coloneqq  
\sum_{\substack{\kappa : G\to \setP \\ \kappa \text{ proper}} } x_{\kappa(1)}\dotsm x_{\kappa(n)} q^{\asc(\kappa)},
\]
where the sum runs over all proper colorings of the graph $G$. 
A coloring is proper there are no edges whose endpoints receive the same color.
The function $\chrom_G(\xvec;q)$ is quasi-symmetric in $\xvec$.

From the definition, it is clear that
\[
\chrom_G(1^k;1)
\]
is the number of ways to color $G$ with $k$ colors, 
so the classical \defin{chromatic polynomial} $\chi_G(k)$ is given by the specialization $\chrom_G(1^k;1)$.
See \cite{SaganVatter2021} for a nice article with bijective proofs of properties of $\chi_G(k)$.


Note that every labeling induces an orientation, by orienting edges in the direction of largest label.
This labeled model is what was introduced in \cite{ShareshianWachs2012,ShareshianWachs2016}.

\subsection[chromaticClawFree]{Claw-free graphs}

If $G$ is a claw-free graph, \name{Vesselin Gasharov} (see \cite{Stanley1998}) conjecture that $\chrom_G(\xvec;1)$ is Schur-positive.
There is no known $q$-analog of this conjecture.

Some progress in this direction has been made in \cite{ShelburneWilligenburg2024x},
using a type of involution inspired by the \hyperref[gesselSlinkyRule]{slinky rule}.
They prove Schur positivity for the family of graphs called \defin{generalized nets}.

\subsection[chromaticQuasisymmetricUnitIntervalGraph]{Unit interval graphs}

The following definition using area sequences is from \cite{AlexanderssonPanova2018}.

A \defin{circular unit arc digraph} (or circular Dyck path) 
$\Gamma_\avec$ is a directed graph with vertex set $[n]$ and edges
\begin{align}
(i-a_i) \to i, \; (i-a_i + 1) \to i,  \; (i-a_i + 2) \to i, \; \dotsc \; , (i-1) \to i
%   i \to i+1, \; i \to i+2,\; \dotsc, i \to i+a_i
\end{align}
for all $i = 1,\dotsc,n$, where indices are taken modulo $n$, 
and the integers $\avec=(a_1,\dotsc,a_n)$ satisfy

\begin{itemize}
\item $0 \leq a_i \leq n-1 $ for $1 \leq i \leq n$, 
\item $a_{i+1} \leq a_{i} + 1$ for $1 \leq i \leq n$, (index mod $n$).
\end{itemize}

Whenever $a_1=0$, $\Gamma_\avec$ is a \defin{unit interval graph}.
The sequence $\avec = (a_1,a_2,\dotsc,a_n)$ is called the \defin{area sequence} of the graph.

There are Catalan $\frac{1}{n+1}\binom{2n}{n}$ many unit interval graphs on $n$ vertices, 
and a total of 
\[
(n+2)\binom{2n-1}{n-1} - 2^{2n-1}
\]
circular unit arc digraph on $n$ vertices, see \cite{AlexanderssonLinussonPotka2019} for a proof.
This sequence appears as \oeis{A194460} in the OEIS.


\begin{example}
The area sequences $(0,1,2,1,1)$ and $(3,2,1,2,2)$ can be 
presented as \emph{Dyck diagrams}, or in the second circular case,
a \defin{circular Dyck diagram}:

\begin{figure}
\begin{ytableau}
\square & \square & \square &  & 5 \\
\square & \square &   & 4 \\
  &   &  3 \\
  &  2 \\
1
\end{ytableau}
\begin{ytableau}
\none & \none & \none & \none & \square & \square & & & 5 \\
\none & \none & \none & \square & \square & & & 4 \\
\none & \none & \square & \square & \square & & 3 \\
\none & \square & \square & & & 2 \\
\square &  & & & 1
\end{ytableau}
\end{figure}

The labeled boxes represent vertices, the empty boxes are edges 
(connecting the vertex in the same column with the vertex in the same row),
and the squares represent non-edges. The first graph has 5 edges, while the second has 9 edges.
Note that circular Dyck diagrams \enquote{wrap around} at top and bottom.
Edges are directed upwards (or rightwards) in these diagrams.
\end{example}


The \defin{Abelian unit interval graphs} on vertices $\{1,2,\dotsc,n\}$
are the unit interval graphs where there is no vertex $j$ such that both $(1,j)$ and $(j,n)$ are non-edges.
The number of Dyck paths of size $n$ with an Abelian unit interval graph is $2^{n-1}$.



\section[chromaticQuasisymmetricProperties]{Properties of chromatic quasisymmetric functions}


\subsection[chromaticQuasisymmetricSymmetric]{Symmetry}

Whenever $G$ is a unit arc digraph as above, $\chrom_G(\xvec;q)$ is symmetric 
(for labeled graphs, \cite{ShareshianWachs2012,ShareshianWachs2016} and for oriented graphs, \cite{Ellzey2017,AlexanderssonPanova2018}).
Furthermore, the coefficients in $\setZ[q]$ in the monomial basis are palindromic.
To be precise,
\[
\chrom_\avec(\xvec;q)=\sum_\lambda c_\lambda(q) \monomial_\lambda(\xvec) \qquad \Longrightarrow \qquad
c_\lambda(q) = q^{|\avec|}c_\lambda(1/q),
\]
where $|\avec|$ denotes the sum $a_1+a_2+\dotsb + a_n$.


\subsection[chromaticQuasisymmetricReciprocity]{Reciprocity}

Let $\chi_G(k)$ be the chromatic polynomial associated with the graph $G$ on the vertex set $[n]$.
\name{Olivier Bernardi} and \name{Philippe Nadeau} \cite{BernardiNadeau2020} gave the following interpretation of
the specialization $(-1)^{n-i}\chi^{(i)}_G(-j)$.
\begin{theorem}
Let $i,j \geq 0$. Then $(-1)^{n-i}\chi^{(i)}_G(-j)$ counts the number of tuples
\[
\left( (V_1, \theta_1),\dotsc,(V_{i+j}, \theta_{i+j}) \right)
\]
such that 
\begin{itemize}
\item $V_1,\dotsc,V_{i+j}$ is a set-partition of the vertices of $G$,
\item $\theta_\ell$ is an acyclic orientation of the graph induced by $V_\ell$,
\item for all $\ell \in [i]$, $V_\ell$ is non-empty and $\theta$ has a unique source, given by $\min V_\ell$.
\end{itemize}
\end{theorem}





\subsection[chromaticQuasisymmetricLLT]{Relationship with LLT polynomials}

Carlson--Mellit \cite[Prop.3.4]{CarlssonMellit2017} proved the following \hyperref[plethysm]{plethystic} relationship between 
chromatic quasisymmetric functions and \hyperref[LLT]{unicellular LLT polynomials} indexed by unit interval graphs 
on $n$ vertices:
\[
(q-1)^{-n} \LLT_\avec[\xvec(q-1);q] =  \chrom_\avec(\xvec;q).
\]
This relationship is not always true in the circular case, that is, for area sequences where $a_1 \gt 0$.
The relationship is reminiscent of the technique 
introduced by \name{Jim Haglund}, \name{Mark Haiman} 
and \name{Nicholas Loehr} \cite{HaglundHaimanLoehr2005}.

This plethystic relationship is generalized in \cite[Thm. 6.2]{NovelliThibon2019},
where a non-commutative lift is given. \name[J.-C. Novelli]{Jean-Christophe Novelli} and \name[J.-Y. Thibon]{Jean-Yves Thibon} also introduce 
a non-commutative lift of the unicellular LLT polynomials.


\subsection[chromaticQuasisymmetricGesselExpansion]{Fundamental quasisymmetric positivity}

It is easy to show that
\[
\chrom_G(\xvec;q) = \sum_{\theta \in AO(G)} q^{\asc(\theta)} K_{P(\theta)}(\xvec)
\]
where $K_{P(\theta)}(\xvec)$ is the generating function 
of \emph{strict} \hyperref[pPartition]{order-preserving maps} 
from the poset $P(\theta)$ given by the transitive closure of the directed edges in $\theta$,
to the set of positive integers.

From this, it follows that $\chrom_G(\xvec;q)$ is 
positive in the \hyperref[gessel]{fundamental quasisymmetric basis}.


\subsection[chromaticQuasisymmetricPowerSumExpansion]{Power-sum positivity}

In \cite{AlexanderssonSulzgruber2019}, 
it is proved that $\omega \chrom_G(\xvec;q)$ is always $\qPsi$-positive, 
that is, positive in a \hyperref[qPsi]{quasisymmetric powersum} basis.
Consequently, $\omega \chrom_G(\xvec;q)$ 
is $\powerSum$-positive whenever $\chrom_G(\xvec;q)$ is symmetric. 
This was proved earlier in \cite{Ellzey2017}, and in the case of unit interval graphs $G$ in \cite{Athanasiadis2015}.
A combinatorial formula in the unit-interval case was conjectured earlier by Shareshian--Wachs.


We have the following expansion, proved in \cite{AlexanderssonSulzgruber2019}:
\begin{theorem}[Alexandersson, Sulzgruber (2019)]
Let $G$ be a graph on $n$ vertices.
The quasisymmetric power-sum expansion of $\omega \chrom_G(\xvec;q)$ is given by
\[
\omega \chrom_G(\xvec;q) = \sum_{\theta \in AO(G)} q^{\asc(\theta)} 
\sum_{f \in \opsurj^{\ast}(\theta)} \frac{\qPsi_{\type(f)}(\xvec)}{z_{\type(f)}}
\]
where the first sum ranges over acyclic orientations of $G$.
The set $\opsurj_{\alpha}^{\ast}(\theta)$
consists of of order-preserving surjections $f$ from $\theta$ to some $[k]$,
such that $\theta$ restricted to the vertices $f^{-1}(j)$ has a unique sink.
The type of $f$ is the composition $\alpha$ given by $\alpha_j \coloneqq |f^{-1}(j)|$.
\end{theorem}
In particular, for $q=1$, we have that $\chrom_G(\xvec)  = \chrom_G(\xvec;1)$ is symmetric.
Thus, the formula above gives that
\begin{align}
\omega \chrom_G(\xvec) &= \sum_{\theta \in AO(G)} 
\sum_{f \in \opsurj^{\ast}(\theta)}
\frac{\powerSum_{\type(f)}(\xvec)}{z_{\type(f)}}.
\end{align}
This can further be simplified,
to give 
\begin{align}
\omega \chrom_G(\xvec) &=\sum_{\theta \in AO(G)} \powerSum_{\type(\theta)}(\xvec),
\end{align}
where $\type(\theta)$ is given by the \emph{source-components}
of $\theta$, see \cite{BernardiNadeau2020}.
This formula is very similar to the stronger result by \name[C. A. Athanasiadis]{Christos Athanasiadis} \cite{Athanasiadis2015},
which treats the case with general $q$.


%
% For a given $\theta$, we define an ordered set partition $\pi(\theta)$
% as follows (compare with the $\elementaryE$-expansion
% of \hyperref[unicellularLLTEExpansion]{unicellular LLT polynomials}).
% Each sink $s$ of $\theta$ lies in a unique part of $\pi(\theta)$.
% Furthermore, vertex $j$ is in the same part as $s$,
% if $s$ is the largest sink $j$ can reach via a directed path.
% Hence, each part of $\pi(\theta)$ induces an
% orientation with a unique sink --- the largest element in the part.
%
% Recall also that $z_\alpha = \prod_j m_j! j^{m_j}$,
% where there are $m_j$ parts equal to $j\geq  1$ in $\alpha$.
%
% &= \sum_{\theta \in AO(G)} \powerSum_{\type(\theta)}(\xvec)
% where the type of $\theta$ is the partition given by the part sizes of $\pi(\theta)$.


\subsection[chromaticQuasisymmetricSchurExpansion]{Schur positivity}

In \cite{Gasharov1996}, \name[Gasharov]{Vesselin Gasharov} proved that the $\chrom_G(\xvec;1)$ 
are Schur-positive with an explicit expansion,
whenever $G$ is the incomparability graph of a $3+1$-free poset.
In particular, this implies the unit-interval case, as those are given as the incomparability graphs
of $3+1$ and $2+2$-free posets. Gasharov's proof uses a kind of sign-reversing involution.

Gasharov's result was later extended to the Shareshian--Wachs setting (with a $q$-parameter),
by using a different involution.
That is, $\chrom_G(\xvec;q)$ is Schur-positive for unit-interval graphs $G$,
see \cite[Thm. 6.3]{ShareshianWachs2012}.

\begin{example}
We have the following Schur expansions of chromatic quasisymmetric functions, indexed by area sequences:
\begin{align*}
\chrom_{000}(x;q)&=\schurS_{111}+2 \schurS_{21}+\schurS_{3} \\
\chrom_{001}(x;q)&=(1+q) \schurS_{111}+(1+q) \schurS_{21} \\
\chrom_{011}(x;q)&=(1+2 q+q^2) \schurS_{111}+q \schurS_{210} \\
\chrom_{111}(x;q)&=(3 q+3 q^2) \schurS_{111} \\
\chrom_{012}(x;q)&=(1+2 q+2 q^2+q^3) \schurS_{111} \\
\chrom_{112}(x;q)&=(q+4 q^2+q^3) \schurS_{111} \\
\chrom_{122}(x;q)&=(3 q^2+3 q^3) \schurS_{111} \\
\chrom_{222}(x;q)&=6 q^3 \schurS_{111} 
\end{align*}
\end{example}


\begin{problem}
Prove that $\chrom_G(\xvec;q)$ is Schur-positive (with coefficients in $\setN[q]$) 
in the case of \emph{circular} unit arc digraphs.
This conjecture has been verified for all circular unit arc digraphs up to 10 vertices.

Find a bijective proof of Schur-positivitity (using \hyperref[schurPositivityCrystals]{crystals} or 
a version of \hyperref[schurPositivityRSK]{RSK}).
Gasharov's result provides a hint on how a crystal graph structure might be defined.

A step in the direction of a crystal structure is taken in \cite{Ehrhard2022x},
but the \enquote{crystals} described there are not crystals in the usual sense.
\end{problem}


In \cite{WangWang2020}, the authors give a (signed) formula 
for the Schur coefficients of the chromatic symmetric function of any graph.
They do this by inverting the matrix of Kostka coefficients.
The approach is reminiscent of how one can turn a fundamental quasisymmetric expansion
into a Schur expansion, using the \hyperref[gesselSlinkyRule]{slinky rule}.
The authors also characterize all Schur positive complete tripartite graphs.



\subsection[chromaticQuasisymmetricElementaryExpansion]{Elementary symmetric expansion}

The Stanley--Stembridge conjecture (1993) is a long-standing open problem in 
algebraic combinatorics, first formulated in \cite{StanleyStembridge1993} and \cite{Stanley1995} (for $q=1$).
A refinement using the general $q$-parameter was presented in \cite[Conj. 4.9]{ShareshianWachs2012},
which we now state.

\todo{Add Hikita's result}

\begin{conjecture}[Shareshian--Wach (2012)]
Let $G$ be a unit interval graph. Find a statistic $\mu$ on acyclic orientations of $G$,
such that
\[
\chrom_G(\xvec;q) = \sum_{\theta \in AO(G)} q^{\asc(\theta)} \elementaryE_{\mu(\theta)}(\xvec).
\]
That is, prove that $\chrom_G(\xvec;q)$ are $\elementaryE$-positive.
\end{conjecture}
See \hyperref[unitIntervalAO]{the weighed Dyck path enumeration page} for more info.

This conjecture extends to the circular unit arc digraph case,
see \cite{Ellzey2017,AlexanderssonPanova2018}.
According to A. Mellit (personal communication), 
this would imply Conjecture 5.4 in \url{https://arxiv.org/pdf/1506.08188.pdf}.

Positivity has been proved for several families of unit-interval graphs,
see the \hyperref[chromaticEExpansionCurrentStatus]{separate subsection on the current status}.


\begin{conjecture}[See \cite[Conj. 5.1]{ShareshianWachs2016}]
The following unimodality conjecture is posed:
Write 
\[
\chrom_{G}(\xvec;q) = \sum_{j=0}^m q^j a_j(\xvec),
\]
where $m$ is the number of edges of $G$.
Then $a_{j+1}(\xvec) - a_{j}(\xvec)$ is $\elementaryE$-positive for all $0 \leq j \lt (m-1)/2$.
\end{conjecture}
This conjecture extends to the circular unit arc digraphs.


The main theorem of \cite{ChoHongLee2020x}, gives an explicit condition
for when a coefficient in the $\elementaryE$-expansion is positive.


\subsection[chromaticEExpansionCurrentStatus]{Current status of the Stanley--Stembridge/Shareshian--Wach conjecture}

The following references give proofs of $\elementaryE$-positivity for some families of graphs.

\begin{itemize}
\item 
Combinatorial interpretation of line graph, 
complete graph and cycle graph, \cite{Ellzey2017,AlexanderssonPanova2018}.
\item 
When the complement graph is also a unit-interval graph, \cite{FoleyHoangMerkel2019}.
\item
Triangular ladders, \cite{Dahlberg2018x}. 
This correspond to area sequences of the form $(0,1,2,2,\dotsc,2)$.
\item 
Lollipop graphs for $q=1$, \cite{DahlbergWilligenburg2018}.
\item 
Melting lollipop graphs, \cite{HuhNamYoo2020}.
\item 
Abelian area sequences with bounce number 2, using a recursive method, \cite{HaradaPrecup2019}.
See also the alternative proof in \cite{ChoHuh2019}, using a sign-reversing involution.
Yet another proof is given in \cite{NadeauTewari2022x}. 
See also \cite{LeeSoh2022x} for a proof.


\item 
For area sequences with bounce number 3, some coefficients considered in \cite{ChoHong2019}. 
This is extended further in \cite{Wang2022x}.

\item Generalized pyramid graphs, and $(claw, 2K_2)$-free unit interval graphs, \cite{LiYang2021}.

\item A generalization of melting lollipop graphs, \cite{Tom2023x}.
This paper includes a signed $\elementaryE$-expansion derived from the $\elementaryE$-expansion
of unicellular LLT polynomials.

\item All $e$-coefficients with at most 2 parts are non-negative,
see \cite{AbreuNigro2023x} 
(using the cohomology of \hyperref[hessenbergVarieties]{Hessenberg varieties}) 
and also \cite{RokSzenes2023x}.

\item The case for $2+1+1$-avoiding unit interval orders. \cite{McDonoughPylyavskyyWang2024x}.
This corresponds to the case when $i-2 \leq a_i$ for all $i$ in the area sequence.
The authors use a novel idea using \emph{strand diagrams} and \hyperref[representationTheory]{representation theory}.


\item Twinning operation applied to several families of graphs, \cite{BanaianCelanoChangLeeColmenarejoGoffKimbleKimpelLentferLiangSundaram2024x}.
Page 7 in this reference has a nice overview of $e$-positivity results.

\item Certain conjoined graphs, \cite{QiTangWang2024x}.

\end{itemize}


Moreover, some larger families of graphs are shown not to be $e$-positive,
see \cite{DahlbergFoleyWilligenburg2017x,FoleyKazdanKrollAlbergaMelnykTenenbaum2018x}.


\subsection[chromaticDeletionContraction]{Deletion contraction property}

There is an extension to \hyperref[extendedChromatic]{extended chromatic symmetric functions}
which admits a deletion-contraction recurrence.


\subsection[rookTheory]{Connection with rook placements}

Chromatic quasisymmetic functions of unit interval graphs has a strong connection with rook placements.
Acyclic orientations and rook placements are in bijection, see for example \cite{AlexanderssonPanova2018}.
This connection is expanded upon in \cite{ColmenarejoMoralesPanova2023}.
In particular, some new results in the Abelian case using $q$-hit numbers are given.


\section[treesChomaticQuasisymmetric]{Chromatic symmetric functions of trees}


\name{Richard Stanley} conjectures \cite{Stanley1995} that the chromatic symmetric function of a tree, 
uniquely determines the tree. In \cite{WangYubZhang2023x} it is shown that trees with exactly 
two vertices of degree greater than 2, are distinguished by their chromatic symmetric functions.

For chromatic symmetric functions (CSF) for trees in the star basis, see \cite{PrietoMierOrellanaZamora2023,GonzalezOrellanaTomba2024x}.
It is shown that trees of diameter less than 5 can be reconstructed from their CSF.
Caterpillar graphs are known to be reconstructed from their CSF, \cite{LoeblSereni2019}.
Certain $q$-caterpillar trees can be reconstructed from their CSF, \cite{ArunkumarNarayananBVSawant2023x}.

In \cite[Thm. 1.3]{HasebeTsujie2017}, it is proved that \emph{ordered trees} are distinguished by
their order quasisymmetric function (where the rooted tree is viewed as a poset, colors are increasing away from the root).
Both the weakly increasing, or strictly increasing version of the quasisymmetric function does the job.
The work by \name{Takahiro Hasebe} and \name{Shuhei Tsujie} is continued in \cite{Zhou2020x},
where one is interested in reconstructing the tree from the quasisymmetric function.

It is still an open problem (posed by Hasabe and Tsujie) to prove that 
the \hyperref[pPartition]{P-partition quasisymmetric function} distinguishes
posets whose Hasse diagrams are trees.

In \cite{AlexanderssonSulzgruber2019}, we conjecture that in the family of directed trees,
the chromatic quasisymmetric function uniquely determines the directed tree.
This conjecture would follow from the conjecture by Hasabe and Tsujie.
See also \cite{FoleyKazdanKrollAlbergaMelnykTenenbaum2021x}.


There is a variant of Stanley's CSF defined for \emph{rooted trees},
and this rooted version is shown to distinguish non-isomorphic trees, see \cite{LoehrWarrington2024}.


The CSF associated with trees are in general not Schur positive,
see the counterexamples presented in \cite{EmmanuellaSandratraRambeloson2020}.
In fact, no tree on 20 vertices with maximal degree 10 is Schur positive.


See also \cite{PrietoMartinWagnerZamora2024x} for various results on polynomial invariants of trees.


\section[chromaticQuasisymmetricGeneralGraphs]{Other families of graphs}

By writing the CSF in different bases, one can obtain various classical statistics of graphs,
see \cite{ChanCrew2023}. For coefficients in the elementary basis in relation to sinks, see \cite{CrewZhang2022x}.


The chromatic symmetric functions for certain \enquote{spider graphs} 
are shown to be Schur positive in \cite{ThibonWang2023x}.
The approach uses non-commutative ideas.



\section[chromaticQuasisymmetricHessenberg]{Connection with Hessenberg varieties}

In \cite[Conj. 5.3]{ShareshianWachs2012}, a connection with cohomology of regular semisimple Hessenberg varieties 
was conjectures.
This conjecture was resolved in in 2015, where \name{Patrick Brosnan} 
and \name{Timothy Chow} proved that for unit-interval graphs $\avec$, 
the symmetric function $\omega \chrom_\avec(\xvec;q)$ 
is the \hyperref[frobeniusCharacteristic]{Frobenius characteristic} of 
a symmetric group action on the cohomology of regular semisimple Hessenberg varieties.
This was also proved independently by Guay-Paquet \cite{GuayPaquet2016x}.
Note that their result implies Schur positivity.

There is a bijection between acyclic orientations of a unit interval graph, and 
the \name{Julianna Tymoczko} cells of the corresponding 
regular nilpotent Hessenberg variety, see \cite{NovelliThibon2024x}.


\subsection[hessenbergVarieties]{Hessenberg varieties}

This section uses information from \href{https://math.dartmouth.edu/~orellana/wachs.pdf}{these slides}.
See also the PhD thesis by \name{Nicholas Teff} \cite{Teff2013}.

\todo{Add more from https://math.dartmouth.edu/~orellana/wachs.pdf}



A \defin{Hessenberg sequence} is a sequence of natural numbers $(m_1,\dotsc,m_n)$
such that $i \leq m_i \leq n$. These are in bijection with area sequences of unit inverval graphs, via $a_i = m_i - i$.

Let $F(n)$ be the set of all flags, $F_1 \subset F_2 \subset \dotsb \subset F_n = \setC^n$,
where $F_i$ has dimension $i$.
Let $D$ be a diagonal $n\times n$-matrix with distinct diagonal entries.

Then the \defin{type $A$ Hessenberg variety} associated with $m$ is defined as
\[
H(m) \coloneqq \{ F \in F(n) : DF_i \subseteq F_{m_i} \text{ for all $i$} \}.
\]

\begin{example}
The area sequence $(0,1,2,1,1)$ correspond to the Hessenberg sequence $(2,3,5,5,5)$.
Note that the Hessenberg sequence is simply the number of 
non-$\square$ boxes in each row in the extended $n\times n$-diagram.

\begin{figure}
\begin{ytableau}
\square & \square & \square &  &  \\
\square & \square &   & & \\
 &   &   &  & \\
 &   &   &  & \\
 &   &   &  &
\end{ytableau}
\end{figure}

\end{example}

There is a torus action on $H(m)$, by left multiplication of invertible $n\times n$-matrices.
We consider $0$-dimensional and $1$-dimensional fixed-points under this torus action.
This is what is called the \defin{moment graph}, associated with $H(m)$.

The combinatorial description of the moment graph is easier to define using 
the corresponding unit-interval graph $\Gamma_\avec$,
with edge set $E$. The vertices are the permutations in $\symS_n$,
and the edge set is 
\[
\{ (\sigma, (i,j)\sigma : \sigma \in \symS_n \text{ and } \{i,j\} \in E \}.
\]
Let us denote this graph by $MG(\avec)$.

One can then define the so called \defin{Tymoczko’s representation} \cite{Tymoczko2008}
where $\symS_n$ act on a certain polynomial ring defined via $MG(\avec)$.


\section[chromaticQuasisymmetricMisc]{Other results related to chromatic symmetric functions}

In \cite{LinPierson2025x}, the authors study a different multiplicative structure (Hopf algebra)
on chromatic symmetric functions and show that graph complement (of unweighted, triangle-free graphs)
corresponds naturally to a Hopf algebra operation.
