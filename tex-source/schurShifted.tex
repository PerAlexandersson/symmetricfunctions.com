\metatitle{Shifted Schur polynomials}
\metadescription{An introduction to shifted Schur polynomials, their definition via falling factorials, properties including vanishing property, combinatorial formula, Jacobi--Trudi identities, and Littlewood--Richardson rule.}
\metakeywords{Shifted Schur polynomials,falling factorials,vanishing property,combinatorial formula,jacobi-trudi,littlewood-richardson rule}

\section[schurShifted]{Shifted Schur polynomials}


\begin{polydata}{schurShifted}
  Name     & Shifted Schur polynomials \\
  Space    & ShiftedSym \\
  Basis    & Yes \\
  Rating   & 3 \\
  Symbol   & $\schurShifted_\mu(\xvec) $ \\
  Bib      & OkounkovOlshanski1996 \\
  Keywords & shifted \\
  Year     & 1996 \\
  Categoty & Schur \\
\end{polydata}


Shifted Schur functions were studied in \cite{OkounkovOlshanski1996}, 
and are closely related to the factorial Schur polynomials.
The shifted Schur functions are specializations of the \hyperref[jackShifted]{shifted Jack functions}.

Define the \defin{falling factorial} as
\[
\fallingFactorial{x}{k} \coloneqq x(x-1)\dotsm (x-k+1), \qquad \fallingFactorial{x}{0} \coloneqq 1.
\]

The \defin{shifted Schur polynomials} are defined as the quotient
\[
\schurShifted_\mu(x_1,\dotsc,x_n) \coloneqq \frac{ \det( \fallingFactorial{x_i+n-i}{\mu_j+n-j} ) }{  \det( \fallingFactorial{x_i+n-i}{n-j} )  }.
\]


\subsection[schurShiftedProperties]{Properties}

The shifted Schur polynomials are stable in the sense that
\[
\schurShifted_\mu(x_1,\dotsc,x_n) = \schurShifted_\mu(x_1,\dotsc,x_n,0).
\]

Let $H(\mu)$ be the product of hook lengths in the diagram $\mu$.
The shifted Schur functions $\schurShifted_\mu(\xvec)$
is the unique family of shifted symmetric functions with degree $\leq |\mu|$
such that 
\[
\schurShifted_\mu(\lambda) = \delta_{\lambda\mu} H(\mu)
\]
for all $\lambda$ such that $|\lambda| \leq |\mu|$.
This is commonly referred to as the \defin{vanishing property}.


Let $l=|\lambda|$ and $k=|\mu|$. 
Then
\[
\frac{f^{\lambda/\mu}}{f^{\lambda}} = \frac{\schurShifted_\mu(\lambda)}{\fallingFactorial{l}{k}}
\]
where $f^{\lambda/\mu}$ is the number of skew standard Young tableaux of shape $\lambda/\mu$.


\subsection[schurShiftedCombinatorialFormula]{Combinatorial formula}

There is a combinatorial formula, see \cite{OkounkovOlshanski1996},
\[
\schurShifted_\mu(\xvec) = \sum_{T \in \mathrm{RSSYT}(\mu)} \prod_{s \in \mu} (x_{T(s)} - c(s))
\]
where $\mathrm{RSSYT}(\mu)$ is the set of reverse semi-standard Young tableaux.
These are fillings of $\mu$ with weakly decreasing rows and strictly decreasing columns.
Here, $c(s)$ is the \emph{content} of the square $s$.

Note that this formula implies that
\[
\schurShifted_\mu(\xvec) = \schurS_\mu(\xvec) + \text{ lower order terms}.
\]


\subsection[schurShiftedJacobiTrudi]{Jacobi--Trudi identities}

Define the shifted elementary and complete homogeneous symmetric functions as
\[
\elementaryE^*_r(\xvec) = \schurShifted_{(1^r)}(\xvec)\qquad
\completeH^*_r(\xvec) = \schurShifted_{(r)}(\xvec).
\]

Let $H^*(u)$ be the formal power series in $u^{-1}$:
\[
H^*(u) \coloneqq \sum_{r\geq 0} \frac{ \completeH^*_r(\xvec) }{ \fallingFactorial{u}{r} } = 
\prod_{i=1}^\infty \frac{u+i}{u+i-x_i}.
\]

Define the automorphism $\phi$ on the algebra of symmetric functions as $\phi H^*(u) = H^*(u-1)$.

In \cite{OkounkovOlshanski1996} it is then proved that
\[
\schurShifted_{\mu}(\xvec) = \det[ \phi^{j-1} \completeH^*_{\mu_i - i +j} ]_{1\leq i,j \leq l}
\qquad 
\schurShifted_{\mu}(\xvec) = \det[ \phi^{1-j} \elementaryE^*_{\mu'_i - i +j} ]_{1\leq i,j \leq m}
\]
where $l \geq \length(\mu)$ and $m \geq \mu_1$.

\subsection[schurShiftedLittlewoodRichardson]{Littlewood--Richardson rule}

Since the \hyperref[schurFactorial]{factorial Schur polynomials} 
has a \hyperref[littlewoodRichardsonModels]{Littlewood--Richardson rule} \cite{MolevSagan1999},
one can find one for the shifted Schur polynomials as well.
For examples of a combinatorial interpretation, see \cite{Molev2009}.

\begin{theorem}
The shifted Schur Littlewood--Richardson coefficients are defined via 
\[
\schurShifted_{\lambda}\schurShifted_{\mu} = \sum_{\nu} c^{\nu}_{\lambda \mu} \schurShifted_{\nu}.
\]
Let $\mu, \nu \subseteq \lambda$. Then
\begin{equation*}
c^{\lambda}_{\mu\nu} = 
\frac{1}{|\lambda|-|\nu|}\left( 
 \sum_{\nu \to \nu^+}  c^{\lambda}_{\mu \nu^+} - \sum_{\lambda^- \to \lambda }  c^{\lambda^-}_{\mu \nu}
\right)
\end{equation*}
where the first sum is taken over all possible ways to add one box to the diagram $\nu$,
and the second sum is over all ways to remove one box from $\lambda$.

This together with the identity 
$c^{\lambda}_{\mu \lambda} = \schurShifted_\mu(\lambda)$ 
(which can be computed via the Jacobi--Trudi identity) gives a recursive method to compute the $c^{\lambda}_{\mu\nu}$.
This recursion has an analog for the \hyperref[jackShiftedLittlewoodRichardsonFormula]{shifted Jack functions}.
\end{theorem}

Note that if $|\nu| = |\lambda|+|\mu|$, then $c^{\lambda}_{\mu\nu}$ is the
classical \hyperref[littlewoodRichardsonModels]{Littlewood--Richardson coefficient} for the Schur functions.




\section[schurFactorial]{Factorial Schur polynomials}


\begin{polydata}{schurFactorial}
  Name   & Factorial Schur polynomials \\
  Space    & Sym \\
  Basis    & Yes \\
  Rating   & 2 \\
  Symbol   & $\schurFactorial_\mu(\xvec)$ \\
  Bib      & BiedenharnLouck1989  \\
  Keywords & factorial, jacobi-trudi, lr-rule, alternant, ssyt, gt-patterns \\
  Year     & 1989 \\
  Categoty & Schur \\
\end{polydata}

\todo{I think BL defined what is now known as shifted Schur functions, and factorial ones are Molev-Sagan.}

The \defin{factorial Schur functions} $\schurFactorial_\lambda(\xvec)$ is a family of non-homogeneous symmetric polynomials, indexed by partitions.
They were introduced by Biedenharn and Louck in \cite{BiedenharnLouck1989}.
The top-degree component is the classical Schur polynomial 
$\schurS_\lambda$. They form a $\setZ$-basis for the space of symmetric polynomials (in $n$ variables).


The factorial Schur functions can be defined using \hyperref[gtpatterns]{Gelfand--Tsetlin patterns}.
We sum over all GT-patterns with $n$ rows and top row $\lambda$ (note that this set is in bijection with 
semi-standard Young tableaux of shape $\lambda$ and maximal entry at most $n$):
\[
 \schurFactorial_\lambda(x_1,\dotsc,x_n) \coloneqq \sum_{(m_{ij}) \in GT(\lambda)} \prod_{j=1}^n \prod_{i=1}^j
 \left( x_j - m_{i,j-1} + i-j \right)_{m_{i,j}-m_{i,j-1}}
\]
where we are using the falling factorial notation $(a)_k = a(a-1)\dotsm (a-k+1)$.
Under the GT-pattern to SSYT bijection, the quantity $m_{i,j}-m_{i,j-1}$ is sent 
to the number of times $j$ appears in row $i$ (in the SSYT).

From this definition, we can see that 
\[
 \schurFactorial_\lambda(x_1,\dotsc,x_n) = \sum_{\mu \downarrow \lambda} \prod_{i=1}^n 
 (x_n - \mu_i - n +i  )_{\lambda_i - \mu_i} \cdot  \schurFactorial_\mu(x_1,\dotsc,x_{n-1}),
\]
where we write $\mu \downarrow \lambda$ for when 
\[
 \lambda_1 \geq \mu_1 \geq \lambda_2 \geq \mu_2 \geq \dotsb \geq \mu_{n-1} \geq \lambda_n.
\]
This is the same as $\mu$ can be in a row immediately below $\lambda$ in a GT-pattern.

The factorial Schur polynomials are special cases of the \hyperref[schubertDouble]{double Schubert polynomials},
indexed by Grassmann permutations.


\defin{Skew factorial Schur functions} can similarly be defined using sums over skew GT-patterns instead 
(see \cite{ChenLouck1993}): 
\[
 \schurFactorial_{\lambda/\mu}(x_1,\dotsc,x_n) \coloneqq \sum_{(m_{ij}) \in GT(\lambda/\mu)} \prod_{j=1}^n \prod_{i=1}^j
 \left( x_j - m_{i,j-1} + i-j \right)_{m_{i,j}-m_{i,j-1}}.
\]
Here, we sum over GT-patterns with $n+1$ rows, with top row $\lambda$ and bottom row $\mu$.


\subsection[schurFactorialAlternant]{Alternant quotient formula}

We have that 
\[
\schurFactorial_{\lambda}(x_1,\dotsc,x_n) = \frac{1}{\Delta(\xvec)} \sum_{\mu} K_{\lambda \mu} \Delta(\xvec-\mu)
\prod_{i=1}^n (x_i)_{\mu_i}
\]
where $\Delta(\xvec)$ is the Vandermonde determinant, see \cite[Thm. 5]{BiedenharnLouck1990}.


Macdonald proved that the above formula leads to the simpler 
\[
\schurFactorial_{\lambda}(x_1,\dotsc,x_n) = \frac{\left| (x_i)_{\lambda_j+n-i} \right| }{ \prod_{i \lt j} x_j - x_j  },
\]
which is analogous to the alternant quotient formula for Schur functions,
see \cite[Thm. 3.2]{ChenLouck1993} for a proof.


\subsection[schurFactorialJT]{Jacobi--Trudi}

There is a Jacobi--Trudi identity for skew factorial Schur functions, see \cite{ChenLouck1993}.

\todo{http://docplayer.net/163152902-Combinatorial-aspects-of-generalizations-of-schur-functions.html}



\subsection[schurFactorialMolevSagan]{Generalized factorial Schur functions}

Molev--Sagan (see \cite{MolevSagan1999}) considers the \defin{generalized factorial Schur functions} 
defined via the alternant formula
\[
\schurFactorial_{\lambda}(x_1,\dotsc,x_n | \avec) = \frac{\left| (x_i|a)_{\lambda_j+n-i} \right|}{ \prod_{i \lt j} x_j - x_j  },
\]
where $(y|a)_k \coloneqq (y-a_1)(y-a_2)\dotsm (y-a_k)$.
Thus, we have that the classical factorial Schur functions are a specialization of the Molev--Sagan family:
\[
\schurFactorial_{\lambda}(x_1,\dotsc,x_n ) = \schurFactorial_{\lambda}(x_1,\dotsc,x_n | 0,1,2,3,\dotsc).
\]

These also admits a tableau definition:
\[
\schurFactorial_{\lambda}(x_1,\dotsc,x_n | \avec) =
\sum_{T \in \SSYT(\lambda)} \prod_{\alpha \in \lambda} \left( x_{T(\alpha)} - a_{T(\alpha)+c(\alpha)}  \right)
\]
where $c(\alpha)$ is the \hyperref[partitionCores]{content} of the box $\alpha$.
These are more or less the same as the \emph{double Schur functions}, see \cite{Molev2009co}.



\subsection[schurFactorialLR]{Littlewood--Richardson rule}

The factorial Schur functions admits a Littlewood--Richardson rule, see \cite{MolevSagan1999}.


