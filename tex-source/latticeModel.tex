\metatitle{Lattice models}
\metadescription{Brief overview of what a lattice model is}



\todo{https://arxiv.org/pdf/2404.13221.pdf}

\section[latticeModel]{Introduction}

From \cite{Karhadkar2022x}: A \defin{lattice model} is based on an edge-labeled graph 
(usually a rectangular grid) with some local constraints. 
Each vertex is assigned a weight, depending on its adjacent edges. 
If every vertex satisfies the constraints, we say that this assignment is an \emph{admissible state}.
The goal is to conclude some sort of global behavior.

The \defin{partition function} of a lattice model, is the sum over all admissible states:
\[
 Z = \sum_{S \in \mathrm{ states}} \prod_{v \in V} \mathrm{weight}_S(v).
\]

Depending on the conditions, various functions and identities may be recovered.
For example, we can obtain the \hyperref[schurS]{Schur functions},
the \hyperref[schurCauchyFormula]{dual Cauchy identity}, etc.

From now on, we assume that we are in the rectangular grid 
setting (each vertex has four adjacent neighbors) unless stated otherwise.

\subsection[latticeModelVertex]{Vertex models}


The terms \defin{five-vertex model}, \defin{six-vertex model} or \defin{eight-vertex model},
refer to the number of admissible local labelings around a vertex.
For example, the figure below illustrates the six labelings in the six-vertex model.
\begin{figure}
<img src="./svg-images/6-vertex-model.svg" alt="The 6 admissible local labelings for the 6-vertex model." style="width:95%; margin:5px;"/>
\end{figure}

Each such state is then assigned a weight, so in the $n$-vertex model, 
there are $n$ possible weights associated with a vertex.
This weight is sometimes referred to as the \defin{Boltzmann weight},
and the matrix which records the weights is called the \defin{R-matrix}.

It is usually interesting to compute the partition function 
with some fixed \defin{boundary condition} or \defin{wall condition}
that is, labels on the edges "exiting" the rectangular grid.


\subsection[YangBaxter]{The Yang--Baxter equation}

In order to discuss the Yang--Baxter equation. 
the edges in the graph can be seen as forming strands, or braidings.

The \defin{Yang--Baxter equation} is a relation describes how 
certain quantities are preserved as strands are permuted.
This is useful for proving symmetries.

Models which satisfy the Yang--Baxter equation are 
called \defin{integrable} or \defin{exactly solvable}. 


\subsection[latticeModelList]{Symmetric functions via lattice models}

In \cite{CurranFrechetteYostWolffZhangZhang2021x},
the 6-vertex model for (supersymmetric) \hyperref[LLT]{LLT polynomials} 
and they prove a Cauchy identity for the spin LLT polynomials $\LLTG^{(k)}_{\lambda/\mu}(\xvec;q)$.
The same model is considered in \cite{Hardt2021x}, and it is shown 
that the partition function is more or less is unique in a certain sense.


For metaplectic Whittaker functions, see \cite{BrubakerBuciumasBumpGustafsson2020}.

Lattice models for Grothendieck polynomials,
\cite{BrubakerFrechetteHardtTiborWeber2020x,BuciumasScrimshaw2020x}.
