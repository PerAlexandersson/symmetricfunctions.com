\metatitle{Kostka coefficients, Kostka--Foulkes polynomials and charge}
\metadescription{Kostka--Foulkes polynomials and charge}


\section[kostkaCoefficients]{Kostka coefficients}


The \defin{Kostka coefficients} appear in the monomial expansion of \hyperref[schurS]{Schur functions},
and the Schur expansion of the complete homogeneous symmetric functions:
\[
\schurS_\lambda = \sum_\mu K_{\lambda\mu} \monomial_\mu,
\qquad 
\completeH_\mu = \sum_\lambda K_{\lambda\mu} \schurS_\lambda.
\]
Hence, $K_{\lambda\mu}$ is equal to the number of \hyperref[prelimTableaux]{semi-standard Young tableaux}
of shape $\lambda$ and content $\mu$.

From this, we immediately have that $K_{\lambda\lambda}=1$, $K_{\lambda,1^n} = f^{\lambda}$
and that $K_{\lambda\mu}=0$ unless $\lambda \trianglerighteq \mu$ 
in \hyperref[partialOrderingsOnPartitions]{dominance order}.


\begin{example*}[Kostka matrices]
Below is the matrix $(K_{\lambda\mu})$ and its inverse, with rows and columns indexed by
the partitions 5, 41, 32, 311, 221, 2111, 11111.
\[
(K_{\lambda\mu})=
\begin{pmatrix}
 1 & 1 & 1 & 1 & 1 & 1 & 1 \\
 0 & 1 & 1 & 2 & 2 & 3 & 4 \\
 0 & 0 & 1 & 1 & 2 & 3 & 5 \\
 0 & 0 & 0 & 1 & 1 & 3 & 6 \\
 0 & 0 & 0 & 0 & 1 & 2 & 5 \\
 0 & 0 & 0 & 0 & 0 & 1 & 4 \\
 0 & 0 & 0 & 0 & 0 & 0 & 1 \\
\end{pmatrix}
,
\quad 
(K_{\lambda\mu})^{-1}=
\begin{pmatrix}
 1 & -1 & 0 & 1 & 0 & -1 & 1 \\
 0 & 1 & -1 & -1 & 1 & 1 & -2 \\
 0 & 0 & 1 & -1 & -1 & 2 & -2 \\
 0 & 0 & 0 & 1 & -1 & -1 & 3 \\
 0 & 0 & 0 & 0 & 1 & -2 & 3 \\
 0 & 0 & 0 & 0 & 0 & 1 & -4 \\
 0 & 0 & 0 & 0 & 0 & 0 & 1 \\
\end{pmatrix}
\]
\end{example*}

See \hyperref[symmetricTransisionMatrices]{the overview of transition matrices}
for a combinatorial interpretation of the entries in the inverse matrix.



\subsection[kostkaFromLR]{Kostka coefficients from Littlewood--Richardson}
\begin{proposition}
For skew Kostka coefficients, we have the equality
\[
K_{\lambda/\mu,\nu} = \langle \schurS_{\mu} \cdot \completeH_{\nu} ,  \schurS_\lambda \rangle =
\langle \schurS_{\mu} \cdot \schurS_{(\nu_1)} \dotsm \cdot \schurS_{(\nu_\ell)} ,  \schurS_\lambda \rangle
\]
and the latter is a \hyperref[littlewoodRichardsonModels]{Littlewood--Richardson coefficient} $c^{\tau}_{\sigma\lambda}$
where $\tau/\sigma$ is the skew shape obtained by "concatenating"
$\ell$ horizontal strips of sizes given by $\nu$, and the shape $\mu$.
See \cite[p.338]{StanleyEC2} for proof.
\end{proposition}


\subsection[kostkaRecursion]{Recursions for Kostka coefficients}


From the definition of Kostka coefficients, there is a straightforward recursion.
This recursion is equivalent to the \hyperref[schurPieriRule]{Pieri rule}.
Let $\mu = (\mu_1,\dotsc,\mu_\ell)$ be a partition with $\ell \geq 2$ parts, and let $\mu^{(\ell)}$
denote the partition $(\mu_1,\dotsc,\mu_{\ell-1})$ (this is $\mu$ with the last part removed).
Then
\[
 K_{\lambda \mu} = \sum_{\nu} K_{\nu,\mu^{(\ell)}}
\]
where the sum is taken over all partitions $\nu$ of size $|\lambda|-\mu_\ell$,
such that $\lambda/\nu$ is a horizontal strip.

See also the \hyperref[kostkaIdentities]{recursion for Kostka--Foulkes polynomials}.


In \cite[p.327]{Macdonald1995}, the following recursion is 
described. In fact, it generalizes to \hyperref[jack]{Jack polynomials}.
\begin{theorem}
First, let $d(\lambda)\coloneqq \partitionN(\lambda')-\partitionN(\lambda)$.
Let $\mu$ be a partition of length $\ell$. 
Then we have the recursion 
\[
 K_{\lambda, \mu} 
=
\frac{1}{ d(\lambda) - d(\mu)}
 \sum_{1 \leq i \lt j \leq \ell}
 \sum_{1 \leq r \leq \mu_j}
 (\mu_i - \mu_j) K_{\lambda, (\mu+r \evec_i - r \evec_j)^*}
\]
where
$(\mu+r \evec_i - r \evec_j)^*$ denotes the partition
obtained from $\mu$ by decreasing entry $i$ by $r$,
and increasing entry $j$ by $r$, and
rearrange in decreasing fashion.
The initial conditions are given by $K_{\mu, \mu}=1$,
and $K_{\lambda, \mu}=0$ unless $\lambda \geq \mu$ in \hyperref[partialOrderingsOnPartitions]{dominance order}.
\end{theorem}


\subsection[kostkaInverseRecursion]{Recursions for inverse Kostka coefficients}

A convenient recursive formula for the entries in $(K_{\lambda\mu})^{-1}$
is given in \cite{EgeciogluRemmel1990}.
\begin{theorem}[Eğecioğlu--Remmel, 1990]
Let $\lambda = (r_1^{j_1}, r_2^{j_2},\dotsc,r_k^{j_k})$ where $r_1 \lt r_2 \lt \dotsb$.
Let $\mu = (\mu_1,\dotsc,\mu_\ell)$ in \emph{increasing} order.
We have that
\[
K^{-1}_{\lambda\mu} = \sum_{\substack{1 \leq j \leq d \\  r_j = \mu_i+j-1}} (-1)^{i-1}
K^{-1}_{\lambda[j], (\mu_1-1,\mu_2-1,\dotsc,\mu_{i-1}-1,\mu_{i+1},\dotsc,\mu_{\ell})}
\]
where $\lambda[j]$ denotes the partition obtained from $\lambda$ with one part equal to $j$ removed.
\end{theorem}
An implementation of this recursion in Mathematica can be found in
this \href{https://github.com/PerAlexandersson/Mathematica-packages}{github repository}.

In \cite{Carbonara1998}, a $t$-version of the Eğecioğlu--Remmel result is proved, 
where $(K_{\lambda\mu}(t))^{-1}$ is given a combinatorial interpretation.


A similar-looking but different recursion is given in \cite[Thm. 3]{Duan2003}.

\begin{theorem}[H. Duan, 2003]
Let $\lambda = (r_1^{j_1}, r_2^{j_2},\dotsc,r_k^{j_k})$ where $r_1 \lt r_2 \lt \dotsb$.
Let $\mu = (\mu_1,\dotsc,\mu_\ell)$ in \emph{increasing} order.
We have that
\[
K^{-1}_{\lambda\mu} = \sum_{r_j \geq \mu_\ell} (-1)^{r_j-\mu_\ell}
\sum_{\nu} K^{-1}_{\lambda[j], \nu}
\]
where $\lambda[j]$ denotes the partition obtained from $\lambda$ with one part equal to $j$ removed,
and the inner sum is taken over all $\nu$ such that $\mu/\nu$ is a vertical strip of size $r_j-\mu_\ell$.
\end{theorem}



See also \cite{WheelerZinnJustin2018}, where the coefficients appearing in 
\[
\schurS_\mu \hallLittlewoodP_\nu(t) = \sum_\lambda \bar{K}^\lambda_{\mu\nu}(t) \schurS_{\nu}
\]
are studied.


\todo{
Quantum Kostka and inverse
https://link.springer.com/content/pdf/10.1007/s10468-016-9640-5.pdf
}




\section[kostantPartitionFunction]{The Kostant partition function}

See \name{P. Harris}, \name{E. Insko} and \name{M. Omar} \cite{HarrisInskoOmar2018} for a short introduction.

Let $\evec_{ij}$ denote the vector with $1$ at index $i$ 
and $-1$ at index $j$ and remaining entries equal to $0$. 
We refer to this collection of vectors as \hyperref[root-systems]{positive roots} or just roots for short.
Given an integer vector $w$ of length $n$, 
let \defin{Kostant's partition function} $\kostantP(w)$ be the number of solutions to
\[
w = \sum_{(i,j) \in \binom{[n]}{2}} a_{ij} \evec_{ij}
\]
with $a_{ij}\geq 0$ being non-negative integers.
This is the definition in root system type $A_n$ --- there are analogues of 
the partition function in other types.
Clearly $\kostantP(w)=0$ unless $w_1+w_2+\dotsb+w_n=0$.

There is a $q$-analogue of $\kostantP(w)$, denoted $\kostantP_q(w)$
introduced by Lustzig \cite{Lusztig1983}. It is defined as 
\[
\kostantP_q(w) \coloneqq \sum_{ (a_1,\dotsc,a_\ell) } q^{a_1+\dotsb +  a_\ell}
\]
where we sum over all non-negative solutions $(a_1,\dotsc,a_\ell)$
that expresses $w$ as a non-negative linear combination of roots.

From the definition above, we have that
\[
\prod_{1 \leq i \lt j \leq n} \left( 1 - q\frac{x_i}{x_j} \right)^{-1} 
= \sum_{ w \in \setZ^n } \kostantP_q(w) \xvec^w.
\]


This has a close relationship with the \hyperref[schurJacobiTrudi]{Jacobi--Trudi identity}, see
the excellent introduction in \cite{DesarmenienLeclercThibon1994}.


\begin{example}
Let $w=(5,2,-7)$, and our roots are $\evec_{12}$, $\evec_{13}$ and $\evec_{23}$.
We have that $\kostantP(w) = 6$, since there are $6$ solutions with non-negative coefficients:
\[
052, 143, 234, 325, 416, 507
\]
For example, $(5,2,-7) = 3\evec_{12} + 2\evec_{13} + 5\evec_{23}$.
Furthermore, 
\[
\kostantP_q(w) = q^{12}+q^{11}+q^{10}+q^9+q^8+q^7.
\]
\end{example}


\begin{lstlisting}
KostantPartitionFunction[w_List] /; Tr[w] != 0 = {};
KostantPartitionFunction[w_List] := Module[
   {a, n = Length[w], roots, idx, vars, sol},
   idx = Subsets[Range[n], {2}];
   roots = Table[SparseArray[ss->{1,-1},{n}], {ss,idx}];
   vars = a /@ idx;
   sol = Solve[
     And @@ Join[Thread[vars >= 0],
       Thread[w == vars.roots]]
     , vars, Integers];
   If[Length[sol] == 0, {}, vars /. sol]
];
KostantPartitionFunction[{5,2,-7}]
\end{lstlisting}


Kostant's partition function is related to juggling sequences,
see \cite{BenedettiHanusaHarrisMoralesSimpson2020}.

The number of summands used in expressing $w$,converges to a Gaussan distribution,
see \cite{HarrisRahmoellerSchneider2019x}. Similar properties for other root systems hold.



\section[kostantPartitionFunctionOtherTypes]{Kostant partition function in general types}

In more generality, let $\Phi^+ \subset \Phi$ be a set of positive roots in a root system.
The \defin{Kostant partition function} for $\Phi^+$ is defined via the relation
\begin{equation*}
\prod_{\alpha \in \Phi^+}  \frac{1}{1-e^\alpha} = \sum_{w} \kostantP(w) e^w.
\end{equation*}
Note that $e$ is just a formal variable in this context, but with this convention,
$e^{\alpha} e^{\beta} = e^{\alpha + \beta}$.
The $q$-analogue is then defined via
\begin{equation*}
\prod_{\alpha \in \Phi^+}  \frac{1}{1-q e^\alpha} = \sum_{w} \kostantP_q(w) e^w.
\end{equation*}
Thus, the coefficient $[q^j]\kostantP_q(w)$ count the number of ways to express $w$
as a sum of exactly $j$ positive roots.

In \cite{RassartThesis}, a different $q$-analog, defined via
\begin{equation*}
\prod_{\alpha \in \Phi^+}  \frac{1+(q-1)e^\alpha}{1- e^\alpha} = \sum_{w} \kostantP^*(w;q) e^w,
\end{equation*}
is considered. Then $[q^j]\kostantP^*(w;q)$ is the number ways to express $w$
as a sum of positive roots, where exactly $j$ roots appear with a positive multiplicity.



See \url{http://de.arxiv.org/pdf/1111.6648.pdf} for a connection with Fibonacci numbers.

\todo{Expand this}


\subsection[kostkaFromKostant]{Kostka coefficients from Kostant's partition function}

Let $\lambda$ and $\mu$ be integer partitions of $n$.
Then the \defin{Kostant's multiplicity formula} (see \cite[(1.1.5)]{Kostant1959}) states that
\[
K_{\lambda,\mu} = \sum_{\sigma \in \symS_n} 
(-1)^{\length(\sigma)} \kostantP\left( \sigma(\lambda + \rho) - (\mu+\rho) \right)
\]
where $\rho = (n-1,n-2,\dotsc,2,1,0)$.
This formula is used to prove polynomiality of stretched Kostka coefficients, see \cite{BilleyGuilleminRassart2004}.
Note that the above formula is really not an efficient way to compute Kostka coefficients.


The \defin{Kostka--Foulkes polynomials} $K_{\lambda,\mu}(q)$ can be computed via the $q$-analogue:
\[
K_{\lambda,\mu}(q) = \sum_{\sigma \in \symS_n} 
(-1)^{\length(\sigma)} \kostantP_q\left( \sigma(\lambda + \rho) - (\mu+\rho) \right).
\]

\begin{theorem}[See \cite{DesarmenienLeclercThibon1994}]
We have that
\[
\completeH_\mu(\xvec) = \sum_{\alpha \in \setZ^n} \kostantP(\alpha)\schurS_{\mu + \alpha}(\xvec).
\]
\end{theorem}


\subsection[kostkaFromContingency]{Kostka coefficients from contingency tables}

Let $\lambda$ and $\mu$ be partitions of lengths at most $n$
and let and $\rho = (n-1,n-2,\dotsc,2,1)$.
Moreover, let $M_\alpha[\beta]$ be the number of non-negative integer matrices
with row sums given by $(\alpha_1,\dotsc,\alpha_n)$
and column sums given by $(\beta_1,\dotsc,\beta_n)$.
Thus, $M_\alpha[\beta]$ count certain contingency tables.

By using the Jacobi--Trudi identity, it is then 
straightforward to derive the following identity for skew Kostka coefficients:
\[
 K_{\lambda/\mu,\nu} = [\monomial_\nu] \schurS_{\lambda/\mu} = 
 \sum_{\sigma \in \symS_n} 
   (-1)^\sigma M_\nu[\sigma(\lambda+\rho) - (\mu+\rho)].
\]

\begin{proof*}
We do the proof in the non-skew case. The skew case is similar.
Recall that for $\lambda = (\lambda_1,\dotsc,\lambda_\ell)$,
the Jacobi--Trudi identity states that
\begin{equation}
 \schurS_\lambda(\xvec) = 
 \left| \completeH_{\lambda_i - i + j}(\xvec) \right|_{1 \leq i,j \leq \ell} .
\end{equation}
By letting $\rho \coloneqq (n-1,n-2,\dotsc,2,1,0)$,
we can write this as 
\begin{equation}
 \schurS_\lambda(\xvec) = 
 \left| \completeH_{\lambda_i + \rho_i - \rho_j}(\xvec) \right|_{1 \leq i,j \leq \ell} 
\end{equation}
Expanding the determinant, we recognize this as
\begin{equation}
 \schurS_\lambda(\xvec) 
   = \sum_{\sigma \in \symS_n} 
   (-1)^\sigma \prod_{j=1}^{\ell} \completeH_{\sigma(\lambda+\rho)_j - \rho_j}(\xvec) 
   = \sum_{\sigma \in \symS_n} 
   (-1)^\sigma \completeH_{\sigma(\lambda+\rho) - \rho}(\xvec).
\end{equation}
Recall now that 
\[
  \completeH_\alpha(\xvec) = \sum_{\beta} M_\alpha[\beta] m_\beta(\xvec),
\]
where $M_\alpha[\beta]$ count the number of non-negative integer 
matrices with row sums given by $\alpha$ and column sums given by $\beta$.
This observation completes the proof.
\end{proof*}

\subsection[littlewoodRichardsonFromKostant]{Littlewood--Richardson coefficients from the partition function}


Let $\lambda$, $\mu$ and $\nu$ be partitions of length at most $k$,
and such that $|\lambda|+|\mu| = |\nu|$.
Set
\[
\delta \coloneqq \frac{1}{2}(k-1,k-3,\dotsc,3-k,1-k).
\]

\begin{theorem}[Steinbergs formula \cite{Steinberg1961,RassartThesis}]
We have that the \hyperref[littlewoodRichardsonModels]{Littlewood--Richardson coefficient} 
$c^{\nu}_{\lambda \mu}$ is given by
\[
c^{\nu}_{\lambda \mu} = 
\sum_{\sigma, \tau \in \symS_k} (-1)^{\inv(\sigma \tau)}
\cdot 
\kostantP\left(
\sigma(\lambda+\delta)
+
\tau(\mu+\delta)
-
(\nu+2\delta)
\right).
\]
\end{theorem}


\begin{lstlisting}
LittlewoodRichardson[lam_List, mu_List, nu_List] := 
  Module[{perms, nlam, nmu, nnu, k, delta},
   k = Max[Length /@ {lam, mu, nu}];
   perms = Permutations@Range[k];
   {nlam, nmu, nnu} = PadRight[#, k] & /@ {lam, mu, nu};
   delta = 1/2 Table[j, {j, k - 1, -(k - 1), -2}];
   Sum[
    With[{
      vec = (nlam + delta)[[sigma]]
        +
        (nmu + delta)[[tau]]
        -
        (nnu + 2 delta)},
     (-1)^Inversions[sigma[[tau]]]
      Length[KonstantPartitionFunction[vec]]
     ]
    , {sigma, perms}, {tau, perms}]
   ];

(* This should return 1 *)
LittlewoodRichardson[{2, 2}, {3}, {4, 2, 1}]
(* This should return 2 *)
LittlewoodRichardson[{2, 1, 1}, {3, 1}, {4, 2, 1, 1}]

\end{lstlisting}


A similar expression for the \hyperref[schurP]{Schur P} structure 
coefficients can be found in \cite[Thm. 6.2.5]{RassartThesis}.
This can perhaps be seen as a type $B$ analog of Steinbergs formula.


In \cite{BerensteinZelevinsky1988} the authors
attributes the following formula to \name{B. Kostant}:
\begin{theorem}[Kostant]
Let $\lambda,\mu,\nu$ be partitions of length at most $\ell$, such that $|\nu|=|\lambda|+|\mu|$. Then
\begin{equation}
c^{\nu}_{\lambda \mu} = \sum_{\sigma \in \symS_\ell} \varepsilon(\sigma) K_{\mu, \sigma(\nu+\rho)-(\lambda+\rho)},
\end{equation}
where $\rho = \tfrac12 (\ell-1,\ell-3,\dotsc,1-\ell)$.
\end{theorem}
Shrivastava \cite{Shrivastava2022x} provides a modern proof of this formula.

A generalization of this formula exists for \hyperref[toricSchurIntoSchur]{Gromow--Witten invariants},
which show up in the study of \hyperref[schurToric]{toric Schur functions}.



\section[kostkaStretched]{Stretched Kostka coefficients}

See also \hyperref[lrStretched]{stretched Littlewood--Richardson coefficients}.

By using the \hyperref[gtKoskta]{Gelfand--Tsetlin polytope model for semistandard Young tableaux},
we can obtain the Kostka coefficients $K_{\lambda,\mu}$ as the number of lattice points 
in certain non-integral (some vertices might not have integer coordinates) polytopes.
This implies that the map $k \mapsto K_{k\lambda,k\mu}$ is a quasi-polynomial in $k$.
However, we have the stronger statement, due to \name{Derksen} and \name{Weyman} \cite{DerksenWeyman2002}:
\begin{theorem}
The map $k \mapsto K_{k\lambda,k\mu}$ is a polynomial in $k$.
\end{theorem}
In fact, this follows from the analogous statement for Littlewood--Richardson coefficients.

The proof idea (both for Kostka and Littlewood--Richardson coefficients) 
is to express $K_{k\lambda,k\mu}$ using \hyperref[kostkaFromKostant]{Kostant's partition function}.
It then remains to prove that
for fixed vectors $\alpha \in \setN^n$ and $\beta \in \setZ^n$,
the map $k \mapsto \kostantP\left( k \alpha + \beta \right)$ is 
a polynomial for large enough $k$.
This fact follows \cite[Thm. 1]{Sturmfels1995} together with the fact that the 
vector partition function $\phi_A$ in type $A$ (another name for $\kostantP$)
arises from a unimodular matrix (all minors have determinant in $\{-1,0,1\}$).
Unimodularity for the matrix corresponding to Kostant's partition function
follows easily from \cite[Appendix]{HellerTompkins1957}.


\hyperref[schurFlagged]{Flagged skew Kostka coefficients} can be expressed using contingency tables.
From this expression, polynomiality also follows, see \cite{AlexanderssonOguz2023x}.



\section[kostkaFoulkes]{Kostka--Foulkes polynomials from charge}

For an excellent background on charge, see \cite[Chapter 2.4]{Butler1994}.

Here is a longer explanation on how to compute the coefficients $K_{\lambda,\mu}(q)$,
the Kostka--Foulkes polynomials by using charge.
These coefficients appear in the expansion of Schur polynomials into 
\hyperref[hallLittlewoodP]{Hall--Littlewood polynomials}.
The combinatorial model using charge was first described by Lascoux and Schützenberger
in \cite{LascouxSchutzenberger78}. 

First we need to introduce the \defin{charge} of a permutation.
For $\sigma \in \symS_k$, we let 
\begin{equation*}
 \charge(\sigma) \coloneqq \maj(\rev(\sigma^{-1})) = 
 \sum_{i \notin \DES(\sigma^{-1})} (k-i).
\end{equation*}
For example, 
\[
\charge(198423765) = \maj(\rev(156498732) ) = \maj(237894651) = 20.
\]

The \defin{cocharge} of a permutation, $\cocharge(\pi)$ is 
defined as $\binom{n}{2}-\charge(\pi)$.

\begin{example*}[Computing cocharge]
Computing $\cocharge(198423765)$ can be done as follows.
Write the permutation and write a 0 as a subscript for $1$.
\[
1_0\; 9\; 8\; 4\; 2\; 3\; 7\; 6\; 5
\]
We then move to the next larger element in the permutation with no subscript.
If moving to the left, the subscript is increased by one, otherwise it stays the same.
The first two steps here is to the right, so the subscripts are set to $0$.
\[
1_0\; 9\; 8\; 4\; 2_0\; 3_0\; 7\; 6\; 5
\]
We then move to the left,
\[
1_0\; 9\; 8\; 4_1\; 2_0\; 3_0\; 7\; 6\; 5,
\]
and then to the right:
\[
1_0\; 9\; 8\; 4_1\; 2_0\; 3_0\; 7\; 6\; 5_1.
\]
Proceeding in this fashion, we arrive at
\[
1_0\; 9_5\; 8_4\; 4_1\; 2_0\; 3_0\; 7_3\; 6_2\; 5_1.
\]
The cocharge is now the sum of the subscripts, $0+5+4+1+0+0+3+2+1 = 16$.
This agrees with the fact that $16+20 = 36 = \binom{9}{2}$.
\end{example*}



\subsection[standardSubwords]{Standard subwords}

Given a word $w$ with content $\mu \vdash n$,
partition its entries into \defin{standard subwords} as follows.
Start from the right of $w$ and mark the first occurrence of $1$.
Proceed to the left and mark the first occurrence of $2$,
then $3$ and so on, wrapping around the end if nessecary,
until $\mu'_1$ entries have been marked.
This subword is the first standard subword of $w$.
Remove this subword, and repeat the process to find the second standard subword,
of length $\mu'_2$.

For example, the first standard subword in $w = 2 1 1 2 3 5 4 3 4 1 1 2 2 3$
has been underlined.
\[
2, 1, 1, \underline{2}, 3, \underline{5}, 4, 3, \underline{4}, 1, \underline{1}, 2, 2, \underline{3}.
\]
In total, we have four standard subwords in $w$, with corresponding charge values
\[
 \charge(25413) = 3,\quad 
 \charge(2431)=2 \quad 
 \charge(132)=2  \quad 
 \charge(12) = 1,
\]
and we define $\charge(w)$ as the sum of the charge values of the standard subwords.
Hence  $\charge(w) = 8$ for our particular word.


\subsection[computingKostkaFoulkes]{Charge of SSYT}

For a semistandard Young tableau $T \in \SSYT(\lambda,\mu)$, the reading word of $T$, $\rw(T)$
is given by concatenating the rows of $T$, starting with the bottom-most row.
We then define $\charge(T) = \charge(\rw(T))$,
and the \defin{Kostka--Foulkes polynomial} $K_{\lambda,\mu}(q)$ may be computed as
\[
 K_{\lambda,\mu}(q) = \sum_{T \in \SSYT(\lambda,\mu)} q^{\charge(T)}.
\]

Similarly, the \defin{modified Kostka--Foulkes polynomials} 
are defined as
\[
 \tilde{K}_{\lambda,\mu}(q) = \sum_{T \in \SSYT(\lambda,\mu)} q^{\cocharge(T)}.
\]
We have the relation 
\[
\tilde{K}_{\lambda,\mu}(q) = q^{\partitionN(\mu)} K_{\lambda,\mu}(q^{-1}).
\]

\begin{example}[Computing a Kostka--Foulkes polynomial]

We shall now compute $K_{421,3211}(q)$. 
There are four tableaux in $\SSYT(421,3211)$:

\begin{figure}
\ytableaushort{1114,22,3}
\ytableaushort{1113,22,4}
\ytableaushort{1112,24,3}
\ytableaushort{1112,23,4}
\end{figure}



The corresponding standard subwords and charge values are given by
\begin{equation*}
 \underset{1+0+0}{ 3214,\; 21,\; 1 }
 \qquad
 %
\underset{2+0+0}{4213,\; 21,\; 1 }
 \qquad
 %
 \underset{1+1+0}{ 3241,\; 12,\; 1}
 \qquad
 %
 \underset{2+1+0}{4231,\; 12,\; 1}
\end{equation*}
Hence, $K_{\lambda\mu}(q) = q+2q^2+q^3$.
\end{example}



\begin{example*}[Table of $K_{\lambda,\mu}(q)$, for size $4$]
Here are all $K_{\lambda,\mu}(q)$ for partitions of size $4$.
The rows are indexed by $\lambda$.

\begin{array}{rrrrrr}
\toprule
 \; & \textbf{4} & \textbf{31} & \textbf{22} & \textbf{211} & \textbf{1111} \\
\midrule
 \textbf{4} & 1 & 0 & 0 & 0 & 0 \\
 \textbf{31} & q & 1 & 0 & 0 & 0 \\
 \textbf{22} & q^2 & q & 1 & 0 & 0 \\
 \textbf{211} & q^3 & q^2+q & q & 1 & 0 \\
 \textbf{1111} & q^6 & q^5+q^4+q^3 & q^4+q^2 & q^3+q^2+q & 1 \\
 \bottomrule
 \end{array}

 \end{example*}


\subsection[chargeProperties]{Properties of charge}


The \defin{elementary Knuth transforms} are transformations on three 
adjacent letters on words. We allow the transformations $yzx \leftrightarrows yxz $ whenever $x \lt y\leq z$
and $xzy \leftrightarrows zxy$ whenever $x \leq y \lt z$.
Two words are \defin{Knuth-equivalent}, denoted $\equiv$ if one can obtain one from the other via a sequence
of elementary Knuth transforms. 
See also the close relationship with the \hyperref[rsk]{Robinson--Schensted--Knuth correspondence}.

For words of partition weight, we have some extra nice properties:
Each equivalence class contains a unique word which is the reading word of a semi-standard Young tableau.
Furthermore, one can show that $w \equiv w'$ then $\charge(w)=\charge(w')$,
see \cite[Cor. 2.4.38]{Butler1994}.



\begin{theorem}[Proved in \cite{LascouxSchutzenberger78}]

Charge on words is the unique statistic with the following properties:
\begin{itemize}
\item $\charge(\emptyset) = 0$, 
\item $\charge(w) = \charge(\sigma\cdot w)$, where $\sigma$ 
act via \hyperref[crystal-operators]{Lascoux--Schützenberger involutions},
\item if $x \neq 1$ and $wx$ has partition-content, then $\charge(wx) = \charge(xw)+1$,
\item if $w1^m$ has partition-content and $w$ contains no ones, then $\charge(w1^m) = \charge(w)$,
\item if $w$ and $w'$ are Knuth-equivalent, then $\charge(w) = \charge(w')$.
\end{itemize}
\end{theorem}
This is stated as above in \cite[p.54]{Nelsen2005}.

One can also obtain the charge via crystal operators \cite{Patimo2021x}.


\subsection[skewKostkaFoulkes]{Skew Kostka--Foulkes polynomials}

The \defin{skew Kostka--Foulkes polynomials} $K_{\lambda/\mu,\nu}(q)$
are defined as the coefficients in the expansion
\[
\schurS_{\lambda/\mu}(\xvec) = \sum_{\nu}K_{\lambda/\mu,\nu}(q) \hallLittlewoodP_{\nu}(\xvec;q).
\]

One can show that the above formula generalizes, see \cite[Corr. 2.5.8]{Butler1994}.
\[
K_{\lambda/\mu,\nu}(q) =  \sum_{T \in \SSYT(\lambda/\mu,\nu)} q^{\charge(T)}.
\]


One can also use \hyperref[riggedConfigurations]{rigged configurations}
to compute Kostka--Foulkes polynomials in an efficient manner.
 
 
  \subsection[cyclage]{Cyclage}
 

In \cite{LascouxSchutzenberger1981},  \name{Lascoux} and \name{Schützenberger} introduce 
the notion of \defin{cyclage}. Let $\mu$ be a partition,
and consider the set $\SSYT(\cdot,\mu) \coloneqq \bigcup_\lambda \SSYT(\lambda,\mu)$.
That is, the set of all semistandard Young tableaux with content $\mu$.
They define a poset structure on $\SSYT(\cdot,\mu)$ via the covering relations
\[
 T \prec T' \text{ if there is $i\geq 2$ such that } \rw(T)\cdot i \equiv i\cdot \rw(T'),
\]
where $\equiv$ denotes Knuth equivalence. This is in fact a graded poset,
whose grading is given by cocharge.
See \cite[Section 5.6]{Lothaire2002} for more properties of cyclage.

\begin{example*}[Figure of a cyclage poset]
The charge is displayed on the right hand side of each tableau.
\begin{figure}
  \svgimg[width=0.4\textwidth]{svg-images/cyclagePoset.svg}{Cyclage poset}
\end{figure}
\end{example*}
 
 
\subsection[catabolism]{Catabolism}
 
 Catabolism is defined in \cite[Exercise 5.6.1]{Lothaire2002} as follows.
 Given a tableau $T$ of shape $\lambda$, let $F$ be the first row of length $\lambda_1$
 and $R$ be the tableau consisting of the remaining rows.
 Perform row insertion on $F \cdot \rw(R)$. 
 This gives a new tableau with the same weight as $T$, 
 and charge has increased by $|\lambda|-\lambda_1$.
 Tableaux with only one row are the only fixed points under catabolism.
 
\begin{example*}[Catabolism of a tableau]
In this example, we start with
\begin{figure}
\ytableaushort{11,23,4}
\end{figure}
The top row is moved to the bottom, to create a skew shape.
\begin{figure}
\ytableaushort{{\none}{\none}23,{\none}{\none}4,11}
\end{figure}
We then perform \hyperref[jeu-de-taquin]{jeu-de-taquin} to get a straight shape.

\begin{figure}
\ytableaushort{{\none}{\none}23,{\none}{\none}4,11}
\ytableaushort{{\none}{\none}23,{\none}14,1}
\ytableaushort{{\none}{\none}23,114}
\ytableaushort{{\none}123,1{\none}4}
\ytableaushort{1123,{\none}{\none}4}
\ytableaushort{1123,4}
\end{figure}

The final tableau is the result of catabolism.
This procedure is equivalent with the one described above.
\end{example*}

\begin{example*}[Catabolism poset]
The charge is displayed on the right hand side of each tableau.
\begin{figure}
  \svgimg[width=0.4\textwidth]{svg-images/catabolism.svg}{Catabolism poset}
\end{figure} 
\end{example*}
 
\todo{
 See \url{https://math.berkeley.edu/~mhaiman/ftp/cdm/cdm.pdf}.
}
 
 
\subsection[kostkaIdentities]{Identities}

Using the \hyperref[rsk]{Robinson--Schensted--Knuth correspondence}, one can show that
\[
\sum_{\lambda \vdash n}  K_{\lambda \mu} K_{\lambda \nu} = N_{\mu\nu},
\]
where $N_{\mu\nu}$ is the number of $n \times n$-matrices with entries in $\setN$,
row sums are determined by $\mu$ and column-sums are determined by $\nu$.

There is a $q$-analogue of this in the case $\nu=1^n$, see \cite[Corolloary 1.3]{Kirillov2000}.
\[
\sum_{\lambda \vdash n}  K_{\lambda \mu} K_{\lambda, 1^n}(q) = q^{n(\mu')} \qbinom{n}{\mu_1,\dotsc,\mu_\ell}_q.
\]


\begin{theorem}[See \cite{BryanJing2016}]
For a partition $\lambda$ let 
\[
\lambda^{(i)} \coloneqq (\lambda_1+1,\lambda_2+1,\dotsc,\lambda_{i-1}+1,\lambda_{i+1},\dotsc,\lambda_{\ell}).
\]
Furthermore, $\lambda^{[i]}\coloneqq (\lambda_{i+1},\dotsc,\lambda_\ell)$.
With these definitions, we have the following recursion for Kostka--Foulkes polynomials:
\[
K_{\lambda\mu}(q) = \sum_{i=1}^{\mu_1} (-1)^{i-1} q^{\lambda_1-\mu_1-i+1} \sum_{\tau^i} K_{\tau^i,\mu^{[1]}}(q)
\]
where the sum runs over all partitions $\tau^i$ such 
that $\tau^i/\lambda^{(i)}$ are horizontal $(\lambda_1-\mu_1-i+1)$-strips.
\end{theorem}

 
\section[kostkaFoulkesParabolic]{Parabolic Kostka polynomials}

The parabolic Kostka polynomials $K_{\lambda,R}(q)$ 
(or generalized Kostka polynomials) generalize the Kostka--Foulkes polynomials.
The following definition appear in \cite{ShimozonoWeyman2000}.

Let $R = (R^1,\dotsc,R^k)$ be a sequence of partitions,
and let $\lambda$ be a partition such that $|\lambda| = \sum_j |R^j|$.
Let $\gamma$ be the partition whose parts is the multiset of parts of the $R^j$.
Furthermore, let $\eta_j \coloneqq \length(R^j)$ and $N \coloneqq \sum_j \eta_j$.
Let $M$ be the $N\times N$ matrix with diagonal blocks of size $\eta_j$,
and let $R^+_\eta \subset [N]^2$ be the subset of entries in $M$ strictly above the blocks.


\begin{example*}
This is from \href{https://www.aimath.org/WWN/kostka/gksw.pdf}{this paper}.

Let $R=((4,4),(3),(1,1,1))$. Then $\eta = (2,1,3)$, $N=6$ and $\gamma = 443111$.
The matrix $M$ is as follows where $R^+_\eta$ consists of the lightgreen boxes.
\begin{figure}
\begin{ytableau}
*(lightgray) & *(lightgray) & *(lightgreen) & *(lightgreen) & *(lightgreen) & *(lightgreen) \\
*(lightgray) & *(lightgray) & *(lightgreen) & *(lightgreen) & *(lightgreen) & *(lightgreen) \\
  &   & *(lightgray) & *(lightgreen) & *(lightgreen) & *(lightgreen) \\
  &   &   & *(lightgray) & *(lightgray) & *(lightgray) \\
  &   &   & *(lightgray) & *(lightgray) & *(lightgray) \\
  &   &   & *(lightgray) & *(lightgray) & *(lightgray) 
\end{ytableau}
\end{figure}
Hence,
\[
R^+_\eta = \{13,14,15,16, \; 23,24,25,26, \; 34, 35, 36 \}.
\]
\end{example*}


Introduce $\kostantP^\eta_q(w)$, which is a generalization of the $q$-analogue of the 
\hyperref[kostkaFromKostant]{Kostant partition function}, via the relation
\begin{equation*}
\sum_{w \in \setZ^n} \kostantP^\eta_q(w) = \prod_{(i,j) \in R^+_\eta} \frac{1}{1-q\frac{x_i}{x_j}}.
\end{equation*}
Finally, the \defin{parabolic Kostka polynomial} $K_{\lambda,R}(q)$ is defined as
\begin{equation*}
K_{\lambda,R}(q) = \sum_{\sigma \in \symS_n} 
(-1)^{\length(\sigma)} \kostantP^\eta_q\left( \sigma(\lambda + \rho) - (\mu+\rho) \right),
\end{equation*}
where $\rho = (N-1,\dotsc,2,1,0)$.
It is straightforward from this definition to see that if each 
partition $R_i = (\mu_i)$, a single part, then $K_{\lambda,R}(q) = K_{\lambda,\mu}(q)$,
the standard \hyperref[kostkaFromKostant]{Kostka--Foulkes polynomial}.

One can show \cite[Prop. 8]{ShimozonoWeyman2000} that 
\[
K_{\lambda,R}(1) = \langle \schurS_\lambda, \schurS_{R^1} \schurS_{R^2} \dotsm \schurS_{R^k} \rangle,
\]
which is a type of generalized Littlewood--Richardson coefficient.


\subsection[kostkaFoulkesRect]{Rectangular special case}

When all $R^j$ are rectangles, the $K_{\lambda,R}(q)$ are \defin{Poincaré polynomials}.

Several conjecture regarding this special case appear in \href{https://www.aimath.org/WWN/kostka/gksw.pdf}{these notes}.
In particular, there is a connection with \hyperref[LLT]{LLT polynomials}.

When the part sizes in the rectangles $R^j$ are weakly decreasing, we have a \defin{dominant sequence of rectangles}.
In this case, it is knonw that $K_{\lambda,R}(q) \in \setN[q]$, via a rigged configuration proof.
A charge formula is conjectured in \cite[Conj. 27]{ShimozonoWeyman2000},
and later proved by M. Shimozono \cite[Thm. 11]{Shimozono2001}.
There,
\[
K_{\lambda,R}(q) = \sum_{T \in LRT(\lambda,R)} q^{\charge_R(T)}
\]
where $\charge_R$ is a generalization of charge, and $LRT(\lambda,R)$ is the set of certain Littlewood--Richardson tableaux.
See \cite{KirillovShimozono2002,KirillovSchillingShimozono2001,SchillingWarnaar1999} 
for more background.
 

\todo{
The algorithm I use to compute $K_{\lambda,\mu}$
is to count chains in the partition poset, where layer $j$
consist of partitions of size $\mu_1+\mu_2+\dotsb+\mu_j$,
and the top layer is just $\lambda$. Counting paths from bottom to top in this graph
can then be done efficiently with built-in functions in Mathematica.
}

\todo{For Sn-modules with grade Frob series, see Garsia-Procesi, 
https://doi.org/10.1016/0001-8708(92)90034-I
}


\subsection[kostkaFoulkesAtoms]{Atomic decomposition}


\href{http://fpsac2019.fmf.uni-lj.si/resources/Posters/76poster.pdf}{See this poster.}



\section[quantumKostka]{Quantum Kostka coefficients}

The quantum Kostka coefficients arises in the study of the
quantum cohomology of the Grassmanian, \cite{PetrovicRadovanovic2016}.
A combinatorial formula for these coefficients first appeared in \cite{BertramFontanineFulton1999}.

Let $\lambda \subseteq n \times k$ and suppose $\nu$ satisfies
$|\nu| = |\lambda| + m(n+k)$. Then the \defin{quantum Kostka coefficient} $K_{\lambda \nu}^{k,n}$
is given by
\[
K_{\lambda \nu}^{k,n} = \sum_{\rho} \sign(\rho/\lambda) K_{\rho \nu}
\]
where $K_{\rho \nu}$ is a regular Kostka coefficient and the sum ranges over
all partitions $\rho$ occupying at most $n$ columns,
obtained by adding $m$ ribbons (each containing $(n+k)$ boxes).
The sign $\sign(\rho/\lambda)$ is given by multiplying $(-1)^{n-width(r_j)}$
for all ribbons $r_j$ where the width is the number of columns the ribbon occupies.



\todo{
\section[kostkaFoulkesGeneralizations]{Generalizations}
See https://arxiv.org/pdf/math/9803006.pdf
This has charge in other types.
https://www.albany.edu/faculty/lenart/articles/charge.pdf
}

\todo{
Monoid of KF-polys:
https://doi.org/10.1007/s10801-019-00878-7
}


\todo{
Overview of q-identities.
https://www.emis.de/journals/SLC/opapers/s32leclerc.pdf
https://hal.archives-ouvertes.fr/hal-00123035v2/document
}