\metatitle{Border strip tableaux and the Littlewood map}
\metadescription{An introduction to border strip tableaux, k-ribbon tableaux, partition quotients, and the Littlewood quotient map, including definitions, examples, and connections to representation theory and combinatorics.}

\section[borderStripTableaux]{Border strip tableaux and the Littlewood map}

\todo{Add this nice reformulation of mn-rule: https://arxiv.org/pdf/1908.03741.pdf}

\todo{https://arxiv.org/pdf/2006.00035.pdf }

\todo{Add this extremely efficient recursion for character values. https://arxiv.org/pdf/1712.08023.pdf}

The notion of border-strip tableaux, also known as rim-hook tableaux,
show up in several areas, most notably
the \hyperref[murnaghanNakayamaRule]{Murnaghan--Nakayama rule} 
and the original definition of \hyperref[LLT]{LLT polynomials}.


\subsection[borderStripTableauxDefinition]{Border strip tableaux}

A \defin{border strip tableau} (also known as \defin{ribbon tableaux} or \defin{rim-hook tableax})
of shape $\lambda$ and weight $\mu$
is a filling of the diagram $\lambda$ such that exactly $\mu_i$ boxes are labeled $i$,
and rows and columns are weakly increasing.
Moreover, for each $i$, the boxes labeled $i$ must form a \hyperref[skewShapes]{border strip}.
We let \defin{$\BST(\lambda,\mu)$} denote this set of border strip tableaux.

\begin{example*}[Border-strip tableaux for $\lambda=44321$, $\mu=5333$.]
Here are the 9 tableaux in $\BST(44321, 5333)$.
\begin{figure}
\begin{ytableau}
1 & 2 & 2 & 2\\
1 & 3 & 4 & 4\\
1 & 3 & 4\\
1 & 3\\
1\\
\end{ytableau}
\begin{ytableau}
1 & 1 & 1 & 1\\
1 & 3 & 4 & 4\\
2 & 3 & 4\\
2 & 3\\
2\\
\end{ytableau}
\begin{ytableau}
1 & 1 & 1 & 1\\
1 & 2 & 4 & 4\\
2 & 2 & 4\\
3 & 3\\
3\\
\end{ytableau}
\begin{ytableau}
1 & 2 & 2 & 3\\
1 & 2 & 3 & 3\\
1 & 4 & 4\\
1 & 4\\
1\\
\end{ytableau}
\begin{ytableau}
1 & 2 & 2 & 2\\
1 & 3 & 3 & 3\\
1 & 4 & 4\\
1 & 4\\
1\\
\end{ytableau}
\begin{ytableau}
1 & 1 & 1 & 1\\
1 & 3 & 3 & 3\\
2 & 4 & 4\\
2 & 4\\
2\\
\end{ytableau}
\begin{ytableau}
1 & 1 & 1 & 1\\
1 & 2 & 2 & 2\\
3 & 4 & 4\\
3 & 4\\
3\\
\end{ytableau}
\begin{ytableau}
1 & 1 & 1 & 1\\
1 & 2 & 3 & 3\\
2 & 2 & 3\\
4 & 4\\
4\\
\end{ytableau}
\begin{ytableau}
1 & 1 & 1 & 1\\
1 & 2 & 2 & 2\\
3 & 3 & 3\\
4 & 4\\
4\\
\end{ytableau}
\end{figure}
\end{example*}
The \defin{height} of such a tableau is the sum of the heights of the border strips,
where the height of a border strip is the number of rows it touches, minus one.
We let $\mathrm{ht}(B)$ denote the height of $B$.

When all strips have the same size, several additional properties hold,
and we let \defin{$\BST(\lambda,d)$} denote the set of 
border-strip tableaux of shape $\lambda$ and all strips have size $d$.



\subsection[kRibbonTableaux]{$k$-ribbon tableaux}

There is a semi-standard version of border strip tableaux,
called \defin{$k$-ribbon tableaux}. Let $k \geq 1$ be an integer.
A \defin{$k$-horizontal strip} is a skew shape formed by a 
disjoint union of $k$-ribbons such that all their heads are in different columns.
A \defin{$k$-ribbon tableau} of shape $\lambda/\mu$ and weight $\nu$, 
is a sequence of shapes 
\[
\mu = \mu^0 \subseteq \mu^1 \subseteq \dotsb \subseteq \mu^{\ell} = \lambda
\]
such that each $\mu^{i+1}/\mu^{i}$ is a horizontal $k$-ribbon
consisting of $\nu_i$ ribbons labeled $i$.
This definition is from \cite{Ougz2020}; see \cite{Descouens2008} for an alternative 
way to phrase this.

We let $\SSYT^{(k)}(\lambda/\mu,\nu)$ denote the set of $k$-ribbon tableaux of shape $\lambda/\mu$
and weight $\nu$. Moreover, we let $K^{(k)}_{\lambda/\mu,\nu}$ be the cardinality of this set.
Note that for $k=1$ we recover the set of semi-standard Young tableaux, and the Kostka coefficients.

\begin{example*}[k-ribbon tableaux]
The following figure shows the elements in 
$\SSYT^{(3)}(4422,1111)$ where we have put the label of each $3$-ribbon on its head.
\begin{figure}
  \svgimg[width=0.8\textwidth]{svg-images/k-ribbon2211-1111.svg}{The 6 3-ribbon tableaux with shape 4422 and weight 1111.}
\end{figure}

The following figure shows the elements in $\SSYT^{(3)}(4422,211)$.
\begin{figure}
  \svgimg[width=0.4\textwidth]{svg-images/k-ribbon2211-211.svg}{The 3 3-ribbon tableaux with shape 4422 and weight 211.}
\end{figure}
\end{example*}


The notion of $k$-ribbon tableaux appear frequently when studying plethysm coefficients,
evaluations at roots of unity, and in the study of LLT polynomials.

We have \cite[p. 29]{DesarmenienLeclercThibon1994} that
\[
 \schurS_\nu[ \powerSum_k \circ \completeH_\mu ] = \sum_{\lambda}
 \epsilon_k(\lambda/\mu) K^{(k)}_{\lambda/\nu,\mu} \schurS_\lambda.
\]


\begin{theorem}[See \cite[p.29]{DesarmenienLeclercThibon1994}]
Let $\lambda/\mu$ be a skew shape and $\nu$ a weak composition.
Let $\xi$ be a primitive $j^\thsup$ root of unity.
Then
 \[
  K^{(j)}_{\lambda/\mu,\nu} =  (-1)^{|\nu|(j-1)}\varepsilon_k(\lambda/\mu) K_{\lambda/\mu,\nu^j}(\xi),
 \]
where $K_{\lambda/\mu,\nu^j}(q)$ is a \hyperref[kostkaFoulkes]{Kostka--Foulkes polynomial}.
\end{theorem}
For example, $K_{4422,2^31^31^3}(q)$ is given by
\begin{align*}
 &q^{25}+q^{24}+4 q^{23}+5 q^{22}+10 q^{21}+13 q^{20}+21 q^{19}+ \\
 &24 q^{18}+33 q^{17}+34 q^{16}+39 q^{15}+36 q^{14}+36 q^{13}+ \\
 &27 q^{12}+23 q^{11}+14 q^{10}+9 q^9+4 q^8+2 q^7
\end{align*}
and it evaluates to $-3$ at $q=e^{2\pi i/3}$, so $K^{(3)}(4422,211)=3$.



\subsection[partitionQuotient]{Partition quotients}

Background on this subsection can be found in \cite[p.12]{Macdonald1995},
and also \cite[Chapter 2.7]{JamesKerber1984}.
See also the \href{https://ghseeli.github.io/grad-school-writings/class-notes/algebraic-combinatorics.pdf}{lecture notes by G. Seelinger (taught by J. Morse)}.

Let $\ell \geq 2$ be a fixed integer. The \defin{$\ell$-core} of 
a partition $\lambda$, denoted $\widetilde{\lambda}$,
is the shape which remains after removing all possible $\ell$-ribbons from $\lambda$.
One can show that the $\ell$-core is independent of the choice of ribbons.
A partition is an $\ell$-core if and only if no 
box in $\ell$ has \hyperref[partitionCores]{hook-length} divisible by $\ell$.


The \defin{$\ell$-quotient} of a partition $\lambda$
is an $\ell$-tuple $(\lambda^0,\lambda^1,\dotsc,\lambda^{\ell-1})$
with several nice properties: 
\begin{itemize}
\item The $\ell$-quotient map
\[
\lambda \to \left( \widetilde{\lambda}; \; \lambda^0,\lambda^1,\dotsc,\lambda^{\ell-1} \right)
\]
is injective, and $|\lambda|  = |\widetilde{\lambda}| + \ell |\lambda^0| + \dotsb  + \ell |\lambda^{\ell-1}|$.
This explains the usage of the word \emph{quotient}.
\item 
Conjugating $\lambda$ conjugates the core and all partitions in the quotient.
\end{itemize}

\emph{Warning!} 
Other definitions of the quotient might give another cyclic rotation of the entries in the $\ell$-tuple.
See the discussion in Macdonald's book. 

There are several ways to compute the quotient, see \cite[p.12]{Macdonald1995}.
The following example shows one possible method.

\begin{example}[Computing the quotient]
Let $\lambda = 8766641$ and $\ell=3$. We label the path outlining the partition
as follows. Each east step is labeled with the content (row-index minus column-index) 
of the box above it, while north steps are labeled with the content of the box to its right. 
We only consider the remainder of the content mod $\ell$, which gives the labeling as in the figure.

To find $\lambda^j$, simply consider all boxes in $\lambda$, such that the label of the horizontal step in the same row,
and the vertical step in the same column both are labeled $j$.
After removing empty columns and rows, these boxes form the shape $\lambda^j$.
\begin{figure}
\svgimg[width=0.4\textwidth]{svg-images/partition-quotient.svg}{A partition and the 3-quotient}

\ytableaushort{{\;}{\;}{\;},{\;}{\;}{\;}}

\ytableaushort{{\;}{\;}}

\ytableaushort{{\;}{\;},{\;},{\;}}

\end{figure}

The $3$-quotient of $\lambda$ is therefore $(33,\;2,\; 211)$ in this case.
For example, $\lambda^0$ has shape $33$ as the boxes marked $\blacksquare$ form this shape.

The partition $\lambda$ defines a binary word, by considering the outline,
starting from the first column. A vertical step is represented by a 0, and a horizontal one by a $1$.
In this case, we have the word $\dotsc0010111011000101011\dotsc$.
Taking every third entry in this word gives the outline associated to one of the $\lambda^j$ ---
this can easily be seen from the above definition of quotient.
\end{example}


The quotient map extends to the so-called \hyperref[littlewoodMap]{Littlewood quotient map} described below.
The name \emph{Littlewood map} appears in \cite[p.92]{qtCatalanBook} so I stick with that.

See also the article \cite{BrunatNath2019}, which explores the quotient map from 
a \hyperref[prelimPartitions]{Frobenius coordinate} perspecive.



\subsection[littlewoodMap]{Littlewood's quotient map}

The \defin{Littlewood quotient map} sends an element in $\BST(\lambda,\ell)$
to an $\ell$-tuple $(T_0,\dotsc,T_{\ell-1})$ of standard Young tableax,
but where the entries $1,2\dotsc,$ are distributed among the the tableaux.
The shape of $T_j$ is $\lambda^j$, the $j^\thsup$ element in the $\ell$-quotient of $\lambda$.

It is easiest to explain the map with an example.

\begin{example}
Let $\lambda = 8766641$ and $\ell=3$. The $3$-core of $\lambda$ is $(2)$,
and the $3$-quotient is $(33,\;2,\; 211)$.

\begin{figure}
  \svgimg[width=0.3\textwidth]{svg-images/border-strip-example.svg}{A border-strip tableau of shape 8766641 with 3-strips.}
\begin{ytableau}
1 & 3 & 8 \\
2 & 11 & 12\\
\end{ytableau}
\begin{ytableau}
  6 & 7 \\
\end{ytableau}
\begin{ytableau}
  4 & 9 \\
  5 \\
  10 \\
\end{ytableau}
\end{figure}

The figure shows a border-strip tableau $B$, where the label has been written in each strip.
Recall that the content of a box $(r,c)$ is given by $c-r$.
In each strip, there is a unique box with minimal content --- this is where the label has been written.
Each label has a content whose remainder is in $\{0,1,\dotsc,\ell-1\}$, mod $\ell$.
Labels whose content-remainder is $j$, are in correspondence with boxes in $T_j$.
Furthermore, labels with the same content in $B$, also have the same content in $T_j$.

For example, the strips with labels in $\{1,2,3,8,11,12\}$ have content-remainder $0$,
and are therefore the labels in $T_0$.
\end{example}


\begin{lstlisting}
(* This code follows the description in [Macdonald1995] *)
PartitionCore[lam_List, d_Integer] := 
  Module[{m = Length@lam, xi, xis, mr, xiTildes},
   xi = PadRight[lam, m] + Range[m - 1, 0, -1];
   xiTildes = Sort[
     Join @@ Table[
       xis = Select[xi, Mod[# - r, d] == 0 &];
       mr = Length[xis];
       Table[d s + r, {s, 0, mr - 1}]
       , {r, 0, d - 1}], Greater];
   DeleteCases[MapIndexed[#1 - m + #2[[1]] &, xiTildes], 0]
   ];
PartitionQuotient[lam_List, d_Integer] := 
  Module[{m = Length@lam, xi, xis, mr},
   xi = PadRight[lam, m] + Range[m - 1, 0, -1];
   Table[
    xis = Select[xi, Mod[# - r, d] == 0 &];
    mr = Length[xis];
    MapIndexed[
     #1 - mr + #2[[1]] &,
     Sort[(xis - r)/d, Greater]
     ], {r, 0, d - 1}]
   ];
(* Skew shapes are given by skewing the quotients. *)
PartitionQuotient[{lam_List, mu_List}, d_Integer] :=
  MapThread[Join, {
    List /@ PartitionQuotient[lam, d],
    List /@ PartitionQuotient[mu, d]}, 1];
(* Example, from J.Haglund's qt-Catalan book *)
PartitionQuotient[{5, 5, 5, 5, 5, 5}, 3] == {{2, 2}, {2, 2}, {1, 1}}
\end{lstlisting}

\begin{example}
The following examples appear in the literature,
in \cite[p.13]{Lascoux97ribbontableaux}, \cite[p.94]{qtCatalanBook} and  \cite[Fig. 2.6]{Pak2000Ribbon},
respectively.

\begin{tabular}{rrl}
\toprule
 $\lambda$ & $\textbf{3-core}$ & $\textbf{3-quotient}$  \\
\midrule
$87^241^5$ & $211$ & $21,\; 22,\; 2$ \\
$5^6$ & $\emptyset$ & $22,\; 22,\; 11$   \\
$987^34$ & $\emptyset$ &  $33,\; 21,\; 32$  \\
\bottomrule
\end{tabular}
Note that I. Pak has a different labeling convention, so one 
needs to cyclically rotate the entries in the quotient for the calculations to agree.
\end{example}

It is straightforward to generalize this map to skew shapes,
sending elements in $\BST(\lambda/\mu,\ell)$ to $\ell$-tuples of skew tableaux,
see \cite{Wildon2018}. This is possible as long as $\lambda$ and $\mu$ have the same $\ell$-core,
otherwise $\BST(\lambda/\mu,\ell)$ is empty.
The Littlewood quotient map can be extended to 
semistandard $k$-ribbon tableaux.
This shows why the original definition 
of \hyperref[lltDefinitionRibbons]{LLT polynomials (using ribbon tableaux)} agree 
with the \hyperref[lltDefinitionTuples]{Bylund--Haiman combinatorial model} for LLT polynomials.


By using the Littlewood map, one can give an alternative proof of Thurston's theorem \cite{Thurston1990},
which shows that the flip graph of domino tilings of rectangles is connected.


\subsection[borderStripAbacus]{The abacus model}

Partition quotients and border-strip tableaux can also be described using abaci,
see \cite{Wildon2018}.


\subsection[borderStripHookFormulas]{Hook formulas for border strip tableaux}

Using Littlewood's \hyperref[littlewoodMap]{quotient map} the following result follows.

\begin{theorem}[Hook formula for border-strip tableaux, see \cite{JamesKerber1984,FominLulov1997}]
Let $\mu \vdash d\ell$. Then 
\[
|\BST(\mu,\ell)| = \binom{\ell}{|\mu^0|,|\mu^1|,\dotsc,|\mu^d|} f^{\mu^0} f^{\mu^1}\dotsm f^{\mu^{d-1}},
\]
where the $\mu^j$ are the partitions in the \hyperref[partitionQuotient]{$d$-quotient} of $\mu$.
\end{theorem}
There are analogs of this formula for shifted shapes as well as for trees.


The above formula can also be phrased as follows.
\begin{theorem}[The modular hook formula, see \cite{FominLulov1997}]
Let $\mu \vdash d\ell$. Then 
\[
|\BST(\mu,\ell)| = \frac{d!}{\prod_{\substack{s \in \mu \\ \ell | h(s) } } h(s)/\ell }
\]
where the product is over all boxes in $\mu$ such that the hook value is divisible by $\ell$.
\end{theorem}
A $q$-analog of this formula is given in \cite{MendesLindbloomAirey2019}.



\begin{theorem}[See \cite[Corollary 9]{White1983} and \cite[Thm. 2.7.27]{JamesKerber1984}]
Let $\lambda \vdash d\ell$. Then 
$|\BST(\lambda,d)| = |\chi^{\lambda}(d^\ell)|$.
In other words, the \hyperref[schurMurnaghanNakaygama]{Murnaghan--Nakayama rule} is cancellation-free.

In fact, \cite[Thm. 2.7.27]{JamesKerber1984} is the following statement.
Let $\lambda \vdash d\ell + m$ where the $d$-core $\widetilde{\lambda}$ has size $m$.
Suppose $\sigma \in \symS_n$ such that the type of $\sigma$ is $d^\ell,\mu$ where $\mu \vdash m$.
Then
\[
\chi^{\lambda}(\sigma) = \sign(\sigma) |\BST(\lambda/\tilde{\lambda},d)| \chi^{\tilde{\lambda}}(\mu) .
\]
\end{theorem}


