\metatitle{Other combinatorial families}
\metadescription{Set-valued tableaux, Catalan numbers, and so on}


\section[combinatorialObjects]{Other combinatorial objects}


\section[fussCatalanObjects]{Generalizations of Catalan numbers}

The \defin{$k$-Catalan numbers} or \defin{Fuß--Catalan numbers} are defined as
\[
  \catalan^k(n) \coloneqq \frac{1}{kn+1}\binom{kn+1}{n}.
\]
There are many interpretations of these,
see e.g. \url{https://www.sciencedirect.com/science/article/pii/S0012365X07009107}



The \defin{s-binomial coefficients} are defined via 
\[
  (1+x+x^2+\dotsb+x^s)^n = \sum_{j=0}^{sn} \binom{n}{j}_s x^j.
\]
The \defin{s-Catalan numbers} are then defined as 
\[
\catalan^s(n) \coloneqq \binom{2n}{sn}_s  - \binom{2n}{sn}_{s+1}.
\]
There is a relationship between s-Catalan numbers
and \hyperref[littlewoodRichardsonModels]{Littlewood--Richardson coefficients},
see \cite{Linz2021x}.
In fact, let $\delta = (n,n-1,\dotsc,2,1)$.
Then
\[
c^{2 \delta_{2n}}_{2 \delta_{2n-1} , (sn,sn)} = \binom{2n}{sn}_s  - \binom{2n}{sn}_{s+1}.
\]
Note that the left hand side is a polynomial in $s$, due to 
a result by E.~Rassart \cite{Rassart2004}.


\section[skewSYT]{Skew standard Young tableaux}

For formulas for the number of skew SYT, see \cite{MoralesZhu2020x}.


\section[parkingFunctions]{Parking functions}

See \cite{CarlsonChristensenHarrisJonesRodriguez2021} for many variations of parking functions.


\section[setValuedTableaux]{Set-valued tableaux}

\name{Anders Buch} introduced set-valued tableaux in \cite{Buch2002}, 
in order to study the K-theoretical Grassmannian.

In \cite{Drube2018}, it is shown that $\catalan^k(n)$
is equal to the number of set-valued SYT of shape $(n,n)$,
where each box in the first row has one entry, while the boxes in the second row
all have exactly $k-1$ entries. Note that for $k=2$, we recover the classical 
Catalan numbers, which count the number of SYT of shape $(n,n)$.
Note that the total number of entries is $kn$.
