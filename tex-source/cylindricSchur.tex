\metatitle{Cylindric Schur polynomials}
\metadescription{An introduction to cylindric Schur polynomials, including their definition via cylindric semi-standard tableaux, Schur expansions, connections to Gromov--Witten invariants, and positivity properties.}
\metakeywords{Cylindric Schur polynomials,Gromov--Witten invariants,Schur-positivity,toric Schur polynomials,affine Schur functions}

\section[schurCylindric]{Cylindric Schur polynomials}


\begin{polydata}{schurCylindric}
  Name   & Cylindric Schur polynomials \\
  Space  & Sym \\
  Basis  & No \\
  Rating & 5 \\
  Year   & 2005 \\
  Bib    & Postnikov2005 \\
\end{polydata}

\todo{http://igm.univ-mlv.fr/~fpsac/FPSAC05/ARTICLES/P21.pdf}


\todo{Cyclic sieving relation ? This has some cool stuff with Cylindric descents. https://arxiv.org/pdf/1710.06664.pdf}



The family of (skew) cylindric Schur functions were introduced by A. Postnikov in \cite{Postnikov2005}.
This is motivated by the fact that the coefficients in the Schur expansion 
of such cylindric Schur functions are given by 3-point Gromov--Witten invariants of the Grassmanian.
See also \cite{McNamara2006} for an introduction to this family of symmetric functions.

The cylindric Schur functions are a subset of \hyperref[schurAffine]{skew affine Schur functions},
which was proved by T. Lam, in \cite[Prop. 33]{Lam2006}.


\subsection[cylindricSchurDefinition]{Definition}

A \defin{cylindric shape} $\lambda$ is an infinite lattice path in $\setZ^2$,
invariant under shifts $(n-m,m)$ where $m \in [1,n-1]$ and 
let $C^{n,m}$ be the set of such paths.
Suppose $\lambda,\mu \in C^{n,m}$ and that $\mu$ always lie weakly to the left of $\lambda$,
then we say that $\mu \subseteq \lambda$ and $\lambda / \mu$ is a \defin{cylindric skew shape}.


A \defin{cylindric semi-standard tableau} of shape $\lambda / \mu$ is a sequence
\[
\mu = \lambda^0 \subseteq \lambda^1 \subseteq \dotsb \subseteq \lambda^{\ell} = \lambda
\]
such that each shape $\lambda^i/\lambda^{i-1}$ contains at most one box in each column.
Furthermore, we can think of placing the value $i$ in the boxes determined by 
$\lambda^i/\lambda^{i-1}$. 
The weight $(a_1,a_2,\dotsc,a_{\ell})$ of the tableau
is given by letting $a_i$ be the number of boxes in any $n-m$ 
consecutive columns of $\lambda^i/\lambda^{i-1}$.


The \defin{cylindric Schur function} $\schurCylindric_{\lambda/\mu}(\xvec)$
is then defined as
\[
\schurCylindric_{\lambda/\mu}(\xvec) = \sum_{T \in \mathrm{CSSYT(\lambda/\mu)}} \xvec^T
\]
where the sum is over all cylindric semi-standard tableaux of shape $\lambda/\mu$.
One can show via a variant of the Bender--Knuth involution 
that all $\schurCylindric_{\lambda/\mu}(\xvec)$ are symmetric functions.


\begin{example*}
In the following examples, $n=7$, $m=3$. 
This implies invariance when shifting $3$ steps up and $4$ steps to the right.
We show two cylindric semistandard tableaux, of two different shapes,
where exactly one representative of boxes under the shift action has been assigned an integer.
Note that the second example shows that the \hyperref[schurSkew]{skew Schur functions} is a proper subset 
of cylindric Schur functions.

\begin{figure}
\begin{ytableau}
\none & \none & \none &\none &\none &\none & \none  &\times & \times \\
\none & \none & \none &\none &\none &\times& \times &\times\\
\none & \none & \none &\none &\none &\times& \times &\times\\
\none & \none & \none &  1 & 1 & \times & \times  \\
\none & 1 & 2 & 2 \\
\none & 2 & 3 & 4\\
\times& 3 & 5 
\end{ytableau}
\begin{ytableau}
\none & \none &\none &\none &\none &\none & \none & \times \\
\none & \none &\none &\none &\none &\none & \times& \times \\
\none & \none &\none &\none &\times&\times& \times& \times \\
\none & \none &  \none & 1 \\
\none & \none &  1 & 2 \\
2 & 2 & 3 & 4
\end{ytableau}
\end{figure}

The first tableau has weight $(3,3,2,1,1)$ and the second one $(2,3,1,1)$.
The Schur expansion of the first shape gives 
\[
\schurCylindric_{\lambda/\mu} = \schurS_{433} + 
\schurS_{4321} - 
\schurS_{33211} + 
\schurS_{2221111} - 
\schurS_{211111111}.
\]
\end{example*}

A. Postnikov uses the notation $\lambda/d/\mu$ to describe a cylindric shape,
where $\lambda$ and $\mu$ are partitions and $d\geq 0.$
The cylindric shape is constructed as follows:
Draw the outline of the partition $\lambda$ in the plane, but shifted $d$ steps right and $d$
steps down, then the outline of $\mu$ without shift, and extend both outlines periodically.

In the example above, the first shape can be described as $(6,3,3)/1/(6,3,1)$
and the second shape is just $(4,4,4)/0/(3,2,0)$, so that when $d=0$
the shape is a regular skew shape.

For the poset structure on cylindric diagrams and connected to root systems and the Bruhat order,
see \cite{NakadaSuzukiToyosawa2023x}.

\todo{Give an example from McNamara2006, p.12, to explain Postnikov's notation}


Every cylindric Schur polynomial can be obtained as a \hyperref[schurAffineSkew]{skew affine Schur function}.
This implies in particular, that the support of the monomials in a skew Schur function form an \hyperref[MConvexity]{M-convex set}.

Note that cylindric Schur polynomials are not in general special cases of \hyperref[pPartitionDefinition]{$(P,w)$-partitions};
see e.g. \cite{McNamara2006} (this is easily seen from the fact that all $(P,w)$-partitions are positive in the 
fundamental quasisymmetric basis, but \hyperref[cylindricSchurPositivity]{"most" cylindric Schur functions are not}.


\section[cylindricAffineLittlewood]{Affine bounded Littlewood identities}

In \cite{HuhKimKrattenthalerOkada2023x}, the authors 
study bounded \hyperref[schurSums]{Littlewood identities} where the cylindric Schur functions appear.
They give proofs of \defin{affine Littlewood identities} which are analogs of 
the Littlewood identities for cylindric Schur functions.


\section[cylindricEquivariantQuantumPieri]{Equivariant Quantum Pieri rule}

For an equivariant quantum Pieri rule for the Grassmanian on cylindric shapes,
see \cite{BertigerEhrlichMilicevicTaipale2022x}.



\section[schurToric]{Toric Schur polynomials}

\begin{polydata}{schurToric}
  Name   & Toric Schur polynomials \\
  Space  & Sym \\
  Basis  & No \\
  Rating & 5 \\
  Year   & 2005 \\
  Bib    & Postnikov2005 \\
\end{polydata}

A shape $\lambda/\mu \in C^{n,m}$ is called \defin{toric} if
every column contains at most $m$ boxes. 
Cylindric Schur functions indexed by such shapes 
are referred to as \defin{toric Schur polynomials}.
All the usual skew shapes are toric.

The specialization
\[
\schurCylindric_{\lambda/d/\mu}(\xvec)(x_1,\dotsc,x_m) \coloneqq 
\schurCylindric_{\lambda/d/\mu}(x_1,\dotsc,x_m,0,\dotsc)
\]
is non-zero if and only of the shape $\lambda/d/\mu$ is toric.
These polynomials, in a finite set of variables, 
are referred to as \defin{toric Schur polynomials}.



\subsection[toricSchurIntoSchur]{Schur expansion}

Postnikov proves that when $\lambda/d/\mu$ is a toric shape in $C^{n,m}$
the coefficients $C_{\mu,\nu}^{\lambda,d}$ in the Schur expansion
\[
\schurCylindric_{\lambda/d/\mu}(x_1,\dotsc,x_m) = 
\sum_\nu C_{\mu,\nu}^{\lambda,d} \schurS_\nu(x_1,\dotsc,x_m)
\]
are non-negative integers given by certain \defin{Gromov--Witten invariants}.


\begin{problem}[Postnikov, \cite{Postnikov2005}]
It is an open problem to find a (positive) combinatorial rule for $C_{\mu,\nu}^{\lambda,d}$.
Note that such a rule would generalize the classical Littlewood--Richardson rule.
\end{problem}

Postnikov also introduces a certain family of \defin{toric Specht modules}
and conjecture that they decompose into irreducible Specht modules with
multiplicities given by the $C_{\mu,\nu}^{\lambda,d}$.


The coefficients $C_{\mu,\nu}^{\lambda,d}$ can be computed as an
\hyperref[littlewoodRichardsonFromKostant]{alternating sum of Kostka coefficients},
see \cite[Cor. 1.4]{Korff2009x}.

Another formula uses an alternating sum of regular Littlewood--Richardson coefficients, see
\cite[Eq. (21)]{BertramFontanineFulton1999}.
They show that for $\lambda, \mu,\nu \subseteq \ell \times k$,
$|\lambda|+|\mu| = |\nu|+mn$,
\[
C_{\lambda,\mu}^{\nu,m} = \sum_{\rho} \sign(\rho/\nu) c^{\nu}_{\lambda,\mu}, \qquad
\sign(\rho/\nu) \coloneqq \prod_{j} (-1)^{k-\mathrm{width}(r_j)},
\]
where the sum is over all diagrams $\rho$ obtained from
$\nu$ by adding $m$ ribbons, each ribbon $r_j$ containing $n$ boxes and starting from
the first row.


The $C_{\mu,\nu}^{\lambda,d}$ are closely \defin{fusion coefficients}
as seen in \cite{Korff2009x}.
For some positivity results on fusion coefficients, see \cite{MorseSchilling2012}.


Another relation notion are the \hyperref[quantumKostka]{quantum Kostka coefficients}.


\subsection[cylindricSchurPositivity]{Cylindric Schur positivity}

P. McNamara \cite{McNamara2006} asked for which cylindric shapes $\schurCylindric_{\lambda/d/\mu}(\xvec)$ are
Schur-positive (as proper symmetric functions in an infinite alphabet) and conjectured that
each cylindric skew Schur function can be positively expanded in 
certain \emph{non-skew} cylindric Schur functions, with coefficients being Gromov--Witten invariants.
P. NcNamara proves that only ordinary skew Schur functions are Schur-positive.
In fact, these are the only cylindric Schur functions that are positive in
the \hyperref[gessel]{Gessel fundamental quasisymmetric basis}, see \cite[Thm. 5.7]{McNamara2006}.


McNamara's conjecture was recently answered in \cite{Lee2017} in the following theorem:
\begin{equation*}
\schurCylindric_{\lambda/d/\mu}(\xvec) = \sum_{\nu \in C^{n,m}, e\geq 0}
c^{\lambda/d/\mu}_{\nu/e/\emptyset} \schurCylindric_{\nu/e/\emptyset}(\xvec).
\end{equation*}
The coefficients are all non-negative, and satisfy
$c^{\lambda/d/\mu}_{\nu/e/\emptyset} = c^{\lambda/d-1/\mu}_{\nu/e-1/\emptyset}$ whenever $e \gt 0$.
Furthermore, we have that $c^{\lambda/d/\mu}_{\nu/0/\emptyset} = C_{\mu,\nu}^{\lambda,d}$.



\section[completeHCylindric]{Cylindric complete homogeneous polynomials}

\begin{polydata}{completeHCylindric}
  Name     & Cylindric complete homogeneous polynomials \\
  Category & Elementary \\
  Space    & CSym \\
  Basis    & No \\
  Rating   & 1 \\
  Year     & 2005 \\
  Bib      & KorffPalazzo2020 \\
\end{polydata}

Introduced in \cite{KorffPalazzo2020}.


\section[elementaryECylindric]{Cylindric elementary polynomials}

\begin{polydata}{elementaryECylindric}
  Name    & Cylindric elementary polynomials \\
  Category& Elementary \\
  Space  & CSym \\
  Basis  & No \\
  Rating & 1 \\
  Year   & 2005 \\
  Bib    & KorffPalazzo2018 \\
\end{polydata}


Introduced in \cite{KorffPalazzo2020}.


\section[cylindricSchurSeeAlso]{See also}

C. Korff discusses connection with a \hyperref[6vertexModel]{6-vertex model} and Hecke characters, see \cite{Korff2020}.

